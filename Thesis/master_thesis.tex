\documentclass[12pt]{article}

%%%%%%%%%%%%%%%%%%%%%%%%%%%%%%%%%%%%%%%%%%%%%%%%%%%%%%%%%%%%%%%%%%%%%%%%%%%%%
%%%%%%%%%%%%%%%%%%%%%%%%%%%%%%%%%%%%%%%%%%%%%%%%%%%%%%%%%%%%%%%%%%%%%%%%%%%%%
% Packages
\usepackage{fontspec} % If used, change compiler to XeLaTex
\usepackage[utf8]{inputenc}
\usepackage[british]{babel}
\usepackage[british]{isodate}
\usepackage{csquotes}
\usepackage{mathtools}
\usepackage{geometry}
\usepackage[
backend=biber,
style=apa,
sorting=nyt, % Sort by name, year, title
backref=true,
dateabbrev=false,
language=british,
urldate=long
]{biblatex}
\usepackage{xcolor}
\usepackage[toc,page]{appendix}
\usepackage{svg}
\usepackage{float}
\usepackage{longtable}
\usepackage{multirow}
\usepackage{caption}
\usepackage{subcaption}
\usepackage{setspace}
\usepackage[symbols, nopostdot, automake, nonumberlist]{glossaries}
\usepackage{fancyhdr}
\usepackage{enumitem}

%%%%%%%%%%%%%%%%%%%%%%%%%%%%%%%%%%%%%%%%%%%%%%%%%%%%%%%%%%%%%%%%%%%%%%%%%%%%%
%%%%%%%%%%%%%%%%%%%%%%%%%%%%%%%%%%%%%%%%%%%%%%%%%%%%%%%%%%%%%%%%%%%%%%%%%%%%%
% Custom Settings
\addbibresource{master_thesis.bib} % Add bib source

\setlength\bibitemsep{1.5\itemsep} % Adjust distance of reference entries

\renewcommand*\appendixpagename{\Large Appendices} % Adjust size of title of the appendices section

\renewcommand*{\finentrypunct}{}
\renewbibmacro*{pageref}{%
  \addperiod% NEW
  \iflistundef{pageref}
    {}
%    {\printtext[parens]{% DELETED
    {\newline\footnotesize\printtext[parens]{% NEW
       \ifnumgreater{\value{pageref}}{1}
         {\bibstring{backrefpages}\ppspace}
     {\bibstring{backrefpage}\ppspace}%
 %      \printlist[pageref][-\value{listtotal}]{pageref}}}}% DELETED
       \printlist[pageref][-\value{listtotal}]{pageref}}}}% NEW
\DefineBibliographyStrings{english}{
  backrefpage={Cited on page},
  backrefpages={Cited on pages}
}

\setmainfont{Times New Roman} % If the usepackage fontspec is active, you can set Times New Roman here

\geometry{ % Adjust paper geometry
 a4paper,
 right=25mm,
 bottom=20mm,
 left=25mm,
 top=25mm}
\urlstyle{same} % Make URL font the same with the rest of the document
\onehalfspacing % Adjust spacing
\allowdisplaybreaks % Allow for breaks in equations across pages
\addto\captionsbritish{ % Replace "english" with the language you use
  \renewcommand{\contentsname}%
    {Table of Contents}%
} % Adjust title of the table of contents
\numberwithin{equation}{section} % Number equations with sections
\setcounter{tocdepth}{2}			% levels which are displayed in the table of contents

\pagestyle{plain}

% Define a custom name format that prints names in bold
\DeclareNameWrapperFormat{sortname}{\mkbibbold{#1}}
\DeclareNameWrapperAlias{author}{sortname}
\DeclareNameWrapperAlias{editor}{sortname}
\DeclareNameWrapperAlias{translator}{sortname}

%%%%%%%%%%%%%%%%%%%%%%%%%%%%%%%%%%%%%%%%%%%%%%%%%%%%%%%%%%%%%%%%%%%%%%%%%%%%%
%%%%%%%%%%%%%%%%%%%%%%%%%%%%%%%%%%%%%%%%%%%%%%%%%%%%%%%%%%%%%%%%%%%%%%%%%%%%%
% Glossary
\makeglossaries 

% Glossary entries
\newglossaryentry{nk}{
name=NK,
description={New Keynesian}
}

\newglossaryentry{ha}{
name=HA,
description={Heterogeneous Agent}
}

\newglossaryentry{hank}{
name=HANK,
description={Heterogeneous Agent New Keynesian}
}

\newglossaryentry{gfc}{
name=GFC,
description={Great Financial Crisis}
}

\newglossaryentry{ltv}{
name=LTV,
description={Loan-to-Value}
}

\newglossaryentry{dsge}{
name=DSGE,
description={Dynamic Stochastic General Equilibrium}
}

%%%%%%%%%%%%%%%%%%%%%%%%%%%%%%%%%%%%%%%%%%%%%%%%%%%%%%%%%%%%%%%%%%%%%%%%%%%%%
%%%%%%%%%%%%%%%%%%%%%%%%%%%%%%%%%%%%%%%%%%%%%%%%%%%%%%%%%%%%%%%%%%%%%%%%%%%%%
% hyperref package
\usepackage{hyperref} % Load after everything else
\usepackage{footnotebackref} % Brings you back to where the footnote is in the text
\hypersetup{ 
    colorlinks,
    linkcolor=blue,
    filecolor=blue,  
    citecolor=blue,
    urlcolor=blue} % Setup with the hyperref package

%%%%%%%%%%%%%%%%%%%%%%%%%%%%%%%%%%%%%%%%%%%%%%%%%%%%%%%%%%%%%%%%%%%%%%%%%%%%%
%%%%%%%%%%%%%%%%%%%%%%%%%%%%%%%%%%%%%%%%%%%%%%%%%%%%%%%%%%%%%%%%%%%%%%%%%%%%%
% Formalities
\title{Financial Constraints and Household Heterogeneity\\in the Macroeconomy}
\author{}
\date{}

%%%%%%%%%%%%%%%%%%%%%%%%%%%%%%%%%%%%%%%%%%%%%%%%%%%%%%%%%%%%%%%%%%%%%%%%%%%%%
%%%%%%%%%%%%%%%%%%%%%%%%%%%%%%%%%%%%%%%%%%%%%%%%%%%%%%%%%%%%%%%%%%%%%%%%%%%%%
%%%%%%%%%%%%%%%%%%%%%%%%%%%%%%%%%%%%%%%%%%%%%%%%%%%%%%%%%%%%%%%%%%%%%%%%%%%%%
%%%%%%%%%%%%%%%%%%%%%%%%%%%%%%%%%%%%%%%%%%%%%%%%%%%%%%%%%%%%%%%%%%%%%%%%%%%%%
%%%%%%%%%%%%%%%%%%%%%%%%%%%%%%%%%%%%%%%%%%%%%%%%%%%%%%%%%%%%%%%%%%%%%%%%%%%%%
%%%%%%%%%%%%%%%%%%%%%%%%%%%%%%%%%%%%%%%%%%%%%%%%%%%%%%%%%%%%%%%%%%%%%%%%%%%%%
% Begin Document
\begin{document}
\selectlanguage{british}
\pagenumbering{gobble} % Turn page numbering off
\maketitle % Create title

\vspace{4cm}
\begin{center}
Master Thesis Presented to the\\
Department of Economics at the\\
Rheinische Friedrich-Wilhelms-Universität Bonn\\
\vspace{1cm}
In Partial Fulfillment of the Requirements for the Degree of\\
Master of Science (M.Sc.)
\end{center}

\vspace{5cm}
\begin{center}
\begin{tabular}{ l l }
Supervisors: & Prof.~Dr.~Keith Kuester \\
& Dr.~Gregor Böhl \\
& \\
& \\
Date of Submission: & \printdate{2023-8-25}\\
Author: & Andreas Koundouros\\
Matriculation Number: & 3466868
\end{tabular}
\end{center}

%%%%%%%%%%%%%%%%%%%%%%%%%%%%%%%%%%%%%%%%%%%%%%%%%%%%%%%%%%%%%%%%%%%%%%%%%%%%%
%%%%%%%%%%%%%%%%%%%%%%%%%%%%%%%%%%%%%%%%%%%%%%%%%%%%%%%%%%%%%%%%%%%%%%%%%%%%%
\newpage
\pagenumbering{roman} % Turn page numbering to small roman
{ \hypersetup{hidelinks} \tableofcontents } % Hide links in table of contents

\newpage
{ \hypersetup{hidelinks} \listoffigures } % Hide links in list of figures
{ \hypersetup{hidelinks} \listoftables } % Hide links in list of tables

% Add here list of abbreviations and list of symbols 
\newpage
\setlist[description]{leftmargin=!, labelwidth=5em} % Change for glossaries
\printglossary[title=List of Abbreviations]
\setlist[description]{style=standard} % Reset settings back to default

%%%%%%%%%%%%%%%%%%%%%%%%%%%%%%%%%%%%%%%%%%%%%%%%%%%%%%%%%%%%%%%%%%%%%%%%%%%%%
%%%%%%%%%%%%%%%%%%%%%%%%%%%%%%%%%%%%%%%%%%%%%%%%%%%%%%%%%%%%%%%%%%%%%%%%%%%%%
\newpage
\begin{abstract} % Create abstract
Abstract goes here.
\end{abstract}

%%%%%%%%%%%%%%%%%%%%%%%%%%%%%%%%%%%%%%%%%%%%%%%%%%%%%%%%%%%%%%%%%%%%%%%%%%%%%
%%%%%%%%%%%%%%%%%%%%%%%%%%%%%%%%%%%%%%%%%%%%%%%%%%%%%%%%%%%%%%%%%%%%%%%%%%%%%
\newpage
\pagenumbering{arabic} % Turn page numbering to small arabic
\pagestyle{fancy}
\fancyhf{}
\fancyhead[C]{\leftmark}
\fancyfoot[C]{\thepage}
\thispagestyle{plain}
\section{Introduction}
\label{sec:introduction}

Households face limits to borrowing. In the real world, these limits take various forms and vary across households (e.g.~by credit-worthiness) and asset types. For example, overdrawing one's checking account is limited to some value and mortgage debt might be constrained by regulatory prescriptions on loan-to-value ratios (\Gls{ltv}). 

In the absence of complete insurance markets, the presence of borrowing limits constrains households in their ability to smooth shocks to consumption. Moreover, borrowing limits combined with idiosyncratic income risk induce precautionary savings, i.e.~a build-up of assets in order to prevent running into the borrowing constraints. 

Borrowing limits play an important role not only for individuals but for the aggregate economy as well.

The following question arises: how does time-variation in those limits affect the economic outcomes on aggregate and household levels? To answer this question, this thesis follows and extends the work by \textcite{gl2017} and analyses shocks to the household-level borrowing limits in a heterogeneous agent New Keynesian model (\Gls{hank}).

This limit to borrowing is usually much harsher than the so-called natural borrowing limit, which is commonly defined as the expected net present value of an individual's income stream. 

Since at least the Great Depression economists have aknowledged the importance of debt. The Great Financial Crisis (\Gls{gfc}) was a vivid reminder of how developments in finance, including credit and limits to credit, affect households and the macroeconomy. 

A heterogeneous-agent (\Gls{ha}) model naturally lends itself to the analysis of household-level borrowing limits for at least two reasons. First, models which neglect or only parsimoniously model household heterogeneity cannot speak to interactions between borrowers and lenders.

First, unlike their representative-agent counterparts, prototypical \Gls{ha} models explicitly incorporate borrowing limits. Second, \Gls{ha} models speak to the differential effect of borrowing constraints across the distribution of households.

I further include some financial constraints on the side of the firms and banks as done in \textcite{zoch2022}.

The model and its dynamics are solved with the Python package \texttt{Econpizza} by \textcite{boehl2023econpizza} and the methods therein.

\subsubsection*{Outline}
\label{sec:outline}
The remainder of this thesis is structured as follows. Section \ref{sec:literature} reviews the related literature. Section \ref{sec:model} lays down the macroeconomic model, the numerical implementation and calibration of which is discussed in section \ref{sec:implementation}. Thereafter, section \ref{sec:results} presents the results and section \ref{sec:conclusion} concludes.

\section{Related Literature}
\label{sec:literature}

This section reviews the two strands of the literature which this thesis is most closely related to. I focus first on the macroeconomic literature revolving around household inequality and constraints to household finance. Second, I dwell on the most prominent works on financial constraints and the macroeconomy, where constraints are usually placed in the firm or bank sectors.

\subsection{Household Heterogeneity in the Macroeconomy}
\label{sec:literature-hetero}

\textcite{gl2017} build a household-focused incomplete-markets model and find that a permanent tightening in the household-level borrowing constraint evokes deleveraging and increased precuationary savings. Combined, these two effects depress output and interest rates in the economy. The recession is aggravated and the economy falls into a liquidity trap when the model is augmented with nominal wage rigidities and the zero lower bound. 

\subsection{Financial Constraints in the Macroeconomy}
\label{sec:literature-finance}

In the real world, it is not only households who face financial constraints. Other agents, such as firms and financial intermediaries, can be subject to limits to their borrowing capacity.

BG and BGG are among the earliest and most widely used approaches to the study of financial constraints on the firm side.

GK focus on the economy's financial intermediaries. In their framework, banks are limited in the amount of leverage they can hold because of the possibility of banks' owners to liquidate the business.

\section{Model}
\label{sec:model}

The model fuses the incomplete-markets approach, pioneered \textit{inter alia} by \textcite{aiyagari1994}, with a New Keynsian dynamic stochastic and general equilibrium (\Gls{dsge}) framework. In the words of \textcite{kaplan2018}, the resulting model is a HANK model.

\subsection{Households}
\label{sec:model-hh}

The household side is modeled following the standard incomplete-markets approach, initiated \textit{inter alia} by. In this framework, households are ex-ante identical but heterogeneous ex-post as they are subject to idiosyncratic, i.e.~household-level, income risk. Markets are incomplete and thus, households can only self-insure by saving and borrowing in a one-period, rsik-free bond. This bond is the only asset in the economy and is perfectly liquid, i.e.~there are no costs involved when purchasing or selling bonds. The bond is to be understood as unsecured borrowing.\footnote{Secured borrowing would be collateralised.}

Households have utility function $U$, which is strictly increasing in household consumption $c_i$. Households choose consumption, labour and asset holdings so as to maximise their utiliy, given the prices they face in the economy.

The bond is supplied by households and by the government. %and is in zero net supply.

A note on the asset structure is in order. The one-asset framework seems to be in contradiction to what we find in reality, where households can choose from a variety of assets to self-insure. What is more, there is government insurance and insurance from employers, see Heathcote et al.

\textcite{kaplan2018} provide an insightful critique of the one-asset structure. 

Note that in incomplete markets models it is required that the interest rate times the discount fatcor is smaller than one.

\subsection{Banks}
\label{sec:model-banks}

Heterogeneous banks are modeled in the paper Bewley banks by.

\subsection{Firms}
\label{sec:model-firms}

Firm profits bear the problem that they have to be distributed. The way this is done might matter for households and their differential consumption-savings behaviour. 

\subsection{Fiscal and Monetary Policy}
\label{sec:model-policy}

\subsection{Equilibrium}
\label{sec:model-eq}

\section{Implementation and Calibration}
\label{sec:implementation}

The grid is not allowed to change over time as this mimeo notes. Otherwise, the interpolation would be between repeatedly different grid points, making the comparison of over time inaccurate. 

The purpose of the present exercise is not to accurately match all moments of the empirical data but rather to flesh out the main mechanisms by which a shock to the borrowing limit affects individual-level and aggregate behaviour.

\section{Results}
\label{sec:results}

Note however that the responses in \textcite{gl2017} are also not very persistent. Their model also lacks capital, hanit formation or adjustment costs.

\subsection{Comparison of Steady States}
\label{sec:results-stst}

\begin{center}
\begin{tabular}{lccc}
%\toprule
     Variable &  Initial Steady State &  Terminal Steady State &  Rel. Change \\
     \hline
%\midrule
            B &                3.0000 &                 3.0000 &         0.00 \\
         beta &                0.9700 &                 0.9700 &         0.00 \\
            C &                1.0000 &                 1.0000 &         0.00 \\
        D\_o\_Y &                0.9699 &                 0.4819 &       -50.31 \\
          div &                0.1000 &                 0.1000 &         0.00 \\
lower\_bound\_a &               -2.0000 &                -1.0000 &       -50.00 \\
          MPC &                0.2481 &                 0.1143 &       -53.93 \\
            N &                1.0000 &                 1.0000 &         0.00 \\
           pi &                1.0000 &                 1.0000 &         0.00 \\
            R &                1.0158 &                 1.0144 &        -0.14 \\
           Rn &                1.0158 &                 1.0144 &        -0.14 \\
           Rr &                1.0158 &                 1.0144 &        -0.14 \\
        Rstar &                1.0158 &                 1.0144 &        -0.14 \\
          tax &                0.0473 &                 0.0431 &        -8.88 \\
       Top10A &                0.8162 &                 0.7231 &       -11.41 \\
       Top10C &                0.1908 &                 0.1897 &        -0.58 \\
            w &                0.9000 &                 0.9000 &         0.00 \\
            y &                1.0000 &                 1.0000 &         0.00 \\
       y\_prod &                1.0000 &                 1.0000 &         0.00 \\
            z &                1.0000 &                 1.0000 &         0.00 \\
%\bottomrule
\end{tabular}
\end{center}

\subsection{Transitional Dynamics}
\label{sec:results-transition}

\section{Sensitivity Analyses}
\label{sec:sensitivity}

\section{Conclusion}
\label{sec:conclusion}

%%%%%%%%%%%%%%%%%%%%%%%%%%%%%%%%%%%%%%%%%%%%%%%%%%%%%%%%%%%%%%%%%%%%%%%%%%%%%
%%%%%%%%%%%%%%%%%%%%%%%%%%%%%%%%%%%%%%%%%%%%%%%%%%%%%%%%%%%%%%%%%%%%%%%%%%%%%
% References section
\newpage
\thispagestyle{plain}
\pagenumbering{Roman}
\printbibliography[heading=bibintoc] % Insert references

%%%%%%%%%%%%%%%%%%%%%%%%%%%%%%%%%%%%%%%%%%%%%%%%%%%%%%%%%%%%%%%%%%%%%%%%%%%%%
%%%%%%%%%%%%%%%%%%%%%%%%%%%%%%%%%%%%%%%%%%%%%%%%%%%%%%%%%%%%%%%%%%%%%%%%%%%%%
% Appendices section
\newpage
\thispagestyle{plain}
\pagenumbering{arabic}% resets `page` counter to 1
\renewcommand*{\thepage}{A-\arabic{page}}
\begin{appendices}
\section{Description of Computer Codes}
\label{sec-app:codes}

\section{The Case of a Shock to Income Risk}
\label{sec-app:income}
The literature has given quite some attention to shocks to labour income risk in \Gls{hank} models in order to produce fluctuations which resemble business cycles. In this appendix, I show that a permanent shock to labour income risk produces impulse responses which are similar to the ones obtained from the analysis above.

\end{appendices}

%%%%%%%%%%%%%%%%%%%%%%%%%%%%%%%%%%%%%%%%%%%%%%%%%%%%%%%%%%%%%%%%%%%%%%%%%%%%%
%%%%%%%%%%%%%%%%%%%%%%%%%%%%%%%%%%%%%%%%%%%%%%%%%%%%%%%%%%%%%%%%%%%%%%%%%%%%%
\newpage
\thispagestyle{plain}
\pagenumbering{gobble} % Turn page numbering off
\section*{Statement of Authorship}
I hereby confirm that the work presented has been performed and interpreted solely by myself except for where I explicitly identified the contrary. I assure that this work has not been presented in any other form for the fulfillment of any other degree or qualification. Ideas taken from other works in letter and in spirit are identified in every single case.

\vspace{2cm}
\noindent
\rule{8cm}{0.4pt}\\
Andreas Koundouros\\
Bonn, the \printdate{2023-8-25}
\end{document}
% End Document
%%%%%%%%%%%%%%%%%%%%%%%%%%%%%%%%%%%%%%%%%%%%%%%%%%%%%%%%%%%%%%%%%%%%%%%%%%%%%
%%%%%%%%%%%%%%%%%%%%%%%%%%%%%%%%%%%%%%%%%%%%%%%%%%%%%%%%%%%%%%%%%%%%%%%%%%%%%
%%%%%%%%%%%%%%%%%%%%%%%%%%%%%%%%%%%%%%%%%%%%%%%%%%%%%%%%%%%%%%%%%%%%%%%%%%%%%
%%%%%%%%%%%%%%%%%%%%%%%%%%%%%%%%%%%%%%%%%%%%%%%%%%%%%%%%%%%%%%%%%%%%%%%%%%%%%
%%%%%%%%%%%%%%%%%%%%%%%%%%%%%%%%%%%%%%%%%%%%%%%%%%%%%%%%%%%%%%%%%%%%%%%%%%%%%
%%%%%%%%%%%%%%%%%%%%%%%%%%%%%%%%%%%%%%%%%%%%%%%%%%%%%%%%%%%%%%%%%%%%%%%%%%%%%
