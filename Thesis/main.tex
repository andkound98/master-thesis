\documentclass[12pt]{article} % Set documentclass article and set font size

%%%%%%%%%%%%%%%%%%%%%%%%%%%%%%%%%%%%%%%%%%%%%%%%%%%%%%%%%%%%%%%%%%%%%%%%%%%%%
%%%%%%%%%%%%%%%%%%%%%%%%%%%%%%%%%%%%%%%%%%%%%%%%%%%%%%%%%%%%%%%%%%%%%%%%%%%%%
%%%%%%%%%%%%%%%%%%%%%%%%%%%%%%%%%%%%%%%%%%%%%%%%%%%%%%%%%%%%%%%%%%%%%%%%%%%%%
% Packages
\usepackage{fontspec} % If used, change compiler to XeLaTex
\usepackage[utf8]{inputenc}
\usepackage[british]{babel}
\usepackage[british]{isodate}
\usepackage{csquotes}
\usepackage{mathtools}
\usepackage{amssymb}
\usepackage{unicode-math}
\usepackage{geometry}
\usepackage[
backend=biber,
style=apa,
sorting=nyt, % Sort by name, year, title
backref=true,
dateabbrev=false,
language=british,
urldate=long
]{biblatex}
\usepackage{xcolor}
\usepackage[toc,page]{appendix}
\usepackage{svg}
\usepackage{float}
\usepackage{longtable}
\usepackage{multirow}
\usepackage{multirow,multicol, makecell, booktabs}
\usepackage{caption}
\usepackage{subcaption}
\usepackage{setspace}
\usepackage{fancyhdr}
\usepackage{enumitem}
\usepackage{tikz}
\usepackage{lscape}
\usepackage{hyperref} % Load after everything else
\usepackage[symbols, nopostdot, automake, nonumberlist]{glossaries}

%%%%%%%%%%%%%%%%%%%%%%%%%%%%%%%%%%%%%%%%%%%%%%%%%%%%%%%%%%%%%%%%%%%%%%%%%%%%%
%%%%%%%%%%%%%%%%%%%%%%%%%%%%%%%%%%%%%%%%%%%%%%%%%%%%%%%%%%%%%%%%%%%%%%%%%%%%%
%%%%%%%%%%%%%%%%%%%%%%%%%%%%%%%%%%%%%%%%%%%%%%%%%%%%%%%%%%%%%%%%%%%%%%%%%%%%%
% Custom Settings
% \DeclareFieldFormat{doi}{% % put doi into new line in references
%   \newline
%   \mkbibacro{DOI}\addcolon\space
%     \ifhyperref
%       {\href{http://dx.doi.org/#1}{\nolinkurl{#1}}}
%       {\nolinkurl{#1}}}

\addbibresource{master_thesis.bib} % Add bibliography source

\setlength\bibitemsep{1.5\itemsep} % Adjust distance of reference entries

\renewcommand*\appendixpagename{\Large Appendices} % Adjust size of title of the appendices section

\renewcommand*{\finentrypunct}{}
\renewbibmacro*{pageref}{%
  \addperiod% NEW
  \iflistundef{pageref}
    {}
    {\newline\footnotesize\printtext[parens]{%
       \ifnumgreater{\value{pageref}}{1}
         {\bibstring{backrefpages}\ppspace}
     {\bibstring{backrefpage}\ppspace}%
       \printlist[pageref][-\value{listtotal}]{pageref}}}}%
\DefineBibliographyStrings{english}{
  backrefpage={Cited on page},
  backrefpages={Cited on pages}
}

\setmainfont{Times New Roman} % If the usepackage fontspec is active, you can set Times New Roman here

\geometry{ % Adjust paper geometry
 a4paper,
 right=25mm,
 bottom=20mm,
 left=25mm,
 top=25mm}
\urlstyle{same} % Make URL font the same with the rest of the document
\onehalfspacing % Adjust spacing
%\setstretch{1.75}
%\doublespacing
\allowdisplaybreaks % Allow for breaks in equations across pages
\addto\captionsbritish{ % Replace "english" with the language you use
  \renewcommand{\contentsname}%
    {Table of Contents}%
} % Adjust title of the table of contents
\numberwithin{equation}{section} % Number equations with sections
\numberwithin{figure}{section}
\numberwithin{table}{section}
%\setcounter{tocdepth}{2} % Set maximum level to which sections are displayed in the table of contents

\newcommand{\appendixsubsection}[1]{
    \stepcounter{subsection}
    \subsection*{\Alph{section}.\arabic{subsection}\hspace{1em}{#1}}
}


\pagestyle{plain}

% Define a custom name format that prints names of authors in references in bold
\DeclareNameWrapperFormat{sortname}{\mkbibbold{#1}}
\DeclareNameWrapperAlias{author}{sortname}

% Add definition format
\newtheorem{definition}{Definition}

% Make links of glossary entries black
\makeatletter
\newcommand*{\glsplainhyperlink}[2]{%
    \begingroup%
      \hypersetup{hidelinks}%
      \hyperlink{#1}{#2}%
    \endgroup%
}
\let\@glslink\glsplainhyperlink
\makeatother

%%%%%%%%%%%%%%%%%%%%%%%%%%%%%%%%%%%%%%%%%%%%%%%%%%%%%%%%%%%%%%%%%%%%%%%%%%%%%
%%%%%%%%%%%%%%%%%%%%%%%%%%%%%%%%%%%%%%%%%%%%%%%%%%%%%%%%%%%%%%%%%%%%%%%%%%%%%
%%%%%%%%%%%%%%%%%%%%%%%%%%%%%%%%%%%%%%%%%%%%%%%%%%%%%%%%%%%%%%%%%%%%%%%%%%%%%
% Glossary
\makeglossaries 

% Glossary entries
\newglossaryentry{nk}{
name=NK,
description={New Keynesian}
}

\newglossaryentry{ha}{
name=HA,
description={Heterogeneous Agents}
}

\newglossaryentry{hank}{
name=HANK,
description={Heterogeneous Agent New Keynesian}
}

\newglossaryentry{gfc}{
name=GFC,
description={Great Financial Crisis}
}

\newglossaryentry{ltv}{
name=LTV,
description={Loan-to-Value}
}

\newglossaryentry{dsge}{
name=DSGE,
description={Dynamic Stochastic General Equilibrium}
}

\newglossaryentry{egm}{
name=EGM,
description={Endogenous Gridpoints Method}
}

\newglossaryentry{iou}{
name=IOU,
description={I Owe You}
}

\newglossaryentry{ghh}{
name=GHH,
description={Greenwood-Hercowitz-Huffman (\cite{ghh1988})}
}

\newglossaryentry{ar}{
name=AR,
description={Autoregressive}
}

\newglossaryentry{zlb}{
name=ZLB,
description={Zero Lower Bound}
}

\newglossaryentry{mpc}{
name=MPC,
description={Marginal Propensity to Consume}
}

\newglossaryentry{fof}{
name=FoF,
description={Flow of Funds}
}

\newglossaryentry{nkpc}{
name=NKPC,
description={New Keynesian Phillips Curve}
}

\newglossaryentry{foc}{
name=FOC,
description={First Order Condition}
}

\newglossaryentry{irf}{
name=IRF,
description={Impulse Response Function}
}

\newglossaryentry{crra}{
name=CRRA,
description={Constant Relative Risk Aversion}
}

\newglossaryentry{gdp}{
name=GDP,
description={Gross Domestic Product}
}

%%%%%%%%%%%%%%%%%%%%%%%%%%%%%%%%%%%%%%%%%%%%%%%%%%%%%%%%%%%%%%%%%%%%%%%%%%%%%
%%%%%%%%%%%%%%%%%%%%%%%%%%%%%%%%%%%%%%%%%%%%%%%%%%%%%%%%%%%%%%%%%%%%%%%%%%%%%
%%%%%%%%%%%%%%%%%%%%%%%%%%%%%%%%%%%%%%%%%%%%%%%%%%%%%%%%%%%%%%%%%%%%%%%%%%%%%
% hyperref package
\usepackage{footnotebackref} % Brings you back to where the footnote is in the text
\hypersetup{ 
    colorlinks,
    linkcolor=blue,
    filecolor=blue,  
    citecolor=blue,
    urlcolor=blue} % Setup with the hyperref package

%%%%%%%%%%%%%%%%%%%%%%%%%%%%%%%%%%%%%%%%%%%%%%%%%%%%%%%%%%%%%%%%%%%%%%%%%%%%%
%%%%%%%%%%%%%%%%%%%%%%%%%%%%%%%%%%%%%%%%%%%%%%%%%%%%%%%%%%%%%%%%%%%%%%%%%%%%%
%%%%%%%%%%%%%%%%%%%%%%%%%%%%%%%%%%%%%%%%%%%%%%%%%%%%%%%%%%%%%%%%%%%%%%%%%%%%%
% Formalities
\title{Financial Constraints and Household Heterogeneity\\in the Macroeconomy}
\author{}
\date{}

%%%%%%%%%%%%%%%%%%%%%%%%%%%%%%%%%%%%%%%%%%%%%%%%%%%%%%%%%%%%%%%%%%%%%%%%%%%%%
%%%%%%%%%%%%%%%%%%%%%%%%%%%%%%%%%%%%%%%%%%%%%%%%%%%%%%%%%%%%%%%%%%%%%%%%%%%%%
%%%%%%%%%%%%%%%%%%%%%%%%%%%%%%%%%%%%%%%%%%%%%%%%%%%%%%%%%%%%%%%%%%%%%%%%%%%%%
%%%%%%%%%%%%%%%%%%%%%%%%%%%%%%%%%%%%%%%%%%%%%%%%%%%%%%%%%%%%%%%%%%%%%%%%%%%%%
%%%%%%%%%%%%%%%%%%%%%%%%%%%%%%%%%%%%%%%%%%%%%%%%%%%%%%%%%%%%%%%%%%%%%%%%%%%%%
%%%%%%%%%%%%%%%%%%%%%%%%%%%%%%%%%%%%%%%%%%%%%%%%%%%%%%%%%%%%%%%%%%%%%%%%%%%%%
% Begin Document
\begin{document}
\selectlanguage{british}
\pagenumbering{gobble} % Turn page numbering off
\maketitle % Create title

\vspace{4cm}
\begin{center}
Master Thesis Presented to the\\
Department of Economics at the\\
Rheinische Friedrich-Wilhelms-Universität Bonn\\
\vspace{1cm}
In Partial Fulfillment of the Requirements for the Degree of\\
Master of Science (M.Sc.)
\end{center}

\vspace{5cm}
\begin{center}
\begin{tabular}{ l l }
Supervisors: & Prof.~Dr.~Keith Kuester \\
& Dr.~Gregor Böhl \\
& \\
& \\
Date of Submission: & \printdate{2023-8-25}\\
Author: & Andreas Koundouros\\
Matriculation Number: & 3466868
\end{tabular}
\end{center}

%%%%%%%%%%%%%%%%%%%%%%%%%%%%%%%%%%%%%%%%%%%%%%%%%%%%%%%%%%%%%%%%%%%%%%%%%%%%%
%%%%%%%%%%%%%%%%%%%%%%%%%%%%%%%%%%%%%%%%%%%%%%%%%%%%%%%%%%%%%%%%%%%%%%%%%%%%%
%%%%%%%%%%%%%%%%%%%%%%%%%%%%%%%%%%%%%%%%%%%%%%%%%%%%%%%%%%%%%%%%%%%%%%%%%%%%%
\newpage
\pagenumbering{roman} % Turn page numbering to small roman
\setcounter{tocdepth}{2}
{ \hypersetup{hidelinks} \tableofcontents } % Hide links in table of contents

\newpage
{ \hypersetup{hidelinks} \listoftables } % Hide links in list of tables
{ \hypersetup{hidelinks} \listoffigures } % Hide links in list of figures

% Add here list of abbreviations and list of symbols 
\newpage
\setlist[description]{leftmargin=!, labelwidth=5em} % Change for glossaries
\printglossary[title=List of Acronyms]
\setlist[description]{style=standard} % Reset settings back to default


%%%%%%%%%%%%%%%%%%%%%%%%%%%%%%%%%%%%%%%%%%%%%%%%%%%%%%%%%%%%%%%%%%%%%%%%%%%%%
%%%%%%%%%%%%%%%%%%%%%%%%%%%%%%%%%%%%%%%%%%%%%%%%%%%%%%%%%%%%%%%%%%%%%%%%%%%%%
%%%%%%%%%%%%%%%%%%%%%%%%%%%%%%%%%%%%%%%%%%%%%%%%%%%%%%%%%%%%%%%%%%%%%%%%%%%%%
\newpage
\begin{abstract} % Create abstract
As borrowers and lenders are connected through their balance sheets, a credit crunch might not only affect debtors but also lenders and, in extension, the macroeconomy. This thesis considers two shocks leading to a credit crunch -- a reduction in the debt limit and an increase in the premium on the borrowing rate -- and traces the consequences of these disturbances. In an economy with idiosyncratic risk and heterogeneous agents, both shocks prompt borrowing households to delever and lending households to increase their demand for assets. To restore equilibrium, the interest rate must fall but might not do so sufficiently to avert a drop in aggregate consumption and prices. The macroeconomic fallout can be similar across the two shocks. Importantly however, the distributional effects from a deleveraging episode might differ depending on whether the maximum quantity of debt tightens or whether debt becomes more costly. To show this, I draw upon a small-scale \Gls{hank} model in which borrowers and lenders emerge naturally and the real interest rate intermediates between the two.

\begin{center}
\rule{10cm}{0.4pt}
\end{center}

\noindent
\textit{JEL Classification:} E12, E21, E32 % JEL topic classification of paper

\noindent
\textit{Keywords:} Borrowers and Lenders, Household Debt, Deleveraging, \Gls{hank} % Keywords of paper
\end{abstract}

%%%%%%%%%%%%%%%%%%%%%%%%%%%%%%%%%%%%%%%%%%%%%%%%%%%%%%%%%%%%%%%%%%%%%%%%%%%%%
%%%%%%%%%%%%%%%%%%%%%%%%%%%%%%%%%%%%%%%%%%%%%%%%%%%%%%%%%%%%%%%%%%%%%%%%%%%%%
%%%%%%%%%%%%%%%%%%%%%%%%%%%%%%%%%%%%%%%%%%%%%%%%%%%%%%%%%%%%%%%%%%%%%%%%%%%%%
\newpage
\pagenumbering{arabic} % Turn page numbering to small arabic
\pagestyle{fancy}
\fancyhf{}
\fancyhead[C]{\leftmark}
\fancyfoot[C]{\thepage}

%%%%%%%%%%%%%%%%%%%%%%%%%%%%%%%%%%%%%%%%%%%%%%%%%%%%%%%%%%%%%%%%%%%%%%%%%%%%%
%%%%%%%%%%%%%%%%%%%%%%%%%%%%%%%%%%%%%%%%%%%%%%%%%%%%%%%%%%%%%%%%%%%%%%%%%%%%%
%%%%%%%%%%%%%%%%%%%%%%%%%%%%%%%%%%%%%%%%%%%%%%%%%%%%%%%%%%%%%%%%%%%%%%%%%%%%%
\thispagestyle{plain}
\section{Introduction}
\label{sec:introduction}

One important dimension along which households differ is their asset positions. Some households are net lenders, i.e.~their positive asset holdings exceed their liabilities, while others are net borrowers, i.e.~they owe more in debt than they own in assets. Crucially, the balance sheets of the two groups of households are connected: the borrowers' debt holdings are the lenders' asset holdings. This link between borrowing and lending households suggests that their respective merits are closely intertwined.\footnote{Note two things. First, over time, households might switch from being a borrower to being a lender and vice versa. Second, entities other than households can intermediate funds from and to the household sector, e.g.~financial institutions. However, I here focus on the intermediation of credit between households. The latter can be thought of as frictionless intermediation through perfectly competitive banks.} 

For example, interconnected balance sheets imply that the consequences of a shock to the indebted households' borrowing capacity -- whatever its cause might be -- might not only be borne by the indebted households themselves but also by the lending households. In turn, this suggests that a tightening in the financial conditions under which borrowers obtain credit, a so-called "credit crunch", affects households with very different asset positions and thus unequal abilities to absorb shocks. This raises the question as to whether and how financial shocks to debtors' balance sheets impact not only on the consumption, lending and borrowing decisions of individuals, but also on the macroeconomy as a whole. 

Hence, this thesis asks: what are the consequences of a credit crunch? In particular, how does such a shock work through borrowers' and lenders' balance sheets, respectively, and how does this consequently impact, first, the two groups' consumption and saving decisions and, second, the aggregate state of the economy? In other words, how are consumption and asset holdings redistributed and how are macroeconomic variables disturbed when debtors' access to credit is suddenly curtailed?

To answer these questions and to investigate the presence of these mechanisms, I follow the modelling approach of \textcite{gl2017} by employing a heterogeneous-agent (\Gls{ha}) model. Such a framework naturally lends itself to the analysis of household-level financial shocks because borrowing and lending asset positions emerge from it endogenously. In particular, the model I propose features a simple New Keynesian (\Gls{nk}) aggregate state space, resulting in a small-scale one-asset heterogeneous-agent New Keynesian (\Gls{hank}) model, and allows for two shocks to households' financial constraints: shocks to the debt limit and shocks to the wedge between borrowing and lending interest rates.

The investigation of these shocks delivers two key insights. First, permanent shocks to the borrowing limit and to the interest rate wedge produce similar \textit{aggregate} adjustments qualitatively. Second, I show that the \textit{distributional} consequences from an episode of forced deleveraging crucially depend on the origins of the shock, i.e.~whether the debt limit tightens or whether debt becomes more costly.

In more detail, the first insight derives from the fact that -- under a plausible calibration of the model -- permanent shocks to both, the borrowing limit and the interest rate wedge, prompt rapid deleveraging on the borrowers' side, aggregate consumption and inflation to fall and the central bank to cut the nominal interest rate, even up to the zero-lower bound (\Gls{zlb}) if the initial interest rate was low to start with. Importantly however, in both cases, the recession is rather short-lived and typically does not last longer than the shock takes to resolve.\footnote{Presumably, this is due to the fact that the model excludes productive capital and does not include various other frictions that are known to create persistence, such as habit formation in household consumption.} 

The aggregate mechanisms come about through the described linkages in household balance sheets and the implied differences in households' resilience to shocks. As debtors are forced to delever, the supply of assets for lenders is reduced. Concurrently, asset-rich households would like to increase their precautionary savings amid tighter financial constraints. To restore equilibrium in asset and goods markets, the interest rate on assets must fall. Even if nominal interest rates are far away from the \Gls{zlb}, this does not preclude a recession, which -- as expected -- is deeper however, if the \Gls{zlb} does bind during the transition. 

\textcolor{red}{The second insight pertains to the distributional effects of financial shocks. The results show that the two financial shocks can manifest themselves in very different distributional outcomes and dynamics. For example, the wealth distribution in a state of a permanently tighter credit limit features more agents directly at the constraint than in a state of relatively easy borrowing. In contrast, in a state of permanently higher costs of credit intermediation, there are significantly fewer agents at the constraint, reflecting the fact that in the latter case, debt is more expensive to hold at all levels of debt.}

The general mechanisms identified in this thesis have been studied in the theoretical literature before, \textit{inter alia} by \textcite{egg2012}, \textcite{riosrull2015} and \textcite{gl2017}. My work is closest to \textcite{egg2012} and \textcite{gl2017}. Both papers show how an unexpected tightening in the household-level borrowing limit provokes depressed levels of consumption and interest rates close or at the \Gls{zlb}, which are features of macroeconomic aggregates that resemble those witnessed during the Great Recession of 2007--2009. Crucially, both studies emphasise the endogenous reaction of the interest rate and highlight how real rigidity and household heterogeneity influence the internal adjustment of the economy.

Before proceeding to the analysis of the above issues, be aware of the following caveat: this thesis is \textit{not} about identifying the roots of the Great Recession. In fact, the present investigation is entirely agnostic about the origins of financial constraints and the causes of why they might change over time. Rather, the focus is on the \textit{effects} that exogenous variation in these constraints has on aggregates as well as on consumption and savings outcomes at the household level. These effects might resemble stylised facts of the Great Recession but this can not be taken as evidence in support of the hypothesis that financial shocks caused the Great Recession.\footnote{Whether financial shocks were the main cause of the Great Recession or merely an accelerator of events is still a matter of discussion in the literature. For example, \textcite{justiniano2015} show that, in their two-agent model, house prices are the more likely driver of the events of the Great Recession than changes in borrowing constraints, modelled through constraints in the loan-to-value (\Gls{ltv}) ratio.}

%%%%%%%%%%%%%%%%%%%%%%%%%%%%%%%%%%%%%%%%%%%%%%%%%%%%%%%%%%%%%%%%%%%%%%%%%%%%%
\subsubsection*{Outline}
\label{sec:outline}
The remainder of this thesis is structured as follows. Section \ref{sec:literature} reviews the literature related to financial constraints, household debt and household heteroegeneity from a macroeconomic perspective. Thereafter, section \ref{sec:model} lays down the \Gls{hank} model and thereby discusses in detail the household sector, the asset structure, borrowers' and lenders' interactions and characteristics as well as the included financial constraints and shocks. Section \ref{sec:model} also deals with the numerical implementation of the model and its calibration. 

Next, section \ref{sec:limit} studies a permanent shock to the household-level borrowing constraint. I first explain what to expect from such a shock. Then, I present the quantitative findings and discuss the underlying mechanisms. Section \ref{sec:wedge} proceeds analogously for shocks to the interest rate wedge. Subsequently, section \ref{sec:sensitivity} explores alternative model specifications and model extensions. Finally, section \ref{sec:conclusion} concludes.

%%%%%%%%%%%%%%%%%%%%%%%%%%%%%%%%%%%%%%%%%%%%%%%%%%%%%%%%%%%%%%%%%%%%%%%%%%%%%
%%%%%%%%%%%%%%%%%%%%%%%%%%%%%%%%%%%%%%%%%%%%%%%%%%%%%%%%%%%%%%%%%%%%%%%%%%%%%
%%%%%%%%%%%%%%%%%%%%%%%%%%%%%%%%%%%%%%%%%%%%%%%%%%%%%%%%%%%%%%%%%%%%%%%%%%%%%
%\newpage
\section{Related Literature}
\label{sec:literature}

Since at least the Great Depression of the 1930s, economists have been thinking about the role of finance and indebtedness in the macroeconomy. For example, the work of \textcite{fisher1933} points to the debt-deflation spiral as one of the key drivers of the Great Depression and the prolonged economic weakness that ensued. In that view, high levels of (nominal) debt have to be reduced once they prove unsustainable. However, this deleveraging kicks off a vicious circle of reduced consumption, subsequent deflation, a mechanic increase in the real debt burden, which finally aggravates the need for deleveraging.

Later, formal macroeconomic models captured financial frictions on the \textit{firm} side or in the \textit{banking} sector of the economy. The typical approach within the former model class is to emphasise that entrepreneurs have private information over their projects which their lenders can not observe \textit{ex ante}, which implies that companies' net worth plays a role in their access to credit (see e.g.~\cite{bg1989}, \cite{bgg1999}). Motivated by the events of the Great Recession of 2007--2009, the literature included constraints to banks' leverage, which are often motivated by the possibility of banks' managers to liquidate the bank (see e.g.~\cite{gertler2011}). Studies in both directions typically find that such financial frictions prove to be accelerators of an economy's reaction to a variety of shocks.\footnote{See \textcite{brunnermeier2012} and \textcite{christiano2022} for surveys on the approaches to model financial frictions in macroeconomic models.}

Recently, the literature has increasingly turned to constraints to \textit{household} finance and debt as potentially important factors in business cycle analysis. This development is motivated by empirical evidence which suggests that household asset positions, access to credit and indebtedness are important for understanding, on the one hand, the behaviour of individuals and, on the other hand, fluctuations in macroeconomic aggregates. Pertaining to the former, \textit{inter alia} \textcite{campbell1989} (with aggregate US time series) and \textcite{zeldes1989} (with survey data on US families) show that certain fractions of the population are indeed financially constrained in their access to liquidity.\footnote{Note that such findings are often taken as evidence against the classic permanent-income view of consumption behaviour, see e.g.~\textcite{kaplan2014}.}

\textcite{gross2002} use granular credit card data which allow them to control for credit demand. Indeed, the authors uncover that households significantly and quickly respond to exogenous changes in their borrowing limit. If the supply of credit is exogenously increased, households increase debt and consumption. These consumption responses are heterogeneous as they are more pronounced for households at or close to their borrowing limit. Further, \textcite{gross2002} show that borrowers' debt holdings move inversely to interest rate changes.

Clearly, households have diverse balance sheets. \textcite{kaplan2014} demonstrate how household portfolios differ in their liquidity composition and document that this influences households' behaviour. For example, households that are wealthy in illiquid assets might exhibit consumption patterns that resembles the ones of the wealth-poor. This suggests that household balance sheets might be an integral part of the analysis of individuals' consumption decisions.

Regarding aggregate outcomes, \textcite{mian2010} show that, on the U.S.~county level, the increase in household leverage before the Great Recession and the reductions in credit card debt availability during the Great Recession significantly inform about the severity of the recession in that county. Further, in a panel-data sample on 30 mostly advanced economies, \textcite{mian2017} find that (total) household debt is a significant statistical predictor for output growth.

%Much of the work works on total household debt, something I will not do. My model actually considers unsecured credit, e.g.~through credit cards. Total household debt of course importantly includes secured credit as well. In the US for example, mortgage debt makes up around $60\%$--$70\%$ of total debt.

Prompted by the apparent relevance of financial constraints, balance sheets and debt in the data, the theoretical literature sought to develop models which give centre stage to household-level financial constraints and which allow for a meaningful inclusion of household debt. Among others, \textcite{egg2012}, \textcite{riosrull2015}, \textcite{justiniano2015} and \textcite{gl2017} employ dynamic macroeconomic models which allow for borrowers and lenders to coexist and their balance sheets to be interlinked. Despite sharing a number of similarities, these studies naturally differ in their modelling approaches and in the mechanisms they emphasise.

For example, in the framework of \textcite{egg2012} differences in patience between two household types induce the impatient agents to borrow -- in nominal terms -- from the relatively more patient ones. An unexpected tightening in the real debt limit for borrowers forces indebted households to reduce their borrowing within one period, which they do by cutting back on consumption. To induce lending households to make up for this fall in consumption, the interest rate must fall. Because of the \Gls{zlb}, the interest rate might be unable to do so, prompting a recession which is exacerbated through the debt-deflation mechanism à la \textcite{fisher1933}.

%Claims are nominal, but the debt limit that borrowing households are subject to is in real terms, opening the door for a debt-deflation spiral à la \textcite{fisher1933}. An unexpected, permanent tightening in this limit forces the borrowing households to delever. As balance sheets are connected, the reduction of the borrowers' debt holdings leads to an excess demand for asset holdings on the part of the lenders. To induce patient households to lend less, the interest rate has to fall in order to restore equilibrium. If the interest rate is unable to do so, e.g.~because of the \Gls{zlb}, the economy enters a recession in the short run, exacerbated through the a debt-deflation mechanism.

In contrast to the two-agent structure of \textcite{egg2012}, \textcite{riosrull2015} and \textcite{gl2017} consider \Gls{ha} models in which the distribution of agents is free to adjust after a financial disturbance, a potentially important mechanism. While \textcite{riosrull2015} emphasise endogenous changes in consumers' search efforts and productivity after a shock to the debt limit, \textcite{gl2017} shares with \textcite{egg2012}, as do I, the focus on the endogenous adjustment of the interest rate.

\textcolor{red}{\textcite{gl2017} put forward an incomplete-markets model which largely abstracts from an aggregate state space but induces borrowing and lending between households through exposing them to idiosyncratic risk. The baseline model of \textcite{gl2017} does not feature price rigidities or the \Gls{zlb} but still, consumption falls considerably after the beginning of a credit crunch episode. A strongly reduced interest rate cannot prevent this because lenders strive to increase their savings in precaution to the now tighter debt limit, which implies that lenders do not increase consumption strongly enough after the interest rate drops. Incidentally, a two-agent model can not capture this channel of increased precautionary savings.}

\textcolor{red}{Moreover, across the distribution, the wealth-poor suffer much larger consumption losses after the credit tightening relative to their wealthier peers. \textcite{gl2017} argue that this is because the poor are more sensitive in their consumption response and the wealthy react more to interest rate changes. Also, including nominal rigidities and the \Gls{zlb} significantly worsens the recession in \textcite{gl2017}.}

\textcolor{red}{Relative to \textcite{gl2017}, I also motivate borrowing and lending through idiosyncratic risk but I add a small-scale \Gls{nk} structure to the \Gls{ha} block of the economy. Thus, the model introduces several key features of the textbook approach of \Gls{nk} monetary economics, mainly an output-inflation trade-off in the form of a Phillips Curve.} 

\textcolor{red}{Allowing for the Fisher equation, lets interesting mechanisms to arise: borrowers, especially those in need of deleveraging, are hit even more as real rates spike.}

%\textcolor{red}{Using the fully-fledged two-asset \Gls{hank} model of \textcite{kaplan2018}, \textcite{kv2018} show that a shock to the borrowing capacity of indebted households can be understood as a microfoundation for a shock to the households' discount factor. The latter shock is a tool in representative-agent models to simulate a sudden change in aggregate demand. They use a a transitory increase in the wedge between borrowing and lending rates. I investigate such a shock in my model framework in section \ref{sec:wedge}.}

Naturally, the insight that forced deleveraging might spark considerable and unequal consumption losses raises the question of which role policy can play in ameliorating the fallout of a credit crunch. While \textcite{egg2012} and \textcite{gl2017} show how fiscal policy can intervene with government debt issuance and spending after the shock arises, \textcite{korinek2016} sketch how regulation can contain \textit{ex-ante} excessive leverage and thus prevent or dampen the consequences of a deleveraging shock. \textcite{korinek2016} study prescriptions to maximum leverage, taxes on debt and insurance subsidies for borrowers as potential policy tools.

Finally, my work is of course related to the fast-growing literature on \Gls{hank} models which embed the incomplete-markets framework, pioneered \textit{inter alia} by \textcite{bewley1986}, \textcite{im1989}, \textcite{huggett1993} and \textcite{aiyagari1994}, in \Gls{nk} environments. An increasing number of questions in macroeconomics is investigated using \Gls{hank} models. For example, \textcite{mckay2016} and \textcite{kaplan2018} study monetary policy through the lens of \Gls{hank}, \textcite{bayer2023} investigate fiscal policy and the supply of liquid assets by the government and \textcite{bayer2019} analyse shocks to income risk.\footnote{\textcite{heath2009} provide an overview of modelling heterogeneous households in macroeconomics. \textcite{kv2018} discuss the advances with \Gls{hank}. \textcite{cherrier2023} offer a historical overview of the advances and different approaches in the macroeconomic analysis with heterogeneity in the 1980s and 1990s.}

%\textcite{egg2012} also find that a deleveraging shock can be completely undone by fiscal policy by targeting short-run tax cuts and long-run tax increases at the borrowing fraction of the population, thereby effectively using its own ability to borrow to help the borrowers who are affected by the deleveraging shock. This result, however, relies very much on the assumption that the debt limit is exogenously set.

%As \textcite{egg2012} note, it is reasonable to assume that the debt limit depends to some degree on current conditions, such as current income.

%Finally, my work is of course related to the literature which studies macroeconomic fluctuations in models with heterogeneous agents. The origins of these models, in which households can only self-insure against idiosyncratic risk by means of precautionary savings, were pioneered \textit{inter alia} by \textcite{bewley1986}, \textcite{im1989}, \textcite{huggett1993} and \textcite{aiyagari1994}.\footnote{Truman Bewley is credited with proving the existence of equilibrium in models where agents face uninsurable idiosyncratic risk (see \cite{cherrier2023}). \textcite{im1989} compares the costs of business cycles in imperfect-insurance models to those implied by perfect-insurance models and finds that the former exhibits higher costs than the latter in the case of no borrowing and lower costs when borrowing is permitted. The \textcite{huggett1993} economy features borrowers and lenders and offers a potential explanation as to why the observed risk-free rate is so much lower than the representative-agent models would suggest. The paper also shows that the risk-free rate is decreasing in the borrowing limit. \textcite{aiyagari1994} shows, in a framework where asset supply is given by heterogeneous households and asset demand by firms with a neoclassical production function, that the degree by which the presence of idiosyncratic risk influences aggregate savings and the equilibrium interest rate depends on the calibration of the parameters governing the desire for precautionary savings. \textcite{heath2009} provide an overview of modelling heterogeneous households in macroeconomics. See \textcite{cherrier2023} for a historical overview of the advances and different approaches in the macroeconomic analysis with heterogeneous agents in the 1980s and 1990s.} More recently, the \Gls{ha} framework has been fused with the New Keynesian approach of modelling the economy, giving rise to the already mentioned \Gls{hank} model class. An increasing number of questions is investigated using the \Gls{hank} framework. For example, \textcite{mckay2016} and \textcite{kaplan2018} use \Gls{hank} for the analysis of monetary policy, \textcite{bayer2023} for the investigation of fiscal policy and the supply of liquid assets by the government and \textcite{bayer2019} for the analysis of shocks to income risk.

%%%%%%%%%%%%%%%%%%%%%%%%%%%%%%%%%%%%%%%%%%%%%%%%%%%%%%%%%%%%%%%%%%%%%%%%%%%%%
%%%%%%%%%%%%%%%%%%%%%%%%%%%%%%%%%%%%%%%%%%%%%%%%%%%%%%%%%%%%%%%%%%%%%%%%%%%%%
%%%%%%%%%%%%%%%%%%%%%%%%%%%%%%%%%%%%%%%%%%%%%%%%%%%%%%%%%%%%%%%%%%%%%%%%%%%%%
\section{Baseline Model}
\label{sec:model}

This section puts forward the baseline \Gls{hank} model used in sections \ref{sec:limit} and \ref{sec:wedge}. The model is heavily inspired by the one-asset \Gls{hank} models in \textcite{auclert2021} and \textcite{boehl2023econpizza}, so I base the description of the present model to large extent on these studies.

Overall, the model features a household block, a corporate sector and a government which conducts monetary and fiscal policies. Section \ref{sec:model-hh} begins with the household sector. Section \ref{sec:model-sectors} deals with the firms, the fiscal and monetary authorities as well as with equilibrium. Finally, section \ref{sec:model-solution} touches upon the numerical solution method and explains the model calibration. 

In all what follows, time, $t$, is discrete, refers to quarters of a year and runs forever.

%%%%%%%%%%%%%%%%%%%%%%%%%%%%%%%%%%%%%%%%%%%%%%%%%%%%%%%%%%%%%%%%%%%%%%%%%%%%%
%%%%%%%%%%%%%%%%%%%%%%%%%%%%%%%%%%%%%%%%%%%%%%%%%%%%%%%%%%%%%%%%%%%%%%%%%%%%%
\subsection{Households}
\label{sec:model-hh}

Households in this model choose optimal paths of consumption, labour hours and savings, given prices. However, individual labour income varies stochastically. To smooth consumption in the presence of incomplete markets, households save in government bonds and lend to and borrow from each other. This private credit intermediation is subject to a debt limit and a premium on the borrowing rate. The present model allows for shocks to both of these constraints.

%%%%%%%%%%%%%%%%%%%%%%%%%%%%%%%%%%%%%%%%%%%%%%%%%%%%%%%%%%%%%%%%%%%%%%%%%%%%%
\subsubsection{Households' Environment}
\label{sec:model-hh-environment}

This subsection presents the formal environment that households are in. Specifically, the economy is populated by a mass-one continuum of \textit{ex-ante} identical households, indexed by $i$. Each household is subject to idiosyncratic labour income risk as its labour productivity, denoted by $e_{i,t}$, fluctuates over time. Specifically, $e_{i,t}$ evolves as a discrete Markov process, generated from the following first-order autoregressive (\Gls{ar}(1)) process:\footnote{The \textcite{rouwenhorst1995} method transforms the continuous process into a discrete one.}
\begin{equation}
    log(e_{i,t}) = \rho_e log(e_{i,t-1}) + \varepsilon^e_{i,t}, \label{eq:hh-income-process}
\end{equation}
where $\varepsilon^e_{i,t}$ is a mean-zero shock to agents' skill level with variance $\sigma_e$. Idiosyncratic productivity shocks die out with persistence $\rho_e$. Average productivity is one.\footnote{For simplicity, I ignore the possibility of unemployment here, in contrast to \textcite{gl2017}.}

Household-specific productivity, $e_{i,t}$, signifies how well household $i$ can turn its working hours into output in period $t$. For the household, process \eqref{eq:hh-income-process} is entirely exogenous and can not be influenced. $e_{i,t}$ enters the labour income of household $i$ in $t$, which is composed of idiosyncratic productivity, individual labour hours, $n_{i,t}$ and the aggregate wage rate, $w_t$.

Crucially, households face incomplete markets as they lack access to state-contingent insurance against idiosyncratic risk stemming from $\varepsilon^e_{i,t}$. Instead, households can only self-insure by accumulating precautionary savings in a non-state-contingent and perfectly liquid asset which matures after one period.\footnote{'Non-state-contingent' means that the asset pays out in any state of the (model) world. In particular, the payment is independent of $e_{i,t}$. 'Perfectly liquid' means that there are no costs involved in trading the asset.} This asset is real, which means that it is a claim to one unit of the consumption good in the next period. I denote the period-$t$ choice on holdings of this asset for period $t+1$ as $b_{i,t}$. As explained below, the model allows households to borrow. If agent $i$ chooses to borrow, $b_{i,t}$ is negative to denote the holding of debt.\footnote{Note that, as in most models (e.g.~\cite{kaplan2018}), asset and debt holdings alike here refer to \textit{net} positions, i.e.~$b_{i,t} > 0$ means that the asset side of the household's balance sheet is longer than its liability side.}

Given the prevailing prices in the economy, given the process in \eqref{eq:hh-income-process} and given the asset structure, households maximise their expected life-time utility:
\begin{equation}
    \max_{c_{i,t}, n_{i,t}, b_{i,t}} \ \mathbb{E}_{e,t} \Bigg\{ \sum_{t=0}^{\infty} \beta^t u(x_{i,t}) \Bigg\}, \label{eq:hh-objective}
\end{equation}
by choosing individual (non-durable) consumption in period $t$, $c_{i,t}$, individual labour supply in $t$, $n_{i,t}$, and next-period's asset holdings, $b_{i,t}$, for all periods. Note that $\mathbb{E}_{e,t}$ denotes expectations taken at time $t$ over the skill process \eqref{eq:hh-income-process}, $\beta$ is the factor by which households discount future utility, $u( \cdot )$ is the period-utility function and $x_{i,t}$ is a composite good of $c_{i,t}$ and $n_{i,t}$. 

The functional forms of $u( \cdot )$ and $x_{i,t}$ are specified below but, naturally, households derive utility from consumption and experience a utility loss from working. Thus, I require, first, $u( \cdot )$ to be increasing in $x_{i,t}$ and, second, $x_{i,t}$ to be increasing in $c_{i,t}$ and decreasing in $n_{i,t}$.\footnote{Formally: $\frac{\partial u( \cdot )}{\partial x_{i,t}} > 0$, $\frac{\partial x_{i,t}}{\partial c_{i,t}} > 0$, $\frac{\partial x_{i,t}}{\partial n_{i,t}} < 0$.}

In maximising \eqref{eq:hh-objective}, households are subject to a sequence of budget constraints and a sequence of borrowing constraints. Intuitively, the budget constraint states that, in any given period $t$, household $i$ must not spend more than it earns. The period-$t$ budget constraint for household $i$ reads as follows: 
\begin{equation}
c_{i,t} + b_{i,t} = RR (RR_t, b_{i,t-1}) b_{i,t-1} + w_t e_{i,t} n_{i,t} + \Pi_t \bar{\Pi} (e_{i,t}) - \tau_t \bar{\tau} (e_{i,t}), \label{eq:hh-budget}
\end{equation}
where $w_t$ is the aggregate real wage paid for each unit of effective labour supply, $e_{i,t} n_{i,t}$, and where $RR (b_{i,t-1}, RR_t)$ is the real interest rate schedule, which determines the return on or cost of previous period's asset or debt holdings, $b_{i,t-1}$. These two are the crucial prices guiding the households' decisions on working, consuming and saving. 

$\Pi_t$ are firm dividends and $\tau_t$ is a tax rate set by the government, discussed in subsequent sections. As in \textcite{mckay2016}, $\bar{\Pi} (e_{i,t})$ and $\bar{\tau} (e_{i,t})$ are incidence rules which transform aggregate dividends and taxes into household-level values in a lump-sum fashion.\footnote{It is well-known that, in \Gls{ha} models, the rule which distributes taxes and profits is of importance for the results of these models (see e.g.~\cite{kaplan2018}). As \textcite{mckay2016} explain, the present approach ensures that profits and taxes are distributed in a way analogous to lump-sum procedures in representative-agent models.} Thus, the left-hand side of \eqref{eq:hh-budget} captures the expenditure of household $i$ in $t$, comprised of goods consumption and asset acquisitions. Earnings, the right-hand side of \eqref{eq:hh-budget}, are composed of asset income, labour income and dividend income from firms, less taxes.

%The household takes all of these prices, profits and taxes as given when making its consumption-working-savings decisions. 

As already noted, the real interest rate schedule in \eqref{eq:hh-budget} is the key price for a household's consumption-savings decision. I specify the real interest rate schedule to take the following form:
\begin{equation}
    RR (RR_t, b_{i,t-1}) = \begin{cases}
	RR_t \ \ \ \ \ \ \ \ \ \ \ \ \ \text{if} \ b_{i,t-1} \ge 0 \\
	RR_t + \kappa_t \ \ \ \ \text{if} \ b_{i,t-1} < 0,
	\end{cases} \label{eq:hh-interest-rate-wedge}
\end{equation}
where $RR_t$ is the gross \textit{ex-post} real interest rate, to be defined momentarily, and where $\kappa_t \ge 0$ is a premium paid for negative asset positions. That is, at positive asset holdings, household $i$ earns the return $RR_t$. However, as soon as the household chooses to borrow, $b_{i,t-1} < 0$, it has to pay $RR_t$ plus $\kappa_t$, if the latter is strictly positive. This wedge is an additional cost to borrowing and is thus a \textit{financial constraint} to household financing, discussed in depth in subsection \ref{sec:model-hh-constraints}.

As in \textcite{auclert2021}, $RR_t$ in \eqref{eq:hh-interest-rate-wedge} adheres to the following Fisher relationship:
\begin{equation}
    RR_t = \frac{R_{t-1}}{\pi_t}, \label{eq:hh-fisher-equation}
\end{equation}
where $R_{t-1}$ is last period's gross nominal interest rate and $\pi_t$ is the current gross rate of inflation, which are determined by monetary policy and by firms, respectively. I discuss $R_{t-1}$ and $\pi_t$ in detail in the corresponding subsections \ref{sec:model-policy} and \ref{sec:model-firms}.

Finally, when choosing next period's asset holding, $b_{i,t}$, household $i$ faces a borrowing constraint à la \textcite{huggett1993}:
\begin{equation}
    b_{i,t} \ge \phi_t, \label{eq:borrowing-limit}
\end{equation}
where $\phi_t \le 0$ is an exogenously imposed limit to borrowing. $\phi_t = 0$ prohibits borrowing, $\phi_t < 0$ allows borrowing up to the specified limit. This limit to indebtedness is another \textit{financial constraint}. Subsection \ref{sec:model-hh-constraints} discusses the borrowing limit in detail.

%or, equivalently, the inverse of the intertemporal elasticity of substitution

%%%%%%%%%%%%%%%%%%%%%%%%%%%%%%%%%%%%%%%%%%%%%%%%%%%%%%%%%%%%%%%%%%%%%%%%%%%%%
\subsubsection{Solution to the Household Problem and Aggregation}
\label{sec:model-hh-solution}

Equations \eqref{eq:hh-income-process} -- \eqref{eq:borrowing-limit} constitute the formal setup of households. Before I turn to a discussion of the key mechanisms and implications of this framework, this subsection succinctly presents the solution to the household problem and aggregates household-level outcomes. Consult appendix \ref{sec-app:full-derivations-households} for the detailed derivations.

To solve the optimisation problem, I follow a series of studies in the \Gls{ha} literature to specify the functional forms of $u( \cdot )$ and $x_{i,t}$, see e.g.~\textcite{bayer2023}. The utility function in $t$ is of the constant relative-risk aversion (\Gls{crra}) type:
\begin{equation}
    u(x_{i,t}) = \frac{x_{i,t}^{1-\sigma_c}}{1-\sigma_c}, \label{eq:hh-utility-fun}
\end{equation}
where $\sigma_c$ is the coefficient of relative risk aversion, which reflects the degree with which households want to smooth consumption over time. The composite good, $x_{i,t}$, combines $c_{i,t}$ and $n_{i,t}$ as follows:
\begin{equation}
    x_{i,t} = c_{i,t} - e_{i,t}\frac{n_{i,t}^{1+\sigma_l}}{1+\sigma_l},
    \label{eq:hh-composite-good}
\end{equation}
where $\sigma_l$ is the inverse Frisch elasticity of labour supply. Solving \eqref{eq:hh-income-process} -- \eqref{eq:hh-composite-good} yields two first-order conditions (\Gls{foc}) and an envelope condition. The first \Gls{foc} and the envelope condition determine the households' optimal consumption, $c_{i,t}$, and savings policies, $b_{i,t}$, given last period's bond holdings, $b_{i,t-1}$, and the current period's productivity realisation, $e_{i,t}$:
\begin{align}
    x_{i,t}^{-\sigma_c} &\ge \beta \mathbb{E}_{e,t} \{ V_{b,t+1} (e_{i,t+1}, b_{i,t}) \}, \label{eq:foc-euler} \\
    V_{b,t} (e_{i,t}, b_{i,t-1}) &= RR_t (b_{i,t-1}) x_{i,t}^{-\sigma_c}, \label{eq:foc-envelope} 
\end{align}
where $V_{b,t}$ is the marginal continuation value of marginally increasing bond holdings for the next period. The Euler equation \eqref{eq:foc-euler} holds with equality whenever household $i$ is away from the borrowing limit. The second \Gls{foc} pins down the optimal labour supply of household $i$, given the real wage:
\begin{equation}
n_{i,t}^{\sigma_l} = w_t, \label{eq:foc-labour-supply}
\end{equation}
which, due to \Gls{ghh} preferences, is independent of household $i$'s consumption level and asset position, see appendix \ref{sec-app:full-derivations-households}.

The fact that households transverse different paths of productivity realisations from \eqref{eq:hh-income-process}, induces them to accumulate and decumulate asset holdings at varying speeds. This gives rise to \textit{ex-post} household heterogeneity with respect to productivity states and asset holdings. That is, at time $t$, there is a distribution of agents over skill levels and wealth, which I denote by $\Psi_t (b,e)$, where, for notational convenience, $b=b_{i,t-1}$ and $e=e_{i,t}$.\footnote{Note the timing: The distribution in $t$ is determined by how much wealth (or debt) agents in $t-1$ chose to hold in $t$ and by where the productivity process dictates them to be in $t$.}

% In a discretised setup, $\Psi_t (b,e)$ essentially returns Pr[].

To obtain economy-wide consumption, labour-supply and savings, I thus simply integrate individual-level outcomes over the distribution of agents across productivity levels and previous period's asset holdings. Formally:
\begin{equation}
    C_t \equiv \int c_{i,t} d \Psi_t (b,e), \ \ \ N_t \equiv \int n_{i,t} d \Psi_t (b,e), \ \ \ B_t^d \equiv \int b_{i,t} d \Psi_t (b,e), \label{eq:hh-aggregation}
\end{equation}
where $C_t$ is aggregate consumption, $N_t$ is aggregate supply of labour hours and $B_t^d$ is aggregate asset demand. Note that due to the fact that all households supply the same amount of labour, individual and aggregate supplies co-move one-to-one and \eqref{eq:foc-labour-supply} generalises to the aggregate labour supply condition:
\begin{equation}
    N_t^{\sigma_l} = w_t. \label{eq:hh-agg-labour-supply}
\end{equation}

%, because \Gls{iou}s cancel each other out from household balance sheets. Thus, $B_t^d$ corresponds to aggregate demand for government bonds.

%%%%%%%%%%%%%%%%%%%%%%%%%%%%%%%%%%%%%%%%%%%%%%%%%%%%%%%%%%%%%%%%%%%%%%%%%%%%%
\subsubsection{Incomplete Markets, Assets, Borrowers and Lenders}
\label{sec:model-hh-assets}

Next, the following two subsections aim at clarifying the key mechanisms of the chosen household framework. Here, I discuss the implications of the chosen asset structure and how borrowers and lenders emerge in the model. Thereafter, subsection \ref{sec:model-hh-constraints} explains how the financial constraints introduced above affect households' consumption behaviour.  

In the model, households' main objective is to smooth consumption over time. However, they are subject to idiosyncratic labour productivity risk from \eqref{eq:hh-income-process} and thus fear the possibility of obtaining low labour income, which hurts their consumption. Further, households dread the possibility of reaching the borrowing limit in \eqref{eq:borrowing-limit}, e.g.~after repeatedly being in low productivity states. Being at the debt limit would be particularly harmful for households because it bars them from taking on any additional debt to absorb adverse productivity shocks, forcing them to adjust consumption one-to-one with the reduction in their income (see e.g.~\cite{carroll2021}). 

Clearly, households have an incentive to insure themselves against labour income shocks.\footnote{The chosen \Gls{ghh} preferences eliminate the possibility of households using working hours as insurance against low productivity. Section \ref{sec:sensitivity-end-labour} explores the case where households can adjust labour supply individually.} However, in an environment of incomplete markets, contracts which insure against specific realisations of labour productivity, high or low, are unavailable.\footnote{Incidentally, assuming complete markets and hence the presence of state-contingent claims would reduce the problem to a representative-agent setup, rendering the investigation of the present issues vacuous.} As \textcite{heath2009} remark, households thus resort to self-insurance by means of precautionary savings which they can draw down when confronted with (potentially repeatedly) low labour income. The existence of the borrowing limit further strengthens the wish to save. Crucially, the desire for precautionary savings contributes to the \textit{demand for assets} by households. %\footnote{\textcite{heath2009} understand precautionary savings as a buildup of assets with the purpose of providing a buffer against the risk of low productivity in the future.}

Some households, however, do remain in states of low productivity for prolonged periods of time. In order to nevertheless sustain a certain level of consumption, these households have a high willingness to tap into debt to finance the related expenditures, even if this involves owing the borrowed funds and the return $RR_t + \kappa_t$. Thus, these households have a \textit{demand for debt} and are allowed to become indebted as long as $\phi_t < 0$.

Now, it is important to realise that the demand for debt corresponds to a source of \textit{asset supply}. This is because a debt contract is a vehicle for the lender to postpone consumption to the future as such a contract specifies that the borrowing household will, in the next period, return the borrowed funds plus some interest to the lending household. Effectively, borrowing creates a private liquid asset in the form of an "I Owe You" (\Gls{iou}), i.e.~a debt contract which represents a lender's claim to a future unit of consumption which the borrower has to provide. Consequently, the act of borrowing reflects private credit intermediation (see e.g.~\cite{bayer2023}).

In this setup, \Gls{iou}s are mutually beneficial since the borrowers' desire to trade consumption today against a promise to repay in the future is matched by the lenders' wish to save today in return for a claim to future consumption. On the one hand, lenders are willing to supply these funds as they have experienced a series of positive productivity shocks and thus demand precautionary savings. On the other hand, borrowers have gone through low productivity and would like to obtain consumption today in return for providing claims to future consumption to the lenders, which for the latter is insurance against future shocks.

Observe then that each positive \Gls{iou} that a lender holds is a negative \Gls{iou} for some borrower. That is, the assets that the lenders hold on the asset side of their balance sheets are one-to-one the debt that borrowing households have on the liability side of their balance sheet. Importantly, this implies that the two groups of households, which are seemingly disconnected, are tightly linked through their balance sheets. This crucially implies that when borrowers are forced, out of some reason, to curtail their leverage and thus their liabilities, lenders will be affected directly as the supply of \Gls{iou}s they can invest in is reduced. Evidently, the credit crunch will, through interlinked balance sheets, have repercussions that outsize the initial effect on borrowers. This is part of the investigations in subsequent sections of this thesis.

As lenders' and borrowers' asset positions net out, \Gls{iou}s are in zero net supply. Therefore, up to this point, the asset structure is essentially the one found in the \textcite{huggett1993} economy, where aggregate net liquidity is zero. However, I follow a series of papers (e.g.\cite{gl2017}, \cite{bayer2023}) by introducing an exogenous and time-invariant supply of government bonds, $B$. These bonds are perfect substitutes to the privately intermediated \Gls{iou}s and are thus an additional asset which households can save in. %Here, essentially the government is the borrower with a fixed amount of debt.

Clearly, the real interest rate, $RR_t$, plays a key role in determining the willingness of households to save or borrow as it reflects the price of doing so. Naturally, borrowers like low interest rates on their debt, while lenders prefer high returns on their savings. To match these contrarious interests, $RR_t$ adjusts endogenously so that, in equilibrium, saving and borrowing households do not have an incentive to save more or less or to borrow more or less. This yields equilibrium in the asset market.

\textcolor{red}{Note that the demand for bonds is essentially demand for consumption tomorrow instead of consumption today. The interest rate makes people with relatively high consumption content to wait and to postpone consumption into the future. Now, when the supply of bonds is reduced and the demand for bonds goes up, then there is more demand for consumption tomorrow than there is supply. To bring consumption forward and thus the asset market into equilibrium, the interest rate must fall.}

\textcolor{red}{Note the different specification of the real interest in the present environment versus the one in the model of \textcite{gl2017}. In their model, the timing of asset and debt holdings is different and the real interest rate is just the inverse of the price of bonds tomorrow. In my model, the interest rate is expost on the bonds of yesterday and is pinned down by the Fisher equation, i.e.~by inflation and the nominal interest rate. Essentially, however, the same mechanism will be at play.}

%Note crucially that the \Gls{iou}s in this model correspond to unsecured borrowing. That is, borrowers do not post any collateral against their debt. One of the reasons why lenders are nevertheless content with conceding funds to the borrowers is because the model does not consider the risk of borrowers defaulting.\footnote{Thus, indebted households always repay their debt. In part, this is the case here because the borrowing limit in \eqref{eq:borrowing-limit} is, in all chosen calibrations, tighter than the \textit{natural borrowing limit}, see appendix \ref{sec:app-hh-borrowing} for details.}

%This makes clear that in order to include household debt, macroeconomic models need to include at least some degree of household heterogeneity.

%What is more, there is government insurance and insurance from employers, see \textcite{heath2009}. \textcite{kaplan2018} provide an insightful critique of the one-asset structure. 

%First, models which neglect or only parsimoniously model household heterogeneity cannot speak to rich interactions between borrowers and lenders. There is either no scope at all for a borrowing and a lending fraction of the population (representative-agent models) or a limitation as to how the two fractions can endogenously evolve (two-agent models). The latter naturally limits the scope of questions that can be answered by a model like the one from \textcite{egg2012}.

%\footnote{As \textcite{kv2018} note, \Gls{hank} models are furthermore capable of giving insights into the changes in inequality upon aggregate shocks. This might also be a reason for using a model with a rich treatment of heterogeneity over a standard representative-agent model.}

%%%%%%%%%%%%%%%%%%%%%%%%%%%%%%%%%%%%%%%%%%%%%%%%%%%%%%%%%%%%%%%%%%%%%%%%%%%%%
\subsubsection{Financial Constraints and Consumption Behaviour}
\label{sec:model-hh-constraints}

In obtaining credit from lenders, borrowing households in the present economy are subject to two financial constraints: an outright limit to debt and a premium on the lending rate they pay on the borrowed funds. How do these frictions work and how do they affect household behaviour?

Equation \eqref{eq:borrowing-limit} caps the maximum \textit{quantity} of debt a household can hold on its balance sheet to amounts smaller or equal to $\phi_t$. In everyday life, households often encounter such limits. For example, credit card holders experience limits to how much they can spend with their credit cards. Also, home buyers are often not able to take out loans which cover the full price of the house they want to purchase and are thus subject to a mortgage loan limit.\footnote{The example of credit card debt is more fitting in terms of the model as it usually is unsecured debt, just as the \Gls{iou}s of the model. Mortgages count to secured, i.e.~collateralised, debt.}

As stated in \textcite{carroll2021}, a borrowing limit constrains households with unfortunate income realisations in their ability to smooth prolonged shocks to their consumption levels. This is because the debt limit introduces an asymmetry at $b_{i,t} = \phi_t$: households can always choose to retire debt -- e.g.~if they experience an increase in income -- but are restricted in taking up more debt in case of an unfortunate realisation of skills and income. 

This asymmetry, or kink, in the savings decision of households influences their consumption behaviour. Importantly, the borrowing constraint forces households that are \textit{at} the constraint to consume in a hand-to-mouth fashion as they can only consume their labour and profit incomes, less taxes and interest payments. This implies that, for these households, a negative shock to productivity translates into a one-to-one fall in consumption, exposing these agents in an extreme way to fluctuations in their labour skills. 

Consequently, the affected households have a very high inclination to consume if given a small windfall of liquid wealth. In other words, because they are at the debt limit and would probably like to borrow more, these agents will consume all of any small amount of extra wealth or, equivalently, less debt and thus remain constrained. Hence, individuals at the constraint have very high marginal propensities to consume (\Gls{mpc}) out of a marginal windfall of liquid wealth.\footnote{I define and calculate the \Gls{mpc} as follows: $mpc_{i,t} = \frac{\partial c_{i,t}}{\partial b_{i,t}}$, see \textcite{gl2017}.} The data confirm the existence of such behaviour, see \textit{inter alia} \textcite{gross2002}.

Even households that are away from the borrowing limit are affected by its existence as they anticipate the risk of reaching the limit in finite time, which would expose them to the same asymmetry as the actually constrained agents. The fear of being constrained in the future prompts even wealthier and more productive households to engage in precautionary savings. However, the further away the household is from the constraint, the less probability it assigns to the possibility of being constrained. These households in turn behave increasingly like fully unconstrained agents, which the literature calls permanent-income consumers. Crucially, these agents feature significantly lower \Gls{mpc}s out of liquid wealth as they were able to already reach rather high levels of consumption which they can shield well with their high wealth. In total, borrowers -- especially those close at the debt limit -- are expected to have a much higher marginal propensity to consume (\Gls{mpc}) out of liquid wealth than lenders do. This proves an important point in the subsequent analyses.

Apart from outright limits to borrowing, households may be forced to pay a premium, $\kappa_t$, on the lending rate for their borrowed funds, as specified in \eqref{eq:hh-interest-rate-wedge}.\footnote{Note that $\kappa_t = \infty$ would be equivalent to $\phi_t=0$.} This is a feature often encountered in the real world as well. Financial institutions typically charge higher rates on accounts with negative balances than they pay on accounts with positive balances. As the present model does not feature any institution which sets the premium $\kappa_t$ (e.g.~in order to maximise profits), the wedge reflects a wasted cost in private credit intermediation as in e.g.~\textcite{bayer2023}. It thus operates as a stand-in for the notion that there is a kink at zero assets.

\textcolor{red}{As the online appendix of \textcite{achdou2022} notes, such a kink can account for empirical facts of the wealth distribution.\footnote{Specifically, I refer to appendix G.3 of the online appendix of \textcite{achdou2022}, which is available here: \url{https://benjaminmoll.com/wp-content/uploads/2019/07/HACT_appendix.pdf}, last retrieved \printdate{2023-07-27}.} Here, I employ it as a device to capture another household-level financial constraint. In particular, it is an extra \textit{cost} of holding debt.}

\textcolor{red}{In essence, it is a constraint that affects all borrowers and thus a larger fraction of the population. Also, it introduces a kink in the savings decision of the household already at zero assets rather than at the borrowing limit and might thus be interesting to study.}

\textcolor{red}{It might also lead to more \Gls{mpc} heterogeneity.}

Clearly, financial constraints such as the borrowing limit and the interest rate wedge might very well vary across household characteristics (e.g.~income, wealth, credit-worthiness) and asset types (e.g.~credit card debt vs mortgage debt). Importantly however, these constraints can vary over time, reflecting the fact that credit can become more or less abundant in the course of time. For example, \textcite{anenberg2018} empirically document how the limits to mortgage loan amounts do not only differ across borrower types but also that they evolve over time.\footnote{\textcite{anenberg2018} use a frontier estimation approach to show \textit{inter alia} how lending constraints became looser in the run-up to the Great Recession and tightened during and after the crisis.} Section \ref{sec:limit} and \ref{sec:wedge} aim to capture the effects of the exogenous component of such variation over time in order to make causal statements about the mechanisms induced by a credit crunch.\footnote{In general, financial constraints and their changes over time are of course endogenous. The model here abstracts from this and sets the constraints exogenously. This allows to study the effects of exogenous variation of the borrowing limit and the interest rate wedge. In a more comprehensive model, both constraints would be the result of the optimisation of some agents, e.g.~financial institutions.}

In summary, the present model features household subject to idiosyncratic risk. Some households are "lucky", while others turn out to be "unlucky". The former lend to the latter, which is mutually beneficial. The real interest rate plays the decisive role of intermediating between the two sides. Finally, borrowers and lenders crucially differ in their consumption behaviour.

%First, unlike their representative-agent counterparts, prototypical \Gls{ha} models explicitly incorporate meaningful borrowing limits.\footnote{Prototypical representative-agent models impose some kind of transversality condition to rule out the possibility of a household running an indefinite Ponzi scheme.} Second, \Gls{ha} models speak to the differential effect of borrowing constraints across the distribution of households.

%%%%%%%%%%%%%%%%%%%%%%%%%%%%%%%%%%%%%%%%%%%%%%%%%%%%%%%%%%%%%%%%%%%%%%%%%%%%%
%%%%%%%%%%%%%%%%%%%%%%%%%%%%%%%%%%%%%%%%%%%%%%%%%%%%%%%%%%%%%%%%%%%%%%%%%%%%%
\subsection{Firms, Government and Equilibrium}
\label{sec:model-sectors}

This section closes the model with a corporate sector and a government which controls a fiscal and a monetary authority. Further, I formally define equilibrium for the present model.

%%%%%%%%%%%%%%%%%%%%%%%%%%%%%%%%%%%%%%%%%%%%%%%%%%%%%%%%%%%%%%%%%%%%%%%%%%%%%
\subsubsection{Firms}
\label{sec:model-firms}

The corporate sector of the model is kept minimal and close to the standard \Gls{nk} literature. My modelling approach is close to the one in \textcite{mckay2016}, but makes use of some ingredients from \textcite{gust2017wp}. More specifically, I assume that a representative and perfectly competitive firm produces the final good by using the inputs from a continuum of monopolistically competitive intermediate-goods firms which produce using labour from households as single input. The intermediate-goods firms can set prices but in doing so are subject to convex price adjustment costs à la \textcite{rotemberg1982}. 

Appendix \ref{sec-app:full-derivations-firms} formally states the firms' maximisation problems and solves for the equations which determine their optimal behaviour. Here, I simply state the equations relevant for the model implementation. First, aggregate production of the final good, $y_t^f$, occurs according to the following simple production function:
\begin{equation}
    y_t^f = z N_t, \label{eq:firms-production}
\end{equation}
where $z$ is technology and $N_t$ is aggregate labour input. Aggregate marginal costs are thus given by $mc_t = \frac{w_t}{z}$. 

Costly price adjustment makes the intermediate-goods firms' problem dynamic, leading to a \Gls{nk} Phillips Curve (\Gls{nkpc}) of the form:
\begin{equation}
    \psi \Bigg( \frac{\pi_t}{\pi_{ss}} - 1 \Bigg) \frac{\pi_t}{\pi_{ss}} = (1-\theta) + \theta mc_t + \psi \beta \Bigg( \frac{\pi_{t+1}}{\pi_{ss}} - 1 \Bigg) \frac{\pi_{t+1}}{\pi_{ss}} \frac{y_{t+1}^f}{y_t^f}, \label{eq:firms-nkpc}
\end{equation}
where $\pi_t \equiv \frac{P_t}{P_{t-1}}$ is the change in the price level in period $t$, i.e.~the gross rate of inflation in period $t$. $\pi_{ss}$ is steady-state gross inflation, $\theta$ is the elasticity of substitution between the varieties of intermediate goods in the production of the final good and $\psi$ is the parameter which determines the costs to price adjustment.  

% \begin{equation*}
%     \pi_t \equiv \frac{P_t}{P_{t-1}} %\label{eq:firms-inflation}
% \end{equation*}

The \Gls{nkpc} pins down inflation in this model and reflects the trade-off between changes in economic activity and changes in prices. In essence, it captures the mechanism that a reduction in goods demand leads to a reduction in inflation -- or deflation -- through lower marginal costs, which come about through lower wages as labour demand adjusts to lower consumption demand. This trade-off gives rise to a channel where aggregate demand, through firms and production, directly influences the \textit{ex-post} real interest rate, $RR_t$, making transparent the tight link between aggregate demand and the real rate in the model.

%is a well-known relationship at the core of modern \Gls{nk} business cycle models. It pins down inflation in this model and reflects the trade-off between changes in economic activity and changes in prices. In essence, it captures the mechanism that a reduction in goods demand leads to a reduction in inflation -- or deflation -- through lower marginal costs, which come about through lower wages as labour demand adjusts to lower consumption demand.

Finally, monopolistic competition in the market for inputs implies that intermediate-goods firms can sell their output with a markup, $\mu_t$, resulting in firm profits:
\begin{equation}
    \Pi_t = \Bigg[ 1 - \frac{\psi}{2} \Bigg( \frac{\pi_t}{\pi_{t-1}} - 1 \Bigg)^2 \Bigg] y_t^f - w_t N_t, \label{eq:firms-profits}
\end{equation}
which, as explained above, are distributed lump-sum to households. Note that this setup implicitly assumes that households own equal shares in the firms, which they can not trade.

%Equations \eqref{eq:firms-production} -- \eqref{eq:firms-profits} describe the corporate sector of the model. Evidently, I abstract from many aspects, such as investment and capital, for the sake of simplicity. %For example, including capital as a production input would require a two-asset structure.

%%%%%%%%%%%%%%%%%%%%%%%%%%%%%%%%%%%%%%%%%%%%%%%%%%%%%%%%%%%%%%%%%%%%%%%%%%%%%
\subsubsection{Fiscal and Monetary Policies}
\label{sec:model-policy}

The government conducts fiscal and monetary policy. As alluded to in section \ref{sec:model-hh-assets}, the fiscal authority supplies a time-invariant quantity of government bonds, $B$, which imposes interest payments on the government. To run a balanced budget, the fiscal authority levies the already specified tax on households, $\tau_t$. Thus, the government budget constraint reads:
\begin{equation}
    \tau_t + B = RR_t B \ \ \ \Leftrightarrow \ \ \ \tau_t = (RR_t - 1)B, \label{eq:fiscal-budget}
\end{equation}
which implies that the tax rate moves one-to-one with the \textit{ex-post} real interest rate.

The monetary authority sets the notional gross nominal interest rate, $R_{t}^n$, in order to stabilise inflation and output. It does so according to a standard Taylor rule:
\begin{equation}
    R_{t}^n = ( R_{t-1}^n )^{\rho} \Bigg[ R_{ss} \Bigg( \frac{\pi_t}{\pi_{ss}} \Bigg)^{\phi_{\pi}} \Bigg( \frac{y_t}{y_{ss}} \Bigg)^{\phi_y} \Bigg]^{(1-\rho)}, \label{eq:taylorrule}
\end{equation}
where $R_{ss}$ is the steady state gross nominal interest rate, to be defined momentarily, and $y_t$ is total "effective" output of the economy, specified in section \ref{sec:model-eq}.

Equation \eqref{eq:taylorrule} implies that the central bank responds -- with persistence $\rho$ -- to deviations of inflation and output from their steady state values. The strengths of these responses are given by the parameters $\phi_{\pi}$ and $\phi_y$, respectively. However, in setting the notional rate, the central bank is subject to the \Gls{zlb} on nominal interest rates. Hence, the gross nominal interest rate, $R_t$, is pinned down by the maximum out of the notional gross nominal interest rate and $1$: 
\begin{equation}
    R_t = \text{max} \ \{ 1, \ R_{t}^n \}. \label{eq:zlb}
\end{equation}

%In contrast to \textcite{gl2017}, the real interest rate here is pinned down by the Fisher equation:
%which implies that firms' price setting and the central bank influence and determine the real interest rate. This offers a new mechanism by which the economy can adjust. 

This completes the description of the models' agents. For the reader's convenience, I depict the baseline model graphically in figure \ref{fig:graphical-model} in appendix \ref{sec-app:full-graphical}.

%%%%%%%%%%%%%%%%%%%%%%%%%%%%%%%%%%%%%%%%%%%%%%%%%%%%%%%%%%%%%%%%%%%%%%%%%%%%%
\subsubsection{Equilibrium and Market Clearing}
\label{sec:model-eq}

Next, I define equilibrium and state the conditions by which markets clear.\footnote{For the definition, I follow \textcite{gl2017} and.} Recall that $\Psi_t (b,e)$ is the distribution of agents over $b_{i,t-1}$ and $e_{i,t}$ at $t$.

\begin{definition}[Equilibrium]
\label{def:eq}
An equilibrium in this economy is defined as a sequence of household consumption and savings policies, $\{ c_{i,t} (b, e) \}_{t=0}^{\infty}$ and $\{ b_{i,t} (b, e) \}_{t=0}^{\infty}$, a sequence of distributions, $\{ \Psi_t (b,e) \}_{t=0}^{\infty} $, and a sequence of aggregate outcomes $\{ B_t^d$, $C_t$, $mc_t$, $N_t$, $\pi_t$, $\Pi_t$, $RR_t$, $R_t$, $R_t^n$, $\tau_t$, $w_t$, $y_t$, $y_t^f \}$ such that:
\begin{enumerate}
    \item given prices, the policies, $c_{i,t} (b, e)$ and $b_{i,t} (b, e)$, solve the households' maximisation problem given in equations \eqref{eq:hh-income-process} -- \eqref{eq:borrowing-limit},
    \item together with the Markov process for productivity, the policies induce a law of motion for the distribution, denoted by $\Gamma$,
    \item given an initial distribution $\Psi_0$, the distribution solves the forward equation of the law of motion:
    \begin{equation*}
        \Psi_{t+1} (b_{i,t},e_{i,t+1}) = \Psi_{t} (b_{i,t-1},e_{i,t}) \Gamma,
    \end{equation*}
    \item all markets clear.
\end{enumerate}
\end{definition}

The labour market clears when condition \eqref{eq:hh-agg-labour-supply} holds. The bond and goods markets clear, respectively, when:
\begin{align}
    B_t^d &= B, \label{eq:bond-clearing} \\
    y_t &= C_t \label{eq:goods-clearing}
\end{align}
hold, where $y_t$ is the output that remains after subtracting costs of price adjustment and of wasteful credit intermediation:
\begin{equation}
    y_t \equiv \Bigg( 1 - \frac{\psi}{2} \Bigg( \frac{\pi_t}{\pi_{t-1}} - 1 \Bigg)^2 \Bigg) y_t^f - \kappa_t D_t, \label{eq:effective-gdp}
\end{equation}
where $D_t$ denotes aggregate debt holdings:
\begin{equation}
    D_t \equiv - \int b_{i,t} \mathbb{1}_{[\phi_t \le b_{i,t} \le 0]} d \Psi_t (b,e), \label{eq:agg-debt}
\end{equation}
where $\mathbb{1}_{[\phi_t \le b_{i,t} \le 0]}$ is an indicator function with a value of one for negative asset holdings and zero otherwise. Note that \eqref{eq:bond-clearing} holds because private \Gls{iou}s cancel each other out and so, from the household perspective, net liquidity is given by the quantity of government bonds. 

%%%%%%%%%%%%%%%%%%%%%%%%%%%%%%%%%%%%%%%%%%%%%%%%%%%%%%%%%%%%%%%%%%%%%%%%%%%%%
%%%%%%%%%%%%%%%%%%%%%%%%%%%%%%%%%%%%%%%%%%%%%%%%%%%%%%%%%%%%%%%%%%%%%%%%%%%%%
\subsection{Solution Method and Calibration}
\label{sec:model-solution}

The \Gls{hank} model does not allow for an analytical solution. This subsection therefore briefly discusses the numerical solution approach I use and the calibration of model parameters I choose.

%%%%%%%%%%%%%%%%%%%%%%%%%%%%%%%%%%%%%%%%%%%%%%%%%%%%%%%%%%%%%%%%%%%%%%%%%%%%%
\subsubsection*{Numerical Solution Method}

As is well-known to the literature, \Gls{ha} models are notoriously difficult to solve since general equilibrium and rational expectations require that agents are able to forecast market-clearing prices which depend on the sequence of distributions, $\{ \Psi_t (b,e) \}_{t=0}^{\infty} $, each element of which is an infinite-dimensional object. By now, a variety of solution methods for this challenge exists but choosing among methods still involves trade-offs. The present thesis makes use of the approach laid out by \textcite{boehl2023econpizza} and made available through the Python package \texttt{Econpizza}.\footnote{The package is maintained by Dr.~Gregor Boehl on \href{https://github.com/gboehl/econpizza}{GitHub}, last retrieved on \printdate{2023-7-17}.} % along different dimensions

In short, the method by \textcite{boehl2023econpizza} builds on recent work by \textcite{auclert2021} and implements a perfect-foresight approach to find the non-linear transitions of \Gls{ha} models after one-time, unexpected ("MIT") shocks. To do so, the method makes use of a Newton root-finding procedure that searches in sequence space for aggregate outcomes, truncated at some distant terminal period, which are consistent with the household policies and the law of motion for the distribution, both of which are given on discretised grids of productivity states and asset holdings. The household policies are found with the endogenous-gridpoints method (\Gls{egm}) by \textcite{carroll2006}, which iterates on equations \eqref{eq:foc-euler} and \eqref{eq:foc-envelope} to find the households' consumption and savings policies.

On the one hand, the implementation by \textcite{boehl2023econpizza} offers advantages over competitor methods. For example, it flexibly accommodates non-linearities in the aggregate state space of the model and is typically very fast due to the use of automatic differentiation. On the other hand, meaningful uncertainty in aggregates -- as opposed to idiosyncratic uncertainty -- is disregarded. In particular, the method assumes that households know the exact trajectory of the economy once the disturbance to the model is announced. Nevertheless, accounting for non-linearities, such as the \Gls{zlb}, appears important for the present use case, so the method by \textcite{boehl2023econpizza} seems appropriate.
% a technique widely used in fitting machine learning models

Appendix \ref{sec-app:codes} briefly dwells on some caveats regarding the implementation of the two specific shocks with the \texttt{Econpizza} package.

%%%%%%%%%%%%%%%%%%%%%%%%%%%%%%%%%%%%%%%%%%%%%%%%%%%%%%%%%%%%%%%%%%%%%%%%%%%%%
\subsubsection*{Calibration of Model Parameters}

Numerically solving a model implies that the model parameters must be calibrated. Table \ref{tab:calibration-baseline} reports the calibration of the baseline model. Overall, I proceed pragmatically by using values that have been used in the literature before and I state wherever I do so.\footnote{Calibration is always to some degree arbitrary. Estimation of model parameters is preferred but beyond the scope of this thesis.}

\begin{table}[t]
\centering
\caption{Calibration of the Baseline Model}
\label{tab:calibration-baseline}
\begin{tabular}{llcc}
Parameter &   &  Value & Source/Target \\
\hline
\hline
\multicolumn{2}{l}{\textbf{Households}} & & \\
$\beta$ &  discount factor & $0.99056$ & $RR_{ss}-1 = 0.15\%$ \\
$\sigma_c$ & coefficient of relative risk aversion & $2$ & \textcite{auclert2021} \\
$\sigma_l$ & inverse Frisch elasticity & $2$ & \textcite{auclert2021} \\
$\rho_e$ & persistence in productivity shocks & $0.966$ & \textcite{auclert2021} \\
$\sigma_e$ & variance of productivity shocks & $0.5$ & \textcite{auclert2021} \\
$N_e$ & number of productivity grid points & $8$ & Appendix \ref{sec-app:codes-ep} \\
$N_b$ & number of asset grid points & $219$ & Appendix \ref{sec-app:codes-ep} \\
& & & \\
\multicolumn{2}{l}{\textbf{Firms}} & & \\
$z$ &  technology & $1$ & \\
$\theta$ & elasticity of substitution & $6$ & \textcite{auclert2021} \\
$\psi$ & parameter on price adjustment costs & $100$ & Flat \Gls{nkpc} \\
& & & \\
\multicolumn{2}{l}{\textbf{Fiscal Policy}} & & \\
$B$ & supply of government bonds & $5.21$ & $B/y_{ss} = 5.71$ \\
& & & (\cite{mckay2016}) \\
& & & \\
\multicolumn{2}{l}{\textbf{Monetary Policy}} & & \\
$\pi_{ss}$ & steady state gross inflation & $1$ & No steady-state net inflation \\
$\rho$ & persistence in notional interest rate & $0.8$ & Standard \\
$\phi_{\pi}$ & Taylor rule coefficient on inflation & $2$ & See text \\
$\phi_{y}$ & Taylor rule coefficient on output & $0.15$ & \\
& & & \\
\multicolumn{2}{l}{\textbf{Financial Constraints}} & & \\
$\phi_{ss}$ & initial borrowing limit & $-2.3485$ & $D_{ss}/y_{ss} = 57\%$ \\
& & & (\cite{bayer2023}) \\
$\kappa_{ss}$ & initial interest rate wedge & 1e-8 & No initial wedge \\
\hline
\multicolumn{4}{l}{\footnotesize \multirow{2}{15cm}{\textit{Note:} The table displays the calibration of the model parameters in the baseline version of the model. All numbers refer to quarterly values. $p.p.$ stands for percentage points.}} \\
& & & \\
\end{tabular}
\end{table}

I choose $\beta$, the households' discount factor, such that the net real interest rate is $0.15\%$ in steady state. This arguably low target aims to capture the possibility of the \Gls{zlb} binding during the credit crunch, given the importance which the literature assigns to this mechanism (e.g.~\cite{egg2012}, \cite{gl2017}). Appendix \ref{sec-app:robust-beta-calibration} re-calibrates $\beta$ so that the \Gls{zlb} does not bind after a shock to the borrowing limit.

For the parameters in the household utility function \eqref{eq:hh-utility-fun}, I use a standard value of $2$ for both, the coefficient of relative risk aversion and the inverse Frisch elasticity of labour supply. For the skills process \eqref{eq:hh-income-process}, I set $\rho_e = 0.95$ and $\sigma_e = 0.5$.\footnote{Consult appendix \ref{sec-app:codes-ep} for details on the choices on discretising the productivity process and the asset grid.} \textcite{auclert2021} use the same values for all those parameters.

%\footnote{Appendix \ref{sec-app:robust-beta} shows that the impulse responses for the baseline model in the case that $\beta$ is chosen such that the steady-state real interest rate is high enough so the \Gls{zlb} does not bind during the transition after the credit crunch.}

On the production side, I set $\theta = 6$ as in \textcite{auclert2021} and $\psi = 100$. I do so to achieve a rather flat \Gls{nkpc} with a slope of $0.06$, which seems in line with empirical estimates.\footnote{See e.g.~\cite{schorfheide2008} and recent cross-regional evidence from the U.S.~by \textcite{hazell2022}, which suggests an even flatter Phillips Curve.} Note that in this simple setup, the choice of $\theta$ pins down all of the following: $\mu_{ss}$, $mc_{ss}$, $w_{ss}$, $N_{ss}$. As long as $\kappa$ does not change, also consumption and output, $C_{ss}$ and $y_{ss}$, remain the same across steady states. In turn, this implies that, after a permanent disturbance to the model, the interest rate must adjust to clear the goods and bond markets.

I set $B$, the supply of real government bonds, along the lines of \textcite{mckay2016} and \textcite{gl2017}. In particular, $B$ matches the ratio of total liquid assets held by households in the US to US real gross domestic product (\Gls{gdp}), averaged over $1951$ -- $2023$.\footnote{See appendix \ref{sec-app:data} for a description of the data and the sources.} Appendix \ref{sec-app:robust-bonds} shows the main results for a calibration of $B$ following the approach by \textcite{bayer2023}.

The parameters for monetary policy are mostly standard. I set steady-state gross inflation to $1.0$, implying no net inflation in steady state and $\rho = 0.8$, which is a standard value. I choose $\phi_{\pi} = 2$ and $\phi_y = 0.15$. The former is arguably a strong reaction of monetary policy to inflation, but lies inbetween the values found in two recent estimations of fully-fledged \Gls{hank} models, \textcite{boehl2022} and \textcite{bayer2023}.

Finally, consider the choices for the debt limit and the interest rate wedge in the steady state of relatively "easy" credit. I set  $\kappa_{ss} = $ 1e-8, so that there is virtually no premium on borrowing to begin with. Akin to the approaches in \textcite{gl2017} and \textcite{bayer2023}, I choose $\phi_{ss}$, the debt limit in the initial steady state, so that the ratio of debt which households hold in the model to output is equal to the fraction of unsecured consumer credit over \Gls{gdp}, both for the US and averaged over $1951$ -- $2023$.\footnote{See appendix \ref{sec-app:data} again for details.} % which turns out to be $57\%$.

%%%%%%%%%%%%%%%%%%%%%%%%%%%%%%%%%%%%%%%%%%%%%%%%%%%%%%%%%%%%%%%%%%%%%%%%%%%%%
%%%%%%%%%%%%%%%%%%%%%%%%%%%%%%%%%%%%%%%%%%%%%%%%%%%%%%%%%%%%%%%%%%%%%%%%%%%%%
%%%%%%%%%%%%%%%%%%%%%%%%%%%%%%%%%%%%%%%%%%%%%%%%%%%%%%%%%%%%%%%%%%%%%%%%%%%%%
\section{A Credit Crunch through the Borrowing Limit}
\label{sec:limit}

This section confronts the \Gls{hank} model with a credit crunch through a permanent tightening of the household-level borrowing limit. To that end, section \ref{sec:limit-stst-loose} begins by briefly characterising the steady state of relatively loose borrowing, i.e.~with $\phi_{ss}$ and $\kappa_{ss}$ as in table \ref{tab:calibration-baseline}. Then, section \ref{sec:limit-stst-tight} describes the steady state with a tighter credit limit and compares the two states. Thereafter, section \ref{sec:limit-transition} considers the dynamics from easy to tight credit.

%%%%%%%%%%%%%%%%%%%%%%%%%%%%%%%%%%%%%%%%%%%%%%%%%%%%%%%%%%%%%%%%%%%%%%%%%%%%%
%%%%%%%%%%%%%%%%%%%%%%%%%%%%%%%%%%%%%%%%%%%%%%%%%%%%%%%%%%%%%%%%%%%%%%%%%%%%%
\subsection{Steady State with a Loose Borrowing Limit}
\label{sec:limit-stst-loose}

%%%%%%%%%%%%%%%%%%%%%%%%%%%%%%%%%%%%%%%%%%%%%%%%%%%%%%%%%%%%%%%%%%%%%%%%%%%%%
\subsubsection*{Household Policies and Asset Distribution}

Figure \ref{fig:init-stst-dist-pol} displays cross-sectional characteristics of the initial steady state with $\phi_{ss}$ and $\kappa_{ss}$. In particular, panel \ref{fig:init-stst-dist-pol-a-dist} shows the steady-state distribution of households over bond and \Gls{iou} holdings. This distribution features households with positive as well as negative asset positions. All agents situated to the left of $0$ are net debtors: they owe more in debt than they own in assets. These households have gone through repeated unfortunate productivity realisations and subsequently tap into debt to sustain consumption. In the present calibration, roughly $33.6\%$ of households are net borrowers, see table \ref{tab:stst_comparison_baseline_limit_permanent}. $4.9\%$ of the population end up at the borrowing limit, i.e.~for them $b_{i,ss} = \phi_{ss}$. These individuals might want to borrow more but are prohibited from doing so because of the borrowing limit.

\begin{figure}[t]
    \caption{The Initial Steady State}
    \label{fig:init-stst-dist-pol}
    \centering
    \begin{subfigure}[b]{0.49\textwidth}
    \caption{Asset Distribution}
    \label{fig:init-stst-dist-pol-a-dist}
         \centering
         \includesvg[inkscapelatex=false, width=\textwidth]{Results/stst_dist_2d_bar_baseline_limit_permanent_initial.svg}
     \end{subfigure}
     \hfill
     \begin{subfigure}[b]{0.49\textwidth}
     \caption{Consumption Policies}
     \label{fig:init-stst-dist-pol-c}
         \centering
         \includesvg[inkscapelatex=false, width=\textwidth]{Results/stst_policies_c_baseline_limit_permanent_initial.svg}
     \end{subfigure}
     \begin{flushleft}
     \footnotesize
	\textit{Note}: In figure \ref{fig:init-stst-dist-pol-a-dist}, the distribution is cut off at an asset level of $50$ for readability. As the model is solved on a logarithmic grid, values at high bond holdings overestimate the true density at large grid points. In figure \ref{fig:init-stst-dist-pol-c}, the labels $\theta\_0$ -- $\theta\_7$ represent the idiosyncratic labour skill states from \eqref{eq:hh-income-process}.
	\end{flushleft}
\end{figure}

The rest of the population have assets in excess of their liabilities. Net lenders have experienced a series of positive productivity realisations and build positive asset positions in order to smooth consumption and to save in precaution of potentially worse skill endowments. Some of these households invest their funds in bonds, others in the \Gls{iou}s originated by borrowers.

As section \ref{sec:model-hh} explains, it is to be expected that asset positions have implications for consumption behaviour. Figure \ref{fig:init-stst-dist-pol-c} shows that this is indeed the case in the present setup by plotting the household consumption policies over assets, separately for each state of idiosyncratic productivity. Naturally, lower asset holdings and/or productivity states translate into lower consumption levels. 

Importantly however, the curvature of the consumption policies also depends on the asset level. For very low levels of wealth and high debt burdens, the consumption policies are concave and much steeper than for higher asset levels. Intuitively, this means that consumption of households at or near the borrowing limit strongly responds to changes in the asset positions of these households. As expected, these individuals thus have high \Gls{mpc}s.

Indeed, table \ref{tab:stst_comparison_baseline_limit_permanent} confirms that the average borrower has an \Gls{mpc} of $0.16$, while the average lender features an \Gls{mpc} of merely $0.02$.\footnote{The aggregate \Gls{mpc} equals $0.0663$, which is quite low when compared to empirical estimates, see \textcite{kaplan2018}. One reason for this is that the model disregards the wealthy hand-to-mouth of \textcite{kaplan2014} and \textcite{kaplan2018}.} Thus, an average borrower would increase consumption by $17$ cents if given one dollar more in assets (or less in debt). Note that this distinction still masks substantial heterogeneity in the \Gls{mpc}s among borrowers, see figure \ref{fig:baseline-limit-initial-mpcs} in appendix \ref{sec-app:figures}. Overall, this suggests that borrowers -- especially those at or very close to the limit $\phi_{ss}$ -- are much more sensitive to changes in their asset positions and incomes, as alluded to in section \ref{sec:model-hh}.\footnote{The fact that \Gls{mpc}s are heterogeneous is empirically plausible (see e.g.~\cite{gross2002}), but their magnitudes might not be exactly matched by the simple model.}

% EVEN BORROWERS' MPCS ARE HETEROGENEOUS

%Indeed, table \ref{tab:stst_Baseline_Limit_Permanent} confirms this heterogeneity in \Gls{mpc}s.

%In contrast to borrowers and asset-poor households, asset-rich individuals behave approximately as permanent-income consumers. Their consumption functions are almost exactly linear, implying that they increase or decrease their consumption by the same constant and small fraction of any positive or negative windfall in income.

% As explained in \textcite{kv2018}, the one-asset structure chosen here generates too many poor hand-to-mouth households, in comparison to the data, see \textcite{kaplan2014}.

%%%%%%%%%%%%%%%%%%%%%%%%%%%%%%%%%%%%%%%%%%%%%%%%%%%%%%%%%%%%%%%%%%%%%%%%%%%%%
\subsubsection*{Aggregates}

The aggregate dimension of the initial steady state does not feature many noteworthy aspects. The real and nominal interest rates are $0.15\%$ and household debt over output equals $57\%$, as calibrated. The aggregate wage rate and aggregate labour hours imply an average quarterly labour income of $0.76$, which means that the initial debt limit allows households to borrow up to $3.1$ times their average quarterly labour income. 

%%%%%%%%%%%%%%%%%%%%%%%%%%%%%%%%%%%%%%%%%%%%%%%%%%%%%%%%%%%%%%%%%%%%%%%%%%%%%
%%%%%%%%%%%%%%%%%%%%%%%%%%%%%%%%%%%%%%%%%%%%%%%%%%%%%%%%%%%%%%%%%%%%%%%%%%%%%
\subsection{Steady State with a Tight Borrowing Limit}
\label{sec:limit-stst-tight}

To induce a state of tight credit, I choose a new, smaller borrowing limit, $\phi_{ss}' < \phi_{ss}$, such that the household debt-to-output ratio is five percentage points lower in the terminal steady state than it is in the initial state. Taking into account the observed developments in the fraction of unsecured credit to \Gls{gdp} over the last decades (see figure \ref{fig:app-calibration-debt} in appendix \ref{sec-app:data}), this appears to be a magnitude which might realistically be experienced in reality.\footnote{\textcite{gl2017} consider a ten percentage points-reduction of the ratio of debt to \textit{annual} \Gls{gdp}. This translates into a drop of $40$ percentage points in quarterly terms, i.e.~a much larger change than I consider.} 

\begin{figure}[t]
    \caption{Baseline Model -- Shock to the Borrowing Limit: Comparison of Steady States}
    \label{fig:comparison-stst-baseline}
    \centering
    \begin{subfigure}[b]{0.49\textwidth}
    \caption{Asset Distributions}
         \centering
         \includesvg[inkscapelatex=false, width=\textwidth]{Results/stst_dist_2d_comparison_baseline_limit_permanent.svg}
     \end{subfigure}
     \hfill
     \begin{subfigure}[b]{0.49\textwidth}
     \caption{Asset Accumulation}
         \centering
         \includesvg[inkscapelatex=false, width=\textwidth]{Results/stst_asset_acc_comparison_baseline_limit_permanent.svg}
     \end{subfigure}
     \begin{flushleft}
     \footnotesize
	\textit{Note}: %As the model is solved on a logarithmic grid, values at high bond holdings overestimate the true density at large grid points.
	\end{flushleft}
\end{figure}

%Looking at the time series of household debt over time, figure \ref{fig:app-calibration-debt} in the appendix, reveals that changes of five percentage points are well within the realm of observed short-term changes, $40$ much less so.

% where the $\phi_{ss}$ is the terminal steady-state borrowing limit. The shock occurs unexpectedly at time $t=1$. Once the shock is announced, the economy moves deterministically to the terminal steady state. 

%That is, the solution of transitions via the perfect-foresight approach implies that agents perfectly anticipate future developments in distributions as well as prices and can thus act accordingly. This assumption

% \begin{table}[H] 
% \centering
% \caption{Calibration of the Shock Processes}
% \label{tab:calibration-shocks}
% \begin{tabular}{llcc}
% Parameter &   &  Value & Source/Target \\
% \hline
% \hline
% \multicolumn{2}{l}{\textbf{A) Shock to the Borrowing Limit}} & & \\
% $\phi_{ss}$ & initial borrowing limit & $-2.352$ & $D_{ss}/y_{ss} = 57\%$ \\
% & & & (\cite{bayer2023}) \\
% $\phi_{ss}'$ & terminal borrowing limit & $-2.18$ & $\Delta (D_{ss}'/y_{ss}) = -5 \ p.p.$ \\
% & & & (See text) \\
% $\rho_{\phi}$ & persistence in borrowing limit & $0.5$ & Shock duration six quarters \\
% & & & \\
% \multicolumn{2}{l}{\textbf{B) Shock to the Interest Rate Wedge}} & & \\
% $\kappa_{ss}$ & initial interest rate wedge & 1e-8 & No initial wedge \\
% $\kappa_{ss}'$ & terminal interest rate wedge & $0.634\%$ & $RR_{ss}' - 1 = 0.02\%$ \\
% & & & (See text) \\
% $\rho_{\kappa}$ & persistence in interest rate wedge & $0.25$ & Shock duration six quarters \\
% \hline
% \multicolumn{4}{l}{\footnotesize \multirow{3}{14cm}{\textit{Note:} The table displays the calibration of the model parameters for the processes of the permanent shocks to the borrowing limit and the interest rate wedge, respectively. $p.p.$ stands for percentage points.}} \\
% & & & \\
% & & & \\
% \end{tabular}
% \end{table}

%The new aggregate \Gls{mpc} is higher, consistent with the observation of \textcite{kaplan2018} that reductions in the supply of assets available to households increase the aggregate \Gls{mpc}.

% The purpose of the present exercise is not to accurately match all moments of the empirical data but rather to flesh out the main mechanisms by which a shock to the borrowing limit affects individual-level and aggregate behaviour.

%%%%%%%%%%%%%%%%%%%%%%%%%%%%%%%%%%%%%%%%%%%%%%%%%%%%%%%%%%%%%%%%%%%%%%%%%%%%%
\subsubsection*{Comparing Steady States}

Gross liquidity is reduced by the shock, reducing the possibility for "lucky" households to transfer consumption into the next period and to self-insure.

% \begin{table}[t]
% \centering
% \caption{Baseline Model -- Shock to the Borrowing Limit: Comparison of Steady States}
% \label{tab:stst_Baseline_Limit_Permanent}
% \begin{tabular}{lrrr}
%                 Variable &  Initial &  Terminal &  Change \\
% \hline
% \hline
% \multicolumn{2}{l}{\textbf{Shock Process}} & & \\
%               $\phi_t$ &  -0.9900 &   -0.7300 &  0.2600 \\
%               & & & \\
% \multicolumn{2}{l}{\textbf{Aggregates}} & & \\
%                        $C_t$ &   0.9487 &    0.9487 &  0.0000 \\
%                      div &   0.0949 &    0.0949 &  0.0000 \\
%                       DY &   0.3519 &    0.2520 & -0.0999 \\
%                gr\_liquid &   1.8338 &    1.7390 & -0.0948 \\
%                       mc &   0.9000 &    0.9000 &  0.0000 \\
%                      MPC &   0.1472 &    0.1601 &  0.0129 \\
%                        $N_t$ &   0.9487 &    0.9487 &  0.0000 \\
%                       pi &   1.0000 &    1.0000 &  0.0000 \\
%                  pitilde &   1.0000 &    1.0000 &  0.0000 \\
%                        R &   1.0019 &    1.0002 & -0.0017 \\
%                     Rbar &   0.0000 &    0.0000 &  0.0000 \\
%                       Rn &   1.0019 &    1.0002 & -0.0017 \\
%                       $RR_t$ &   1.0019 &    1.0002 & -0.0017 \\
%                  Rrminus &   1.0019 &    1.0002 & -0.1697 \\
%                      tau &   0.0028 &    0.0003 & -0.0025 \\
%                        w &   0.9000 &    0.9000 &  0.0000 \\
%                & & & \\
% \multicolumn{2}{l}{\textbf{Cross-Section}} & & \\
%       Frac. of Borrowers &  46.5300 &   44.3800 & -2.1500 \\
% Frac. at Borrowing Limit &  10.9800 &   12.0900 &  1.1100 \\
%     Frac. at Zero Assets &   0.5600 &    0.5000 & -0.0600 \\
% \hline
% \end{tabular}
% \end{table}

\begin{table}[H]
\centering
\caption{Baseline Model -- Shock to the Borrowing Limit: Comparison of Steady States}
\label{tab:stst_comparison_baseline_limit_permanent}
\begin{tabular}{lccc}
                Variable & Initial & Terminal &  Change \\
\hline
\hline
\multicolumn{2}{l}{\textbf{Financial Constraints}} & & \\
$\phi$ & -2.3485 &  -2.1775 &  0.1710 \\
$\kappa$ & 1e-8 &  1e-8 & -- \\
& & & \\
\multicolumn{2}{l}{\textbf{Aggregates}} & & \\
$C$ &  0.9129 &   0.9129 &  -- \\
$D / y$ &    $57\%$ & $52\%$ & $-5 \ p.p.$ \\
$B + D$ &  5.7304 &   5.6847 & -0.0457 \\
$RR - 1$ &  $0.15\%$ &  $0.13\%$ & -0.0002 \\
$\tau$ &  0.0078 &   0.0069 & -0.0009 \\
%$w$ &  0.8333 &   0.8333 &  0.0000 \\
& & & \\
\multicolumn{2}{l}{\textbf{Cross-Section}} & & \\
                  Share of Assets Held by Top $10\%$ &  0.4834 &   0.4774 & -1.2412 \\
                  Share of Consumption by Top $10\%$ &  0.1729 &   0.1731 &  0.1157 \\
                  %  Top1A &  0.0852 &   0.0841 & -1.2911 \\
                  %  Top1C &  0.0222 &   0.0222 &  0.0000 \\
                  % Top25A &  0.8253 &   0.8155 & -1.1874 \\
                  % Top25C &  0.3741 &   0.3744 &  0.0802 \\
Fraction of Borrowers &   33.66 &     32.9 & -0.7600 \\
Fraction at Borrowing Limit &    4.89 &     5.07 &  0.1800 \\
Fraction at Zero Assets &    0.33 &     0.35 &  0.0200 \\
MPC of Borrowers &    0.16 &     0.17 &  0.0100 \\
MPC of Lenders &    0.02 &     0.02 &  0.0000 \\
\hline
\multicolumn{4}{l}{\footnotesize \multirow{2}{12cm}{\textit{Note:} The table displays selected values of the initial steady state with $\phi_{ss}$ and the terminal steady state with $\phi_{ss}'$. The change is either in absolute or in percentage terms. The numbers are rounded.}} \\
& & & \\
\end{tabular}
\end{table}
% , $b_{i,ss} = 0$

%%%%%%%%%%%%%%%%%%%%%%%%%%%%%%%%%%%%%%%%%%%%%%%%%%%%%%%%%%%%%%%%%%%%%%%%%%%%%
%%%%%%%%%%%%%%%%%%%%%%%%%%%%%%%%%%%%%%%%%%%%%%%%%%%%%%%%%%%%%%%%%%%%%%%%%%%%%
\subsection{Credit Crunch Dynamics}
\label{sec:limit-transition}

This subsection turns to the dynamics of the model induced by the change in the borrowing limit. The shock arises unexpectedly in $t=1$. Thereafter, $\phi_t$ evolves deterministically according to an \Gls{ar}(1) process:
\begin{equation}
    \phi_t = \phi_{ss} \Bigg( \frac{\phi_{t-1}}{\phi_{ss}} \Bigg)^{\rho_{\phi}} exp(\varepsilon_{\phi}), \label{eq:borrowing-limit-process}
\end{equation}
where $\rho_{\phi}$ is the persistence of changes in the borrowing limit.\footnote{$\varepsilon_{\phi}$ is a one-time shock to the borrowing limit. In this thesis, $\varepsilon_{\phi} = 0$ $\forall t$.} I follow \textcite{gl2017} in choosing a duration of the credit crunch of six quarters, i.e.~one and a half years, which requires $\rho_{\phi} = 0.5$. Doing so is meant to capture the notion that a financial shock might not force households to delever fully within one period but might take some time to manifest itself in household balance sheets.\footnote{Section \ref{sec:sensitivity-persistence} explores alternative choices for the speed of the credit crunch episode.} 

Next, I present and explain the dynamic consequences of the credit crunch episode, i.e.~the transition from a state of loose borrowing (with $\phi_{ss}$) to a state of tight borrowing (with $\phi_{ss}'$). I proceed as follows. First, I dwell on the effects of the shock on asset policies and the asset market, thereby emphasising the endogenous adjustment of the real interest rate. Thereafter, I discuss the response of consumption and further incomes. 

%%%%%%%%%%%%%%%%%%%%%%%%%%%%%%%%%%%%%%%%%%%%%%%%%%%%%%%%%%%%%%%%%%%%%%%%%%%%%
\subsubsection{Savings Policies and the Interest Rate}

Figure \ref{fig:baseline-permanent-limit-agg-phi} shows the development of the borrowing limit. The first direct effect of the reduction in the debt limit is the deleveraging forced on the immediately constrained agents.

%%%%%%%%%%%%%%%%%%%%%%%%%%%%%%%%%%%%%%%%%%%%%%%%%%%%%%%%%%%%%%%%%%%%%%%%%%%%%
\subsubsection{Consumption, Employment and Incomes}

%Figure \ref{fig:baseline-permanent-limit-agg} presents the responses of some key variables to the shock in the borrowing limit. First, panel \ref{fig:baseline-permanent-limit-agg-phi} shows the development of the borrowing limit.

Clearly, the response of aggregate consumption is not very persistent and does not last much longer than the shock itself. Note however that the responses in \textcite{gl2017} are also not very persistent. Their model also lacks capital, habit formation or investment adjustment costs. Interestingly however, the response of household debt shows strong persistence. 

%%%%%%%%%%%%%%%%%%%%%%%%%%%%%%%%%%%%%%%%%%%%%%%%%%%%%%%%%%%%%%%%%%%%%%%%%%%%%
% \subsubsection*{Aggregate Dynamics}

Clearly, here the responses of borrowers and lenders do not wash out in the aggregate as in \textcite{justiniano2015}.

The figures with the \Gls{zlb} not binding also show that, unlike in \textcite{egg2012}, the \Gls{zlb} is not the crucial amplification mechanism.

This creates the perfect storm for borrowers: their incomes fall, taxes increase, they have to delever, the rate on their current debt spikes. And they do not have insurance through working more due to \Gls{ghh} preferences.

A higher \textit{ex-post} real interest rate in $t=1$ is a windfall profit for lenders and a windfall cost for borrowers. Going forward, lenders anticipate however that the real rate has to fall.

The target-rate of the central bank adjusts immediately. That is, the central bank knows immediately in $t=1$ where the real rate will settle in the long run.

In the short-run, household debt-to-GDP spikes as output falls faster than households' debt positions. This is something that \textcite{justiniano2015} also find but argue that this increase was much larger in the Great Recession. 

Household debt-to-GDP \textit{increases} on impact, before starting to fall to its new lower level.

%\footnote{The reader might refer to \textcite{mian2017} to object this as that paper finds that an \textit{increase} in debt leads to a fall in GDP. However, GDP in their statistical analyses falls only as soon as debt falls after an exogenous increase.}

Inflation falls. Note that this does not induce a a debt-deflation spiral à la \textcite{fisher1933} because assets are in real terms, meaning that their value immediately adjusts with inflation.

\begin{figure}[H]
    \centering
    \caption{Baseline Model -- Shock to the Borrowing Limit: Aggregate Dynamics}
    \label{fig:baseline-permanent-limit-agg}
     \begin{subfigure}[b]{0.49\textwidth}
     \caption{Borrowing Limit, $\phi$}
     \label{fig:baseline-permanent-limit-agg-phi}
         \centering
         \includesvg[inkscapelatex=false, width=\textwidth]{Results/transition_phi_baseline_limit_permanent.svg}
     \end{subfigure}
    \hfill
    \begin{subfigure}[b]{0.49\textwidth}
    \caption{Consumption, $C$}
         \centering
         \includesvg[inkscapelatex=false, width=\textwidth]{Results/transition_C_baseline_limit_permanent.svg}
     \end{subfigure}
     \hfill
     \begin{subfigure}[b]{0.49\textwidth}
     \caption{Debt-to-Output, $\frac{D}{y}$}
         \centering
         \includesvg[inkscapelatex=false, width=\textwidth]{Results/transition_DY_baseline_limit_permanent.svg}
     \end{subfigure}
     \hfill
     \begin{subfigure}[b]{0.49\textwidth}
     \caption{\textit{Ex-Post} Real Interest Rate, $RR$}
         \centering
         \includesvg[inkscapelatex=false, width=\textwidth]{Results/transition_Rr_baseline_limit_permanent.svg}
     \end{subfigure}
     \hfill
     \begin{subfigure}[b]{0.49\textwidth}
     \caption{Nominal and Notional Interest Rates, $R$ and $R^n$}
         \centering
         \includesvg[inkscapelatex=false, width=\textwidth]{Results/transition_R_Rn_baseline_limit_permanent.svg}
     \end{subfigure}
     \hfill
     \begin{subfigure}[b]{0.49\textwidth}
     \caption{Inflation, $\pi$}
         \centering
         \includesvg[inkscapelatex=false, width=\textwidth]{Results/transition_pi_baseline_limit_permanent.svg}
     \end{subfigure}
     \begin{flushleft}
     \footnotesize
	\textit{Note}: Percent deviations mean percent deviations from the respective initial steady state. The time horizon in all plots is fixed to 12 quarters.
	\end{flushleft}
\end{figure}

%%%%%%%%%%%%%%%%%%%%%%%%%%%%%%%%%%%%%%%%%%%%%%%%%%%%%%%%%%%%%%%%%%%%%%%%%%%%%
\subsubsection*{Distributional Dynamics}

A shock to the model is effectively an exogenous shock to the supply of IOUs. This can be due to a lot of reasons, \textit{inter alia} due to lost confidence in the ability of borrowers to repay their debt.

As noted by \textcite{gl2017}, this shock has the effect of forcing some to save more and the desire of others to save more, too. This reduced the bond market-clearing real interest rate. However, note that this reduction again incentivises consumption.

In \textcite{gl2017}, just less than $1\%$ of the population are at the borrowing limit in the initial steady state. However, note that their stationary distributions feature comparable fractions of borrowers overall: $33\%$ and $27\%$, respectively.\footnote{Based on own calculations from the replication codes by \textcite{gl2017}. The MATLAB codes are available here: \url{https://sites.northwestern.edu/guidolor/}, last retrieved \printdate{2023-6-28}.}

As \textcite{gl2017} note, the consumption response will depend on the aggregate MPC. If the MPC is large, then the consumption response will be large.

Note how the fraction of borrowers after the shock is endogenously reduced, a feature that cannot be captured by a two-asset model structure, where the fraction of lenders and borrowers remain constant over time.

%%%%%%%%%%%%%%%%%%%%%%%%%%%%%%%%%%%%%%%%%%%%%%%%%%%%%%%%%%%%%%%%%%%%%%%%%%%%%
%%%%%%%%%%%%%%%%%%%%%%%%%%%%%%%%%%%%%%%%%%%%%%%%%%%%%%%%%%%%%%%%%%%%%%%%%%%%%
%%%%%%%%%%%%%%%%%%%%%%%%%%%%%%%%%%%%%%%%%%%%%%%%%%%%%%%%%%%%%%%%%%%%%%%%%%%%%
\section{A Credit Crunch through the Interest Rate Wedge}
\label{sec:wedge}

Having understood the consequences of a shock to the household-level borrowing limit, this sections turns to a credit crunch arising through an increase in the \textit{price} of debt. Thus, I am at identifying the similarities and differences in the dynamics induced by the two financial shocks. Crucially, this section finds that the consequences are similar qualitatively, but that important quantitative and distributional divergences exist. 

Analogously to \eqref{eq:borrowing-limit-process}, let the process of the interest rate wedge, $\kappa_t$, be given by:
\begin{equation}
    \kappa_t = \kappa_{ss} \Bigg( \frac{\kappa_{t-1}}{\kappa_{ss}} \Bigg)^{\rho_{\kappa}} exp(e_{\kappa}), \label{eq:wedge-process}
\end{equation}
where $\rho_{\kappa}$ is again the persistence of the process. $\rho_{\kappa} = 0.3$ results in the duration of the adjustment being complete within six periods. 

%%%%%%%%%%%%%%%%%%%%%%%%%%%%%%%%%%%%%%%%%%%%%%%%%%%%%%%%%%%%%%%%%%%%%%%%%%%%%
%%%%%%%%%%%%%%%%%%%%%%%%%%%%%%%%%%%%%%%%%%%%%%%%%%%%%%%%%%%%%%%%%%%%%%%%%%%%%
% \subsection{Steady States with Cheap and Expensive Borrowing}
% \label{sec:wedge-stst}

%%%%%%%%%%%%%%%%%%%%%%%%%%%%%%%%%%%%%%%%%%%%%%%%%%%%%%%%%%%%%%%%%%%%%%%%%%%%%
% \subsubsection*{Policies and Distributions}

%%%%%%%%%%%%%%%%%%%%%%%%%%%%%%%%%%%%%%%%%%%%%%%%%%%%%%%%%%%%%%%%%%%%%%%%%%%%%
% \subsubsection*{Aggregates}

\begin{table}[h]
\centering
\caption{Permanent Shock to the Interest Rate Wedge: Comparison of Steady States}
\label{tab:stst_Baseline_Wedge_Permanent}
\begin{tabular}{lrrr}
                Variable &  Initial &  Terminal &  Change \\
\hline
\hline
              borr\_limit &  -0.9900 &   -0.9900 &  0.0000 \\
                       C &   0.9487 &    0.9460 & -0.2846 \\
                     div &   0.0949 &    0.0949 &  0.0000 \\
                      DY &   0.3519 &    0.2858 & -0.0661 \\
               gr\_liquid &   1.8338 &    1.7704 & -0.0634 \\
                      mc &   0.9000 &    0.9000 &  0.0000 \\
                     MPC &   0.1472 &    0.1141 & -0.0331 \\
                       n &   0.9487 &    0.9487 &  0.0000 \\
                      pi &   1.0000 &    1.0000 &  0.0000 \\
                 pitilde &   1.0000 &    1.0000 &  0.0000 \\
                       R &   1.0019 &    1.0001 & -0.0018 \\
                    Rbar &   0.0000 &    0.0100 &  0.0100 \\
                      Rn &   1.0019 &    1.0001 & -0.0018 \\
                      Rr &   1.0019 &    1.0001 & -0.0018 \\
                 Rrminus &   1.0019 &    1.0101 &  0.8184 \\
                     tau &   0.0028 &    0.0002 & -0.0026 \\
                       w &   0.9000 &    0.9000 &  0.0000 \\
                       y &   0.9487 &    0.9460 & -0.2846 \\
                  y\_prod &   0.9487 &    0.9487 &  0.0000 \\
      Frac. of Borrowers &  46.5300 &   42.1300 & -4.4000 \\
Frac. at Borrowing Limit &  10.9800 &    7.7400 & -3.2400 \\
    Frac. at Zero Assets &   0.5600 &    0.2800 & -0.2800 \\
\hline
\end{tabular}
\end{table}

%%%%%%%%%%%%%%%%%%%%%%%%%%%%%%%%%%%%%%%%%%%%%%%%%%%%%%%%%%%%%%%%%%%%%%%%%%%%%
%%%%%%%%%%%%%%%%%%%%%%%%%%%%%%%%%%%%%%%%%%%%%%%%%%%%%%%%%%%%%%%%%%%%%%%%%%%%%
% \subsection{Transition to Expensive Borrowing}
% \label{sec:wedge-transition}

%A more involved model would of course model the intermediation costs as profits to a financial intermediary. As those profits would then be distributed to households, the effect should be dampened. This, however, depends also on the way that profits are distributed.

% \subsubsection*{Aggregate Dynamics}



%%%%%%%%%%%%%%%%%%%%%%%%%%%%%%%%%%%%%%%%%%%%%%%%%%%%%%%%%%%%%%%%%%%%%%%%%%%%%
%%%%%%%%%%%%%%%%%%%%%%%%%%%%%%%%%%%%%%%%%%%%%%%%%%%%%%%%%%%%%%%%%%%%%%%%%%%%%
%%%%%%%%%%%%%%%%%%%%%%%%%%%%%%%%%%%%%%%%%%%%%%%%%%%%%%%%%%%%%%%%%%%%%%%%%%%%%
\section{The Asymmetry of Financial Shocks}
\label{sec:asymmetry}

This section explores the asymmetry of financial shocks.

%%%%%%%%%%%%%%%%%%%%%%%%%%%%%%%%%%%%%%%%%%%%%%%%%%%%%%%%%%%%%%%%%%%%%%%%%%%%%
%%%%%%%%%%%%%%%%%%%%%%%%%%%%%%%%%%%%%%%%%%%%%%%%%%%%%%%%%%%%%%%%%%%%%%%%%%%%%
%%%%%%%%%%%%%%%%%%%%%%%%%%%%%%%%%%%%%%%%%%%%%%%%%%%%%%%%%%%%%%%%%%%%%%%%%%%%%
\section{Sensitivity Analyses}
\label{sec:sensitivity}

This section completes the formal analysis of the present thesis by exploring the effects of two crucial aspects in the preceding analysis: the speed with which a credit crunch takes effect and the ability of households to absorb shocks via individual labour supply. First, subsection \ref{sec:sensitivity-persistence} varies the persistence of the shock processes \eqref{eq:borrowing-limit-process} and \eqref{eq:wedge-process} in order to show the differences in the economy's responses to the varying speeds of the transition to tighter credit. Second, subsection \ref{sec:sensitivity-end-labour} adopts a preference specification for households which allows households to self-insure through labour supply.

%%%%%%%%%%%%%%%%%%%%%%%%%%%%%%%%%%%%%%%%%%%%%%%%%%%%%%%%%%%%%%%%%%%%%%%%%%%%%
%%%%%%%%%%%%%%%%%%%%%%%%%%%%%%%%%%%%%%%%%%%%%%%%%%%%%%%%%%%%%%%%%%%%%%%%%%%%%
\subsection{Duration of the Credit Crunch}
\label{sec:sensitivity-persistence}

To that end, I vary the speed of adjustment to the new state of tight credit. In particular, I compare the baseline scenario to two alternatives: first, the shock resolves in ten quarters and second, the shock lasts for only three quarters.

Note that the grids differ across the models due to the computational challenges with the changing grid. 

%%%%%%%%%%%%%%%%%%%%%%%%%%%%%%%%%%%%%%%%%%%%%%%%%%%%%%%%%%%%%%%%%%%%%%%%%%%%%
%%%%%%%%%%%%%%%%%%%%%%%%%%%%%%%%%%%%%%%%%%%%%%%%%%%%%%%%%%%%%%%%%%%%%%%%%%%%%
\subsection{Insurance through Endogenous Labour Supply}
\label{sec:sensitivity-end-labour}

\Gls{ghh} preferences preclude households from adjusting working hours in order to absorb shocks to their consumption. This section explores an extension of the model which allows households to do so. 

This allows wealth-poor households to potentially insure against reductions in income by supplying more labour hours.

Households now have \Gls{crra}-type preferences:
\begin{equation}
    u(c_{i,t}, n_{i,t}) = \frac{c_{i,t}^{1-\sigma_c}}{1-\sigma_c} - \phi \frac{n_{i,t}^{1+\sigma_l}}{1+\sigma_l}, \label{eq:hh-crra-utility}
\end{equation}
where $\phi$ is the weight on the disutility of labour. I follow \textcite{mckay2016} and set $\phi=1$.\footnote{There is a wide range of values in use for this parameter.}

Note that in this setup, output is an entirely endogenous object. 

%%%%%%%%%%%%%%%%%%%%%%%%%%%%%%%%%%%%%%%%%%%%%%%%%%%%%%%%%%%%%%%%%%%%%%%%%%%%%
%%%%%%%%%%%%%%%%%%%%%%%%%%%%%%%%%%%%%%%%%%%%%%%%%%%%%%%%%%%%%%%%%%%%%%%%%%%%%
%%%%%%%%%%%%%%%%%%%%%%%%%%%%%%%%%%%%%%%%%%%%%%%%%%%%%%%%%%%%%%%%%%%%%%%%%%%%%
\section{Concluding Remarks}
\label{sec:conclusion}

Policy considerations abound.

Would be interesting to extend the model so  that there is also borrowing by other entities, especially firms.

An interesting extension of the model would be a small open economy as then funds could potentially be obtained elsewhere.

Borrowing limits are, of course, endogenous outcomes. The present approach abstracted from this endogeneity. Uncovering the \textit{sources} of time-variations in the borrowing limits (also in the cross-section) requires addressing the endogeneity and explicitly modelling borrowing limits as an outcome of decisions by market participants.

Future work should aim at endogenising borrowing limits and interest rate wedges. Some advances in the latter can be found in Boehl 2023 The micro and macro.

As \textcite{kv2018} remark, an empirical investigation of the shocks driving the  Great Recession would ideally take into account how consumption responded along the entire wealth distribution in order to discern whether the shocks analysed here were a driving force of the recession.

%%%%%%%%%%%%%%%%%%%%%%%%%%%%%%%%%%%%%%%%%%%%%%%%%%%%%%%%%%%%%%%%%%%%%%%%%%%%%
%%%%%%%%%%%%%%%%%%%%%%%%%%%%%%%%%%%%%%%%%%%%%%%%%%%%%%%%%%%%%%%%%%%%%%%%%%%%%
%%%%%%%%%%%%%%%%%%%%%%%%%%%%%%%%%%%%%%%%%%%%%%%%%%%%%%%%%%%%%%%%%%%%%%%%%%%%%
% References section
\newpage
\thispagestyle{plain}
\pagenumbering{Roman}
\printbibliography[heading=bibintoc] % Insert references

%%%%%%%%%%%%%%%%%%%%%%%%%%%%%%%%%%%%%%%%%%%%%%%%%%%%%%%%%%%%%%%%%%%%%%%%%%%%%
%%%%%%%%%%%%%%%%%%%%%%%%%%%%%%%%%%%%%%%%%%%%%%%%%%%%%%%%%%%%%%%%%%%%%%%%%%%%%
%%%%%%%%%%%%%%%%%%%%%%%%%%%%%%%%%%%%%%%%%%%%%%%%%%%%%%%%%%%%%%%%%%%%%%%%%%%%%
% Appendices section
\newpage
\begin{refsection}
\thispagestyle{plain}
\pagenumbering{arabic}  % Turn page numbering to arabic
\renewcommand*{\thepage}{A-\arabic{page}} % Add 'A' to each page number for appendices section
\addtocontents{toc}{\protect\setcounter{tocdepth}{1}} % This hides the appendix subsections in the table of contents
\begin{appendices}
%%%%%%%%%%%%%%%%%%%%%%%%%%%%%%%%%%%%%%%%%%%%%%%%%%%%%%%%%%%%%%%%%%%%%%%%%%%%%
%%%%%%%%%%%%%%%%%%%%%%%%%%%%%%%%%%%%%%%%%%%%%%%%%%%%%%%%%%%%%%%%%%%%%%%%%%%%%
\section{Full Models}
\label{sec-app:full}

This appendix present the full models employed throughout the main text and the subsequent appendices. Appendix \ref{sec-app:full-derivations} derives the equations characterising optimal behaviour of households and firms. Thereafter, appendix \ref{sec-app:full-equations} displays the full system of equations passed to the \texttt{Econpizza} package (see appendix \ref{sec-app:codes} for details). Finally, appendix \ref{sec-app:full-graphical} presents a graphical representation of the baseline model.

%%%%%%%%%%%%%%%%%%%%%%%%%%%%%%%%%%%%%%%%%%%%%%%%%%%%%%%%%%%%%%%%%%%%%%%%%%%%%
\subsection{Derivations for Households and Firms}
\label{sec-app:full-derivations}

This subsection derives the equations that characterise optimal behaviour by households and firms. It begins with the household side. Next, the corporate sector's problem is dealt with.

\subsubsection{Households}
\label{sec-app:full-derivations-households}

The household setup is explained in section \ref{sec:model-hh} of the main text. Here, I present the derivations which yield the final equations necessary to determine optimal household behaviour. For the baseline model, which features \Gls{ghh} preferences, the final equations are \eqref{eq:foc-euler} -- \eqref{eq:foc-labour-supply}. For the model with \Gls{crra}-type preferences, the final equations are ...

The maximisation problem of the households can be cast into the following Bellman equation:
\begin{equation}
    V_t (e_{i,t}, b_{i,t-1}) = \max_{c_{i,t}, n_{i,t}, b_{i,t}} \Bigg\{ u(x_{i,t}) + \beta \mathbb{E}_{e,t} \{ V_{t+1} (e_{i,t+1}, b_{i,t}) \} \Bigg\} \label{eq:app-hh-bellman}
\end{equation}
subject to \eqref{eq:hh-budget} and \eqref{eq:borrowing-limit}. As appendix \ref{sec-app:full-derivations-households} shows, using the specifications for the composite good in equation \eqref{eq:hh-composite-good} and the utility function in equation \eqref{eq:hh-utility-fun}, this formulation leads to two first-order conditions (\Gls{foc}) and an envelope condition, which together characterise optimal household behaviour for all $i$. Note first the conditions with respect to the consumption-savings decision:

Households have \Gls{ghh} preferences over the composite good, $x_{i,t}$. \Gls{ghh} preferences simplify the household decision problem as it makes household $i$'s labour supply decision exogenous with respect to $i$'s consumption. As \textcite{auclert2017} note, this is because, with \Gls{ghh} preferences, the marginal utility of consumption and the marginal disutility of working move proportionally, thus leaving $n_{i,t}$ unchanged when $c_{i,t}$ changes. In the end, this implies that all households work the same hours, despite consuming different amounts.

\subsubsection{Firms}
\label{sec-app:full-derivations-firms}
A representative and perfectly competitive firm produces the final good, $y_t^f$, by combining different intermediate goods, $Y_{j,t}$. The technology it uses for that process features constant returns to scale and is given by:
\begin{equation}
    y_t^f = \Bigg( \int_0^1 Y_{j,t}^{\frac{\theta - 1}{\theta}} dj \Bigg)^{\frac{\theta}{\theta-1}},
\end{equation}
where $\theta$ is the elasticity of substitution for the intermediate goods in the production process. Profit maximisation and a zero-profits condition lead to input demand:
\begin{equation}
    Y_{j,t} = \Bigg( \frac{P_{j,t}}{P_t} \Bigg)^{-\theta} y_t^f,
\end{equation}
and a price aggregator of the form:
\begin{equation}
    P_t = \Bigg( \int_0^1 P_{j,t}^{1-\theta} dj \Bigg)^{\frac{1}{1-\theta}}
\end{equation}

A continuum of intermediate-goods firms, indexed by $j$, sell their output, produced with the production function:
\begin{equation}
    y_{j,t} = z N_{j,t}
\end{equation}
to the representative final-good firm in a monopolistically competitive market by setting the nominal price $P_{j,t}$. In adjusting their prices, the intermediate-goods firms are subject to price adjustment costs à la \textcite{rotemberg1982}.

%%%%%%%%%%%%%%%%%%%%%%%%%%%%%%%%%%%%%%%%%%%%%%%%%%%%%%%%%%%%%%%%%%%%%%%%%%%%%
\subsection{Full System of Equations}
\label{sec-app:full-equations}

This subsection lists all equations from the baseline model.

%%%%%%%%%%%%%%%%%%%%%%%%%%%%%%%%%%%%%%%%%%%%%%%%%%%%%%%%%%%%%%%%%%%%%%%%%%%%%
\subsection{Graphical Summary of the Model}
\label{sec-app:full-graphical}

This subsection presents a visualisation of the baseline model presented in section \ref{sec:model}. The figure is meant to be an aid in understanding the flows and mechanisms of the model and clearly does not capture essential aspects of the model, such as its dynamic nature.

\begin{figure}[H]
\centering
\caption{Graphical Representation of the Baseline Model}
\label{fig:graphical-model}
\begin{tikzpicture}
  % Big rectangle
  \draw[fill=white] (0,) rectangle (15,8);
  \node at (7.5,7.5) {\Large Households};

  \node at (3,6) {"unlucky"};
  \node at (12,6) {"lucky"};

  \node at (2.25,4) {Borrowers};
  \node at (2.25,3.6) {\scriptsize negative \Gls{iou}s};
  \node at (8,4) {Lenders};
  \node at (8,3.6) {\scriptsize positive \Gls{iou}s};
  \node at (12,4) {Bond Holders};

  \node at (13.85,-2) {Bonds $B$};
  \node at (14.1,-3) {Interest $RR$};
  
  \node at (7.5,9) {Productivity Risk $\varepsilon^e$};

  \node at (2,9) {Financial Constraints $\phi$, $\kappa$};
  
  \node at (2.5,-4.25) {Firms};

  \node at (4.3,-4.5) {$\pi$};
  \node at (6.1,-4.5) {$R^n$};
  
  \node at (8.25,-4.25) {Monetary Policy};

  \node at (13,-4.25) {Fiscal Policy};
  
  % Arrows
  \draw[->] (7.5,8.75) -- (7.5,8.25); % to households

  \draw[->] (7.5,7.25) -- (3,6.25); % to unlucky
  \draw[->] (7.5,7.25) -- (12,6.25); % to lucky

  \draw[->] (3,5.75) -- (3,4.25); % to borrowers
  \draw[->] (12,5.75) -- (8,4.25); % to lenders
  \draw[->] (12,5.75) -- (12,4.25); % to bond holders
  
  \draw[->, red] (0.5,8.75) -- (2.25,4.25); % to borrowers

  \draw[->] (7,4.1) -- (3.75,4.1);
  \draw[->] (3.75,3.9) -- (7,3.9); 

  \draw[->] (13,-4) -- (13,3.75); 

  \draw[-] (3.2,-4.25) -- (6.5,-4.25); % btw MP an firms
  \draw[->] (5.5,-4.25) -- (5.5,2.5); % to interest rate btw borr and lend
  %\node at (4.65,2.45) {Interest $RR$};
  
  \draw[->] (5.5,-3) -- (13,-3); 

  \draw[->] (2.65,1) -- (2.65,-4); % to firms
  %\node at (2.7,-2) {Labour $N$};
  \draw[->] (2.4,-4) -- (2.4,1); % from firms
  \node at (1.4,-1) {Income $wN$};
  \node at (3.6,-1) {Labour $N$};
  %\node at (1.65,-1.5) {$wN$};

  \node at (1.2,-2) {Consumption};
  \node at (1.4,-2.5) {Goods $C$};
  
  \node at (1.4,0) {Profits $\Pi$};

  \draw[->] (12.2,1) -- (12.2,-4); % to FP
  \node at (11.4,-2) {Taxes $\tau$};

  \node at (5.5,4.5) {\footnotesize "Consumption"};
  \node at (5.5,3.5) {\footnotesize "Insurance"};
  \node at (5.5,3) {\footnotesize + Interest $RR$};

\end{tikzpicture}
\begin{flushleft}
\footnotesize
\textit{Note}: This figure displays the agents and some key mechanisms of the baseline model from section \ref{sec:model}. To ease notation, time subscripts are suppressed.
\end{flushleft}
\end{figure}

%%%%%%%%%%%%%%%%%%%%%%%%%%%%%%%%%%%%%%%%%%%%%%%%%%%%%%%%%%%%%%%%%%%%%%%%%%%%%
\subsection{The Natural Borrowing Limit}
\label{sec:app-hh-borrowing}
The natural borrowing limit is the borrowing limit which just marginally ensures that borrowers can repay and not have negative consumption. This natural borrowing limit can be calculated from forward iterating the household budget constraint \eqref{eq:hh-budget}:
\begin{equation*}
    \frac{wne_{L} + \Pi_t \bar{\Pi} (e_{L}) - \tau_t \bar{\tau} (e_{L})}{RR - 1},
\end{equation*}
where $e_L$ denotes the lowest level of skills attainable from \eqref{eq:hh-income-process}.

%%%%%%%%%%%%%%%%%%%%%%%%%%%%%%%%%%%%%%%%%%%%%%%%%%%%%%%%%%%%%%%%%%%%%%%%%%%%%
%%%%%%%%%%%%%%%%%%%%%%%%%%%%%%%%%%%%%%%%%%%%%%%%%%%%%%%%%%%%%%%%%%%%%%%%%%%%%
%%%%%%%%%%%%%%%%%%%%%%%%%%%%%%%%%%%%%%%%%%%%%%%%%%%%%%%%%%%%%%%%%%%%%%%%%%%%%
%\newpage
\thispagestyle{plain}
\section{Data for Model Calibration}
\label{sec-app:data}

This appendix describes the data series and sources used for the calibration of the baseline model in section \ref{sec:model-solution} and for the calibration of the model in section \ref{sec-app:robust-bonds}.

First, to match $B$ to total liquid assets held by households, I follow \textcite{mckay2016} and \textcite{gl2017} in using information on aggregate household balance sheets, available in the Financial Accounts of the US Z1, known as "Flow of Funds" (\Gls{fof}) and provided by the Federal Reserve.

Second, I acquire data on US nominal \Gls{gdp}, the US price level from the FRED data base, which is maintained by the Federal Reserve of Saint Louis.

\begin{figure}[H]
    \caption{Time Series of Data for the Baseline Model Calibration}
    \centering
    \begin{subfigure}[b]{0.49\textwidth}
    \caption{Liquid Assets Held by Households}
    \label{fig:app-calibration-liquid}
         \centering
         \includesvg[inkscapelatex=false, width=\textwidth]{Results/FRED_l.svg}
     \end{subfigure}
     \hfill
     \begin{subfigure}[b]{0.49\textwidth}
     \caption{Consumer Credit over \Gls{gdp}} 
    \label{fig:app-calibration-debt}
         \centering
         \includesvg[inkscapelatex=false, width=\textwidth]{Results/FRED_d.svg}
     \end{subfigure}
     \begin{flushleft}
     \footnotesize
	\textit{Note}: Data taken from the \href{https://www.federalreserve.gov/releases/z1/}{\Gls{fof} statistics} and the \href{https://fred.stlouisfed.org}{FRED data base}. Sample range: 1951:Q4 -- 2023:Q1. The dotted line is the full sample mean of the respective series. The series in panel \ref{fig:app-calibration-liquid} is in terms of the units from the \textit{baseline} calibration of the model in section \ref{sec:model}, in which $y_{ss} = 0.9129$. 
	\end{flushleft}
\end{figure}

\begin{figure}[H]
    \centering
    \caption{Time Series of Government Bonds}
    \label{fig:app-calibration-bonds}
    \includesvg[inkscapelatex=false, width=0.75\textwidth]{Results/FRED_b.svg}
\end{figure}
\begin{flushleft}
     \footnotesize
	\textit{Note}: Data taken from \href{https://fred.stlouisfed.org}{FRED data base}. Sample range: 1951:Q4 -- 2023:Q1. The dotted line is the full sample mean of the series. The series is in terms of the units from the \textit{baseline} calibration of the model in section \ref{sec:model}, in which $y_{ss} = 0.9129$. 
\end{flushleft}


%%%%%%%%%%%%%%%%%%%%%%%%%%%%%%%%%%%%%%%%%%%%%%%%%%%%%%%%%%%%%%%%%%%%%%%%%%%%%
%%%%%%%%%%%%%%%%%%%%%%%%%%%%%%%%%%%%%%%%%%%%%%%%%%%%%%%%%%%%%%%%%%%%%%%%%%%%%
%%%%%%%%%%%%%%%%%%%%%%%%%%%%%%%%%%%%%%%%%%%%%%%%%%%%%%%%%%%%%%%%%%%%%%%%%%%%%
%\newpage
\thispagestyle{plain}
\section{Computational Implementation}
\label{sec-app:codes}

This appendix has two parts. The first one notes some caveats when implementing the financial shocks of sections \ref{sec:limit} -- \ref{sec:sensitivity} in \texttt{Econpizza}. The second part provides a rough guide for the computer codes producing the results for this thesis. The full set of codes is provided together with the thesis and is available on my GitHub page: \url{https://github.com/andkound98/master-thesis}, last retrieved \printdate{2023-08-25}.

\subsection{Implementation of Shocks in \texttt{Econpizza}}
\label{sec-app:codes-ep}

The choices on the grid sizes naturally involve an accuracy-speed trade-off. I approximate the productivity process on a grid with $8$ grid points. The asset grid has $219$ grid points. I choose the number of grid points roughly such that increasing the number does not change the main results in a meaningful way.

Regarding the implementation of the shocks in the subsequent sections, it is important to note two things. First, in case of permanent shocks, the transition path must be calculated using the Jacobian of the terminal steady state instead of the initial steady state (see \cite{auclert2021}). Second, in the case of a shock to the borrowing limit, a correct implementation requires that the asset grid is not changed over time so as to avoid invalid comparisons based on different interpolations (see this \cite{mellior2016mimeo}). The code considers both of these aspects.

The horizon is set to $200$, i.e.~the solution method assumes that after $200$ quarters the system has returned its steady state.

%Second, in the case of a shock to the borrowing limit, a correct implementation requires that the asset grid is not changed over time, as this mimeo notes. Otherwise, the interpolation would be between repeatedly different grid points, making the comparison of over time inaccurate. 

This is because, from $t=1$, the new regime applies.

\subsection{Guide to Computer Codes}
\label{sec-app:codes-guide}

I now shortly describe the Python code files employed. All these codes were run using the Spyder IDE $5.3.4$ with Python $3.9.12$ and \texttt{Econpizza} $0.6.1$ on macOS $12.6.5$.

\subsection*{\texttt{main.py}}

The file \texttt{main.py} is the main file of the project.

\subsection*{\texttt{custom\_functions.py}}

\subsection*{\texttt{plot\_functions.py}}

\subsection*{\texttt{calibration.py}}

This file uses the data described in appendix \ref{sec-app:data} to calculate the desired calibration targets and to create the figures \ref{fig:app-calibration-liquid}, \ref{fig:app-calibration-debt} and \ref{fig:app-calibration-bonds}.

%%%%%%%%%%%%%%%%%%%%%%%%%%%%%%%%%%%%%%%%%%%%%%%%%%%%%%%%%%%%%%%%%%%%%%%%%%%%%
%%%%%%%%%%%%%%%%%%%%%%%%%%%%%%%%%%%%%%%%%%%%%%%%%%%%%%%%%%%%%%%%%%%%%%%%%%%%%
%%%%%%%%%%%%%%%%%%%%%%%%%%%%%%%%%%%%%%%%%%%%%%%%%%%%%%%%%%%%%%%%%%%%%%%%%%%%%
%\newpage
\thispagestyle{plain}
\section{Supplementary Figures}
\label{sec-app:figures}

This appendix contains supplementary figures, placed here for the sake of brevity.
%that are referenced throughout the text but which are placed here for the sake of brevity.

\subsection{Figures for Section \ref{sec:limit-stst-loose}}
\label{sec-app:figures-limit-loose}

\begin{figure}[H]
    \centering
    \caption{\Gls{mpc}s Across (Low) Asset Holdings}
    \label{fig:baseline-limit-initial-mpcs}
    \includesvg[inkscapelatex=false, width=0.75\textwidth]{Results/stst_policies_mpc_baseline_limit_permanent_initial.svg}
\end{figure}

\subsection{Figures for Section \ref{sec:limit-transition}}
\label{sec-app:figures-limit-transition}

\begin{figure}[H]
    %\ContinuedFloat
    \caption{Supplement for Figure \ref{fig:baseline-permanent-limit-agg}}%{\hypersetup{hidelinks}\ref{fig:baseline-permanent-limit-agg}}}
    \centering
    \begin{subfigure}[b]{0.49\textwidth}
     \caption{Household Debt, $D$}
         \centering
         \includesvg[inkscapelatex=false, width=\textwidth]{Results/transition_D_baseline_limit_permanent.svg}
     \end{subfigure}
     \hfill
    \begin{subfigure}[b]{0.49\textwidth}
     \caption{Labour Hours, $N$}
         \centering
         \includesvg[inkscapelatex=false, width=\textwidth]{Results/transition_N_baseline_limit_permanent.svg}
     \end{subfigure}
     \hfill
     \begin{subfigure}[b]{0.49\textwidth}
     \caption{Wage, $w$}
         \centering
         \includesvg[inkscapelatex=false, width=\textwidth]{Results/transition_w_baseline_limit_permanent.svg}
     \end{subfigure}
     \hfill
    \begin{subfigure}[b]{0.49\textwidth}
    \caption{Tax Rate, $\tau$}
         \centering
         \includesvg[inkscapelatex=false, width=\textwidth]{Results/transition_tau_baseline_limit_permanent.svg}
     \end{subfigure}
     \hfill
     \begin{subfigure}[b]{0.49\textwidth}
     \caption{Dividends, $\Pi$}
         \centering
         \includesvg[inkscapelatex=false, width=\textwidth]{Results/transition_div_baseline_limit_permanent.svg}
     \end{subfigure}
     \begin{flushleft}
     \footnotesize
	\textit{Note}: Percent deviations mean percent deviations from the respective initial steady state. The time horizon in all plots is fixed to 12 quarters.
	\end{flushleft}
\end{figure}

\subsection{Figures for Section \ref{sec:wedge}}
\label{sec-app:figures-wedge}





%%%%%%%%%%%%%%%%%%%%%%%%%%%%%%%%%%%%%%%%%%%%%%%%%%%%%%%%%%%%%%%%%%%%%%%%%%%%%
%%%%%%%%%%%%%%%%%%%%%%%%%%%%%%%%%%%%%%%%%%%%%%%%%%%%%%%%%%%%%%%%%%%%%%%%%%%%%
%%%%%%%%%%%%%%%%%%%%%%%%%%%%%%%%%%%%%%%%%%%%%%%%%%%%%%%%%%%%%%%%%%%%%%%%%%%%%
%\newpage
\thispagestyle{plain}
\section{Robustness and Supplementary Experiments}
\label{sec-app:robust}

This appendix presents various robustness exercises and supplementary experiments alluded to in the main text.

%%%%%%%%%%%%%%%%%%%%%%%%%%%%%%%%%%%%%%%%%%%%%%%%%%%%%%%%%%%%%%%%%%%%%%%%%%%%%
%%%%%%%%%%%%%%%%%%%%%%%%%%%%%%%%%%%%%%%%%%%%%%%%%%%%%%%%%%%%%%%%%%%%%%%%%%%%%
\subsection{Calibration with Low Discount Factor}
\label{sec-app:robust-beta-calibration}

This appendix conducts the same exercise as section \ref{sec:limit} does, but changes the calibration of the model, in particular the value of $\beta$, such that the \Gls{zlb} on the nominal interest rate does not bind during the transition to tighter credit.

As \textcite{gl2017} point out, the interest rate elasticity might still be quite high, so that the \Gls{zlb} binding or not matters a great deal.

%%%%%%%%%%%%%%%%%%%%%%%%%%%%%%%%%%%%%%%%%%%%%%%%%%%%%%%%%%%%%%%%%%%%%%%%%%%%%
%%%%%%%%%%%%%%%%%%%%%%%%%%%%%%%%%%%%%%%%%%%%%%%%%%%%%%%%%%%%%%%%%%%%%%%%%%%%%
\subsection{Calibration of Bonds à la \textcite{bayer2023}}
\label{sec-app:robust-bonds}

This appendix re-calibrates $B$ in the vein of \textcite{bayer2023} by targeting the ratio of the market value of federal government debt to GDP instead of the fraction of total liquid assets over GDP.

To that end, I use data from FRED, as \textcite{bayer2023} do.

I re-set $\beta$ so that the steady state (real) interest rate is $0.15\%$, as before.

Now, \Gls{iou}s make up a larger share of gross liquidity. 

Clearly, the steady-state distributions are different. \Gls{mpc}s are much higher.

%%%%%%%%%%%%%%%%%%%%%%%%%%%%%%%%%%%%%%%%%%%%%%%%%%%%%%%%%%%%%%%%%%%%%%%%%%%%%
%%%%%%%%%%%%%%%%%%%%%%%%%%%%%%%%%%%%%%%%%%%%%%%%%%%%%%%%%%%%%%%%%%%%%%%%%%%%%
%\subsection{Distortionary Labour Income Tax}
%\label{sec-app:robust-labour-tax}

%%%%%%%%%%%%%%%%%%%%%%%%%%%%%%%%%%%%%%%%%%%%%%%%%%%%%%%%%%%%%%%%%%%%%%%%%%%%%
%%%%%%%%%%%%%%%%%%%%%%%%%%%%%%%%%%%%%%%%%%%%%%%%%%%%%%%%%%%%%%%%%%%%%%%%%%%%%
%\subsection{A Permanent Discount Factor Shock}
%\label{sec-app:robust-beta-shock}

%%%%%%%%%%%%%%%%%%%%%%%%%%%%%%%%%%%%%%%%%%%%%%%%%%%%%%%%%%%%%%%%%%%%%%%%%%%%%
%%%%%%%%%%%%%%%%%%%%%%%%%%%%%%%%%%%%%%%%%%%%%%%%%%%%%%%%%%%%%%%%%%%%%%%%%%%%%
\end{appendices}
\newpage
\thispagestyle{plain}
\pagenumbering{Roman}  % Turn page numbering to arabic
\renewcommand*{\thepage}{A-\Roman{page}} % Add 'A' to each page number for appendices section
\thispagestyle{plain}
\printbibliography[heading=subbibliography, title={Appendix References}]
\addcontentsline{toc}{section}{Appendix References}
\end{refsection}

%%%%%%%%%%%%%%%%%%%%%%%%%%%%%%%%%%%%%%%%%%%%%%%%%%%%%%%%%%%%%%%%%%%%%%%%%%%%%
%%%%%%%%%%%%%%%%%%%%%%%%%%%%%%%%%%%%%%%%%%%%%%%%%%%%%%%%%%%%%%%%%%%%%%%%%%%%%
%%%%%%%%%%%%%%%%%%%%%%%%%%%%%%%%%%%%%%%%%%%%%%%%%%%%%%%%%%%%%%%%%%%%%%%%%%%%%
\newpage
\thispagestyle{plain}
\pagenumbering{gobble} % Turn page numbering off
\section*{Statement of Authorship} % Include statement of authorship
I hereby confirm that the work presented has been performed and interpreted solely by myself except for where I explicitly identified the contrary. I assure that this work has not been presented in any other form for the fulfillment of any other degree or qualification. Ideas taken from other works in letter and in spirit are identified in every single case.

\vspace{2cm}
\noindent
\rule{8cm}{0.4pt}\\
Andreas Koundouros\\
Bonn, the \printdate{2023-8-25}
\end{document}
% End Document
%%%%%%%%%%%%%%%%%%%%%%%%%%%%%%%%%%%%%%%%%%%%%%%%%%%%%%%%%%%%%%%%%%%%%%%%%%%%%
%%%%%%%%%%%%%%%%%%%%%%%%%%%%%%%%%%%%%%%%%%%%%%%%%%%%%%%%%%%%%%%%%%%%%%%%%%%%%
%%%%%%%%%%%%%%%%%%%%%%%%%%%%%%%%%%%%%%%%%%%%%%%%%%%%%%%%%%%%%%%%%%%%%%%%%%%%%
%%%%%%%%%%%%%%%%%%%%%%%%%%%%%%%%%%%%%%%%%%%%%%%%%%%%%%%%%%%%%%%%%%%%%%%%%%%%%
%%%%%%%%%%%%%%%%%%%%%%%%%%%%%%%%%%%%%%%%%%%%%%%%%%%%%%%%%%%%%%%%%%%%%%%%%%%%%
%%%%%%%%%%%%%%%%%%%%%%%%%%%%%%%%%%%%%%%%%%%%%%%%%%%%%%%%%%%%%%%%%%%%%%%%%%%%%
