\documentclass[a4paper,12pt]{article} % Set documentclass article and set font size

%%%%%%%%%%%%%%%%%%%%%%%%%%%%%%%%%%%%%%%%%%%%%%%%%%%%%%%%%%%%%%%%%%%%%%%%%%%%%
%%%%%%%%%%%%%%%%%%%%%%%%%%%%%%%%%%%%%%%%%%%%%%%%%%%%%%%%%%%%%%%%%%%%%%%%%%%%%
%%%%%%%%%%%%%%%%%%%%%%%%%%%%%%%%%%%%%%%%%%%%%%%%%%%%%%%%%%%%%%%%%%%%%%%%%%%%%
% Packages
\usepackage{fontspec} % If used, change compiler to XeLaTex
\usepackage[utf8]{inputenc}
\usepackage[british]{babel}
\usepackage[british]{isodate}
\usepackage{csquotes}
\usepackage{mathtools}
\usepackage{amssymb}
\usepackage{unicode-math}
\usepackage{geometry}
\usepackage[
backend=biber,
style=apa,
sorting=nyt, % Sort by name, year, title
backref=true,
dateabbrev=false,
language=british,
urldate=long
]{biblatex}
\usepackage{xcolor}
\usepackage[toc,page]{appendix}
\usepackage{svg}
\usepackage{float}
\usepackage{longtable}
\usepackage{multirow}
\usepackage{multirow,multicol, makecell, booktabs}
\usepackage{caption}
\usepackage{subcaption}
\usepackage{setspace}
\usepackage{ragged2e}
\usepackage{fancyhdr}
\usepackage{enumitem}
\usepackage{tikz}
\usepackage{lscape}
\usepackage{rotating}
\usepackage{hyperref} % Load after everything else
\usepackage[symbols, nopostdot, automake, nonumberlist]{glossaries}

%%%%%%%%%%%%%%%%%%%%%%%%%%%%%%%%%%%%%%%%%%%%%%%%%%%%%%%%%%%%%%%%%%%%%%%%%%%%%
%%%%%%%%%%%%%%%%%%%%%%%%%%%%%%%%%%%%%%%%%%%%%%%%%%%%%%%%%%%%%%%%%%%%%%%%%%%%%
%%%%%%%%%%%%%%%%%%%%%%%%%%%%%%%%%%%%%%%%%%%%%%%%%%%%%%%%%%%%%%%%%%%%%%%%%%%%%
% Custom Settings
% \DeclareFieldFormat{doi}{% % put doi into new line in references
%   \newline
%   \mkbibacro{DOI}\addcolon\space
%     \ifhyperref
%       {\href{http://dx.doi.org/#1}{\nolinkurl{#1}}}
%       {\nolinkurl{#1}}}

\addbibresource{master_thesis.bib} % Add bibliography source

\setlength\bibitemsep{1.5\itemsep} % Adjust distance of reference entries

\renewcommand*\appendixpagename{\Large Appendices} % Adjust size of title of the appendices section

\renewcommand*{\finentrypunct}{}
\renewbibmacro*{pageref}{%
  \addperiod% NEW
  \iflistundef{pageref}
    {}
    {\newline\footnotesize\printtext[parens]{%
       \ifnumgreater{\value{pageref}}{1}
         {\bibstring{backrefpages}\ppspace}
     {\bibstring{backrefpage}\ppspace}%
       \printlist[pageref][-\value{listtotal}]{pageref}}}}%
\DefineBibliographyStrings{english}{
  backrefpage={Cited on page},
  backrefpages={Cited on pages}
}

\setmainfont{Times New Roman} % If the usepackage fontspec is active, you can set Times New Roman here

\geometry{ % Adjust paper geometry
 a4paper,
 right=25mm,
 bottom=20mm,
 left=25mm,
 top=25mm}
\urlstyle{same} % Make URL font the same with the rest of the document
\onehalfspacing % Adjust spacing
%\setstretch{1.75}
%\doublespacing
\allowdisplaybreaks % Allow for breaks in equations across pages
\addto\captionsbritish{ % Replace "english" with the language you use
  \renewcommand{\contentsname}%
    {Table of Contents}%
} % Adjust title of the table of contents
\numberwithin{equation}{section} % Number equations with sections
\numberwithin{figure}{section}
\numberwithin{table}{section}
%\setcounter{tocdepth}{2} % Set maximum level to which sections are displayed in the table of contents

\newcommand{\appendixsubsection}[1]{
    \stepcounter{subsection}
    \subsection*{\Alph{section}.\arabic{subsection}\hspace{1em}{#1}}
}


\pagestyle{plain}

% Define a custom name format that prints names of authors in references in bold
\DeclareNameWrapperFormat{sortname}{\mkbibbold{#1}}
\DeclareNameWrapperAlias{author}{sortname}

% Add definition format
\newtheorem{definition}{Definition}

% Make links of glossary entries black
\makeatletter
\newcommand*{\glsplainhyperlink}[2]{%
    \begingroup%
      \hypersetup{hidelinks}%
      \hyperlink{#1}{#2}%
    \endgroup%
}
\let\@glslink\glsplainhyperlink
\makeatother

%%%%%%%%%%%%%%%%%%%%%%%%%%%%%%%%%%%%%%%%%%%%%%%%%%%%%%%%%%%%%%%%%%%%%%%%%%%%%
%%%%%%%%%%%%%%%%%%%%%%%%%%%%%%%%%%%%%%%%%%%%%%%%%%%%%%%%%%%%%%%%%%%%%%%%%%%%%
%%%%%%%%%%%%%%%%%%%%%%%%%%%%%%%%%%%%%%%%%%%%%%%%%%%%%%%%%%%%%%%%%%%%%%%%%%%%%
% Glossary
\makeglossaries 

% Glossary entries
\newglossaryentry{nk}{
name=NK,
description={New Keynesian}
}

\newglossaryentry{ha}{
name=HA,
description={Heterogeneous Agents}
}

\newglossaryentry{hank}{
name=HANK,
description={Heterogeneous Agent New Keynesian}
}

\newglossaryentry{gfc}{
name=GFC,
description={Great Financial Crisis}
}

\newglossaryentry{ltv}{
name=LTV,
description={Loan-to-Value}
}

\newglossaryentry{dsge}{
name=DSGE,
description={Dynamic Stochastic General Equilibrium}
}

\newglossaryentry{egm}{
name=EGM,
description={Endogenous Gridpoints Method}
}

\newglossaryentry{iou}{
name=IOU,
description={I Owe You}
}

\newglossaryentry{ghh}{
name=GHH,
description={Greenwood-Hercowitz-Huffman (\cite{ghh1988})}
}

\newglossaryentry{ar}{
name=AR,
description={Autoregressive}
}

\newglossaryentry{zlb}{
name=ZLB,
description={Zero Lower Bound}
}

\newglossaryentry{mpc}{
name=MPC,
description={Marginal Propensity to Consume}
}

\newglossaryentry{fof}{
name=FoF,
description={Flow of Funds}
}

\newglossaryentry{nkpc}{
name=NKPC,
description={New Keynesian Phillips Curve}
}

\newglossaryentry{foc}{
name=FOC,
description={First Order Condition}
}

\newglossaryentry{irf}{
name=IRF,
description={Impulse Response Function}
}

\newglossaryentry{crra}{
name=CRRA,
description={Constant Relative Risk Aversion}
}

\newglossaryentry{gdp}{
name=GDP,
description={Gross Domestic Product}
}

%%%%%%%%%%%%%%%%%%%%%%%%%%%%%%%%%%%%%%%%%%%%%%%%%%%%%%%%%%%%%%%%%%%%%%%%%%%%%
%%%%%%%%%%%%%%%%%%%%%%%%%%%%%%%%%%%%%%%%%%%%%%%%%%%%%%%%%%%%%%%%%%%%%%%%%%%%%
%%%%%%%%%%%%%%%%%%%%%%%%%%%%%%%%%%%%%%%%%%%%%%%%%%%%%%%%%%%%%%%%%%%%%%%%%%%%%
% hyperref package
\usepackage{footnotebackref} % Brings you back to where the footnote is in the text
\hypersetup{ 
    colorlinks,
    linkcolor=blue,
    filecolor=blue,  
    citecolor=blue,
    urlcolor=blue} % Setup with the hyperref package

%%%%%%%%%%%%%%%%%%%%%%%%%%%%%%%%%%%%%%%%%%%%%%%%%%%%%%%%%%%%%%%%%%%%%%%%%%%%%
%%%%%%%%%%%%%%%%%%%%%%%%%%%%%%%%%%%%%%%%%%%%%%%%%%%%%%%%%%%%%%%%%%%%%%%%%%%%%
%%%%%%%%%%%%%%%%%%%%%%%%%%%%%%%%%%%%%%%%%%%%%%%%%%%%%%%%%%%%%%%%%%%%%%%%%%%%%
% Formalities
\title{Financial Constraints and Household Heterogeneity\\in the Macroeconomy}
\author{}
\date{}

%%%%%%%%%%%%%%%%%%%%%%%%%%%%%%%%%%%%%%%%%%%%%%%%%%%%%%%%%%%%%%%%%%%%%%%%%%%%%
%%%%%%%%%%%%%%%%%%%%%%%%%%%%%%%%%%%%%%%%%%%%%%%%%%%%%%%%%%%%%%%%%%%%%%%%%%%%%
%%%%%%%%%%%%%%%%%%%%%%%%%%%%%%%%%%%%%%%%%%%%%%%%%%%%%%%%%%%%%%%%%%%%%%%%%%%%%
%%%%%%%%%%%%%%%%%%%%%%%%%%%%%%%%%%%%%%%%%%%%%%%%%%%%%%%%%%%%%%%%%%%%%%%%%%%%%
%%%%%%%%%%%%%%%%%%%%%%%%%%%%%%%%%%%%%%%%%%%%%%%%%%%%%%%%%%%%%%%%%%%%%%%%%%%%%
%%%%%%%%%%%%%%%%%%%%%%%%%%%%%%%%%%%%%%%%%%%%%%%%%%%%%%%%%%%%%%%%%%%%%%%%%%%%%
% Begin Document
\begin{document}
\selectlanguage{british}
\pagenumbering{gobble} % Turn page numbering off
\maketitle % Create title

\vspace{4cm}
\begin{center}
Master Thesis Presented to the\\
Department of Economics at the\\
Rheinische Friedrich-Wilhelms-Universität Bonn\\
\vspace{1cm}
In Partial Fulfillment of the Requirements for the Degree of\\
Master of Science (M.Sc.)
\end{center}

\vspace{5cm}
\begin{center}
\begin{tabular}{ l l }
Supervisors: & Prof.~Dr.~Keith Kuester \\
& Dr.~Gregor Böhl \\
& \\
& \\
Date of Submission: & \printdate{2023-8-25}\\
Author: & Andreas Koundouros\\
Matriculation Number: & 3466868
\end{tabular}
\end{center}

%%%%%%%%%%%%%%%%%%%%%%%%%%%%%%%%%%%%%%%%%%%%%%%%%%%%%%%%%%%%%%%%%%%%%%%%%%%%%
%%%%%%%%%%%%%%%%%%%%%%%%%%%%%%%%%%%%%%%%%%%%%%%%%%%%%%%%%%%%%%%%%%%%%%%%%%%%%
%%%%%%%%%%%%%%%%%%%%%%%%%%%%%%%%%%%%%%%%%%%%%%%%%%%%%%%%%%%%%%%%%%%%%%%%%%%%%
\newpage
\pagenumbering{roman} % Turn page numbering to small roman
\setcounter{tocdepth}{2}
{ \hypersetup{hidelinks} \tableofcontents } % Hide links in table of contents

\newpage
{ \hypersetup{hidelinks} \listoftables } % Hide links in list of tables
{ \hypersetup{hidelinks} \listoffigures } % Hide links in list of figures

% Add here list of abbreviations and list of symbols 
\newpage
\setlist[description]{leftmargin=!, labelwidth=5em} % Change for glossaries
\printglossary[title=List of Acronyms]
\setlist[description]{style=standard} % Reset settings back to default


%%%%%%%%%%%%%%%%%%%%%%%%%%%%%%%%%%%%%%%%%%%%%%%%%%%%%%%%%%%%%%%%%%%%%%%%%%%%%
%%%%%%%%%%%%%%%%%%%%%%%%%%%%%%%%%%%%%%%%%%%%%%%%%%%%%%%%%%%%%%%%%%%%%%%%%%%%%
%%%%%%%%%%%%%%%%%%%%%%%%%%%%%%%%%%%%%%%%%%%%%%%%%%%%%%%%%%%%%%%%%%%%%%%%%%%%%
\newpage
\begin{abstract} % Create abstract

A financial shock to debtors' borrowing capacity affects not only debtors themselves but, through household balance sheets, also lenders and, in extension, the macroeconomy. This thesis studies two such "credit crunch"-scenarios: a reduction in the debt limit and an increase in the borrowing premium. In a New Keynesian economy with heterogeneous agents, both shocks force constrained debtors to delever and thus reduce liquidity in the economy. To restore equilibrium, the interest rate must fall. However, at the same time, unconstrained agents increase their demand for safe assets, prompting a short-run recession and a protracted transition of aggregate debt to lower levels. In the long run, a lower real rate lowers wealth inequality but consumption inequality is higher. In spite of similar aggregate responses, the two shocks produce quantitative differences. Finally, I document an interesting asymmetry: financial tightenings are more recessionary than financial easings are expansionary. 

\begin{center}
\rule{10cm}{0.4pt}
\end{center}

\noindent
\textit{JEL Classification:} E12, E21, E32 % JEL topic classification of paper

\noindent
\textit{Keywords:} Borrowers and Lenders, Household Debt, Deleveraging, \Gls{hank} % Keywords of paper
\end{abstract}

%%%%%%%%%%%%%%%%%%%%%%%%%%%%%%%%%%%%%%%%%%%%%%%%%%%%%%%%%%%%%%%%%%%%%%%%%%%%%
%%%%%%%%%%%%%%%%%%%%%%%%%%%%%%%%%%%%%%%%%%%%%%%%%%%%%%%%%%%%%%%%%%%%%%%%%%%%%
%%%%%%%%%%%%%%%%%%%%%%%%%%%%%%%%%%%%%%%%%%%%%%%%%%%%%%%%%%%%%%%%%%%%%%%%%%%%%
\newpage
\pagenumbering{arabic} % Turn page numbering to small arabic
\pagestyle{fancy}
\fancyhf{}
\fancyhead[C]{\leftmark}
\fancyfoot[C]{\thepage}

%%%%%%%%%%%%%%%%%%%%%%%%%%%%%%%%%%%%%%%%%%%%%%%%%%%%%%%%%%%%%%%%%%%%%%%%%%%%%
%%%%%%%%%%%%%%%%%%%%%%%%%%%%%%%%%%%%%%%%%%%%%%%%%%%%%%%%%%%%%%%%%%%%%%%%%%%%%
%%%%%%%%%%%%%%%%%%%%%%%%%%%%%%%%%%%%%%%%%%%%%%%%%%%%%%%%%%%%%%%%%%%%%%%%%%%%%
\thispagestyle{plain}
\section{Introduction}
\label{sec:introduction}

The vicissitudes of life prompt households to differ in their asset positions. Some households are net lenders, i.e.~their positive asset holdings exceed their liabilities, while others happen to be net borrowers, i.e.~they owe more in debt than they own in assets. Crucially, the debt holdings of borrowers are the lenders' asset holdings, reflecting the fact that debt provides creditors with the liquidity they demand. Thus, the balance sheets of the two groups of households are linked, which in turn suggests that their respective merits are closely intertwined.\footnote{Note two things. First, over time, households might switch from being a borrower to being a lender and vice versa. Second, entities other than households can intermediate funds from and to the household sector, e.g.~financial institutions. However, I here focus on the intermediation of credit between households. The latter can be thought of as frictionless intermediation through perfectly competitive banks.} 

Interconnected balance sheets imply that the consequences of a shock to the indebted households' borrowing capacity -- whatever its cause might be -- might not only be borne by the indebted households themselves but also by the lending households. In particular, this suggests that a tightening in the financial conditions under which borrowers obtain credit, a so-called "credit crunch", affects households with very different asset positions and thus unequal abilities to absorb shocks. This raises the question as to whether and how financial shocks to debtors' balance sheets impact not only on the consumption, lending and borrowing decisions of individuals, but also on the macroeconomy as a whole. 

Hence, this thesis asks: how can a credit crunch arise and what might be its short- and long-run consequences? In particular, how does a financial shock work through borrowers' and lenders' balance sheets, respectively, and how does this then impact, first, the two groups' consumption and savings decisions and, second, the aggregate state of the economy? In other words, how are consumption and wealth redistributed and how are macroeconomic variables disturbed when debtors' access to credit is suddenly curtailed?

To answer these questions, I follow the modelling approach of \textcite{gl2017} by employing a heterogeneous-agent (\Gls{ha}) model. Such a framework naturally lends itself to the analysis of household-level financial shocks because borrowing and lending asset positions emerge from it endogenously. The model is a small-scale one-asset \Gls{ha} New Keynesian (\Gls{hank}) model which allows for two disturbances to households' debt capacity: shocks to the idiosyncratic borrowing limit and shocks to the premium of the borrowing interest rate over the lending rate.

I present three key results. First, confronting the model with a shock to the borrowing limit confirms the presence of the overall mechanisms identified in the theoretical literature on credit tightenings. Second, I demonstrate how equally sized shocks to the credit limit and to the borrowing spread induce qualitatively similar responses but that some interesting quantitative differences emerge. Third, I find that financial shocks can operate asymmetrically in the present framework.

In more detail, the first result derives from reducing -- unexpectedly and permanently -- the household-level borrowing limit in a plausibly calibrated version of the model. This shock forces constrained borrowers to delever mechanically (a mechanism I call \textit{balance-sheet effect}) and elevates the fear of reaching the now more imminent constraint among the unconstrained households (a \textit{precautionary-savings effect}). Both of these forces prompt agents -- the asset-poor in particular -- to cut back on consumption. To alleviate the pressure from the shrunken supply of and the increased demand for liquidity, interest rates must fall. In the long-run, i.e.~after the debt limit settled in its terminal state, this implies lower wealth inequality but higher dispersion in consumption. In the short-run of the adjustment, the economy enters recession with consumption falling strongly and the zero-lower bound (\Gls{zlb}) on nominal rates potentially inhibiting the central bank to stimulate the recovery as needed. Interestingly, outstanding debt takes exceptionally long to reach its new steady state.

% Importantly however, in both cases, the recession in consumption is rather short-lived and typically does not last longer than the shock takes to resolve. %\footnote{Presumably, this is due to the fact that the model excludes productive capital and does not include various other frictions that are known to create persistence, such as habit formation in household consumption.} 

This type of a credit crunch has been studied in the theoretical literature before, most prominently by \textcite{egg2012} and \textcite{gl2017}. Both papers show how such a shock provokes depressed levels of consumption and interest rates close or at the \Gls{zlb}, which are features of macroeconomic aggregates that resemble those witnessed during the Great Recession of 2007--2009. Crucially, both studies emphasise the endogenous reaction of the interest rate in the economy's process of adjustment. In particular, \textcite{gl2017} highlight how rich household heterogeneity gives rise to the precautionary-savings channel of a credit crunch.

In two brief sensitivity analyses I investigate how assumptions about about the transition to tight credit and the choice of households' preferences can influence the results. As expected, the slower financial shocks evolve, the more mildly the economy reacts and allowing households to individually adjust their labour hours provides them with an important tool to absorb shocks to their balance sheets.

The second main result of this thesis derives from the fact that a permanent shock to the interest rate wedge triggers the same mechanisms as shocks to the borrowing limit do and thus, naturally, induces qualitatively similar responses. Yet, I highlight that the reaction of the economy might differ quantitatively depending on the origin of the shock, even if both disturbances are calibrated to yield the same amount of deleveraging.

Finally, my third result establishes that the present model is capable of producing at least some form of asymmetry in its responses to financial shocks. That is, I ask whether the boom triggered by a credit \textit{easing} from the state of tight credit back to that of loose credit is of the same absolute magnitude as the recession caused by the reverse transition. The answer is indeed a clear no: a credit easing is significantly less expansionary than a credit crunch is recessionary. \textcolor{red}{Intuitively, this is because a loosening in financial constraints leaves more flexibility -- to save or to borrow -- than a tightening does.}

Before proceeding to the analysis of the above issues, be aware of the following caveat: this thesis is \textit{not} about identifying the roots of the Great Recession. In fact, the present investigation is entirely agnostic about the origins of financial constraints and the causes of why they might change over time. Rather, the focus is on the \textit{effects} that exogenous variation in these constraints has on aggregates as well as on household-level consumption and savings outcomes. Of course, understanding the mechanisms of a credit crunch is not only an interesting theoretical exercise in and for itself but should, in extension, inform empirical and policy-orientated work. This is important -- independent of whether these effects resemble stylised facts of the Great Recession or not -- but beyond the scope of the present thesis.\footnote{Indeed, even if responses coincide with what was witnessed during the Great Recession, this can not be taken as evidence in support of the hypothesis that financial shocks caused the Great Recession. Whether financial shocks were the main cause of the Great Recession or merely an accelerator of events is still a matter of discussion in the literature. For example, \textcite{justiniano2015} show that, in their two-agent model, house prices are the more likely driver of the events of the Great Recession than changes in borrowing constraints.}

%%%%%%%%%%%%%%%%%%%%%%%%%%%%%%%%%%%%%%%%%%%%%%%%%%%%%%%%%%%%%%%%%%%%%%%%%%%%%
\subsubsection*{Outline}
\label{sec:outline}
The remainder of this thesis is structured as follows. Section \ref{sec:literature} briefly reviews the literature on financial constraints and heterogeneous households from a macroeconomic perspective. Then, section \ref{sec:model} lays down the \Gls{hank} model with a focus on borrowers and lenders, their interactions and characteristics as well as the included financial constraints and shocks. Section \ref{sec:model} also deals with the solution method of the model and its calibration. Section \ref{sec:limit} studies the long- and short-term effects of a permanent shock to the household-level borrowing constraint. The analysis first describes the steady states of relatively easy and tight credit, respectively, and then discusses the economy's transition from the former to the latter. Section \ref{sec:sensitivity} presents some related sensitivity analyses. Subsequently, section \ref{sec:wedge} studies the shock to the interest rate wedge and section \ref{sec:asymmetry} explores the asymmetry of financial shocks. Finally, section \ref{sec:conclusion} concludes.

%%%%%%%%%%%%%%%%%%%%%%%%%%%%%%%%%%%%%%%%%%%%%%%%%%%%%%%%%%%%%%%%%%%%%%%%%%%%%
%%%%%%%%%%%%%%%%%%%%%%%%%%%%%%%%%%%%%%%%%%%%%%%%%%%%%%%%%%%%%%%%%%%%%%%%%%%%%
%%%%%%%%%%%%%%%%%%%%%%%%%%%%%%%%%%%%%%%%%%%%%%%%%%%%%%%%%%%%%%%%%%%%%%%%%%%%%
%\newpage
\section{Related Literature}
\label{sec:literature}

Since at least the Great Depression of the 1930s, economists have been thinking about the role of finance and indebtedness in the macroeconomy. For example, the work of \textcite{fisher1933} points to the debt-deflation spiral as one of the key drivers of the Great Depression and the prolonged economic weakness that ensued. In that view, high levels of (nominal) debt have to be reduced once they prove unsustainable. However, this deleveraging kicks off a vicious circle of reduced consumption, subsequent deflation, a mechanic increase in the real debt burden, which finally aggravates the need for deleveraging.

Later, formal macroeconomic models captured financial frictions on the \textit{firm} side or in the \textit{banking} sector of the economy. The typical approach within the former model class is to let entrepreneurs have private information over their projects which their lenders can not observe \textit{ex ante}, implying that companies' net worth plays a role in their access to credit (see e.g.~\cite{bg1989}, \cite{bgg1999}). Motivated by the events of the Great Recession, the literature introduced constraints to banks' leverage, which are often motivated by the possibility of banks' managers to liquidate the bank (see e.g.~\cite{gertler2011}). Studies in both directions typically find that such financial frictions prove to be accelerators of an economy's reaction to a variety of shocks.\footnote{See \textcite{brunnermeier2012} and \textcite{christiano2022} for surveys on financial frictions in macroeconomics.}

Recently, the literature has increasingly turned to constraints to \textit{household} finance and debt as potentially important factors in business cycle analysis. This development is motivated by empirical evidence suggesting that household asset positions, access to credit and indebtedness are central to the understanding of the behaviour of individuals as well as of fluctuations in macroeconomic aggregates. Pertaining to the former, \textit{inter alia} \textcite{campbell1989} (with aggregate US time series) and \textcite{zeldes1989} (with survey data on US families) show that certain fractions of the population are indeed constrained in their access to liquidity.\footnote{Such findings are often taken as evidence against the classic permanent-income view of consumption behaviour (see e.g.~\cite{kaplan2014}).}

\textcite{gross2002} use granular credit card data which allow them to control for credit demand. The authors document that households significantly and quickly respond to exogenous changes in their borrowing limit. Upon an increase in credit supply, households increase debt and consumption. These consumption responses are heterogeneous as they are more pronounced for households at or close to their borrowing limit. Further, \textcite{gross2002} show that borrowers' debt holdings move inversely to interest rate changes.

Clearly, balance sheets of households are diverse and influence their consumption decisions. \textcite{kaplan2014} demonstrate how household portfolios differ in their liquidity composition and document that this influences households' behaviour. For example, households wealthy in illiquid assets might exhibit consumption patterns resembling the ones of the wealth-poor.

Regarding aggregate outcomes, \textcite{mian2010} show that, on the U.S.~county level, the increase in household leverage before the Great Recession and the reductions in credit card debt availability during the Great Recession significantly inform about the severity of the recession in that county. Further, in a panel-data sample on 30 mostly advanced economies, \textcite{mian2017} find that (total) household debt is a significant statistical predictor for output growth.

Prompted by the apparent relevance of financial constraints, balance sheets and debt in the data, the theoretical literature sought to develop models which give centre stage to household-level financial constraints and allow for a meaningful inclusion of household debt. Among others, \textcite{egg2012}, \textcite{riosrull2015}, \textcite{justiniano2015} and \textcite{gl2017} study borrower-lender interactions and consider how these shape macroeconomic dynamics.

In the framework of \textcite{egg2012}, borrowing and lending arise from differences in patience between two household types of exogenous proportions. The impatient agents borrow -- in nominal terms -- from the relatively patient ones, up to the real debt limit. An unexpected tightening in this limit forces debtors to delever, reducing the supply of liquidity in the economy, triggering the interest rates to fall. However, if the change in the debt constraint is large enough and the \Gls{zlb} is present, the interest rate might be unable to adjust downward sufficiently, the economy enters recession in the short-run. The downturn is exacerbated through debt-deflation à la \textcite{fisher1933}.

In contrast to the two-agent structure of \textcite{egg2012}, \textcite{riosrull2015} and \textcite{gl2017} consider \Gls{ha} models with incomplete markets in which the distribution of agents is free to adjust after a financial disturbance, a potentially important mechanism. While \textcite{riosrull2015} emphasise endogenous changes in consumers' search efforts and productivity after a shock to the debt limit, \textcite{gl2017} share with \textcite{egg2012}, as do I, the focus on the endogenous adjustment of the interest rate.

\textcite{gl2017} induce borrowing and lending by exposing households to idiosyncratic income risk. This gives rise to an increased demand for precautionary savings after a sudden, permanent reduction in the borrowing limit. Together with mechanical deleveraging, this results in a severe mismatch between asset demand and supply, provoking a reduction of aggregate demand even in the absence of nominal rigidities and the \Gls{zlb}. As the wealth distribution adjusts, the real interest rate undershoots its lower, terminal value. The authors further show that, since the wealth-poor are more sensitive in their consumption responses and it is predominantly the wealthy who react to interest-rate changes, the former suffer much larger consumption losses after the credit tightening than their wealthier peers. Considering nominal rigidities, modelled via fixed wages, and the \Gls{zlb} exacerbate the downturn. 

Relative to \textcite{gl2017}, I also motivate borrowing and lending through idiosyncratic risk but I add a small-scale, textbook \Gls{nk} structure to the \Gls{ha} block of the economy. Thus, the model features \textit{inter alia} an output-inflation trade-off in the form of a Phillips Curve, a Taylor rule and a Fisher relationship.

Naturally, the insight that forced deleveraging might spark considerable and unequal consumption losses raises the question of how policy could ameliorate the fallout of a credit crunch. While \textcite{egg2012} and \textcite{gl2017} show how fiscal policy can intervene with government debt issuance and spending after the shock arises, \textcite{korinek2016} sketch how regulation, in the form of maximum leverage, taxes on debt and insurance subsidies for borrowers, can contain excessive leverage \textit{ex-ante} and thus prevent or dampen the consequences of a deleveraging shock.

Finally, my work of course draws on the fast-growing literature embedding the incomplete-markets framework, pioneered \textit{inter alia} by \textcite{bewley1986}, \textcite{im1989}, \textcite{huggett1993} and \textcite{aiyagari1994}, in \Gls{nk} environments. An increasing number of macroeconomic questions is investigated using these \Gls{hank} models, e.g.~\textcite{mckay2016} and \textcite{kaplan2018} study monetary policy through the lens of \Gls{hank}, \textcite{bayer2023} investigate fiscal policy and its supply of liquid assets and \textcite{bayer2019} analyse shocks to income risk.\footnote{\textcite{heath2009} provide an overview of modelling heterogeneous households in macroeconomics. \textcite{kv2018} discuss the advances with \Gls{hank}. \textcite{cherrier2023} offer a historical overview of the advances and different approaches in the macroeconomic analysis with heterogeneity in the 1980s and 1990s.}

%\textcite{egg2012} also find that a deleveraging shock can be completely undone by fiscal policy by targeting short-run tax cuts and long-run tax increases at the borrowing fraction of the population, thereby effectively using its own ability to borrow to help the borrowers who are affected by the deleveraging shock. This result, however, relies very much on the assumption that the debt limit is exogenously set.

%As \textcite{egg2012} note, it is reasonable to assume that the debt limit depends to some degree on current conditions, such as current income.

%Finally, my work is of course related to the literature which studies macroeconomic fluctuations in models with heterogeneous agents. The origins of these models, in which households can only self-insure against idiosyncratic risk by means of precautionary savings, were pioneered \textit{inter alia} by \textcite{bewley1986}, \textcite{im1989}, \textcite{huggett1993} and \textcite{aiyagari1994}.\footnote{Truman Bewley is credited with proving the existence of equilibrium in models where agents face uninsurable idiosyncratic risk (see \cite{cherrier2023}). \textcite{im1989} compares the costs of business cycles in imperfect-insurance models to those implied by perfect-insurance models and finds that the former exhibits higher costs than the latter in the case of no borrowing and lower costs when borrowing is permitted. The \textcite{huggett1993} economy features borrowers and lenders and offers a potential explanation as to why the observed risk-free rate is so much lower than the representative-agent models would suggest. The paper also shows that the risk-free rate is decreasing in the borrowing limit. \textcite{aiyagari1994} shows, in a framework where asset supply is given by heterogeneous households and asset demand by firms with a neoclassical production function, that the degree by which the presence of idiosyncratic risk influences aggregate savings and the equilibrium interest rate depends on the calibration of the parameters governing the desire for precautionary savings. \textcite{heath2009} provide an overview of modelling heterogeneous households in macroeconomics. See \textcite{cherrier2023} for a historical overview of the advances and different approaches in the macroeconomic analysis with heterogeneous agents in the 1980s and 1990s.} More recently, the \Gls{ha} framework has been fused with the New Keynesian approach of modelling the economy, giving rise to the already mentioned \Gls{hank} model class. An increasing number of questions is investigated using the \Gls{hank} framework. For example, \textcite{mckay2016} and \textcite{kaplan2018} use \Gls{hank} for the analysis of monetary policy, \textcite{bayer2023} for the investigation of fiscal policy and the supply of liquid assets by the government and \textcite{bayer2019} for the analysis of shocks to income risk.

%%%%%%%%%%%%%%%%%%%%%%%%%%%%%%%%%%%%%%%%%%%%%%%%%%%%%%%%%%%%%%%%%%%%%%%%%%%%%
%%%%%%%%%%%%%%%%%%%%%%%%%%%%%%%%%%%%%%%%%%%%%%%%%%%%%%%%%%%%%%%%%%%%%%%%%%%%%
%%%%%%%%%%%%%%%%%%%%%%%%%%%%%%%%%%%%%%%%%%%%%%%%%%%%%%%%%%%%%%%%%%%%%%%%%%%%%
\section{Baseline Model}
\label{sec:model}

This section puts forward the baseline \Gls{hank} model used in sections \ref{sec:limit} and \ref{sec:wedge}. The model is heavily inspired by the one-asset \Gls{hank} models in \textcite{auclert2021} and \textcite{boehl2023econpizza}, so I base the description of the present model to large extent on these studies.

Overall, the model features a household block, a corporate sector and a government which conducts monetary and fiscal policies. Section \ref{sec:model-hh} begins with the household sector. Section \ref{sec:model-sectors} deals with the firms, the fiscal and monetary authorities as well as with equilibrium. Finally, section \ref{sec:model-solution} touches upon the numerical solution method and explains the model calibration. 

In all what follows, time, $t$, is discrete, refers to quarters of a year and runs forever.

%%%%%%%%%%%%%%%%%%%%%%%%%%%%%%%%%%%%%%%%%%%%%%%%%%%%%%%%%%%%%%%%%%%%%%%%%%%%%
%%%%%%%%%%%%%%%%%%%%%%%%%%%%%%%%%%%%%%%%%%%%%%%%%%%%%%%%%%%%%%%%%%%%%%%%%%%%%
\subsection{Heterogeneous Households and Financial Constraints}
\label{sec:model-hh}

Households in this model choose optimal paths of consumption, labour hours and savings, given prices. However, individual labour income varies stochastically. To smooth consumption in the presence of incomplete markets, households save in government bonds and lend to and borrow from each other. This private credit intermediation is subject to a debt limit and a premium on the borrowing rate. The present model allows for shocks to both of these constraints.

%%%%%%%%%%%%%%%%%%%%%%%%%%%%%%%%%%%%%%%%%%%%%%%%%%%%%%%%%%%%%%%%%%%%%%%%%%%%%
\subsubsection{Households' Environment}
\label{sec:model-hh-environment}

This subsection presents the formal environment that households are in. Specifically, the economy is populated by a mass-one continuum of \textit{ex-ante} identical households, indexed by $i$. Each household is subject to idiosyncratic labour income risk as its labour productivity, denoted by $e_{i,t}$, fluctuates over time. Specifically, $e_{i,t}$ evolves as a discrete Markov process, generated from the following first-order autoregressive (\Gls{ar}(1)) process:\footnote{The \textcite{rouwenhorst1995} method transforms the continuous process into a discrete one.}
\begin{equation}
    log(e_{i,t}) = \rho_e log(e_{i,t-1}) + \varepsilon^e_{i,t}, \label{eq:hh-income-process}
\end{equation}
where $\varepsilon^e_{i,t}$ is a mean-zero shock to agents' skill level with variance $\sigma_e$. Idiosyncratic productivity shocks die out with persistence $\rho_e$. Average productivity is one. %\footnote{For simplicity, I ignore the possibility of unemployment here, in contrast to \textcite{gl2017}.}

Household-specific productivity, $e_{i,t}$, signifies how well household $i$ can turn its working hours into output in period $t$. For the household, process \eqref{eq:hh-income-process} is entirely exogenous and can not be influenced. $e_{i,t}$ enters the labour income of household $i$ in $t$, which is composed of idiosyncratic productivity, individual labour hours, $n_{i,t}$ and the aggregate wage rate, $w_t$.

Households face incomplete markets, i.e.~they lack access to state-contingent insurance against idiosyncratic risk stemming from $\varepsilon^e_{i,t}$. Instead, households can only self-insure by accumulating precautionary savings in a non-state-contingent and perfectly liquid asset which matures after one period.\footnote{'Non-state-contingent' means that the asset pays out in any state of the (model) world. In particular, the payment is independent of $e_{i,t}$. 'Perfectly liquid' means that there are no costs involved in trading the asset.} This asset is in real terms, i.e.~it is a claim to one unit of the consumption good in the next period. I denote the period-$t$ choice on holdings of this asset for period $t+1$ as $b_{i,t}$. As explained below, the model allows households to borrow. If agent $i$ chooses to borrow, $b_{i,t}$ is negative to denote the holding of debt.\footnote{Note that, as in most models (e.g.~\cite{kaplan2018}), asset and debt holdings alike here refer to \textit{net} positions, i.e.~$b_{i,t} > 0$ means that the asset side of the household's balance sheet is longer than its liability side.}

Given the prevailing prices in the economy, given the process in \eqref{eq:hh-income-process} and given the asset structure, households maximise their expected life-time utility:
\begin{equation}
    \max_{c_{i,t}, n_{i,t}, b_{i,t}} \ \mathbb{E}_{e,t} \Bigg\{ \sum_{t=0}^{\infty} \beta^t u(x_{i,t}) \Bigg\}, \label{eq:hh-objective}
\end{equation}
by choosing individual (non-durable) consumption in period $t$, $c_{i,t}$, individual labour supply in $t$, $n_{i,t}$, and next-period's asset holdings, $b_{i,t}$, for all periods. Note that $\mathbb{E}_{e,t}$ denotes expectations taken at time $t$ over the skill process \eqref{eq:hh-income-process}, $\beta$ is the factor by which households discount future utility, $u( \cdot )$ is the period-utility function and $x_{i,t}$ is a composite good of $c_{i,t}$ and $n_{i,t}$. The functional forms of $u( \cdot )$ and $x_{i,t}$ are specified below but, naturally, households derive utility from consumption and experience a utility loss from working. Thus, I require $u( \cdot )$ to be increasing in $x_{i,t}$ as well as $x_{i,t}$ to be increasing in $c_{i,t}$ and decreasing in $n_{i,t}$.\footnote{Formally: $\frac{\partial u( \cdot )}{\partial x_{i,t}} > 0$, $\frac{\partial x_{i,t}}{\partial c_{i,t}} > 0$, $\frac{\partial x_{i,t}}{\partial n_{i,t}} < 0$.}

In maximising \eqref{eq:hh-objective}, households are subject to a sequence of budget constraints and a sequence of borrowing constraints. Intuitively, the budget constraint states that, in any given period $t$, household $i$ must not spend more than it earns. The period-$t$ budget constraint for household $i$ reads as follows: 
\begin{equation}
c_{i,t} + b_{i,t} = RR (RR_t, b_{i,t-1}) b_{i,t-1} + w_t e_{i,t} n_{i,t} + \Pi_t \bar{\Pi} (e_{i,t}) - \tau_t \bar{\tau} (e_{i,t}), \label{eq:hh-budget}
\end{equation}
where $w_t$ is the aggregate real wage paid for each unit of effective labour supply, $e_{i,t} n_{i,t}$, and where $RR (b_{i,t-1}, RR_t)$ is the real interest rate schedule, which determines the return on or cost of previous period's asset or debt holdings, $b_{i,t-1}$. These two are the crucial prices guiding the households' decisions on working, consuming and saving. 

$\Pi_t$ are firm dividends and $\tau_t$ is a tax rate set by the government, discussed in subsequent sections. As in \textcite{mckay2016}, $\bar{\Pi} (e_{i,t})$ and $\bar{\tau} (e_{i,t})$ are incidence rules which transform aggregate dividends and taxes into household-level values in a lump-sum fashion.\footnote{It is well-known that, in \Gls{ha} models, the rule which distributes taxes and profits is of importance for the results of these models (see e.g.~\cite{kaplan2018}). As \textcite{mckay2016} explain, the present approach ensures that profits and taxes are distributed in a way analogous to lump-sum procedures in representative-agent models.} Thus, the left-hand side of \eqref{eq:hh-budget} captures the expenditure of household $i$ in $t$, comprised of goods consumption and asset acquisitions. Earnings, the right-hand side of \eqref{eq:hh-budget}, are composed of asset income, labour income and dividend income from firms, less taxes.

%The household takes all of these prices, profits and taxes as given when making its consumption-working-savings decisions. 

As already noted, the real interest rate schedule in \eqref{eq:hh-budget} is the key price for a household's consumption-savings decision. I specify the real interest rate schedule to take the following form:
\begin{equation}
    RR (RR_t, b_{i,t-1}) = \begin{cases}
	RR_t \ \ \ \ \ \ \ \ \ \ \ \ \ \text{if} \ b_{i,t-1} \ge 0 \\
	RR_t + \kappa_t \ \ \ \ \text{if} \ b_{i,t-1} < 0,
	\end{cases} \label{eq:hh-interest-rate-wedge}
\end{equation}
where $RR_t$ is the gross \textit{ex-post} real interest rate, to be defined momentarily, and where $\kappa_t \ge 0$ is a premium paid for negative asset positions. That is, at positive asset holdings, household $i$ earns the return $RR_t$. However, as soon as the household chooses to borrow, $b_{i,t-1} < 0$, it has to pay $RR_t$ plus $\kappa_t$, if the latter is strictly positive. This wedge is an additional cost to borrowing and is thus a \textit{financial constraint} to household financing, discussed in depth in subsection \ref{sec:model-hh-constraints}.

As in \textcite{auclert2021}, $RR_t$ in \eqref{eq:hh-interest-rate-wedge} adheres to the following Fisher relationship:
\begin{equation}
    RR_t = \frac{R_{t-1}}{\pi_t}, \label{eq:hh-fisher-equation}
\end{equation}
where $R_{t-1}$ is last period's gross nominal interest rate and $\pi_t$ is the current gross rate of inflation, which are determined by monetary policy and by firms, respectively. I discuss $R_{t-1}$ and $\pi_t$ in detail in the corresponding subsections \ref{sec:model-policy} and \ref{sec:model-firms}.

Finally, when choosing next period's asset holding, $b_{i,t}$, household $i$ faces a borrowing constraint à la \textcite{huggett1993}:
\begin{equation}
    b_{i,t} \ge \phi_t, \label{eq:borrowing-limit}
\end{equation}
where $\phi_t \le 0$ is an exogenously imposed limit to borrowing. $\phi_t = 0$ prohibits borrowing, $\phi_t < 0$ allows borrowing up to the specified limit. This limit to indebtedness is another \textit{financial constraint}. Subsection \ref{sec:model-hh-constraints} discusses the borrowing limit in detail.

%or, equivalently, the inverse of the intertemporal elasticity of substitution

%%%%%%%%%%%%%%%%%%%%%%%%%%%%%%%%%%%%%%%%%%%%%%%%%%%%%%%%%%%%%%%%%%%%%%%%%%%%%
\subsubsection{Solution to the Household Problem and Aggregation}
\label{sec:model-hh-solution}

Equations \eqref{eq:hh-income-process} -- \eqref{eq:borrowing-limit} constitute the formal setup of households. Before I turn to a discussion of the key mechanisms and implications of this framework, this subsection succinctly presents the solution to the household problem and aggregates household-level outcomes. Consult appendix \ref{sec-app:full-derivations-households} for the detailed derivations.

To solve the optimisation problem, I follow a series of studies in the \Gls{ha} literature to specify the functional forms of $u( \cdot )$ and $x_{i,t}$, see e.g.~\textcite{bayer2023}. The utility function in $t$ is of the constant relative-risk aversion (\Gls{crra}) type:
\begin{equation}
    u(x_{i,t}) = \frac{x_{i,t}^{1-\sigma_c}}{1-\sigma_c}, \label{eq:hh-utility-fun}
\end{equation}
where $\sigma_c$ is the coefficient of relative risk aversion, which reflects the degree with which households want to smooth consumption over time. The composite good, $x_{i,t}$, combines $c_{i,t}$ and $n_{i,t}$ as follows:
\begin{equation}
    x_{i,t} = c_{i,t} - e_{i,t}\frac{n_{i,t}^{1+\sigma_l}}{1+\sigma_l},
    \label{eq:hh-composite-good}
\end{equation}
where $\sigma_l$ is the inverse Frisch elasticity of labour supply. Solving \eqref{eq:hh-income-process} -- \eqref{eq:hh-composite-good} yields two first-order conditions (\Gls{foc}) and an envelope condition. The first \Gls{foc} and the envelope condition determine the households' optimal consumption, $c_{i,t}$, and savings policies, $b_{i,t}$, given last period's bond holdings, $b_{i,t-1}$, and the current period's productivity realisation, $e_{i,t}$:
\begin{align}
    x_{i,t}^{-\sigma_c} &\ge \beta \mathbb{E}_{e,t} \{ V_{b,t+1} (e_{i,t+1}, b_{i,t}) \}, \label{eq:foc-euler} \\
    V_{b,t} (e_{i,t}, b_{i,t-1}) &= RR_t (b_{i,t-1}) x_{i,t}^{-\sigma_c}, \label{eq:foc-envelope} 
\end{align}
where $V_{b,t}$ is the marginal continuation value of marginally increasing bond holdings for the next period. The Euler equation \eqref{eq:foc-euler} holds with equality whenever household $i$ is away from the borrowing limit. The second \Gls{foc} pins down the optimal labour supply of household $i$, given the real wage:
\begin{equation}
n_{i,t}^{\sigma_l} = w_t, \label{eq:foc-labour-supply}
\end{equation}
which, due to \Gls{ghh} preferences, is independent of household $i$'s consumption level and asset position, see appendix \ref{sec-app:full-derivations-households}.

The fact that households transverse different paths of productivity realisations from \eqref{eq:hh-income-process}, induces them to accumulate and decumulate asset holdings at varying speeds. This gives rise to \textit{ex-post} household heterogeneity with respect to productivity states and asset holdings. That is, at time $t$, there is a distribution of agents over skill levels and wealth, which I denote by $\Psi_t (b,e)$, where, for notational convenience, $b=b_{i,t-1}$ and $e=e_{i,t}$.\footnote{Note the timing: The distribution in $t$ is determined by how much wealth (or debt) agents in $t-1$ chose to hold in $t$ and by where the productivity process dictates them to be in $t$.}

% In a discretised setup, $\Psi_t (b,e)$ essentially returns Pr[].

To obtain economy-wide consumption, labour-supply and savings, I thus simply integrate individual-level outcomes over the distribution of agents across productivity levels and previous period's asset holdings. Formally:
\begin{equation}
    C_t \equiv \int c_{i,t} d \Psi_t (b,e), \ \ \ N_t \equiv \int n_{i,t} d \Psi_t (b,e), \ \ \ B_t^d \equiv \int b_{i,t} d \Psi_t (b,e), \label{eq:hh-aggregation}
\end{equation}
where $C_t$ is aggregate consumption, $N_t$ is aggregate supply of labour hours and $B_t^d$ is aggregate asset demand. Note that due to the fact that all households supply the same amount of labour, individual and aggregate supplies co-move one-to-one and \eqref{eq:foc-labour-supply} generalises to the aggregate labour supply condition:
\begin{equation}
    N_t^{\sigma_l} = w_t. \label{eq:hh-agg-labour-supply}
\end{equation}

%, because \Gls{iou}s cancel each other out from household balance sheets. Thus, $B_t^d$ corresponds to aggregate demand for government bonds.

%%%%%%%%%%%%%%%%%%%%%%%%%%%%%%%%%%%%%%%%%%%%%%%%%%%%%%%%%%%%%%%%%%%%%%%%%%%%%
\subsubsection{Incomplete Markets, Assets, Borrowers and Lenders}
\label{sec:model-hh-assets}

This subsection aims at clarifying the key implications of the chosen asset structure and how borrowers and lenders emerge in the model. Thereafter, subsection \ref{sec:model-hh-constraints} explains how the financial constraints introduced above affect households' consumption behaviour.  

In the model, households' main objective is to smooth consumption over time. However, they are subject to idiosyncratic labour productivity risk from \eqref{eq:hh-income-process} and thus fear the possibility of obtaining low labour income, which hurts their consumption. Further, households dread the possibility of reaching the borrowing limit in \eqref{eq:borrowing-limit}, e.g.~after repeatedly being in low productivity states. Being at the debt limit would be particularly harmful for households because it bars them from taking on any additional debt to absorb adverse productivity shocks, forcing them to adjust consumption one-to-one with the reduction in their income (see e.g.~\cite{carroll2021}). 

Clearly, households have an incentive to insure themselves against labour income shocks.\footnote{The chosen \Gls{ghh} preferences eliminate the possibility of households using working hours as insurance against low productivity. Section \ref{sec:sensitivity-end-labour} explores the case where households can adjust labour supply individually.} However, in an environment of incomplete markets, contracts which insure against specific realisations of labour productivity, high or low, are unavailable.\footnote{Incidentally, assuming complete markets and hence the presence of state-contingent claims would reduce the problem to a representative-agent setup, rendering the investigation of the present issues vacuous.} As \textcite{heath2009} remark, households thus resort to self-insurance by means of precautionary savings which they can draw down when confronted with low labour income. The existence of the borrowing limit further strengthens the wish to save. Crucially, the desire for precautionary savings contributes to the \textit{demand for assets} by households. 

Some households, however, do remain in states of low productivity for prolonged periods of time. In order to nevertheless sustain a certain level of consumption, these households have a high willingness to tap into debt to finance the related expenditures, even if this involves owing the borrowed funds and the return $RR_t + \kappa_t$. Thus, these households have a \textit{demand for debt}.

Now, it is important to realise that the demand for debt corresponds to a source of \textit{asset supply}. This is because a debt contract offers lenders a vehicle to postpone consumption to the future as it specifies that the borrowing household will, in the next period, return the borrowed funds plus some interest. Borrowing thus creates a private liquid asset in the form of an "I Owe You" (\Gls{iou}), i.e.~a debtor's promise to provide a future unit of consumption. Consequently, the act of borrowing reflects private credit intermediation (see e.g.~\cite{bayer2023}).

\Gls{iou}s are mutually beneficial since the borrowers' desire to trade consumption today against a promise to repay in the future is matched by the lenders' wish to save today in return for a claim to future consumption. On the one hand, lenders are willing to supply these funds as they have experienced a series of positive shocks and thus demand liquid savings as insurance against future shocks. On the other hand, borrowers have gone through low productivity and would like to exchange some of tomorrow's consumption for consumption today.

As the debt that borrowing households have on the liability side of their balance sheet are matched one-to-one by assets that the lenders hold on the asset side of their balance sheets -- each positive \Gls{iou} of a lender is a negative \Gls{iou} of some borrower -- the two groups of households are tightly linked through their balance sheets and the asset market. This crucially implies that when borrowers are forced, out of some reason, to curtail their leverage and thus their liabilities, lenders will be directly affected as the supply of \Gls{iou}s they can invest in is reduced. Evidently, such a credit crunch will, through interlinked balance sheets, have repercussions that outsize the initial effect on borrowers. %This is at the core of the analyses below.

The real interest rate, $RR_t$, is what lenders earn on their positive \Gls{iou}s and what borrowers pay on their negative \Gls{iou}s. Thus, the real rate is the key price guiding households' borrowing-savings decisions. Naturally, debtors prefer to pay low interest, while lenders like high returns on their savings. To match these contrarious interests, $RR_t$ adjusts endogenously so that, in equilibrium, saving and borrowing households have no incentive to save more or less or to borrow more or less. Hence, whenever a situation of excess demand for bonds or, equivalently -- through the budget constraint \eqref{eq:hh-budget} -- of excess demand for tomorrow's relative to today's consumption, arises, the real interest rate has to adjust to restore equilibrium in the asset market.

Unlike in \textcite{gl2017}, the Fisher equation \eqref{eq:hh-fisher-equation} pins down the real interest rate here, highlighting that developments in the asset market influence $RR_t$ through the goods market. Further, the real rate is an \textit{ex-post} rate in this setup, i.e.~it applies to past instead of to future asset holdings. This does not change the main mechanisms relative to \textcite{gl2017} but gives rise to some distinctive features of the analysis in \ref{sec:limit}.

As lenders' and borrowers' asset positions net out, \Gls{iou}s are in zero net supply. Therefore, up to this point, the asset structure is essentially the one found in the \textcite{huggett1993} economy, where aggregate net liquidity is zero. However, I follow a series of papers (e.g.\cite{gl2017}, \cite{bayer2023}) by introducing an exogenous and time-invariant supply of government bonds, $B$. These bonds are perfect substitutes to the privately intermediated \Gls{iou}s and are thus an additional asset which households can save in. The entirety of liquidity available to net lenders, gross liquidity, is the sum of bonds and privately intermediated \Gls{iou}s.

%Note crucially that the \Gls{iou}s in this model correspond to unsecured borrowing. That is, borrowers do not post any collateral against their debt. One of the reasons why lenders are nevertheless content with conceding funds to the borrowers is because the model does not consider the risk of borrowers defaulting.\footnote{Thus, indebted households always repay their debt. In part, this is the case here because the borrowing limit in \eqref{eq:borrowing-limit} is, in all chosen calibrations, tighter than the \textit{natural borrowing limit}, see appendix \ref{sec:app-hh-borrowing} for details.}

%What is more, there is government insurance and insurance from employers, see \textcite{heath2009}. \textcite{kaplan2018} provide an insightful critique of the one-asset structure. 

%%%%%%%%%%%%%%%%%%%%%%%%%%%%%%%%%%%%%%%%%%%%%%%%%%%%%%%%%%%%%%%%%%%%%%%%%%%%%
\subsubsection{Financial Constraints and Consumption Behaviour}
\label{sec:model-hh-constraints}

In obtaining credit from lenders, borrowing households in the present economy are subject to two financial constraints: an outright limit to debt and a premium on the lending rate they pay on the borrowed funds. How do these frictions work and how do they affect household behaviour?

Equation \eqref{eq:borrowing-limit} caps the maximum \textit{quantity} of debt a household can hold on its balance sheet to amounts smaller or equal to $\phi_t$. In everyday life, households often encounter such limits. For example, credit card holders experience limits to how much they can spend with their credit cards. As stated in \textcite{carroll2021}, a borrowing limit constrains households with unfortunate income realisations in their ability to smooth prolonged shocks to their consumption levels. This is because the debt limit introduces an asymmetry at $b_{i,t} = \phi_t$: households can always choose to retire debt -- e.g.~if they experience an increase in income -- but are restricted in taking up more debt in case of an unfortunate realisation of skills and income. 

%Also, home buyers are often not able to take out loans which cover the full price of the house they want to purchase and are thus subject to a mortgage loan limit.\footnote{The example of credit card debt is more fitting in terms of the model as it usually is unsecured debt, just as the \Gls{iou}s of the model. Mortgages count to secured, i.e.~collateralised, debt.}

This asymmetry, or kink, in the savings decision of households influences their consumption behaviour. Importantly, the borrowing constraint forces households that are \textit{at} the constraint to consume in a hand-to-mouth fashion as they can only consume their labour and profit incomes, less taxes and interest payments. This implies that, for these households, a negative shock to productivity translates into a one-to-one fall in consumption, exposing these agents in an extreme way to fluctuations in their labour skills. 

Consequently, the affected households have a very high inclination to consume if given a small windfall of liquid wealth. In other words, because they are at the debt limit and would probably like to borrow more, these agents will consume all of any small amount of extra wealth or, equivalently, less debt and thus remain constrained. Hence, individuals at the constraint have very high marginal propensities to consume (\Gls{mpc}) out of a marginal windfall of liquid wealth.\footnote{I follow \textcite{gl2017} to define and calculate the \Gls{mpc}: $mpc_{i,t} = \frac{\partial c_{i,t}}{\partial b_{i,t}}$. I restrict idiosyncratic \Gls{mpc}s to be weakly smaller than one.} The data confirm the existence of such behaviour, see \textit{inter alia} \textcite{gross2002}.

Even households that are away from the borrowing limit are affected by its existence as they anticipate the risk of reaching the limit in finite time, which would expose them to the same asymmetry as the actually constrained agents. The fear of being constrained in the future prompts even wealthier and more productive households to engage in precautionary savings. However, the further away the household is from the constraint, the less probability it assigns to the possibility of being constrained. These households in turn behave increasingly like fully unconstrained agents, which the literature calls permanent-income consumers. Crucially, these agents feature significantly lower \Gls{mpc}s out of liquid wealth as they were able to already reach rather high levels of consumption which they can shield well with their high wealth. In total, borrowers -- especially those close at the debt limit -- are expected to have a much higher marginal propensity to consume (\Gls{mpc}) out of liquid wealth than lenders do. This proves an important point in the subsequent analyses.

Apart from outright limits to borrowing, households may be forced to pay a premium, $\kappa_t$, on the lending rate for their borrowed funds, as specified in \eqref{eq:hh-interest-rate-wedge}.\footnote{Note that $\kappa_t = \infty$ would be equivalent to $\phi_t=0$.} This premium operates as a stand-in for the often encountered fact that financial institutions typically charge higher rates on accounts with negative balances than they pay on accounts with positive balances. As the present model does not feature any institution which maximises profits by setting $\kappa_t$, the wedge reflects a wasted cost in private credit intermediation as in e.g.~\textcite{bayer2023}.

\textcolor{red}{As the online appendix of \textcite{achdou2022} notes, such a kink can account for empirical facts of the wealth distribution.\footnote{Specifically, I refer to appendix G.3 of the online appendix of \textcite{achdou2022}, which is available here: \url{https://benjaminmoll.com/wp-content/uploads/2019/07/HACT_appendix.pdf}, last retrieved \printdate{2023-07-27}.} Here, I employ it as a device to capture another household-level financial constraint. In particular, it is an extra \textit{cost} of holding debt.}

\textcolor{red}{In essence, it is a constraint that affects all borrowers and thus a larger fraction of the population. Also, it introduces a kink in the savings decision of the household already at zero assets rather than at the borrowing limit and might thus be interesting to study.}

%\textcolor{red}{It might also lead to more \Gls{mpc} heterogeneity.}

Clearly, financial constraints such as the borrowing limit and the interest rate wedge might very well vary across household characteristics (e.g.~income, wealth, credit-worthiness) and asset types (e.g.~credit card debt vs mortgage debt). Importantly however, these constraints can vary over time, reflecting the fact that credit can become more or less abundant in the course of time. For example, \textcite{anenberg2018} empirically document how the limits to mortgage loan amounts do not only differ across borrower types but also that they evolve over time.\footnote{\textcite{anenberg2018} use a frontier estimation approach to show \textit{inter alia} how lending constraints became looser in the run-up to the Great Recession and tightened during and after the crisis.} Section \ref{sec:limit} and \ref{sec:wedge} aim to capture the effects of the exogenous component of such variation over time in order to make causal statements about the mechanisms induced by a credit crunch.\footnote{In general, financial constraints are of course endogenous. However, setting the constraints exogenously allows to study the main channels by which variation of these constraints influence the economy. In a more comprehensive model, $\phi_t$ and $\kappa_t$ would be the result of the optimisation of some agents, e.g.~financial institutions.}

A shock to the financial constraints to household finance can be due to a lot of reasons, \textit{inter alia} due to lost confidence in the ability of borrowers to repay their debt.

In summary, the present model features household subject to idiosyncratic risk. Some households are "lucky", while others turn out to be "unlucky". The former lend to the latter, which is mutually beneficial. The real interest rate plays the decisive role of intermediating between the two sides. Finally, borrowers and lenders crucially differ in their consumption behaviour.

%First, unlike their representative-agent counterparts, prototypical \Gls{ha} models explicitly incorporate meaningful borrowing limits.\footnote{Prototypical representative-agent models impose some kind of transversality condition to rule out the possibility of a household running an indefinite Ponzi scheme.} Second, \Gls{ha} models speak to the differential effect of borrowing constraints across the distribution of households.

%%%%%%%%%%%%%%%%%%%%%%%%%%%%%%%%%%%%%%%%%%%%%%%%%%%%%%%%%%%%%%%%%%%%%%%%%%%%%
%%%%%%%%%%%%%%%%%%%%%%%%%%%%%%%%%%%%%%%%%%%%%%%%%%%%%%%%%%%%%%%%%%%%%%%%%%%%%
\subsection{Firms, Government and Equilibrium}
\label{sec:model-sectors}

This section closes the model with a corporate sector and a government which controls a fiscal and a monetary authority. Further, I formally define equilibrium for the present model.

%%%%%%%%%%%%%%%%%%%%%%%%%%%%%%%%%%%%%%%%%%%%%%%%%%%%%%%%%%%%%%%%%%%%%%%%%%%%%
\subsubsection{Firms}
\label{sec:model-firms}

The corporate sector of the model is kept minimal and close to the \Gls{nk} literature with ingredients from the approaches of \textcite{mckay2016} and \textcite{gust2017wp}. In particular, I assume that a representative and perfectly competitive firm produces the final good by using the intermediate goods from a continuum of monopolistically competitive firms which employ households' labour and set prices subject to convex price adjustment costs à la \textcite{rotemberg1982} with price indexation.\footnote{Thus, I abstract from productive capital. A meaningful inclusion of capital would require a two-asset model in order to avoid a counterfactual increase in investment amid a credit crunch, see \textcite{kv2018}.}

Appendix \ref{sec-app:full-derivations-firms} formally states the firms' maximisation problems and solves for the equations which determine their optimal behaviour. Here, I simply state the equations relevant for the model implementation. First, production of the final good, $y_t^f$, occurs according to the following simple production function:
\begin{equation}
    y_t^f = z N_t, \label{eq:firms-production}
\end{equation}
where $z$ is technology and $N_t$ is aggregate labour input. Aggregate marginal costs are thus given by $mc_t = \frac{w_t}{z}$. 

Costly price adjustment makes the intermediate-goods firms' problem dynamic, leading to a \Gls{nk} Phillips Curve (\Gls{nkpc}) of the form:
\begin{equation}
    \psi \Bigg( \frac{\pi_t}{\tilde{\pi}_{t-1}} - 1 \Bigg) \frac{\pi_t}{\tilde{\pi}_{t-1}} = (1-\theta) + \theta mc_t + \psi \beta \Bigg( \frac{\pi_{t+1}}{\tilde{\pi}_{t}} - 1 \Bigg) \frac{\pi_{t+1}}{\tilde{\pi}_{t}} \frac{y_{t+1}^f}{y_t^f}, \label{eq:firms-nkpc}
\end{equation}
where $\pi_t \equiv \frac{P_t}{P_{t-1}}$ is the change in the price level in period $t$, i.e.~gross inflation in period $t$. $\tilde{\pi}_t \equiv \pi_{ss}^{\omega} \pi_{t}^{1-\omega}$ is indexed inflation with $\omega$ determining the degree of indexation to steady-state inflation. $\theta$ is the elasticity of substitution between the intermediate-goods varieties in the production of $y_t^f$ and $\psi$ is the parameter which determines the costs to price adjustment.  

The \Gls{nkpc} pins down inflation in this model and reflects the trade-off between changes in economic activity and changes in prices. In essence, it captures how a reduction in goods demand leads to a reduction in inflation -- or deflation -- through lower marginal costs, which come about through lower wages as labour demand adjusts to lower goods demand. This trade-off gives rise to a channel where aggregate consumption demand, through firms and their price-setting, directly influences the \textit{ex-post} real interest rate, $RR_t$, making transparent the tight link between the asset and the goods market in the model.

Finally, monopolistic competition in the market for inputs implies that intermediate-goods firms can sell their output with a markup, $\mu_t$, resulting in firm profits:
\begin{equation}
    \Pi_t = \Bigg[ 1 - \frac{\psi}{2} \Bigg( \frac{\pi_t}{\tilde{\pi}_{t-1}} - 1 \Bigg)^2 \Bigg] y_t^f - w_t N_t, \label{eq:firms-profits}
\end{equation}
which, as explained above, are distributed lump-sum to households. Thus, households own equal shares in the firms, which they can not trade.

%%%%%%%%%%%%%%%%%%%%%%%%%%%%%%%%%%%%%%%%%%%%%%%%%%%%%%%%%%%%%%%%%%%%%%%%%%%%%
\subsubsection{Fiscal and Monetary Policies}
\label{sec:model-policy}

The government conducts fiscal and monetary policy. As alluded to in section \ref{sec:model-hh-assets}, the fiscal authority supplies a time-invariant quantity of government bonds, $B$, which imposes interest payments on the government. To run a balanced budget, the fiscal authority levies the already specified tax on households, $\tau_t$. Thus, the government budget constraint reads:
\begin{equation}
    \tau_t + B = RR_t B \ \ \ \Leftrightarrow \ \ \ \tau_t = (RR_t - 1)B, \label{eq:fiscal-budget}
\end{equation}
which implies that the tax rate moves one-to-one with the \textit{ex-post} real interest rate.

The monetary authority sets the notional gross nominal interest rate, $R_{t}^n$, in order to stabilise inflation and output. It does so according to a standard Taylor rule:
\begin{equation}
    R_{t}^n = ( R_{t-1}^n )^{\rho} \Bigg[ R_{ss} \Bigg( \frac{\pi_t}{\pi_{ss}} \Bigg)^{\phi_{\pi}} \Bigg( \frac{y_t}{y_{ss}} \Bigg)^{\phi_y} \Bigg]^{(1-\rho)}, \label{eq:taylorrule}
\end{equation}
where $R_{ss}$ is the steady state gross nominal interest rate, to be defined momentarily, and $y_t$ is total "effective" output of the economy, specified in section \ref{sec:model-eq}. Equation \eqref{eq:taylorrule} implies that the central bank responds -- with persistence $\rho$ -- to deviations of inflation and output from their steady state values. The strengths of these responses are given by $\phi_{\pi}$ and $\phi_y$, respectively. 

In setting the notional rate, however, the central bank is subject to the \Gls{zlb} on nominal interest rates. Hence, the gross nominal interest rate, $R_t$, is pinned down by the maximum out of the notional gross nominal interest rate and $1$: 
\begin{equation}
    R_t = \text{max} \ \{ 1, \ R_{t}^n \}. \label{eq:zlb}
\end{equation}

%In contrast to \textcite{gl2017}, the real interest rate here is pinned down by the Fisher equation:
%which implies that firms' price setting and the central bank influence and determine the real interest rate. This offers a new mechanism by which the economy can adjust. 

This completes the description of the models' agents. For the reader's convenience, I depict the baseline model graphically in figure \ref{fig:graphical-model} in appendix \ref{sec-app:full-graphical}.

%%%%%%%%%%%%%%%%%%%%%%%%%%%%%%%%%%%%%%%%%%%%%%%%%%%%%%%%%%%%%%%%%%%%%%%%%%%%%
\subsubsection{Equilibrium and Market Clearing}
\label{sec:model-eq}

Next, I define equilibrium and state the conditions by which markets clear. Recall that $\Psi_t (b,e)$ is the distribution of agents over $b_{i,t-1}$ and $e_{i,t}$ at $t$.

\begin{definition}[Equilibrium]
\label{def:eq}
An equilibrium in this economy is defined as a sequence of household consumption and savings policies, $\{ c_{i,t} (b, e) \}_{t=0}^{\infty}$ and $\{ b_{i,t} (b, e) \}_{t=0}^{\infty}$, a sequence of distributions, $\{ \Psi_t (b,e) \}_{t=0}^{\infty} $, and a sequence of aggregate outcomes $\{ B_t^d$, $C_t$, $mc_t$, $N_t$, $\pi_t$, $\Pi_t$, $RR_t$, $R_t$, $R_t^n$, $\tau_t$, $w_t$, $y_t$, $y_t^f \}$ such that:
\begin{enumerate}
    \item given prices, the policies, $c_{i,t} (b, e)$ and $b_{i,t} (b, e)$, solve the households' maximisation problem given in equations \eqref{eq:hh-income-process} -- \eqref{eq:borrowing-limit},
    \item together with the Markov process for productivity, the policies induce a law of motion for the distribution, denoted by $\Gamma$,
    \item given an initial distribution $\Psi_0$, the distribution solves the forward equation of the law of motion:
    \begin{equation*}
        \Psi_{t+1} (b_{i,t},e_{i,t+1}) = \Psi_{t} (b_{i,t-1},e_{i,t}) \Gamma,
    \end{equation*}
    \item all markets clear.
\end{enumerate}
\end{definition}

The labour market clears when condition \eqref{eq:hh-agg-labour-supply} holds. The bond and goods markets clear, respectively, when:
\begin{align}
    B_t^d &= B, \label{eq:bond-clearing} \\
    C_t &= y_t \label{eq:goods-clearing}
\end{align}
hold, where $y_t$ is the output that remains after subtracting costs of price adjustment and of credit intermediation:
\begin{equation}
    y_t \equiv \Bigg( 1 - \frac{\psi}{2} \Bigg( \frac{\pi_t}{\pi_{t-1}} - 1 \Bigg)^2 \Bigg) y_t^f - \kappa_t D_t, \label{eq:effective-gdp}
\end{equation}
where $D_t$ denotes aggregate debt holdings:
\begin{equation}
    D_t \equiv - \int b_{i,t} \mathbb{1}_{[\phi_t \le b_{i,t} \le 0]} d \Psi_t (b,e), \label{eq:agg-debt}
\end{equation}
where $\mathbb{1}_{[\phi_t \le b_{i,t} \le 0]}$ is an indicator function for negative asset holdings.

%%%%%%%%%%%%%%%%%%%%%%%%%%%%%%%%%%%%%%%%%%%%%%%%%%%%%%%%%%%%%%%%%%%%%%%%%%%%%
%%%%%%%%%%%%%%%%%%%%%%%%%%%%%%%%%%%%%%%%%%%%%%%%%%%%%%%%%%%%%%%%%%%%%%%%%%%%%
\subsection{Solution Method and Calibration}
\label{sec:model-solution}

The \Gls{hank} model does not allow for an analytical solution. This subsection therefore briefly discusses the numerical solution approach I use and the calibration of model parameters I choose.

As is well-known to the literature, \Gls{ha} models are notoriously difficult to solve since general equilibrium and rational expectations require that agents are able to forecast market-clearing prices which depend on the sequence of distributions, $\{ \Psi_t (b,e) \}_{t=0}^{\infty} $, each element of which is an infinite-dimensional object. By now, a variety of solution methods for this challenge exists but choosing among methods still involves trade-offs. 

The present thesis makes use of the method laid out by \textcite{boehl2023econpizza} which builds on recent work by \textcite{auclert2021} and implements a perfect-foresight approach to find the non-linear transitions of \Gls{ha} models after one-time, unexpected ("MIT") shocks. To do so, the method makes use of a Newton root-finding procedure that searches in sequence space for aggregate outcomes, truncated at some distant terminal period, which are consistent with the household policies and the law of motion for the distribution, both of which are given on discretised grids of productivity states and asset holdings. The household policies are found with the endogenous-gridpoints method (\Gls{egm}) by \textcite{carroll2006}, which iterates on equations \eqref{eq:foc-euler} and \eqref{eq:foc-envelope} to find the households' consumption and savings policies.

On the one hand, the implementation by \textcite{boehl2023econpizza} offers advantages over competitor methods. It flexibly accommodates non-linearities in the model's aggregate state space and is typically very fast due to the use of automatic differentiation. On the other hand, meaningful uncertainty in aggregates -- as opposed to idiosyncratic uncertainty -- is disregarded. In particular, the method assumes that households know the exact trajectory of the economy once the shock to the model is announced. Nevertheless, accounting for non-linearities, such as the \Gls{zlb}, appears important for the present use case, so the method by \textcite{boehl2023econpizza} seems appropriate.

Its implementation is made available through the Python package \texttt{Econpizza}. For the sake of brevity, I delegate a discussion of some caveats regarding the use of the package in the present application to appendix \ref{sec-app:codes}.\footnote{The package is maintained by Dr.~Gregor Boehl on \href{https://github.com/gboehl/econpizza}{GitHub}, last retrieved on \printdate{2023-7-17}.}

Numerically solving a model implies that its parameters must be calibrated. Table \ref{tab:calibration-baseline} reports the baseline calibration together with the sources or targets I use.\footnote{Calibration is always to some degree arbitrary. Estimation of model parameters is preferred but beyond the scope of this thesis.} I choose $\beta$, the households' discount factor, such that the net real interest rate is $0.15\%$ in steady state. This arguably low target aims to capture the possibility of the \Gls{zlb} binding during the credit crunch, given the importance which the literature assigns to this mechanism (e.g.~\cite{egg2012}, \cite{gl2017}). Appendix \ref{sec-app:robust-no-zlb} explores a model version ignoring the \Gls{zlb}.

\begin{table}[t]
\centering
\caption{Calibration of the Baseline Model}
\label{tab:calibration-baseline}
\begin{tabular}{llcc}
Parameter &   &  Value & Source/Target \\
\hline
\hline
\multicolumn{2}{l}{\textbf{Households}} & & \\
$\beta$ &  discount factor & $0.99056$ & $RR_{ss}-1 = 0.15\%$ \\
$\sigma_c$ & coefficient of relative risk aversion & $2$ & \textcite{auclert2021} \\
$\sigma_l$ & inverse Frisch elasticity & $2$ & \textcite{auclert2021} \\
$\rho_e$ & persistence in productivity shocks & $0.966$ & \textcite{auclert2021} \\
$\sigma_e$ & variance of productivity shocks & $0.5$ & \textcite{auclert2021} \\
$N_e$ & number of productivity grid points & $8$ & Appendix \ref{sec-app:codes-ep} \\
$N_b$ & number of asset grid points & $219$ & Appendix \ref{sec-app:codes-ep} \\
& & & \\
\multicolumn{2}{l}{\textbf{Firms}} & & \\
$z$ &  technology & $1$ & $y=N$ \\
$\theta$ & elasticity of substitution & $6$ & \textcite{auclert2021} \\
$\psi$ & parameter on price adjustment costs & $100$ & Flat \Gls{nkpc} \\
$\omega$ & parameter of price ineratia & $0.44$ & \textcite{gust2017wp} \\
& & & \\
\multicolumn{2}{l}{\textbf{Fiscal Policy}} & & \\
$B$ & supply of government bonds & $5.21$ & $B/y_{ss} = 5.71$ \\
& & & (\cite{mckay2016}) \\
& & & \\
\multicolumn{2}{l}{\textbf{Monetary Policy}} & & \\
$\pi_{ss}$ & steady state gross inflation & $1$ & No steady-state net inflation \\
$\rho$ & persistence in notional interest rate & $0.8$ & Standard \\
$\phi_{\pi}$ & Taylor rule coefficient on inflation & $2$ & See text \\
$\phi_{y}$ & Taylor rule coefficient on output & $0.15$ & Common \\
& & & \\
\multicolumn{2}{l}{\textbf{Financial Constraints}} & & \\
$\phi_{ss}$ & initial borrowing limit & $-2.3485$ & $D_{ss}/y_{ss} = 57\%$ \\
& & & (\cite{bayer2023}) \\
$\kappa_{ss}$ & initial interest rate wedge & 1e-8 & No initial wedge \\
\hline
\multicolumn{4}{l}{\footnotesize \multirow{2}{15cm}{\justifying \textit{Note:} The table displays the calibration of the model parameters in the baseline version of the model. Note that all parameters jointly determine the steady states. All numbers refer to quarterly values. $p.p.$ stands for percentage points.}} \\
& & & \\
\end{tabular}
\end{table}

For the household utility function \eqref{eq:hh-utility-fun}, I use a standard value of $2$ for both, the coefficient of relative risk aversion and the inverse Frisch elasticity of labour supply. For the skills process \eqref{eq:hh-income-process}, I set $\rho_e = 0.95$ and $\sigma_e = 0.5$.\footnote{Consult appendix \ref{sec-app:codes-ep} on the details for discretising the skills process and the asset grid.} All of these parameters are as in \textcite{auclert2021}.

On the production side, I choose $\theta = 6$ as in \textcite{auclert2021} and $\psi = 100$ to achieve a rather flat \Gls{nkpc} with a slope of $0.06$, which seems in line with empirical estimates.\footnote{See e.g.~\textcite{schorfheide2008}. Recent cross-regional evidence from the U.S.~by \textcite{hazell2022} suggests an even flatter Phillips Curve.} Note that in this simple setup, the choice of $\theta$ pins down all of the following: $\mu_{ss}$, $mc_{ss}$, $w_{ss}$, $N_{ss}$. As long as $\kappa$ does not change, also consumption and output, $C_{ss}$ and $y_{ss}$, remain the same across steady states. This leaves the interest rate free to adjust to a permanent disturbance to the model.

I set $B$, the supply of real government bonds, along the lines of \textcite{mckay2016} and \textcite{gl2017} by matching it to the ratio of total liquid assets held by U.S.~households to U.S.~real gross domestic product (\Gls{gdp}), averaged over $1951$ -- $2023$.\footnote{See appendix \ref{sec-app:data} for details on the data sources and the required calculations.} Appendix \ref{sec-app:robust-bonds} shows the main results for a calibration of $B$ following the approach by \textcite{bayer2023}.

The parameters for monetary policy are mostly standard. I set steady-state gross inflation to $1.0$, implying no net inflation in steady state. I let $\rho = 0.8$, $\phi_y = 0.15$ and $\phi_{\pi} = 2$. The latter is arguably a strong reaction of monetary policy to inflation deviations, but lies within the range of values found in two recent estimations of fully-fledged \Gls{hank} models, \textcite{boehl2022} and \textcite{bayer2023}.

Finally, consider the choices for the financial constraints in the steady state of relatively "loose" credit. Akin to the approaches in \textcite{gl2017} and \textcite{bayer2023}, I select $\phi_{ss}$, the loose debt limit, so that the ratio of households' debt to output in the model is equal to the fraction of unsecured consumer credit over \Gls{gdp} in the U.S., averaged over $1951$ -- $2023$, which turns out to be $57\%$.\footnote{See again appendix \ref{sec-app:data}.} I set $\kappa_{ss} =$ 1e-8, so that there is virtually no premium on borrowing in the steady state of loose credit. % which turns out to be $57\%$.

%%%%%%%%%%%%%%%%%%%%%%%%%%%%%%%%%%%%%%%%%%%%%%%%%%%%%%%%%%%%%%%%%%%%%%%%%%%%%
%%%%%%%%%%%%%%%%%%%%%%%%%%%%%%%%%%%%%%%%%%%%%%%%%%%%%%%%%%%%%%%%%%%%%%%%%%%%%
%%%%%%%%%%%%%%%%%%%%%%%%%%%%%%%%%%%%%%%%%%%%%%%%%%%%%%%%%%%%%%%%%%%%%%%%%%%%%
\section{A Credit Crunch through the Borrowing Limit}
\label{sec:limit}

This section confronts the \Gls{hank} model with a credit crunch through a permanent tightening of the household-level borrowing limit. To that end, section \ref{sec:limit-stst-loose} begins by briefly characterising the steady state of relatively loose borrowing, i.e.~with $\phi_{ss}$ and $\kappa_{ss}$. Section \ref{sec:limit-stst-tight} defines a state with a tighter credit limit, denoted $\phi_{ss}'$, and compares it to the initial one. Thereafter, section \ref{sec:limit-transition} describes and explains the dynamics from easy to tight credit.

%%%%%%%%%%%%%%%%%%%%%%%%%%%%%%%%%%%%%%%%%%%%%%%%%%%%%%%%%%%%%%%%%%%%%%%%%%%%%
%%%%%%%%%%%%%%%%%%%%%%%%%%%%%%%%%%%%%%%%%%%%%%%%%%%%%%%%%%%%%%%%%%%%%%%%%%%%%
\subsection{A State of Loose Credit}
\label{sec:limit-stst-loose}

Figure \ref{fig:init-stst-dist-pol} displays key cross-sectional characteristics of the initial steady state with $\phi_{ss}$ and $\kappa_{ss}$. In particular, panel \ref{fig:init-stst-dist-pol-a-dist} shows the steady-state distribution of households over bond and \Gls{iou} holdings. This distribution features households with positive as well as negative asset positions. Agents situated to the left of $0$ are net debtors: they owe more in debt than they own in assets. These households have gone repeatedly through unfortunate realisations of labour skills and subsequently tap into debt to sustain consumption. In the present calibration, roughly $33.6\%$ of households are net borrowers, see table \ref{tab:stst_comparison_baseline_limit_permanent}. $4.89\%$ of the population end up directly at the borrowing limit.\footnote{Note that these fractions might not be close to the data (e.g.~compared to the $15\%$ for the borrowing fraction in the population reported in table 5 of \cite[p.~722]{kaplan2018}). However, the initial stationary distribution in the baseline model of \textcite{gl2017} features a comparable fraction of borrowers, about $33\%$. However, their model predicts only roughly one percent of the population to be directly at the initial constraint. These values are from own calculations based on the replication codes by \textcite{gl2017}, whose MATLAB codes are available here: \url{https://sites.northwestern.edu/guidolor/}, last retrieved \printdate{2023-6-28}.} These individuals might want to borrow more but are prohibited from doing so because of the borrowing limit.

\begin{figure}[t]
    \caption{Baseline Model: Initial Steady State}
    \label{fig:init-stst-dist-pol}
    \centering
    \begin{subfigure}[b]{0.49\textwidth}
    \caption{Asset Distribution}
    \label{fig:init-stst-dist-pol-a-dist}
         \centering
         \includesvg[inkscapelatex=false, width=\textwidth]{Results/stst_dist_2d_bar_baseline_limit_permanent_initial.svg}
     \end{subfigure}
     \hfill
     \begin{subfigure}[b]{0.49\textwidth}
     \caption{Consumption Policies}
     \label{fig:init-stst-dist-pol-c}
         \centering
         \includesvg[inkscapelatex=false, width=\textwidth]{Results/stst_policies_c_baseline_limit_permanent_initial.svg}
     \end{subfigure}

    \vspace{10pt}
     
     \justifying
     \footnotesize
	\textit{Note}: In figure \ref{fig:init-stst-dist-pol-a-dist}, the distribution is cut off at an asset level of $50$ for readability. As the model is solved on a logarithmic grid, grid points are distributed unevenly. Grid points are very dense at the lower end of the grid, which makes the colour appear bleak. At high levels of assets, grid points are very sparse, overestimating the true density at high bond holdings. The size of the bar at the borrowing limit is exaggerated. In figure \ref{fig:init-stst-dist-pol-c}, the labels $\theta\_0$ -- $\theta\_7$ represent the idiosyncratic labour skill states from \eqref{eq:hh-income-process}.
\end{figure}

\begin{table}[t]
\centering
\caption{Baseline Model -- Shock to the Borrowing Limit: Comparison of Steady States}
\label{tab:stst_comparison_baseline_limit_permanent}
\begin{tabular}{lccc}
Variable & Initial & Terminal &  Change \\
\hline
\hline
\multicolumn{2}{l}{\textbf{Borrowing Limit}} & & \\
$\phi$ & $-2.3485$ & $ -2.1775$ &  $0.1710$ \\
& & & \\
\multicolumn{2}{l}{\textbf{Aggregates}} & & \\
$C$ &  $0.9129$ &   $0.9129$ &  -- \\
$D / y$ &    $57\%$ & $52\%$ & $-5 \ p.p.$ \\
$B + D$ &  $5.7304$ &   $5.6847$ & $-0.78\%$ \\
$RR - 1$ &  $0.15\%$ &  $0.13\%$ & $-0.02 \ p.p.$ \\
$\tau$ &  $0.0078$ & $0.0069$ & $-0.0009$  \\
& & & \\
\multicolumn{2}{l}{\textbf{Cross-Section}} & & \\
                  Share of Assets Held by Top-$10\%$ &  $0.4834$ &   $0.4774$ & $-0.006 \ p.p.$ \\
                  Share of Consumption by Top-$10\%$ &  $0.1729$ &   $0.1731$ & $0.0002 \ p.p.$ \\
Fraction of Borrowers & $33.66$ & $32.9$ & $-0.76 \ p.p.$ \\
Fraction at Borrowing Limit & $ 4.89$ & $5.07$ &  $0.18 \ p.p.$ \\
\Gls{mpc} of Borrowers &    $0.16$ &     $0.17$ &   \\
\Gls{mpc} of Lenders &    $0.02$ &     $0.02$ & \\
\hline
\multicolumn{4}{l}{\footnotesize \multirow{2}{12cm}{\justifying \textit{Note:} The table contains selected values of the initial steady state with $\phi_{ss}$ and the terminal steady state with $\phi_{ss}'$. All numbers are rounded and refer to quarterly values. $p.p.$ stands for percentage points.}} \\
& & & \\
\end{tabular}
\end{table}

The rest of the population holds assets in excess of its liabilities. Net lenders have experienced a series of positive productivity realisations and build positive asset positions in order to smooth consumption over time and to save in precaution of potentially worse skill endowments. Some of these households invest their funds in bonds, others in the \Gls{iou}s originated by borrowers. Gross liquidity, $B+D$, equals $6.3$ times quarterly output.

As section \ref{sec:model-hh} explains, balance sheets are expected to have implications for consumption behaviour. Figure \ref{fig:init-stst-dist-pol-c} shows that this is the case in the present setup. It plots the household consumption policies over asset holdings, separately for each productivity state. Naturally, lower asset holdings and/or lower productivity states translate into lower consumption. Importantly however, the curvature of the consumption policies also depends on the asset level. For very low levels of wealth and high debt burdens, the consumption policies are concave and much steeper than for higher asset levels. This confirms that consumption of households at or near the borrowing limit strongly responds to changes in the respective households' asset positions and that consequently, these individuals have high \Gls{mpc}s.

Indeed, table \ref{tab:stst_comparison_baseline_limit_permanent} shows that, in the initial steady state, the average borrower has an \Gls{mpc} of $0.16$, while the average lender features an \Gls{mpc} of merely $0.02$.\footnote{The aggregate \Gls{mpc} equals $0.0663$, which is quite low when compared to empirical estimates, see \textcite{kaplan2018}. One reason for this is that the model disregards the wealthy hand-to-mouth of \textcite{kaplan2014}.} Thus, an average borrower would increase consumption by $14$ cents more if given one dollar more in assets (i.e.~less in debt) than the average lender would. This distinction still masks substantial heterogeneity in the \Gls{mpc}s among borrowers, see figure \ref{fig:baseline-limit-initial-mpcs} in appendix \ref{sec-app:figures}. The \Gls{mpc} of borrowers at the debt limit can equal one, while borrowers with only little debt have \Gls{mpc}s comparable to those of lenders. Overall, this suggests that debtors -- especially those at or very close to $\phi_{ss}$ -- are much more sensitive to changes in their asset positions and incomes, as alluded to in section \ref{sec:model-hh}.\footnote{The fact that \Gls{mpc}s are heterogeneous is empirically plausible (see e.g.~\cite{gross2002}), but their magnitudes might not be exactly matched by the simple model.} %This observation proves crucial in the subsequent analysis.

The aggregate dimension of the steady state shows that the real and nominal interest rates are $0.15\%$ and that debt over output equals $57\%$, as calibrated. Aggregate wage and labour hours imply an average quarterly labour income of $0.76$, so the initial debt limit allows households to borrow up to $3.1$ times their average quarterly labour income. 

%%%%%%%%%%%%%%%%%%%%%%%%%%%%%%%%%%%%%%%%%%%%%%%%%%%%%%%%%%%%%%%%%%%%%%%%%%%%%
%%%%%%%%%%%%%%%%%%%%%%%%%%%%%%%%%%%%%%%%%%%%%%%%%%%%%%%%%%%%%%%%%%%%%%%%%%%%%
\subsection{Loose vs Tight Credit: Comparing Steady States}
\label{sec:limit-stst-tight}

Next, I specify a new steady state of the economy, in which credit is tighter due to a reduced borrowing limit. To that end, I choose a borrowing limit, $\phi_{ss}' < \phi_{ss}$, such that in this terminal steady state the household debt-to-output ratio is five percentage points lower than it is in the initial state, i.e.~$\frac{D_{ss}'}{y_{ss}'} = 52\%$. Taking into account the observed developments in the fraction of unsecured credit to \Gls{gdp} over the last decades (see figure \ref{fig:app-calibration-debt} in appendix \ref{sec-app:data}), a change of five percentage points in this ratio appears to be an empirically realistic magnitude.\footnote{\textcite{gl2017} consider a reduction by ten percentage points of the ratio of debt to \textit{annual} \Gls{gdp}. This translates into a drop of $40$ percentage points in quarterly terms, i.e.~a much larger change than I study.} 

How does the steady state with a confined debt limit look like and how does it compare to the initial steady state? What forces and induces households to behave differently in an environment of tight credit relative to one of loose credit? To compare the steady state with $\phi_{ss}'$ to the initial one, consider figures \ref{fig:comparison-stst-baseline-policies} and \ref{fig:comparison-stst-baseline-dist} in appendix \ref{sec-app:figures-limit-tight}, which are analogous to the plots in figure IV in \textcite[p.1443]{gl2017}, as well as table \ref{tab:stst_comparison_baseline_limit_permanent}. 

\begin{figure}[t]
    \caption{Baseline Model -- Shock to the Borrowing Limit: Household Policies}
    \label{fig:comparison-stst-baseline-policies}
    \centering
    \begin{subfigure}[b]{0.49\textwidth}
    \caption{Average Asset Policy}
    \label{fig:comparison-stst-baseline-policies-a}
         \centering
         \includesvg[inkscapelatex=false, width=\textwidth]{Results/stst_a_baseline_limit_permanent.svg}
     \vspace{0.01cm}
     \end{subfigure}
     \hfill
     \begin{subfigure}[b]{0.49\textwidth}
     \caption{Average Consumption Policy}
     \label{fig:comparison-stst-baseline-policies-c}
         \centering
         \includesvg[inkscapelatex=false, width=\textwidth]{Results/stst_c_baseline_limit_permanent.svg}
    \vspace{0.01cm}
     \end{subfigure}
     \justifying
     \footnotesize
	\textit{Note}: The policies are cut off at an asset level of $30$ for readability. In both figures, the differences are a bit difficult to discern. A close look reveals the aspects discussed in the text.
\end{figure}

Figure \ref{fig:comparison-stst-baseline-dist} shows that the stationary distributions of agents over their asset positions look very similar across steady states. Trivially, no mass of agents is admissible between the old and the new borrowing limit in the new steady state, so the density there is zero. The distribution seems to have been moved to the right. However, as in \textcite{gl2017}, the tighter borrowing limit actually compresses the distribution rather than pushing it to the right. On the one hand, at the remaining admissible levels of debt and at low positive asset holdings, the density of the terminal state lies above the initial density. On the other hand, less agents are located at high bond holdings in an environment of tight credit relative to before.

Figure \ref{fig:comparison-stst-baseline-policies-a} sheds light on why this compression takes place by contrasting the average asset accumulation in each steady state.\footnote{Asset accumulation is $b_{i,t} - b_{i,t-1}$, i.e.~the asset policy less the $45$ degree line. The average is over the productivity levels induced by \eqref{eq:hh-income-process}.} Compared to the initial steady state, the average household reduces debt faster when being a borrower and draws down positive asset holdings faster when being a lender. Why do agents behave in this new way? Consider first the asset-poor households. As the borrowing limit has tightened, the danger of reaching this new limit is especially elevated for indebted households and those with little wealth. To reduce the possibility of approaching the debt limit, these agents forgo consumption (see the low-wealth portion of figure \ref{fig:comparison-stst-baseline-policies-c}) and aim at reducing borrowing faster whenever given the chance to do so. Thus, on average, asset-poor individuals pay back debt in a higher pace than in the initial steady state because they want to maintain a safe distance to the credit limit.

All else equal, asset-rich households would likewise aim for higher savings due to the debt limit being a more imminent threat for them as well. These households nevertheless decumulate assets faster rather than doing the opposite as the real interest rate is permanently lower in the terminal steady state than in the initial one (see table \ref{tab:stst_comparison_baseline_limit_permanent}). As these agents are interest-rate sensitive -- recall from subsection \ref{sec:model-hh-constraints} that the asset-rich behave akin to permanent-income consumers -- the lower real interest rate induces them to substitute away from savings towards consumption (see the right end of figure \ref{fig:comparison-stst-baseline-policies-c}). In total, in an environment of tight credit, asset-poor individuals consume less to keep their distance to the credit limit, while wealthy households reduce asset holdings in order to consume more due to a lower real rate. Incidentally, these developments result in reduced long-term wealth inequality but elevated consumption inequality as captured by the shares that the top-$10\%$ save and consume, respectively, see table \ref{tab:stst_comparison_baseline_limit_permanent}.

These observations raise the question as to why the real interest rate is reduced in the terminal steady state. Crucially, this drop is a consequence of two adjustments in the asset market: a reduction in the supply of liquidity and a simultaneous increase in the demand for it. The former is the result of what I call the \textit{balance-sheet effect}: the deleveraging forced onto the households between the initial and the terminal borrowing limits prohibits these agents to hold their initial levels of debt, thus preventing them to issue as many \Gls{iou}s as they did before. Crucially, this mechanical restriction in the quantity of \Gls{iou}s out of debtors' balance sheets reduces the supply of liquid assets in the economy, see the change of $B + D$ in table \ref{tab:stst_comparison_baseline_limit_permanent}. The balance-sheet effect is present in the models of both, \textcite{egg2012} and \textcite{gl2017}. 

However, as alluded to above, a tighter borrowing limit increases all agents' desire for precautionary savings. This is the second development in the asset market. The heightened fear of reaching the borrowing limit thus boosts asset demand from the unconstrained households. This is the \textit{precautionary-savings effect} of a credit crunch. It is muted in \textcite{egg2012} but features prominently in \Gls{ha} models such as the one here and in \textcite{gl2017}. 

Taken together, the curtailed supply of and the heightened demand for liquidity call for a lower real interest rate to equilibrate the asset market. In particular, the reduction in the real rate depresses the desire of borrowers to shorten their balance sheets' liability sides, thus stabilising the supply of \Gls{iou}s, and prompts lenders to demand less assets as they substitute tomorrow's consumption for today's consumption. By consequently dampening the combined balance-sheet and precautionary-savings effects, the lower real rate ensures equilibrium. This is the crucial change occurring in an economy which faces tight credit and adjusts through the interest rate.

Note that, as expected, aggregate consumption is identical across steady states since, in this setup, it is pinned down by the supply side of the model. Lower interest rates (and taxes) enable consumption to remain unchanged, even in an environment of tight credit.\footnote{Hence, the model can not speak to considerations about the long-term connection of credit availability and the level of aggregate consumption.}

%%%%%%%%%%%%%%%%%%%%%%%%%%%%%%%%%%%%%%%%%%%%%%%%%%%%%%%%%%%%%%%%%%%%%%%%%%%%%
%%%%%%%%%%%%%%%%%%%%%%%%%%%%%%%%%%%%%%%%%%%%%%%%%%%%%%%%%%%%%%%%%%%%%%%%%%%%%
\subsection{Transitioning to Tight Credit}
\label{sec:limit-transition}

How does the economy evolve during the transition from the steady state of subsection \ref{sec:limit-stst-loose} to the one described in subsection \ref{sec:limit-stst-tight}? That is, how does the shock to the borrowing limit work through the households' balance sheets over time and then create, in general equilibrium, distributional and macroeconomic dynamics? To answer these questions, section \ref{sec:limit-transition-specific} first defines the transition path of the borrowing limit. Section \ref{sec:limit-transition-summ} briefly summarises the impulse responses of the aggregate variables and section \ref{sec:limit-transition-mechanisms} explains the mechanisms underlying these responses, while taking into account the cross-sectional dimension of the credit crunch. 

%%%%%%%%%%%%%%%%%%%%%%%%%%%%%%%%%%%%%%%%%%%%%%%%%%%%%%%%%%%%%%%%%%%%%%%%%%%%%
\subsubsection{Specifying the Transition}
\label{sec:limit-transition-specific}

Let the economy be in the initial steady state with $\phi_{ss}$ in $t=0$. Latest by $t=200$, the economy settles in the terminal steady state with $\phi_{ss}'$. Inbetween the two endpoints, the shock arises unexpectedly in $t=1$ and $\phi_t$ evolves deterministically thereafter according to an \Gls{ar}(1) process:
\begin{equation}
    \phi_t = \phi_{ss} \Bigg( \frac{\phi_{t-1}}{\phi_{ss}} \Bigg)^{\rho_{\phi}} exp(\varepsilon_{\phi}), \label{eq:borrowing-limit-process}
\end{equation}
where $\rho_{\phi}$ is the persistence of changes in the borrowing limit. $\varepsilon_{\phi}$ is a one-time shock to the borrowing limit but in this thesis I only consider permanent shocks and thus $\varepsilon_{\phi} = 0$ $\forall t$.

I follow \textcite{gl2017} in choosing a duration of the credit crunch of six quarters, which requires $\rho_{\phi} = 0.5$. Doing so is meant to capture the notion that a financial shock might not force households to delever fully within one period but might take some time to manifest itself in household balance sheets. Section \ref{sec:sensitivity-persistence} explores alternative choices for $\rho_{\phi}$.

%%%%%%%%%%%%%%%%%%%%%%%%%%%%%%%%%%%%%%%%%%%%%%%%%%%%%%%%%%%%%%%%%%%%%%%%%%%%%
\subsubsection{Summary of Aggregate Responses}
\label{sec:limit-transition-summ}

Figure \ref{fig:baseline-permanent-limit-agg} presents the dynamics of some key aggregate variables. For the sake of brevity, I refer the reader to figure \ref{fig:baseline-permanent-limit-agg-extra} in appendix \ref{sec-app:figures-limit-transition} for some supplementary responses. First, plot \ref{fig:baseline-permanent-limit-agg-phi} shows the evolution of the borrowing limit from $\phi_{ss} = -2.3485$ to $\phi_{ss}' = -2.1775$. In $t=1$, when the shock is announced, the debt limit contracts by almost half of its total change. Thereafter, $\phi_t$ completes most of its transition within one and a half years, as calibrated.

\begin{figure}[!ht]
    \centering
    \caption{Baseline Model -- Shock to the Borrowing Limit: Aggregate Dynamics}
    \label{fig:baseline-permanent-limit-agg}
     \begin{subfigure}[b]{0.49\textwidth}
     \caption{Borrowing Limit, $\phi$}
     \label{fig:baseline-permanent-limit-agg-phi}
         \centering
         \includesvg[inkscapelatex=false, width=\textwidth]{Results/transition_phi_baseline_limit_permanent.svg}
     \end{subfigure}
     \hfill
    \begin{subfigure}[b]{0.49\textwidth}
    \caption{Consumption, $C$}
         \centering
         \includesvg[inkscapelatex=false, width=\textwidth]{Results/transition_C_baseline_limit_permanent.svg}
     \end{subfigure}
    \hfill
    \begin{subfigure}[b]{0.49\textwidth}
    \caption{Debt, $D$}
    \label{fig:baseline-permanent-limit-agg-d}
         \centering
         \includesvg[inkscapelatex=false, width=\textwidth]{Results/transition_D_baseline_limit_permanent.svg}
     \end{subfigure}
     \hfill
     \begin{subfigure}[b]{0.49\textwidth}
     \caption{Debt-to-Output, $\frac{D}{y}$}
         \centering
         \includesvg[inkscapelatex=false, width=\textwidth]{Results/transition_DY_baseline_limit_permanent.svg}
     \end{subfigure}
     \hfill
     \begin{subfigure}[b]{0.49\textwidth}
     \caption{\textit{Ex-Post} Real Interest Rate, $RR$}
     \label{fig:baseline-permanent-limit-agg-RR}
         \centering
         \includesvg[inkscapelatex=false, width=\textwidth]{Results/transition_Rr_baseline_limit_permanent.svg}
    %\vspace{0.01cm}
     \end{subfigure}
     \hfill
     \begin{subfigure}[b]{0.49\textwidth}
     \caption{Nominal and Notional Interest Rates, $R$ and $R^n$}
     \label{fig:baseline-permanent-limit-agg-RN}
         \centering
         \includesvg[inkscapelatex=false, width=\textwidth]{Results/transition_R_Rn_baseline_limit_permanent.svg}
    %\vspace{0.01cm}
     \end{subfigure}

     \vspace{10pt}
     
     \justifying
     \footnotesize
	\textit{Note}: Wherever the terminal steady state of a variable differs from the initial steady state, the red dotted line displays the former. Percent deviations mean percent deviations from the respective initial steady state. The time horizon in all plots is fixed to 12 quarters, including the initial steady state in $t=0$.
\end{figure}

The figures reveal that, on announcement of the shock, the economy enters a recession. While aggregate debt holdings fall only slightly, consumption drops strongly by roughly three percent. These different speeds of downward adjustment imply that debt over output increases by $1.5$ percentage points.\footnote{The results in \textcite{gl2017} do not capture such an increase. However, \textcite{justiniano2015} argue that an account of the Great Recession has to be consistent with the observation that debt as a fraction of \Gls{gdp} surged in the wake of the recession.} The \textit{ex-post} real interest rate also spikes on impact of the shock. The monetary authority would like to move the nominal interest rate into negative territory but the \Gls{zlb} starts binding. Figure \ref{fig:baseline-permanent-limit-agg-extra} further shows that total labour hours, wages, marginal costs and inflation fall. Taxes as well as dividends surge on impact.

After the initial shock, the debt limit continues to tighten in a slower pace and the economy rebounds. Consumption exceeds its steady-state value for a short period of time, before reaching its original level six quarters after the shock started. Labour hours, wages, marginal costs, inflation and dividends follow along. The real and nominal interest rates undershoot their lower, terminal values and show quite some persistence on their path to those. Outstanding debt takes an exceptionally prolonged period of time -- roughly $15$ years -- to reach its terminal state. 

%%%%%%%%%%%%%%%%%%%%%%%%%%%%%%%%%%%%%%%%%%%%%%%%%%%%%%%%%%%%%%%%%%%%%%%%%%%%%
\subsubsection{Mechanisms}
\label{sec:limit-transition-mechanisms}

This subsection probes into the mechanisms giving rise to the observed responses on impact of the shock and afterwards and answers the following questions: why does aggregate consumption fall? What explains the evolution of the interest rates? When and how does the economy return to steady state?

Consider first the economy's instantaneous response to the credit crunch. On impact, the agents get to know the entire path of the borrowing limit and adjust optimally. Of course, the balance-sheet and precautionary-savings effects described in section \ref{sec:limit-stst-tight} are present in the short-run as well: the tightening in the borrowing limit forces borrowers with net debt holdings between $\phi_{ss}$ and $\phi_1$ to delever. Mechanically, this gives rise to the balance-sheet effect by which the affected borrowers cut back on consumption to consolidate their balance sheets, resulting in a contracting supply of \Gls{iou}s. 

As the shock brings the credit limit closer to all individuals, the precautionary-savings effect prompts even unconstrained borrowers to reduce debt, inducing a further drop in the supply of \Gls{iou}s, and unconstrained lenders to hold more safe assets. To achieve these higher wealth targets, both groups of agents are willing to forgo consumption. Importantly, recall from section \ref{sec:limit-stst-loose} that balance sheets influence consumption behaviour in the form of heterogeneity in \Gls{mpc}s and so, the degree by which agents reduce expenditures will vary along agents' asset positions. Individuals closer to the borrowing limit are expected to curb consumption by much more than agents far away from the constraint.

As documented by \textcite{gl2017}, the balance-sheet and the precautionary-savings effects taken together thus provoke a strong partial-equilibrium reduction in aggregate consumption. Of course, in general equilibrium, there are amplifying as well as off-setting forces. Regarding the former, note that a dwindling goods demand reduces production, labour demand by firms and thus wages. Hence, labour incomes fall for all households fall but they especially hurt the low-skill agents with inadequate insurance against such shortfalls. As these agents feature high \Gls{mpc}s, they strongly reduce their consumption, amplifying the downturn.

Further, depressed wages lead to falling marginal costs and, through the \Gls{nkpc}, to deflation as firms try to maintain demand by adjusting prices downward. Through the Fisher equation \eqref{eq:hh-fisher-equation}, deflation in $t=1$ induces the \textit{ex-post} real interest rate on initial asset positions to increase sharply, see plot \ref{fig:baseline-permanent-limit-agg-RR}. An unexpectedly high real rate redistributes income from high-\Gls{mpc} borrowers to low-\Gls{mpc} lenders as the former face higher interest payments on last period's debt holdings, which translate into capital gains for the latter group of agents.\footnote{Deflation here strains borrowers' budgets through increased interest payments on past debt holdings and thus amplifies their financial trouble. However, this is not a debt-deflation spiral à la \textcite{fisher1933} as the real debt burden does not increase with deflation.} Further, to meet its obligations, the fiscal authority raises the tax burden on households. These adjustments contribute to the pressure on households' budgets and consumption levels.

Consider now the dampening forces in general equilibrium. One of those is that firms' dividends to households increase during the recession.\footnote{This is a typical result in a textbook \Gls{nk} environment with flexible wages, as the latter imply that marginal costs fall faster than the rigid prices do, giving rise to higher margins.} However, the most adjustment is the one of the real interest rate, the price intermediating between the asset and goods markets. As the credit crunch induces a state of excess demand for bonds and \Gls{iou}s or, equivalently, a surplus of demand for consumption tomorrow relative to demand for consumption today,\footnote{This makes the shock to the borrowing limit resemble a discount-factor shock, see \textcite{kv2018}.} the real interest rate must fall from its unexpected surge on impact.

For the consumption-savings decisions of households in $t=1$, the real interest rate in $t=2$ is the key determinant. In turn, the real rate is pinned down by the developments of the nominal interest rate and of inflation. Upon announcement of the shock, the monetary authority adjusts immediately its target rate to $R_{ss}' - 1 = 0.13\%$. Concurrently, the pronounced downward deviations of inflation and output from their steady-state levels call for a lower nominal interest rate. The Taylor rule prescribes a rate of roughly $-0.3$ percent \textit{per annum}. However, the nominal rate can not go below zero and the \Gls{zlb} starts binding, see figure \ref{fig:baseline-permanent-limit-agg-RN}. Still, this induces a downward adjustment in the real rate in $t=2$. Crucially, the lower interest rate dampens the lenders' increased desire for precautionary buffers in $t=1$ and stimulates their expenditures as they substitute some of their increased asset demand for consumption today. This helps reduce the excess-demand pressure in the asset market. 

Of course, the lower interest rate also applies to the liability side of balance sheets, i.e.~debt becomes relatively cheaper. Debtors' decisions on how much debt to hold are thus influenced by counteracting forces: dampened rates encourage borrowing but ever closer limits discourage it. Plot \ref{fig:baseline-permanent-limit-agg-d} shows a mild decline of aggregate debt on impact in the present calibration, suggesting that the latter effect dominates the former only by a little.\footnote{Theoretically, an increase of total debt is possible as the shock to the borrowing limit affects the maximum quantity of debt a household can hold and not the amount of indebtedness in the economy as a whole. Some calibrations, not shown, indeed produce an increase on impact.} That is, to shield their consumption to some degree, unconstrained borrowers make up for at least some of the mechanical reduction in debt. 

This is in itself a noteworthy result: despite the fact that the borrowing limit tightens the most in $t=1$, total debt does not decline much and despite this, the economy enters recession. This hints at two mechanisms. First, while debt between $\phi_{ss}$ and $\phi_1$ is prohibited, an increased mass of agents chooses high levels of borrowing in the admissible range of indebtedness. Second, one might infer that the present model economy features strong internal amplification: a financial shock might not lead to quick deleveraging but still trigger a severe consumption reduction. Presumably, this is because the shock hits exactly those agents with the highest \Gls{mpc}s in the economy.

Note two important aspects concerning the dampening force of the real interest rate. First, the \Gls{zlb} clearly prevents a more forceful reduction of the real rate. This reinforces the reduction in consumption as interest-sensitive individuals are not incentivised sufficiently to expand consumption. However, in the present setup and unlike in the model of \textcite{egg2012}, the \Gls{zlb} is indeed only an amplifier and not the main mechanism behind the drop in aggregate consumption, see the discussion in appendix \ref{sec-app:robust-no-zlb}. Second, recall that the households who are most sensitive to changes in the interest rate are the relatively wealthy. At the same time, these agents feature low \Gls{mpc}s, implying that, even amid falling rates, they might not be willing to substitute away from their increased savings demand sufficiently and might consequently not aid much in the dampening of the recession. 

Indeed, the above considerations raise the question of how the credit crunch impacts agents differently across the asset distribution. To that end, consider figure \ref{fig:baseline-permanent-limit-dist-c-response}, which plots the consumption responses separately for the bottom $25\%$, bottom $50\%$ and top $25\%$ in the distribution. Consistent with the findings in \textcite{gl2017}, the responses are heterogeneous: the recession hits the bottom $25\%$ the hardest, while the consumption losses for the top $25\%$ are less than those for the aggregate.\footnote{The figure also hints at the fact that consumption inequality is higher in the long run, see section \ref{sec:limit-stst-tight}.} 

\begin{figure}[t]
    \centering
    \caption{Consumption Response by Selected Percentiles}
    \label{fig:baseline-permanent-limit-dist-c-response}
    \includesvg[inkscapelatex=false, width=0.75\textwidth]{Results/percentile_transitions_baseline_limit_permanent_C.svg}

    \vspace{10pt}
    
    \begin{minipage}{0.75\textwidth} 
    \footnotesize
    \textit{Note}: The responses are calculated as the product of aggregate consumption in period $t$ and the share that the respective household percentile has in consumption in $t$. Percent deviations mean percent deviations from the respective initial steady state. The time horizon in all plots is fixed to 12 quarters, including the initial steady state in $t=0$.
    \end{minipage}
\end{figure}

How does the economy evolve after the initial shock? The debt limit keeps tightening over the next five quarters, albeit at an ever slower pace. Consistent with the findings of \textcite{gl2017}, the real interest rate continues its downward adjustment and undershoots its terminal value before approaching it from below. The intuition for the undershooting is as follows: as the asset distribution adjusts only slowly towards the terminal density, it features, during the transition, still too many debtors to the left of the final debt limit in comparison to the terminal steady state.\footnote{The code file \texttt{main.py} visualises the transition of the asset distribution over time. I omit it here for the sake of brevity.} These agents are subject to the balance-sheet effect as they must mechanically delever. An undershooting real interest rate guarantees equilibrium amid the pressure from the remaining agents reducing their net debt holdings.

The anticipation of lower real rates stimulates consumption for the unconstrained agents and induces the economy to rebound. As demand picks up, so does inflation, which in turn amplifies the recovery by fuelling the downward adjustment of the real rate. Indeed, inflation pushes the real rate even into negative territory for two quarters. This instigates a small and short-lived boom in consumption, which provokes a counter-reaction of the central bank in the form of higher nominal interest rates and finally a return of the economy to the steady state.

Interestingly, aggregate debt takes a very long time to attain its terminal value. As mentioned above, debt falls only very slightly in $t=1$ as the economy is in recession. By the time the borrowing limit completes its transition in $t=6$, the debt-to-\Gls{gdp} ratio has merely reached $55.5\%$ and the rest of the transition takes until $15$ years after the shock arose, see figure \ref{fig:baseline-permanent-limit-longrun} in the appendix. As the economy rebounds, debt falls in a faster pace since agents use their increasing incomes to pay back their debt. As incomes settle, households smooth the amount of debt repayment over a long horizon. I repeatedly come back to the issue of slow balance sheet adjustment further below.

In summary, a shock to the debt limit triggers a mechanical deleveraging, which reduces the supply of liquidity, and an enforced motive for precautionary savings, which heightens demand for liquidity. The strong drop in consumption initiated by these forces is amplified by various mechanisms, e.g.~unexpected inflation and heterogeneity in \Gls{mpc}s. To maintain equilibrium, the real interest rate falls from its initial peak and undershoots its terminal value as the asset distribution slowly adjusts towards the steady state of tight credit.

%As can be inferred from the sluggish response of aggregate debt in plot \ref{fig:baseline-permanent-limit-agg-d}, in the present calibration, the latter effect seems to outweigh the former by only a little in the first period of the economy's transition.
% This suggests that debtors' decisions on how much debt to hold are influenced by counteracting forces. On the one hand, dampened rates stimulate borrowing, a helpful device to shield consumption from falling incomes -- at least for those unaffected by the mechanical tightening in the credit limit. On the other hand, borrowers want to reduce their debt due to the fear of reaching the shrunken limit.

%Overall, I confirm the general mechanisms identified by \textcite{gl2017}. As both, the model here and the one in \textcite{gl2017} lack capital and frictions such as investment adjustment costs and habit formation, the adjustment in consumption is not very persistent. 

%As pointed out by \textcite{egg2012}, the deflation caused on impact of the shock leaves room for future inflation. The fact that agents (correctly) anticipate inflation to rebound from $t=2$ onward aids the downward adjustment of the \textit{ex-post} real rate in $t=2$.

%In a model where only the \textit{balance-sheet effect} is present, as in the two-agent framework of \textcite{egg2012}, the real rate for $t=2$ could -- if it was in sufficiently large distance from the \Gls{zlb} -- drop enough to avert a recession upon emergence of a deleveraging shock in $t=1$ (see \cite{riosrull2015}). In contrast, as in \textcite{gl2017}, the household heterogeneity of the present \Gls{hank} model allows for the \textit{precautionary-savings effect} to arise. As explained, this significantly contributes to the downward pressure on consumption demand provoked by the \textit{balance-sheet effect}, rendering it less likely that the real rate can drop sufficiently to ensure that aggregate consumption does not fall. 

%Note also that the response of aggregate consumption is not very persistent and does not last much longer than the shock itself. Note however that the responses in \textcite{gl2017} are also not very persistent. Their model also lacks capital, habit formation or investment adjustment costs. Interestingly however, the response of household debt shows strong persistence.  

%As \textcite{gl2017} note, the consumption response will depend on the aggregate MPC. If the MPC is large, then the consumption response will be large. %But crucially, it might also depend on how \Gls{mpc}s are distributed, no???

%%%%%%%%%%%%%%%%%%%%%%%%%%%%%%%%%%%%%%%%%%%%%%%%%%%%%%%%%%%%%%%%%%%%%%%%%%%%%
%%%%%%%%%%%%%%%%%%%%%%%%%%%%%%%%%%%%%%%%%%%%%%%%%%%%%%%%%%%%%%%%%%%%%%%%%%%%%
%%%%%%%%%%%%%%%%%%%%%%%%%%%%%%%%%%%%%%%%%%%%%%%%%%%%%%%%%%%%%%%%%%%%%%%%%%%%%
\section{Sensitivity Analyses}
\label{sec:sensitivity}

This section complements the previous section and explores how two choices made in the preceding analysis influence its
results. First, subsection \ref{sec:sensitivity-persistence} varies the persistence of the shock process \eqref{eq:borrowing-limit-process} in order to show how the economy's responses differ across varying speeds of the transition to tight credit. Second, subsection \ref{sec:sensitivity-end-labour} adopts a specification for $u( \cdot )$ in \eqref{eq:hh-objective} which allows households to self-insure by choosing their own working hours to investigate whether this might be an important insurance mechanism for borrowers.

%%%%%%%%%%%%%%%%%%%%%%%%%%%%%%%%%%%%%%%%%%%%%%%%%%%%%%%%%%%%%%%%%%%%%%%%%%%%%
%%%%%%%%%%%%%%%%%%%%%%%%%%%%%%%%%%%%%%%%%%%%%%%%%%%%%%%%%%%%%%%%%%%%%%%%%%%%%
\subsection{Duration of the Credit Crunch}
\label{sec:sensitivity-persistence}

This section considers different speeds with which the economy adjusts to the new state of tight credit. Recall that the duration of six quarters chosen in section \ref{sec:limit} were motivated by the observation that, in reality, deleveraging shocks might take some time to take full effect. One reason for this might be that many assets in household balance sheets have a maturity of over one quarter. The analysis here upholds this assumption but recognises that, in reality, shocks can unfold differently. Indeed, I show that the shock duration does not affect the overall appearances of the responses but does influence the quantitative results in intuitive ways.

%the sense that quicker adjustments are more hurtful.

I compare the baseline scenario from section \ref{sec:limit-transition} to two alternatives: first, the shock resolves in ten quarters and second, the shock lasts for only four quarters. The former experiment requires setting the persistence of \eqref{eq:borrowing-limit-process}, $\rho_{\phi}$, equal to $0.62$ ("slow" shock), while the latter implies $\rho_{\phi} = 0.3$ ("fast" shock).\footnote{Note that this affects the asset grid on which the model is solved. This is due to the computational challenges associated with shocks to the borrowing limit, see appendix \ref{sec-app:codes-ep}. It is not expected that this influences the results.} Figure \ref{fig:baseline-permanent-limit-duration} contrasts the responses of aggregate consumption and debt across persistence parameters. Appendix figure \ref{fig:baseline-permanent-limit-duration-extra} contains some supplementary responses.

\begin{figure}[t]
    \centering
    \caption{Baseline Model -- Varying Durations of a Credit Crunch}
    \label{fig:baseline-permanent-limit-duration}
     \begin{subfigure}[b]{0.49\textwidth}
     \caption{Consumption, $C$}
     \label{fig:baseline-permanent-limit-duration-C}
         \centering
         \includesvg[inkscapelatex=false, width=\textwidth]{Results/compare_transitions/comparison_C_fast_shock_limit_permanent_baseline_limit_permanent_slow_shock_limit_permanent.svg}
     \end{subfigure}
     \hfill
     \begin{subfigure}[b]{0.49\textwidth}
     \caption{Debt, $D$}
     \label{fig:baseline-permanent-limit-duration-D}
         \centering
         \includesvg[inkscapelatex=false, width=\textwidth]{Results/compare_transitions/comparison_D_fast_shock_limit_permanent_baseline_limit_permanent_slow_shock_limit_permanent.svg}
     \end{subfigure}

    \vspace{10pt}
     
     \justifying
     \footnotesize
	\textit{Note}: Percent deviations mean percent deviations from the respective initial steady state. The time horizon in all plots is fixed to 12 quarters, including the initial steady state in $t=0$.
\end{figure}

Clearly, the quicker the transition to the tight credit limit, the harsher the recession on impact of the shock. Consumption drops by over four percent relative to its initial steady state in the case of a deleveraging transition of four quarters. In contrast, allowing for an adjustment period of ten quarters results in a drop of about two percent on impact. The baseline scenario lies inbetween the two more extreme cases. 

These observations are intuitive: a quicker shock reduces the borrowing limit more forcefully, see plot \ref{fig:baseline-permanent-limit-duration-extra-phi}. This intensifies mechanically the deleveraging forced on constrained agents and intensifies their consumption reductions, thus reinforcing the balance-sheet effect. Similarly, as the debt constraint moves more quickly towards the unconstrained agents, the precautionary-savings effect is more pronounced. The result is that consumption drops more strongly when the transition to tight credit takes effect more rapidly. 

At the same time, plot \ref{fig:baseline-permanent-limit-duration-C} reveals that the recovery is more pronounced in the scenario of a fast shock. This is because, in that case, the real rate falls by more and does so more quickly, see picture \ref{fig:baseline-permanent-limit-duration-Rr}, as it must balance a larger fraction of households deleveraging with an increase in the precautionary-savings motive over and above that present in the baseline case.

The mechanically higher speed in deleveraging also becomes evident in the first ten quarters after the shock arises, see plot \ref{fig:baseline-permanent-limit-duration-D}. As expected, the slower the credit crunch takes effect, the more reluctantly aggregate debt converges to its terminal value. The figure contributes to the discussion of section \ref{sec:limit} regarding the transition of debt to its terminal state. In all three cases, economy-wide deleveraging picks up as soon as aggregate consumption recovers to above-steady-state values and levels off when consumption returns to steady state after its boom. This suggests that rising incomes are used to repay debt. Especially telling is the fact that from $t=10$ on, aggregate debt evolves identically across the three scenarios.

Overall, different speeds of adjustment to the regime of tight credit produce the same short-run patterns but with varying quantitative intensity. Quicker transitions enforce faster deleveraging and leave agents less time to adjust to the tighter credit limit, making the immediate downturn more painful, while leaving the long-run state unaffected.

%%%%%%%%%%%%%%%%%%%%%%%%%%%%%%%%%%%%%%%%%%%%%%%%%%%%%%%%%%%%%%%%%%%%%%%%%%%%%
%%%%%%%%%%%%%%%%%%%%%%%%%%%%%%%%%%%%%%%%%%%%%%%%%%%%%%%%%%%%%%%%%%%%%%%%%%%%%
\subsection{Insurance through Endogenous Labour Supply}
\label{sec:sensitivity-end-labour}

The \Gls{ghh} preferences assumed in the baseline model preclude households from adjusting working hours in order to absorb shocks to their consumption. This section explores an extension of the model which allows households to do so. Indeed, the analysis shows that, amid a credit crunch, the wealth-poor expand their labour effort to shield their consumption expenditures. Expectedly, this dampens the aggregate response of the economy significantly.

Consider households with \Gls{crra}-type preferences:
\begin{equation}
    u(c_{i,t}, n_{i,t}) = \frac{c_{i,t}^{1-\sigma_c}}{1-\sigma_c} - \phi \frac{n_{i,t}^{1+\sigma_l}}{1+\sigma_l}, \label{eq:hh-crra-utility}
\end{equation}
where $\phi$ is the weight on the disutility of labour. This modification changes some aspects of the formal aspects of the model. I refer the reader to appendix \ref{sec-app:full} for the required adjustments.

I follow \textcite{mckay2016} and set $\phi=1$.\footnote{\textcite{gl2017} also use \Gls{crra} preferences. They set $\psi = 12.48$ in their baseline specification but choose $\psi = 1.389$ in their "low-$\psi$" calibration.} I re-calibrate $\beta$, $\phi_{ss}$ and $\phi_{ss}'$ to match the same targets as before. In particular, debt to output is $57\%$ initially and equals $52\%$ in the terminal steady state.\footnote{Specifically: $\beta = 0.98801$, $\phi_{ss} = -1.7956$, $\phi_{ss}' = -1.655$ and $B = 5.39$. $\rho_{\phi}$ is again $0.5$. Note that leaving $\sigma_l$ unchanged is likely to change the average Frisch elasticity in the economy.} The rest of the model remains unchanged, but of course the steady states can be quite different to those described in sections \ref{sec:limit-stst-loose} and \ref{sec:limit-stst-tight}. In the appendix, figure \ref{fig:init-stst-end-L} reproduces plot \ref{fig:init-stst-dist-pol} and table \ref{tab:stst_comparison_end_L_limit_permanent} is the analogue to table \ref{tab:stst_comparison_baseline_limit_permanent}.

Figure \ref{fig:comparison-end-labour-agg} compares the dynamic responses of aggregate consumption and of debt-to-\Gls{gdp} to the tightening in the debt limit. Further responses are in figure \ref{fig:comparison-end-labour-agg-extra}. Across the two models, stark differences in their responses are visible. Importantly, on impact of the shock, consumption drops by only slightly more than half a percent, roughly six times less than in the baseline case with \Gls{ghh} preferences. Further, consumption quickly returns to its pre-shock level by $t=3$ and does not exhibit the same boom as in the baseline, see plot \ref{fig:comparison-end-labour-c}. 

\begin{figure}[t]
    \caption{Endogenous Labour Supply -- Shock to the Borrowing Limit: Aggregate Dynamics}
    \label{fig:comparison-end-labour-agg}
    \centering
    \begin{subfigure}[b]{0.49\textwidth}
    \caption{Consumption, $C$}
    \label{fig:comparison-end-labour-c}
         \centering
         \includesvg[inkscapelatex=false, width=\textwidth]{Results/compare_transitions/comparison_C_baseline_limit_permanent_end_L_limit_permanent.svg}
     \end{subfigure}
     \hfill
    \begin{subfigure}[b]{0.49\textwidth}
    \caption{Debt-to-Output, $\frac{D}{y}$}
    \label{fig:comparison-end-labour-DY}
         \centering
         \includesvg[inkscapelatex=false, width=\textwidth]{Results/compare_transitions/comparison_DY_baseline_limit_permanent_end_L_limit_permanent.svg}
     \end{subfigure}
     
     \vspace{10pt}
     
     \justifying
     \footnotesize
	\textit{Note}: Percent deviations mean percent deviations from the respective initial steady state. The time horizon in all plots is fixed to 12 quarters, including the initial steady state in $t=0$.
\end{figure}

As consumption does not fall as strongly, goods demand deteriorates by less, implying that wages, marginal costs and consequently prices do not drop as much as they do in the baseline.\footnote{Plots \ref{fig:comparison-end-labour-agg-extra-N} and \ref{fig:comparison-end-labour-agg-extrag-w} suggest that, indeed, the average Frisch elasticity might differ from the baseline.} The central bank reduces the nominal interest rate but the \Gls{zlb} does not bind. Also, agents reduce outstanding debt much more quickly, inducing a \textit{fall} in the ratio of debt to output instead of the increase observed in section \ref{sec:limit}.

Why is the short-run fallout from the credit crunch much less pronounced in this model? A glimpse at table \ref{tab:stst_comparison_end_L_limit_permanent} could have suggested the opposite. The fractions of households in debt and at the borrowing constraint are much higher in the present extension and the borrowers exhibit higher \Gls{mpc}s, insinuating a stronger response of the economy to the credit crunch. However, these statistics are measures of the direct consumption effects. The total responses might differ. As already noted, the preferences in \eqref{eq:hh-crra-utility} equip households with another tool to absorb shocks: their labour supply. 

Crucially, this margin of adjustment is used particularly by those hardest hit by the financial shock. To see this, consider figure \ref{fig:end-L-permanent-limit-percentiles}, which displays households' consumption and labour supply responses, separately for the bottom-$25\%$, bottom-$50\%$ and top-$25\%$ in the distribution. The results are intuitive: the bottom-$25\%$ experience the largest consumption as in the baseline (refer back to figure \ref{fig:baseline-permanent-limit-dist-c-response}) but can now individually adjust their labour hours, which they do, see plot \ref{fig:end-L-permanent-limit-percentiles-n}. In contrast to the rich, the wealth-poor expand the amount of work they supply by over three percent on impact. The bottom-$50\%$ also supply more labour relative so steady state.

\begin{figure}[t]
    \caption{Consumption and Labour Supply Responses by Selected Percentiles}
    \label{fig:end-L-permanent-limit-percentiles}
    \centering
    \begin{subfigure}[b]{0.49\textwidth}
    \caption{Consumption}
    \label{fig:end-L-permanent-limit-percentiles-c}
         \centering
         \includesvg[inkscapelatex=false, width=\textwidth]{Results/percentile_transitions_end_L_limit_permanent_C.svg}
     \end{subfigure}
     \hfill
    \begin{subfigure}[b]{0.49\textwidth}
    \caption{Labour Supply}
    \label{fig:end-L-permanent-limit-percentiles-n}
         \centering
         \includesvg[inkscapelatex=false, width=\textwidth]{Results/percentile_transitions_end_L_limit_permanent_N.svg}
     \end{subfigure}
     
     \vspace{10pt}
     
     \justifying
     \footnotesize
	\textit{Note}: The responses are calculated as the product of aggregate consumption (or labour) in period $t$ and the share that the respective household percentile has in consumption (or labour) in $t$. Percent deviations mean percent deviations from the respective initial steady state. The time horizon in all plots is fixed to 12 quarters, including the initial steady state in $t=0$.
\end{figure}

By supplying more labour, especially those with the biggest consumption losses can shield their budgets from even bigger strains. As these agents are typically those with particularly high \Gls{mpc}s, their effort to shield their expenditures stabilises the aggregate very effectively. This explains the much milder responses in the present model version. These results hint at the fact that idiosyncratic labour supply might be a quantitatively important insurance mechanism against shocks to the household-level borrowing constraints. Of course, working (more) still incurs (more) disutility and so agents are still willing to accept consumption losses. Thus, the responses' shapes will be greatly influenced by $\psi$. Exploring the effect of this parameter is, however, beyond the scope of this thesis.

Before closing this section, let me briefly return to the discussion of the transition of debt. Here, outstanding debt returns much faster to its final steady state than in the baseline model. This is due to the fact that the recession is not as deep but also that agents now have an additional tool to delever to their disposal, their labour supply. 

Note that here the long-run effects are different (because now steady state consumption is not determined by the supply side of the model) and that inequality in supplying labour goes down, a bit.

%%%%%%%%%%%%%%%%%%%%%%%%%%%%%%%%%%%%%%%%%%%%%%%%%%%%%%%%%%%%%%%%%%%%%%%%%%%%%
%%%%%%%%%%%%%%%%%%%%%%%%%%%%%%%%%%%%%%%%%%%%%%%%%%%%%%%%%%%%%%%%%%%%%%%%%%%%%
%%%%%%%%%%%%%%%%%%%%%%%%%%%%%%%%%%%%%%%%%%%%%%%%%%%%%%%%%%%%%%%%%%%%%%%%%%%%%
\section{A Credit Crunch through the Interest Rate Wedge}
\label{sec:wedge}

Having understood the consequences of a shock to the household-level borrowing limit, this sections considers a credit crunch through an increase in the premium paid on debt holdings. That is, I consider a permanent shock to the interest rate wedge, $\kappa_t$. By doing so, I aim at identifying the similarities and differences in the dynamics induced by this shock relative to the borrowing limit shock. Crucially, the findings demonstrate that the consequences are similar qualitatively, but that quantitative and distributional divergences do exist. 

Analogously to \eqref{eq:borrowing-limit-process}, let the process of the interest rate wedge, $\kappa_t$, be given by:
\begin{equation}
    \kappa_t = \kappa_{ss} \Bigg( \frac{\kappa_{t-1}}{\kappa_{ss}} \Bigg)^{\rho_{\kappa}} exp(e_{\kappa}), \label{eq:wedge-process}
\end{equation}
where $\kappa_{ss} =$ 1e-8 is as before and $\kappa_{ss}' = 0.00203$ achieves a reduction in the terminal debt-to-output ratio of five percentage points. This makes for a fair comparison to the shock to the borrowing limit. Choosing $\rho_{\kappa}$, the persistence of the borrowing wedge, equal to $0.3$ results in the duration of the adjustment being complete within six periods. 

Table \ref{tab:stst_comparison_baseline_wedge_permanent} compares the steady states and figure \ref{fig:baseline-permanent-wedge-agg} presents some selected dynamics of the economy from the state of loose credit to one of tight credit, while figure \ref{fig:baseline-permanent-wedge-agg-extra} in the appendix contains further responses. Consider first the steady states. Table \ref{tab:stst_comparison_baseline_wedge_permanent} reveals that the increase of the wedge by roughly $0.8$ percentage points in annual terms achieves the same long-run reduction in debt-to-output as the previously analysed shock to the credit limit. Noticeably, the increase in the interest rate wedge leads to a permanently lower real rate on lending, $RR$, in the long run. As is the case with a tight debt limit, this hints at the fact that a shock which concerns only borrowers in the first place, carries consequences for the lenders as well.

\begin{table}[t]
\centering
\caption{Baseline Model -- Shock to the Interest Rate Wedge: \\ Comparison of Steady States}
\label{tab:stst_comparison_baseline_wedge_permanent}
\begin{tabular}{lccc}
Variable & Initial & Terminal &  Change \\
\hline
\hline
\multicolumn{2}{l}{\textbf{Interest Rate Wedge}} & & \\
$\kappa$ & $0\%$ & $0.203\%$ & $0.203 \ p.p.$ \\
& & & \\
\multicolumn{2}{l}{\textbf{Aggregates}} & & \\
                       $C$ &  $0.9129$ &   $0.9119$ & $-0.11\%$ \\
                      $D / y$ &  $57\%$ &  $52\%$ & $-5 \ p.p.$ \\
               $B+D$ &  5.7304 &   5.6841 & -0.0463 \\
                      $Rr-1$ &  $0.15\%$ & $0.12\%$ & -0.0003 \\
                 $(RR+\kappa)-1$ &  $0.15\%$ & $0.33\%$ &  0.0018 \\
                     $\tau$ &  0.0078 &   0.0064 & -0.0014 \\
& & & \\
\multicolumn{2}{l}{\textbf{Cross-Section}} & & \\
Share of Assets Held by Top-$10\%$ &  $0.4834$ &   $0.4742$ & $-0.0092 \ p.p.$ \\
Share of Consumption by Top-$10\%$ &  $0.1729$ &   $0.1734$ &  $0.0005 \ p.p.$ \\
Fraction of Borrowers &   $33.66$ &    $31.87$ & $-1.79 \ p.p.$ \\
Fraction at Borrowing Limit &    $4.89$ &     $4.17$ & $-0.72 \ p.p.$ \\
\Gls{mpc} of Borrowers &    0.16 &     0.14 & -0.0200 \\
\Gls{mpc} of Lenders &    0.02 &     0.02 &  0.0000 \\
\hline
\multicolumn{4}{l}{\footnotesize \multirow{2}{12cm}{\justifying \textit{Note:} The table contains selected values of the initial steady state with $\kappa_{ss}$ and the terminal steady state with $\kappa_{ss}'$. All numbers are rounded and refer to quarterly values. $p.p.$ stands for percentage points.}} \\
& & & \\
\end{tabular}
\end{table}

\begin{figure}[!ht]
    \centering
    \caption{Baseline Model -- Shock to the Interest Rate Wedge: Aggregate Dynamics}
    \label{fig:baseline-permanent-wedge-agg}
     \begin{subfigure}[b]{0.49\textwidth}
     \caption{Interest Rate Wedge, $\kappa$}
     \label{fig:baseline-permanent-wedge-agg-kappa}
         \centering
         \includesvg[inkscapelatex=false, width=\textwidth]{Results/compare_transitions/comparison_Rbar_baseline_limit_permanent_baseline_wedge_permanent.svg}
     \end{subfigure}
     \hfill
     \begin{subfigure}[b]{0.49\textwidth}
     \caption{\textit{Ex-Post} Real Rate on Debt, $RR + \kappa$}
     \label{fig:baseline-permanent-wedge-agg-Rr-on-debt}
         \centering
         \includesvg[inkscapelatex=false, width=\textwidth]{Results/compare_transitions/comparison_Rrminus_baseline_limit_permanent_baseline_wedge_permanent.svg}
     \end{subfigure}
     \hfill
    \begin{subfigure}[b]{0.49\textwidth}
    \caption{Consumption, $C$}
    \label{fig:baseline-permanent-wedge-agg-C}
         \centering
         \includesvg[inkscapelatex=false, width=\textwidth]{Results/compare_transitions/comparison_C_baseline_limit_permanent_baseline_wedge_permanent.svg}
     \end{subfigure}
    \hfill
    \begin{subfigure}[b]{0.49\textwidth}
    \caption{Debt, $D$}
    \label{fig:baseline-permanent-wedge-agg-D}
         \centering
         \includesvg[inkscapelatex=false, width=\textwidth]{Results/compare_transitions/comparison_D_baseline_limit_permanent_baseline_wedge_permanent.svg}
     \end{subfigure}

    \vspace{10pt}
     
     \justifying
     \footnotesize
	\textit{Note}: Percent deviations mean percent deviations from the respective initial steady state. The time horizon in all plots is fixed to 12 quarters, including the initial steady state in $t=0$.
\end{figure}

The mechanisms at play here are a mirror-image to those identified in section \ref{sec:limit}. Unlike in the case of a tighter credit limit, the balance-sheet effect here concerns all debtors as the spread affects all agents with negative net asset holdings. Of course, those with very high debt holdings face especially higher interest payments. Thus, the supply of \Gls{iou}s and of gross liquidity is mechanically reduced. Lenders increase their asset demand in order to avoid being forced to use the now more costly option of borrowing when hit with unfortunate productivity realisations. This combination of balance-sheet and precautionary-savings effects calls for lower real interest rates. This reduction comes in the form of a permanently lower $RR$ for lenders, which in turn is also a lower "base rate" for borrowers.

The different kinds of financial shocks resemble each other also in the cross-sectional dimension. Wealth inequality is lower and consumption inequality is higher after a credit crunch through the interest rate spread. However, for the same decrease in the debt-to-\Gls{gdp} ratio, the fraction of borrowers falls stronger: $-1.79$ here versus $-0.76$ percentage points in section \ref{sec:limit}. A qualitative difference arises regarding the fraction of the population at the respective debt limits. It increases in the case of a shock to the debt limit, while it falls after a credit crunch through the credit spread. The latter two observations are due to the fact that the elevated wedge affects all borrowers.

Next, I turn to the dynamics of the present credit crunch. First, plot \ref{fig:baseline-permanent-wedge-agg-kappa} displays the process of the shock. As it starts from a value very close to zero, the \Gls{ar}(1) process exhibits only a small increase in $\kappa_1$ relative to $\kappa_{ss}$. It subsequently picks up and completes its transition to the terminal value of $0.81\%$ \textit{per annum} within six quarters.

%In contrast to the credit crunch provoked by the borrowing limit, the present shock does not necessarily prohibit agents from taking up more debt. However, it makes holding debt -- at all levels of debt -- more expensive and thus reduces the willingness (and ability) of borrowers to supply \Gls{iou}s.

Interestingly, plot \ref{fig:baseline-permanent-wedge-agg-D} reveals that household debt declines even slower in response to the increase in the interest rate wedge than to the tightening in the borrowing limit. It supports the considerations from section \ref{sec:limit-transition}: as incomes do not rise as quickly, households wish to smooth the deleveraging as much as possible. Also, incomes are permanently lower, so repayment is more difficult.

%%%%%%%%%%%%%%%%%%%%%%%%%%%%%%%%%%%%%%%%%%%%%%%%%%%%%%%%%%%%%%%%%%%%%%%%%%%%%
%%%%%%%%%%%%%%%%%%%%%%%%%%%%%%%%%%%%%%%%%%%%%%%%%%%%%%%%%%%%%%%%%%%%%%%%%%%%%
%%%%%%%%%%%%%%%%%%%%%%%%%%%%%%%%%%%%%%%%%%%%%%%%%%%%%%%%%%%%%%%%%%%%%%%%%%%%%
\section{The Asymmetry of Financial Shocks}
\label{sec:asymmetry}

This section explores the asymmetry of financial shocks. To that end, I consider a simple experiment: what if the borrowing limit, instead of tightening from $\phi_{ss}$ to $\phi_{ss}'$, loosens from $\phi_{ss}'$ to $\phi_{ss}$?\footnote{I call this a "simple" experiment because a more complete analysis would also consider an expansion from $\frac{D}{y} = 57\%$ to $\frac{D}{y} = 62\%$. I avoid doing so here because this would require further adjustments to the asset grid.} Would the economy experience the same dynamics as those in section \ref{sec:limit} just with their sign reversed? To probe into a potential sign dependency, I use the baseline model but compute the transition from $\phi_{ss}'$ to $\phi_{ss}$. That is, I let the household-level borrowing constraint loosen, thus inducing a credit easing. Figure xx shows the dynamic responses of this exercise.

The results are telling: the boom from a credit easing is much less pronounced than the recession caused by a tightening of credit.

%%%%%%%%%%%%%%%%%%%%%%%%%%%%%%%%%%%%%%%%%%%%%%%%%%%%%%%%%%%%%%%%%%%%%%%%%%%%%
%%%%%%%%%%%%%%%%%%%%%%%%%%%%%%%%%%%%%%%%%%%%%%%%%%%%%%%%%%%%%%%%%%%%%%%%%%%%%
%%%%%%%%%%%%%%%%%%%%%%%%%%%%%%%%%%%%%%%%%%%%%%%%%%%%%%%%%%%%%%%%%%%%%%%%%%%%%
\section{Concluding Remarks}
\label{sec:conclusion}

Policy considerations abound.

Would be interesting to extend the model so that there is also borrowing by other entities, especially firms.

An interesting extension of the model would be a small open economy as then funds could potentially be obtained elsewhere.

Borrowing limits are, of course, endogenous outcomes. The present approach abstracted from this endogeneity. Uncovering the \textit{sources} of time-variations in the borrowing limits (also in the cross-section) requires addressing the endogeneity and explicitly modelling borrowing limits as an outcome of decisions by market participants.

Future work should aim at endogenising borrowing limits and interest rate wedges. 

% Some advances in the latter can be found in Boehl 2023 The micro and macro.

As \textcite{kv2018} remark, an empirical investigation of the shocks driving the  Great Recession would ideally take into account how consumption responded along the entire wealth distribution in order to discern whether the shocks analysed here were a driving force of the recession.

%%%%%%%%%%%%%%%%%%%%%%%%%%%%%%%%%%%%%%%%%%%%%%%%%%%%%%%%%%%%%%%%%%%%%%%%%%%%%
%%%%%%%%%%%%%%%%%%%%%%%%%%%%%%%%%%%%%%%%%%%%%%%%%%%%%%%%%%%%%%%%%%%%%%%%%%%%%
%%%%%%%%%%%%%%%%%%%%%%%%%%%%%%%%%%%%%%%%%%%%%%%%%%%%%%%%%%%%%%%%%%%%%%%%%%%%%
% References section
\newpage
\thispagestyle{plain}
\pagenumbering{Roman}
\printbibliography[heading=bibintoc] % Insert references

%%%%%%%%%%%%%%%%%%%%%%%%%%%%%%%%%%%%%%%%%%%%%%%%%%%%%%%%%%%%%%%%%%%%%%%%%%%%%
%%%%%%%%%%%%%%%%%%%%%%%%%%%%%%%%%%%%%%%%%%%%%%%%%%%%%%%%%%%%%%%%%%%%%%%%%%%%%
%%%%%%%%%%%%%%%%%%%%%%%%%%%%%%%%%%%%%%%%%%%%%%%%%%%%%%%%%%%%%%%%%%%%%%%%%%%%%
% Appendices section
\newpage
\begin{refsection}
\thispagestyle{plain}
\pagenumbering{arabic}  % Turn page numbering to arabic
\renewcommand*{\thepage}{A-\arabic{page}} % Add 'A' to each page number for appendices section
\addtocontents{toc}{\protect\setcounter{tocdepth}{1}} % This hides the appendix subsections in the table of contents
\begin{appendices}
%%%%%%%%%%%%%%%%%%%%%%%%%%%%%%%%%%%%%%%%%%%%%%%%%%%%%%%%%%%%%%%%%%%%%%%%%%%%%
%%%%%%%%%%%%%%%%%%%%%%%%%%%%%%%%%%%%%%%%%%%%%%%%%%%%%%%%%%%%%%%%%%%%%%%%%%%%%
\section{Full Models}
\label{sec-app:full}

This appendix present the full models employed throughout the main text and the subsequent appendices. Appendix \ref{sec-app:full-derivations} derives the equations characterising optimal behaviour of households and firms. Thereafter, appendix \ref{sec-app:full-equations} displays the full system of equations passed to the \texttt{Econpizza} package (see appendix \ref{sec-app:codes} for details). Finally, appendix \ref{sec-app:full-graphical} presents a graphical representation of the baseline model.

%%%%%%%%%%%%%%%%%%%%%%%%%%%%%%%%%%%%%%%%%%%%%%%%%%%%%%%%%%%%%%%%%%%%%%%%%%%%%
\subsection{Derivations for Households and Firms}
\label{sec-app:full-derivations}

This subsection derives the equations that characterise optimal behaviour by households and firms. It begins with the household side. Next, the corporate sector's problem is dealt with.

\subsubsection{Households}
\label{sec-app:full-derivations-households}

The household setup is explained in section \ref{sec:model-hh} of the main text. Here, I present the derivations which yield the final equations necessary to determine optimal household behaviour. For the baseline model, which features \Gls{ghh} preferences, the final equations are \eqref{eq:foc-euler} -- \eqref{eq:foc-labour-supply}. For the model with \Gls{crra}-type preferences, the final equations are ...

The maximisation problem of the households can be cast into the following Bellman equation:
\begin{equation}
    V_t (e_{i,t}, b_{i,t-1}) = \max_{c_{i,t}, n_{i,t}, b_{i,t}} \Bigg\{ u(x_{i,t}) + \beta \mathbb{E}_{e,t} \{ V_{t+1} (e_{i,t+1}, b_{i,t}) \} \Bigg\} \label{eq:app-hh-bellman}
\end{equation}
subject to \eqref{eq:hh-budget} and \eqref{eq:borrowing-limit}. As appendix \ref{sec-app:full-derivations-households} shows, using the specifications for the composite good in equation \eqref{eq:hh-composite-good} and the utility function in equation \eqref{eq:hh-utility-fun}, this formulation leads to two first-order conditions (\Gls{foc}) and an envelope condition, which together characterise optimal household behaviour for all $i$. Note first the conditions with respect to the consumption-savings decision:

Households have \Gls{ghh} preferences over the composite good, $x_{i,t}$. \Gls{ghh} preferences simplify the household decision problem as it makes household $i$'s labour supply decision exogenous with respect to $i$'s consumption. As \textcite{auclert2017} note, this is because, with \Gls{ghh} preferences, the marginal utility of consumption and the marginal disutility of working move proportionally, thus leaving $n_{i,t}$ unchanged when $c_{i,t}$ changes. In the end, this implies that all households work the same hours, despite consuming different amounts.

\subsubsection{Firms}
\label{sec-app:full-derivations-firms}
A representative and perfectly competitive firm produces the final good, $y_t^f$, by combining different intermediate goods, $Y_{j,t}$. The technology it uses for that process features constant returns to scale and is given by:
\begin{equation}
    y_t^f = \Bigg( \int_0^1 Y_{j,t}^{\frac{\theta - 1}{\theta}} dj \Bigg)^{\frac{\theta}{\theta-1}},
\end{equation}
where $\theta$ is the elasticity of substitution for the intermediate goods in the production process. Profit maximisation and a zero-profits condition lead to input demand:
\begin{equation}
    Y_{j,t} = \Bigg( \frac{P_{j,t}}{P_t} \Bigg)^{-\theta} y_t^f,
\end{equation}
and a price aggregator of the form:
\begin{equation}
    P_t = \Bigg( \int_0^1 P_{j,t}^{1-\theta} dj \Bigg)^{\frac{1}{1-\theta}}
\end{equation}

A continuum of intermediate-goods firms, indexed by $j$, sell their output, produced with the production function:
\begin{equation}
    y_{j,t} = z N_{j,t}
\end{equation}
to the representative final-good firm in a monopolistically competitive market by setting the nominal price $P_{j,t}$. In adjusting their prices, the intermediate-goods firms are subject to price adjustment costs à la \textcite{rotemberg1982}.

%%%%%%%%%%%%%%%%%%%%%%%%%%%%%%%%%%%%%%%%%%%%%%%%%%%%%%%%%%%%%%%%%%%%%%%%%%%%%
\subsection{Full System of Equations}
\label{sec-app:full-equations}

This subsection lists all equations from the baseline model.

%%%%%%%%%%%%%%%%%%%%%%%%%%%%%%%%%%%%%%%%%%%%%%%%%%%%%%%%%%%%%%%%%%%%%%%%%%%%%
\subsection{Graphical Summary of the Model}
\label{sec-app:full-graphical}

This subsection presents a visualisation of the baseline model presented in section \ref{sec:model}. The figure is meant to be an aid in understanding the flows and mechanisms of the model and clearly does not capture essential aspects of the model, such as its dynamic nature.

\begin{figure}[H]
\centering
\caption{Graphical Representation of the Baseline Model}
\label{fig:graphical-model}
\begin{tikzpicture}
  % Big rectangle
  \draw[fill=white] (0,) rectangle (15,8);
  \node at (7.5,7.5) {\Large Households};

  \node at (3,6) {"unlucky"};
  \node at (12,6) {"lucky"};

  \node at (2.25,4) {Borrowers};
  \node at (2.25,3.6) {\scriptsize negative \Gls{iou}s};
  \node at (8,4) {Lenders};
  \node at (8,3.6) {\scriptsize positive \Gls{iou}s};
  \node at (12,4) {Bond Holders};

  \node at (13.85,-2) {Bonds $B$};
  \node at (14.1,-3) {Interest $RR$};
  
  \node at (7.5,9) {Productivity Risk $\varepsilon^e$};

  \node at (2,9) {Financial Constraints $\phi$, $\kappa$};
  
  \node at (2.5,-4.25) {Firms};

  \node at (4.3,-4.5) {$\pi$};
  \node at (6.1,-4.5) {$R^n$};
  
  \node at (8.25,-4.25) {Monetary Policy};

  \node at (13,-4.25) {Fiscal Policy};
  
  % Arrows
  \draw[->] (7.5,8.75) -- (7.5,8.25); % to households

  \draw[->] (7.5,7.25) -- (3,6.25); % to unlucky
  \draw[->] (7.5,7.25) -- (12,6.25); % to lucky

  \draw[->] (3,5.75) -- (3,4.25); % to borrowers
  \draw[->] (12,5.75) -- (8,4.25); % to lenders
  \draw[->] (12,5.75) -- (12,4.25); % to bond holders
  
  \draw[->, red] (0.5,8.75) -- (2.25,4.25); % to borrowers

  \draw[->] (7,4.1) -- (3.75,4.1);
  \draw[->] (3.75,3.9) -- (7,3.9); 

  \draw[->] (13,-4) -- (13,3.75); 

  \draw[-] (3.2,-4.25) -- (6.5,-4.25); % btw MP an firms
  \draw[->] (5.5,-4.25) -- (5.5,2.5); % to interest rate btw borr and lend
  
  \draw[->] (5.5,-3) -- (13,-3); 

  \draw[->] (2.65,1) -- (2.65,-4); % to firms
  \draw[->] (2.4,-4) -- (2.4,1); % from firms
  \node at (1.4,-1) {Income $wN$};
  \node at (3.6,-1) {Labour $N$};

  \node at (1.2,-2) {Consumption};
  \node at (1.4,-2.5) {Goods $C$};
  
  \node at (1.4,0) {Profits $\Pi$};

  \draw[->] (12.2,1) -- (12.2,-4); % to FP
  \node at (11.4,-2) {Taxes $\tau$};

  \node at (5.5,4.5) {\footnotesize "Consumption"};
  \node at (5.5,3.5) {\footnotesize "Insurance"};
  \node at (5.5,3) {\footnotesize + Interest $RR$};

\end{tikzpicture}
\begin{flushleft}
\footnotesize
\textit{Note}: This figure displays the agents and some key mechanisms of the baseline model from section \ref{sec:model}. To ease notation, time subscripts are suppressed.
\end{flushleft}
\end{figure}

%%%%%%%%%%%%%%%%%%%%%%%%%%%%%%%%%%%%%%%%%%%%%%%%%%%%%%%%%%%%%%%%%%%%%%%%%%%%%
\subsection{The Natural Borrowing Limit}
\label{sec:app-hh-borrowing}
The natural borrowing limit is the borrowing limit which just marginally ensures that borrowers can repay and not have negative consumption. This natural borrowing limit can be calculated from forward iterating the household budget constraint \eqref{eq:hh-budget}:
\begin{equation*}
    \frac{wne_{L} + \Pi_t \bar{\Pi} (e_{L}) - \tau_t \bar{\tau} (e_{L})}{RR - 1},
\end{equation*}
where $e_L$ denotes the lowest level of skills attainable from \eqref{eq:hh-income-process}.

%%%%%%%%%%%%%%%%%%%%%%%%%%%%%%%%%%%%%%%%%%%%%%%%%%%%%%%%%%%%%%%%%%%%%%%%%%%%%
%%%%%%%%%%%%%%%%%%%%%%%%%%%%%%%%%%%%%%%%%%%%%%%%%%%%%%%%%%%%%%%%%%%%%%%%%%%%%
%%%%%%%%%%%%%%%%%%%%%%%%%%%%%%%%%%%%%%%%%%%%%%%%%%%%%%%%%%%%%%%%%%%%%%%%%%%%%
%\newpage
\thispagestyle{plain}
\section{Computational Implementation}
\label{sec-app:codes}

This appendix has two parts. The first one notes some caveats when implementing the financial shocks of sections \ref{sec:limit} -- \ref{sec:sensitivity} in \texttt{Econpizza}. The second part provides a rough guide for the computer codes producing the results for this thesis. The full set of codes is provided together with the thesis and is available on my GitHub page: \url{https://github.com/andkound98/master-thesis}, last retrieved \printdate{2023-08-25}.

\subsection{Implementation of Shocks in \texttt{Econpizza}}
\label{sec-app:codes-ep}

The choices on the grid sizes naturally involve an accuracy-speed trade-off. I approximate the productivity process on a grid with $8$ grid points. The asset grid has $219$ grid points. I choose the number of grid points roughly such that increasing the number does not change the main results in a meaningful way.

Regarding the implementation of the shocks in the subsequent sections, it is important to note two things. First, in case of permanent shocks, the transition path must be calculated using the Jacobian of the terminal steady state instead of the initial steady state (see \cite{auclert2021}). Second, in the case of a shock to the borrowing limit, a correct implementation requires that the asset grid is not changed over time so as to avoid invalid comparisons based on different interpolations (see this \cite{mellior2016mimeo}). The code considers both of these aspects.

The horizon is set to $200$, i.e.~the solution method assumes that after $200$ quarters the system has returned its steady state.

%Second, in the case of a shock to the borrowing limit, a correct implementation requires that the asset grid is not changed over time, as this mimeo notes. Otherwise, the interpolation would be between repeatedly different grid points, making the comparison of over time inaccurate. 

This is because, from $t=1$, the new regime applies.

\subsection{Guide to Computer Codes}
\label{sec-app:codes-guide}

I now shortly describe the Python code files employed. All these codes were run using the Spyder IDE $5.3.4$ with Python $3.9.12$ and \texttt{Econpizza} $0.6.1$ on macOS $12.6.5$.

\subsection*{\texttt{main.py}}

The file \texttt{main.py} is the main file of the project.

\subsection*{\texttt{custom\_functions.py}}

\subsection*{\texttt{plot\_functions.py}}

\subsection*{\texttt{calibration.py}}

This file uses the data described in appendix \ref{sec-app:data} to calculate the desired calibration targets and to create the figures \ref{fig:app-calibration-liquid}, \ref{fig:app-calibration-debt} and \ref{fig:app-calibration-bonds}.

%%%%%%%%%%%%%%%%%%%%%%%%%%%%%%%%%%%%%%%%%%%%%%%%%%%%%%%%%%%%%%%%%%%%%%%%%%%%%
%%%%%%%%%%%%%%%%%%%%%%%%%%%%%%%%%%%%%%%%%%%%%%%%%%%%%%%%%%%%%%%%%%%%%%%%%%%%%
%%%%%%%%%%%%%%%%%%%%%%%%%%%%%%%%%%%%%%%%%%%%%%%%%%%%%%%%%%%%%%%%%%%%%%%%%%%%%
%\newpage
\thispagestyle{plain}
\section{Data for Model Calibration}
\label{sec-app:data}

This appendix describes the data series and sources used for the calibration of the baseline model in section \ref{sec:model-solution} and for the calibration of the model in section \ref{sec-app:robust-bonds}.

First, to match $B$ to total liquid assets held by households, I follow \textcite{mckay2016} and \textcite{gl2017} in using information on aggregate household balance sheets, available in the Financial Accounts of the US Z1, known as "Flow of Funds" (\Gls{fof}) and provided by the Federal Reserve.

Second, I acquire data on US nominal \Gls{gdp}, the US price level from the FRED data base, which is maintained by the Federal Reserve of Saint Louis.

\begin{figure}[H]
    \caption{Time Series of Data for the Baseline Model Calibration}
    \centering
    \begin{subfigure}[b]{0.49\textwidth}
    \caption{Liquid Assets Held by Households}
    \label{fig:app-calibration-liquid}
         \centering
         \includesvg[inkscapelatex=false, width=\textwidth]{Results/FRED_l.svg}
     \end{subfigure}
     \hfill
     \begin{subfigure}[b]{0.49\textwidth}
     \caption{Consumer Credit over \Gls{gdp}} 
    \label{fig:app-calibration-debt}
         \centering
         \includesvg[inkscapelatex=false, width=\textwidth]{Results/FRED_d.svg}
     \end{subfigure}
     \begin{flushleft}
     \footnotesize
	\textit{Note}: Data taken from the \href{https://www.federalreserve.gov/releases/z1/}{\Gls{fof} statistics} and the \href{https://fred.stlouisfed.org}{FRED data base}. Sample range: 1951:Q4 -- 2023:Q1. The dotted line is the full sample mean of the respective series. The series in panel \ref{fig:app-calibration-liquid} is in terms of the units from the \textit{baseline} calibration of the model in section \ref{sec:model}, in which $y_{ss} = 0.9129$. 
	\end{flushleft}
\end{figure}

\begin{figure}[H]
    \centering
    \caption{Time Series of Government Bonds}
    \label{fig:app-calibration-bonds}
    \includesvg[inkscapelatex=false, width=0.75\textwidth]{Results/FRED_b.svg}

    \vspace{10pt}
    
    \begin{minipage}{0.75\textwidth} %{12cm}
    \footnotesize
    \textit{Note}: Data taken from \href{https://fred.stlouisfed.org}{FRED data base}. Sample range: 1951:Q4 -- 2023:Q1. The dotted line is the full sample mean of the series. The series is in terms of the units from the \textit{baseline} calibration of the model in section \ref{sec:model}, in which $y_{ss} = 0.9129$.
    \end{minipage}
\end{figure}

%%%%%%%%%%%%%%%%%%%%%%%%%%%%%%%%%%%%%%%%%%%%%%%%%%%%%%%%%%%%%%%%%%%%%%%%%%%%%
%%%%%%%%%%%%%%%%%%%%%%%%%%%%%%%%%%%%%%%%%%%%%%%%%%%%%%%%%%%%%%%%%%%%%%%%%%%%%
%%%%%%%%%%%%%%%%%%%%%%%%%%%%%%%%%%%%%%%%%%%%%%%%%%%%%%%%%%%%%%%%%%%%%%%%%%%%%
%\newpage
\thispagestyle{plain}
\section{Supplementary Results}
\label{sec-app:figures}

This appendix contains supplementary figures and tables, placed here for the sake of brevity in the main text.

%%%%%%%%%%%%%%%%%%%%%%%%%%%%%%%%%%%%%%%%%%%%%%%%%%%%%%%%%%%%%%%%%%%%%%%%%%%%%
%%%%%%%%%%%%%%%%%%%%%%%%%%%%%%%%%%%%%%%%%%%%%%%%%%%%%%%%%%%%%%%%%%%%%%%%%%%%%
\subsection{Figures for Section \ref{sec:limit-stst-loose}}
\label{sec-app:figures-limit-loose}

\begin{figure}[H]
    \centering
    \caption{\Gls{mpc}s Across Low Asset Holdings}
    \label{fig:baseline-limit-initial-mpcs}
    \includesvg[inkscapelatex=false, width=0.75\textwidth]{Results/stst_policies_mpc_baseline_limit_permanent_initial.svg}

    \vspace{10pt}
    
    \begin{minipage}{0.75\textwidth}
    \footnotesize
    \textit{Note}: The figure is cut off at an asset level of $2$. \Gls{mpc}s calculated as $mpc_{i,t} = \frac{\partial c_{i,t}}{\partial b_{i,t}}$ and are restricted to be weakly smaller than one. The labels $\theta\_0$ -- $\theta\_7$ represent the idiosyncratic labour skill states from \eqref{eq:hh-income-process}.
    \end{minipage}
\end{figure}

%%%%%%%%%%%%%%%%%%%%%%%%%%%%%%%%%%%%%%%%%%%%%%%%%%%%%%%%%%%%%%%%%%%%%%%%%%%%%
%%%%%%%%%%%%%%%%%%%%%%%%%%%%%%%%%%%%%%%%%%%%%%%%%%%%%%%%%%%%%%%%%%%%%%%%%%%%%
\subsection{Figures for Section \ref{sec:limit-stst-tight}}
\label{sec-app:figures-limit-tight}

\begin{sidewaysfigure}
    \caption{Baseline Model -- Shock to the Borrowing Limit: Comparison
 of Asset Distributions}
    \label{fig:comparison-stst-baseline-dist}
    \centering
\includesvg[inkscapelatex=false, width=0.8\textwidth]{Results/stst_dist_2d_comparison_baseline_limit_permanent.svg}
    \\
     \centering
	\textit{Note}: Note that the density is due to numerical errors.
\end{sidewaysfigure}

%%%%%%%%%%%%%%%%%%%%%%%%%%%%%%%%%%%%%%%%%%%%%%%%%%%%%%%%%%%%%%%%%%%%%%%%%%%%%
%%%%%%%%%%%%%%%%%%%%%%%%%%%%%%%%%%%%%%%%%%%%%%%%%%%%%%%%%%%%%%%%%%%%%%%%%%%%%
\subsection{Figures for Section \ref{sec:limit-transition}}
\label{sec-app:figures-limit-transition}

\begin{figure}[H]
    \caption{Supplement for Figure \ref{fig:baseline-permanent-limit-agg}}%{\hypersetup{hidelinks}\ref{fig:baseline-permanent-limit-agg}}}
    \label{fig:baseline-permanent-limit-agg-extra}
    \begin{subfigure}[b]{0.49\textwidth}
     \caption{Labour Hours, $N$}
     \label{fig:baseline-permanent-limit-agg-N}
         \centering
         \includesvg[inkscapelatex=false, width=\textwidth]{Results/transition_N_baseline_limit_permanent.svg}
     \end{subfigure}
     \hfill
     \begin{subfigure}[b]{0.49\textwidth}
     \caption{Wage, $w$}
     \label{fig:baseline-permanent-limit-agg-w}
         \centering
         \includesvg[inkscapelatex=false, width=\textwidth]{Results/transition_w_baseline_limit_permanent.svg}
     \end{subfigure}
     \hfill
     \begin{subfigure}[b]{0.49\textwidth}
     \centering
     \caption{Marginal Costs, $mc$}
     \label{fig:baseline-permanent-limit-agg-mc}
         \includesvg[inkscapelatex=false, width=\textwidth]{Results/transition_mc_baseline_limit_permanent.svg}
     \end{subfigure}
     \hfill
     \begin{subfigure}[b]{0.49\textwidth}
     \centering
     \caption{Inflation, $\pi$}
     \label{fig:baseline-permanent-limit-agg-pi}
         \includesvg[inkscapelatex=false, width=\textwidth]{Results/transition_pi_baseline_limit_permanent.svg}
     \end{subfigure}
     \hfill
    \begin{subfigure}[b]{0.49\textwidth}
    \caption{Tax Rate, $\tau$}
    \label{fig:baseline-permanent-limit-agg-tau}
         \centering
         \includesvg[inkscapelatex=false, width=\textwidth]{Results/transition_tau_baseline_limit_permanent.svg}
     \end{subfigure}
     \hfill
     \begin{subfigure}[b]{0.49\textwidth}
     \caption{Dividends, $\Pi$}
     \label{fig:baseline-permanent-limit-agg-Pi}
         \centering
         \includesvg[inkscapelatex=false, width=\textwidth]{Results/transition_div_baseline_limit_permanent.svg}
     \end{subfigure}

    \vspace{10pt}
     
     \begin{flushleft}
     \footnotesize
	\textit{Note}: Percent deviations mean percent deviations from the respective initial steady state. The time horizon in all plots is fixed to 12 quarters, including the initial steady state in $t=0$.
	\end{flushleft}
\end{figure}

\begin{figure}[H]
    \caption{Baseline Model -- Shock to the Borrowing Limit: Debt in the Long-Run}
    \label{fig:baseline-permanent-limit-longrun}
    \begin{subfigure}[b]{0.49\textwidth}
     \caption{Debt, $D$}
     \label{fig:baseline-permanent-limit-longrun-D}
         \centering
         \includesvg[inkscapelatex=false, width=\textwidth]{Results/D_longrun.svg}
     \end{subfigure}
     \hfill
     \begin{subfigure}[b]{0.49\textwidth}
     \caption{Debt-to-Output, $\frac{D}{y}$}
     \label{fig:baseline-permanent-limit-longrun-DY}
         \centering
         \includesvg[inkscapelatex=false, width=\textwidth]{Results/DY_longrun.svg}
     \end{subfigure}

     \vspace{10pt}
 
     \begin{flushleft}
     \footnotesize
	\textit{Note}: The time horizon in all plots is fixed to $80$ quarters, including the initial steady state in $t=0$. 
	\end{flushleft}
\end{figure}

%%%%%%%%%%%%%%%%%%%%%%%%%%%%%%%%%%%%%%%%%%%%%%%%%%%%%%%%%%%%%%%%%%%%%%%%%%%%%
%%%%%%%%%%%%%%%%%%%%%%%%%%%%%%%%%%%%%%%%%%%%%%%%%%%%%%%%%%%%%%%%%%%%%%%%%%%%%
\subsection{Figures for Section \ref{sec:sensitivity-persistence}}
\label{sec-app:figures-persistence}

\begin{figure}[H]
    \centering
    \caption{Supplement for Figure \ref{fig:baseline-permanent-limit-duration}}
    \label{fig:baseline-permanent-limit-duration-extra}
     \begin{subfigure}[b]{0.49\textwidth}
     \caption{Borrowing Limit, $\phi$}
     \label{fig:baseline-permanent-limit-duration-extra-phi}
         \centering
         \includesvg[inkscapelatex=false, width=\textwidth]{Results/compare_transitions/comparison_phi_fast_shock_limit_permanent_baseline_limit_permanent_slow_shock_limit_permanent.svg}
     \end{subfigure}
     \hfill
     \begin{subfigure}[b]{0.49\textwidth}
     \caption{\textit{Ex-Post} Real Interest Rate, $RR$}
     \label{fig:baseline-permanent-limit-duration-Rr}
         \centering
         \includesvg[inkscapelatex=false, width=\textwidth]{Results/compare_transitions/comparison_Rr_fast_shock_limit_permanent_baseline_limit_permanent_slow_shock_limit_permanent.svg}
     \end{subfigure}

    \vspace{10pt}
     
     \justifying
     \footnotesize
	\textit{Note}: The blue dashed line corresponds to $\rho_{\phi} = 0.3$ ("fast shock"), the orange continuous line corresponds to $\rho_{\phi} = 0.5$ (baseline) and the green dotted line corresponds to $\rho_{\phi} = 0.65$ ("slow shock"). Percent deviations mean percent deviations from the respective initial steady state. The time horizon in all plots is fixed to 12 quarters, including the initial steady state in $t=0$.
\end{figure}

%%%%%%%%%%%%%%%%%%%%%%%%%%%%%%%%%%%%%%%%%%%%%%%%%%%%%%%%%%%%%%%%%%%%%%%%%%%%%
%%%%%%%%%%%%%%%%%%%%%%%%%%%%%%%%%%%%%%%%%%%%%%%%%%%%%%%%%%%%%%%%%%%%%%%%%%%%%
\subsection{Results for Section \ref{sec:sensitivity-end-labour}}
\label{sec-app:figures-end-L}

\begin{figure}[H]
    \caption{Endogenous Labour -- The Initial Steady State}
    \label{fig:init-stst-end-L}
    \centering
    \begin{subfigure}[b]{0.49\textwidth}
    \caption{Asset Distribution}
    \label{fig:init-stst-end-L-dist}
         \centering
         \includesvg[inkscapelatex=false, width=\textwidth]{Results/stst_dist_2d_bar_end_L_limit_permanent_initial.svg}
     \end{subfigure}
     \hfill
     \begin{subfigure}[b]{0.49\textwidth}
     \caption{Consumption Policies}
     \label{fig:init-stst-end-L-pol-c}
         \centering
         \includesvg[inkscapelatex=false, width=\textwidth]{Results/stst_policies_c_end_L_limit_permanent_initial.svg}
     \end{subfigure}

    \vspace{10pt}
     
     \justifying
     \footnotesize
	\textit{Note}: In figure \ref{fig:init-stst-end-L-dist}, the distribution is cut off at an asset level of $50$ for readability. As the model is solved on a logarithmic grid, grid points are distributed unevenly. Grid points are very dense at the lower end of the grid, which makes the colour appear bleak. At high levels of assets, grid points are very sparse, overestimating the true density at high bond holdings. The size of the bar at the borrowing limit is exaggerated. In figure \ref{fig:init-stst-end-L-pol-n}, the labels $\theta\_0$ -- $\theta\_7$ represent the idiosyncratic labour skill states from \eqref{eq:hh-income-process}.
\end{figure}

\begin{figure}[H]
    \centering
    \caption{Endogenous Labour -- Labour Supply Policies}
    \label{fig:init-stst-end-L-pol-n}
    \includesvg[inkscapelatex=false, width=0.75\textwidth]{Results/stst_policies_n_end_L_limit_permanent_initial.svg}

    \vspace{10pt}
    
    \begin{minipage}{0.75\textwidth}
    \footnotesize
    \textit{Note}: The labels $\theta\_0$ -- $\theta\_7$ represent the idiosyncratic labour skill states from \eqref{eq:hh-income-process}.
    \end{minipage}
\end{figure}

\begin{table}[H]
\centering
\caption{Endogenous Labour -- Shock to the Borrowing Limit \\ Comparison of Steady States}
\label{tab:stst_comparison_end_L_limit_permanent}
\begin{tabular}{lccc}
                Variable & Initial & Terminal &  Change \\
\hline
\hline
                       $C$ &   0.944 &   0.9438 & -0.0212 \\
                       D &  0.5381 &   0.4909 & -8.7716 \\
                      DY &    0.57 &   0.5201 & -0.0499 \\
               gr\_liquid &  5.9281 &   5.8809 & -0.7962 \\
                     MPC &  0.1386 &   0.1404 &  0.0018 \\
                       N &   0.944 &   0.9438 & -0.0212 \\
                     phi & -1.7956 &   -1.655 &  0.1406 \\
$Rr-1$ &  1.0015 &   1.0014 & -0.0001 \\
                     tau &  0.0082 &   0.0074 & -0.0008 \\
                  Top10A &  0.5322 &   0.5264 & -0.0058 \\
                  Top10C &  0.1563 &   0.1564 &  0.0001 \\
                  Top10N &  0.2123 &   0.2121 & -0.0002 \\
      Frac. of Borrowers &   39.14 &    38.46 & -0.6800 \\
Frac. at Borrowing Limit &   11.21 &    11.24 &  0.0300 \\
        MPC of Borrowers &    0.33 &     0.34 &  0.0100 \\
          MPC of Lenders &    0.02 &     0.02 &  0.0000 \\
\hline
\end{tabular}
\end{table}

\begin{figure}[H]
    \caption{Supplement for Figure \ref{fig:comparison-end-labour-agg}}%{\hypersetup{hidelinks}\ref{fig:baseline-permanent-limit-agg}}}
    \label{fig:comparison-end-labour-agg-extra}
    \begin{subfigure}[b]{0.49\textwidth}
     \caption{Debt, $D$}
     \label{fig:comparison-end-labour-agg-extra-DY}
         \centering
         \includesvg[inkscapelatex=false, width=\textwidth]{Results/compare_transitions/comparison_D_baseline_limit_permanent_end_L_limit_permanent.svg}
     \end{subfigure}
     \hfill
     \begin{subfigure}[b]{0.49\textwidth}
     \centering
     \caption{Inflation, $\pi$}
     \label{fig:comparison-end-labour-agg-extra-pi}
         \includesvg[inkscapelatex=false, width=\textwidth]{Results/compare_transitions/comparison_pi_baseline_limit_permanent_end_L_limit_permanent.svg}
     \end{subfigure}
     \hfill
    \begin{subfigure}[b]{0.49\textwidth}
     \caption{\textit{Ex-Post} Real Interest Rate, $RR$}
     \label{fig:comparison-end-labour-agg-extra-Rr}
         \centering
         \includesvg[inkscapelatex=false, width=\textwidth]{Results/compare_transitions/comparison_Rr_baseline_limit_permanent_end_L_limit_permanent.svg}
     \end{subfigure}
     \hfill
     \begin{subfigure}[b]{0.49\textwidth}
     \caption{Nominal and Notional Interest Rates, $R$ and $R^n$}
     \label{fig:comparison-end-labour-agg-extrag-Rn}
         \centering
         \includesvg[inkscapelatex=false, width=\textwidth]{Results/compare_transitions/comparison_R_Rn_baseline_limit_permanent_end_L_limit_permanent.svg}
     \end{subfigure}
     \hfill
    \begin{subfigure}[b]{0.49\textwidth}
     \caption{Labour Hours, $N$}
     \label{fig:comparison-end-labour-agg-extra-N}
         \centering
         \includesvg[inkscapelatex=false, width=\textwidth]{Results/compare_transitions/comparison_N_baseline_limit_permanent_end_L_limit_permanent.svg}
     \end{subfigure}
     \hfill
     \begin{subfigure}[b]{0.49\textwidth}
     \caption{Wage, $w$}
     \label{fig:comparison-end-labour-agg-extrag-w}
         \centering
         \includesvg[inkscapelatex=false, width=\textwidth]{Results/compare_transitions/comparison_w_baseline_limit_permanent_end_L_limit_permanent.svg}
     \end{subfigure}

     \vspace{10pt}

     \begin{flushleft}
     \footnotesize
	\textit{Note}: Percent deviations mean percent deviations from the respective initial steady state. The time horizon in all plots is fixed to 12 quarters, including the initial steady state in $t=0$.
	\end{flushleft}
\end{figure}

%%%%%%%%%%%%%%%%%%%%%%%%%%%%%%%%%%%%%%%%%%%%%%%%%%%%%%%%%%%%%%%%%%%%%%%%%%%%%
%%%%%%%%%%%%%%%%%%%%%%%%%%%%%%%%%%%%%%%%%%%%%%%%%%%%%%%%%%%%%%%%%%%%%%%%%%%%%
\subsection{Figures for Section \ref{sec:wedge}}
\label{sec-app:figures-wedge}

\begin{figure}[H]
    \caption{Supplement for Figure \ref{fig:baseline-permanent-wedge-agg}}%{\hypersetup{hidelinks}\ref{fig:baseline-permanent-limit-agg}}}
    \label{fig:baseline-permanent-wedge-agg-extra}
    \begin{subfigure}[b]{0.49\textwidth}
     \caption{Debt-to-Output, $\frac{D}{y}$}
     \label{fig:baseline-permanent-wedge-agg-DY}
         \centering
         \includesvg[inkscapelatex=false, width=\textwidth]{Results/compare_transitions/comparison_DY_baseline_limit_permanent_baseline_wedge_permanent.svg}
     \end{subfigure}
     \hfill
     \begin{subfigure}[b]{0.49\textwidth}
     \centering
     \caption{Inflation, $\pi$}
     \label{fig:baseline-permanent-wedge-agg-pi}
         \includesvg[inkscapelatex=false, width=\textwidth]{Results/compare_transitions/comparison_pi_baseline_limit_permanent_baseline_wedge_permanent.svg}
     \end{subfigure}
     \hfill
    \begin{subfigure}[b]{0.49\textwidth}
     \caption{\textit{Ex-Post} Real Interest Rate, $RR$}
     \label{fig:baseline-permanent-wedge-agg-Rr}
         \centering
         \includesvg[inkscapelatex=false, width=\textwidth]{Results/compare_transitions/comparison_Rr_baseline_limit_permanent_baseline_wedge_permanent.svg}
     \end{subfigure}
     \hfill
     \begin{subfigure}[b]{0.49\textwidth}
     \caption{Nominal and Notional Interest Rates, $R$ and $R^n$}
     \label{fig:baseline-permanent-wedge-agg-Rn}
         \centering
         \includesvg[inkscapelatex=false, width=\textwidth]{Results/compare_transitions/comparison_R_Rn_baseline_limit_permanent_baseline_wedge_permanent.svg}
     \end{subfigure}

     \vspace{10pt}

     \begin{flushleft}
     \footnotesize
	\textit{Note}: Percent deviations mean percent deviations from the respective initial steady state. The time horizon in all plots is fixed to 12 quarters, including the initial steady state in $t=0$.
	\end{flushleft}
\end{figure}

%%%%%%%%%%%%%%%%%%%%%%%%%%%%%%%%%%%%%%%%%%%%%%%%%%%%%%%%%%%%%%%%%%%%%%%%%%%%%
%%%%%%%%%%%%%%%%%%%%%%%%%%%%%%%%%%%%%%%%%%%%%%%%%%%%%%%%%%%%%%%%%%%%%%%%%%%%%
%%%%%%%%%%%%%%%%%%%%%%%%%%%%%%%%%%%%%%%%%%%%%%%%%%%%%%%%%%%%%%%%%%%%%%%%%%%%%
%\newpage
\thispagestyle{plain}
\section{Robustness and Supplementary Experiments}
\label{sec-app:robust}

This appendix presents various robustness exercises and supplementary experiments alluded to in the main text.

%%%%%%%%%%%%%%%%%%%%%%%%%%%%%%%%%%%%%%%%%%%%%%%%%%%%%%%%%%%%%%%%%%%%%%%%%%%%%
%%%%%%%%%%%%%%%%%%%%%%%%%%%%%%%%%%%%%%%%%%%%%%%%%%%%%%%%%%%%%%%%%%%%%%%%%%%%%
\subsection{Abscence of the \Gls{zlb}}
\label{sec-app:robust-no-zlb}

This appendix conducts the same exercise as section \ref{sec:limit} does, but exchanges \eqref{eq:zlb} for:
\begin{equation}
    R_t = R^n_t, \label{eq:no-zlb}
\end{equation}
i.e.~the modified model ignores the \Gls{zlb} on nominal interest rates. I do so in order to gauge the effect that the \Gls{zlb} has on the quantitative results from section \ref{sec:limit}.

%changes the calibration of the model, in particular the value of $\beta$, such that the \Gls{zlb} on the nominal interest rate does not bind during the transition to tighter credit.

%As \textcite{gl2017} point out, the interest rate elasticity might still be quite high, so that the \Gls{zlb} binding or not matters a great deal.

%%%%%%%%%%%%%%%%%%%%%%%%%%%%%%%%%%%%%%%%%%%%%%%%%%%%%%%%%%%%%%%%%%%%%%%%%%%%%
%%%%%%%%%%%%%%%%%%%%%%%%%%%%%%%%%%%%%%%%%%%%%%%%%%%%%%%%%%%%%%%%%%%%%%%%%%%%%
\subsection{Calibration of Bonds à la \textcite{bayer2023}}
\label{sec-app:robust-bonds}

This appendix re-calibrates $B$ following the approach of \textcite{bayer2023}. Specifically, $B$ here targets the ratio of the market value of federal government debt to \Gls{gdp} instead of the fraction of total liquid assets over \Gls{gdp}. To that end, I follow \textcite{bayer2023} by using the exact same use data series from FRED as these authors do, see appendix \ref{sec-app:data} for details of how to obtain the data. The result is that $B$ has to be set to $2.24$ to match the empirical ratio of $2.45$. What is more, I re-set $\beta$ so that the steady state (real) interest rate is $0.15\%$, as before.

Now, \Gls{iou}s make up a larger share of gross liquidity. 

Clearly, the steady-state distributions are different. \Gls{mpc}s are much higher, consistent with the observation by \textcite{kaplan2018} that \Gls{mpc}s rise with less available liquidity.

%%%%%%%%%%%%%%%%%%%%%%%%%%%%%%%%%%%%%%%%%%%%%%%%%%%%%%%%%%%%%%%%%%%%%%%%%%%%%
%%%%%%%%%%%%%%%%%%%%%%%%%%%%%%%%%%%%%%%%%%%%%%%%%%%%%%%%%%%%%%%%%%%%%%%%%%%%%
\end{appendices}
\newpage
\thispagestyle{plain}
\pagenumbering{arabic}  % Turn page numbering to Roman
\renewcommand*{\thepage}{A-\arabic{page}} % Add 'A' to each page number for appendices section
\printbibliography[heading=subbibliography, title={Appendix References}]
\addcontentsline{toc}{section}{Appendix References}
\end{refsection}

%%%%%%%%%%%%%%%%%%%%%%%%%%%%%%%%%%%%%%%%%%%%%%%%%%%%%%%%%%%%%%%%%%%%%%%%%%%%%
%%%%%%%%%%%%%%%%%%%%%%%%%%%%%%%%%%%%%%%%%%%%%%%%%%%%%%%%%%%%%%%%%%%%%%%%%%%%%
%\end{refsection}

%%%%%%%%%%%%%%%%%%%%%%%%%%%%%%%%%%%%%%%%%%%%%%%%%%%%%%%%%%%%%%%%%%%%%%%%%%%%%
%%%%%%%%%%%%%%%%%%%%%%%%%%%%%%%%%%%%%%%%%%%%%%%%%%%%%%%%%%%%%%%%%%%%%%%%%%%%%
%%%%%%%%%%%%%%%%%%%%%%%%%%%%%%%%%%%%%%%%%%%%%%%%%%%%%%%%%%%%%%%%%%%%%%%%%%%%%
\newpage
\thispagestyle{plain}
\pagenumbering{gobble} % Turn page numbering off
\section*{Statement of Authorship} % Include statement of authorship
I hereby confirm that the work presented has been performed and interpreted solely by myself except for where I explicitly identified the contrary. I assure that this work has not been presented in any other form for the fulfillment of any other degree or qualification. Ideas taken from other works in letter and in spirit are identified in every single case.

\vspace{2cm}
\noindent
\rule{8cm}{0.4pt}\\
Andreas Koundouros\\
Bonn, the \printdate{2023-8-25}
\end{document}
% End Document
%%%%%%%%%%%%%%%%%%%%%%%%%%%%%%%%%%%%%%%%%%%%%%%%%%%%%%%%%%%%%%%%%%%%%%%%%%%%%
%%%%%%%%%%%%%%%%%%%%%%%%%%%%%%%%%%%%%%%%%%%%%%%%%%%%%%%%%%%%%%%%%%%%%%%%%%%%%
%%%%%%%%%%%%%%%%%%%%%%%%%%%%%%%%%%%%%%%%%%%%%%%%%%%%%%%%%%%%%%%%%%%%%%%%%%%%%
%%%%%%%%%%%%%%%%%%%%%%%%%%%%%%%%%%%%%%%%%%%%%%%%%%%%%%%%%%%%%%%%%%%%%%%%%%%%%
%%%%%%%%%%%%%%%%%%%%%%%%%%%%%%%%%%%%%%%%%%%%%%%%%%%%%%%%%%%%%%%%%%%%%%%%%%%%%
%%%%%%%%%%%%%%%%%%%%%%%%%%%%%%%%%%%%%%%%%%%%%%%%%%%%%%%%%%%%%%%%%%%%%%%%%%%%%
