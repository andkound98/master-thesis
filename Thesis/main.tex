\documentclass[a4paper,12pt]{article} % Set documentclass article and set font size

%%%%%%%%%%%%%%%%%%%%%%%%%%%%%%%%%%%%%%%%%%%%%%%%%%%%%%%%%%%%%%%%%%%%%%%%%%%%%
%%%%%%%%%%%%%%%%%%%%%%%%%%%%%%%%%%%%%%%%%%%%%%%%%%%%%%%%%%%%%%%%%%%%%%%%%%%%%
%%%%%%%%%%%%%%%%%%%%%%%%%%%%%%%%%%%%%%%%%%%%%%%%%%%%%%%%%%%%%%%%%%%%%%%%%%%%%
% Packages
\usepackage{fontspec} % If used, change compiler to XeLaTex
\usepackage[utf8]{inputenc}
\usepackage[british]{babel}
\usepackage[british]{isodate}
\usepackage{csquotes}
\usepackage{mathtools}
\usepackage{amssymb}
\usepackage{unicode-math}
\usepackage{geometry}
\usepackage[
backend=biber,
style=apa,
sorting=nyt, % Sort by name, year, title
backref=true,
dateabbrev=false,
language=british,
urldate=long
]{biblatex}
\usepackage{xcolor}
\usepackage[toc,page]{appendix}
\usepackage{svg}
\usepackage{float}
\usepackage{longtable}
\usepackage{multirow}
\usepackage{multirow,multicol,makecell,booktabs}
\usepackage[justification=centering]{caption}
\usepackage{subcaption}
\usepackage{setspace}
\usepackage{ragged2e}
\usepackage{fancyhdr}
\usepackage{enumitem}
\usepackage{tikz}
\usepackage{rotating}
\usepackage{hyperref} % Load after everything else
\usepackage[symbols, nopostdot, automake, nonumberlist]{glossaries}

%%%%%%%%%%%%%%%%%%%%%%%%%%%%%%%%%%%%%%%%%%%%%%%%%%%%%%%%%%%%%%%%%%%%%%%%%%%%%
%%%%%%%%%%%%%%%%%%%%%%%%%%%%%%%%%%%%%%%%%%%%%%%%%%%%%%%%%%%%%%%%%%%%%%%%%%%%%
%%%%%%%%%%%%%%%%%%%%%%%%%%%%%%%%%%%%%%%%%%%%%%%%%%%%%%%%%%%%%%%%%%%%%%%%%%%%%
% Custom Settings
% \DeclareFieldFormat{doi}{% % put doi into new line in references
%   \newline
%   \mkbibacro{DOI}\addcolon\space
%     \ifhyperref
%       {\href{http://dx.doi.org/#1}{\nolinkurl{#1}}}
%       {\nolinkurl{#1}}}

\addbibresource{master_thesis.bib} % Add bibliography source

\setlength\bibitemsep{1.5\itemsep} % Adjust distance of reference entries

\renewcommand*\appendixpagename{\Large Appendices} % Adjust size of title of the appendices section

\renewcommand*{\finentrypunct}{}
\renewbibmacro*{pageref}{%
  \addperiod% NEW
  \iflistundef{pageref}
    {}
    {\newline\footnotesize\printtext[parens]{%
       \ifnumgreater{\value{pageref}}{1}
         {\bibstring{backrefpages}\ppspace}
     {\bibstring{backrefpage}\ppspace}%
       \printlist[pageref][-\value{listtotal}]{pageref}}}}%
\DefineBibliographyStrings{english}{
  backrefpage={Cited on page},
  backrefpages={Cited on pages}
}

\setmainfont{Times New Roman} % If the usepackage fontspec is active, you can set Times New Roman here

\geometry{ % Adjust paper geometry
 a4paper,
 right=25mm,
 bottom=20mm,
 left=25mm,
 top=25mm}
\urlstyle{same} % Make URL font the same with the rest of the document
\onehalfspacing % Adjust spacing
%\setstretch{1.75}
%\doublespacing
\allowdisplaybreaks % Allow for breaks in equations across pages
\addto\captionsbritish{ % Replace "english" with the language you use
  \renewcommand{\contentsname}%
    {Table of Contents}%
} % Adjust title of the table of contents
\numberwithin{equation}{section} % Number equations with sections
\numberwithin{figure}{section}
\numberwithin{table}{section}
%\setcounter{tocdepth}{2} % Set maximum level to which sections are displayed in the table of contents

\newcommand{\appendixsubsection}[1]{
    \stepcounter{subsection}
    \subsection*{\Alph{section}.\arabic{subsection}\hspace{1em}{#1}}
}


\pagestyle{plain}

% Define a custom name format that prints names of authors in references in bold
\DeclareNameWrapperFormat{sortname}{\mkbibbold{#1}}
\DeclareNameWrapperAlias{author}{sortname}

% Add definition format
\newtheorem{definition}{Definition}

% Make links of glossary entries black
\makeatletter
\newcommand*{\glsplainhyperlink}[2]{%
    \begingroup%
      \hypersetup{hidelinks}%
      \hyperlink{#1}{#2}%
    \endgroup%
}
\let\@glslink\glsplainhyperlink
\makeatother

%%%%%%%%%%%%%%%%%%%%%%%%%%%%%%%%%%%%%%%%%%%%%%%%%%%%%%%%%%%%%%%%%%%%%%%%%%%%%
%%%%%%%%%%%%%%%%%%%%%%%%%%%%%%%%%%%%%%%%%%%%%%%%%%%%%%%%%%%%%%%%%%%%%%%%%%%%%
%%%%%%%%%%%%%%%%%%%%%%%%%%%%%%%%%%%%%%%%%%%%%%%%%%%%%%%%%%%%%%%%%%%%%%%%%%%%%
% Glossary
\makeglossaries 

% Glossary entries
\newglossaryentry{nk}{
name=NK,
description={New Keynesian}
}

\newglossaryentry{ha}{
name=HA,
description={Heterogeneous Agents}
}

\newglossaryentry{hank}{
name=HANK,
description={Heterogeneous Agent New Keynesian}
}

\newglossaryentry{gfc}{
name=GFC,
description={Great Financial Crisis}
}

\newglossaryentry{ltv}{
name=LTV,
description={Loan-to-Value}
}

\newglossaryentry{dsge}{
name=DSGE,
description={Dynamic Stochastic General Equilibrium}
}

\newglossaryentry{egm}{
name=EGM,
description={Endogenous Gridpoints Method}
}

\newglossaryentry{iou}{
name=IOU,
description={I Owe You}
}

\newglossaryentry{ghh}{
name=GHH,
description={Greenwood-Hercowitz-Huffman (\cite{ghh1988})}
}

\newglossaryentry{ar}{
name=AR,
description={Autoregressive}
}

\newglossaryentry{zlb}{
name=ZLB,
description={Zero Lower Bound}
}

\newglossaryentry{mpc}{
name=MPC,
description={Marginal Propensity to Consume}
}

\newglossaryentry{fof}{
name=FoF,
description={Flow of Funds}
}

\newglossaryentry{nkpc}{
name=NKPC,
description={New Keynesian Phillips Curve}
}

\newglossaryentry{foc}{
name=FOC,
description={First Order Condition}
}

\newglossaryentry{irf}{
name=IRF,
description={Impulse Response Function}
}

\newglossaryentry{crra}{
name=CRRA,
description={Constant Relative Risk Aversion}
}

\newglossaryentry{gdp}{
name=GDP,
description={Gross Domestic Product}
}

%%%%%%%%%%%%%%%%%%%%%%%%%%%%%%%%%%%%%%%%%%%%%%%%%%%%%%%%%%%%%%%%%%%%%%%%%%%%%
%%%%%%%%%%%%%%%%%%%%%%%%%%%%%%%%%%%%%%%%%%%%%%%%%%%%%%%%%%%%%%%%%%%%%%%%%%%%%
%%%%%%%%%%%%%%%%%%%%%%%%%%%%%%%%%%%%%%%%%%%%%%%%%%%%%%%%%%%%%%%%%%%%%%%%%%%%%
% hyperref package
\usepackage{footnotebackref} % Brings you back to where the footnote is in the text
\hypersetup{ 
    colorlinks,
    linkcolor=blue,
    filecolor=blue,  
    citecolor=blue,
    urlcolor=blue} % Setup with the hyperref package

%%%%%%%%%%%%%%%%%%%%%%%%%%%%%%%%%%%%%%%%%%%%%%%%%%%%%%%%%%%%%%%%%%%%%%%%%%%%%
%%%%%%%%%%%%%%%%%%%%%%%%%%%%%%%%%%%%%%%%%%%%%%%%%%%%%%%%%%%%%%%%%%%%%%%%%%%%%
%%%%%%%%%%%%%%%%%%%%%%%%%%%%%%%%%%%%%%%%%%%%%%%%%%%%%%%%%%%%%%%%%%%%%%%%%%%%%
% Formalities
\title{Financial Constraints and Household Heterogeneity\\in the Macroeconomy}
\author{}
\date{}

%%%%%%%%%%%%%%%%%%%%%%%%%%%%%%%%%%%%%%%%%%%%%%%%%%%%%%%%%%%%%%%%%%%%%%%%%%%%%
%%%%%%%%%%%%%%%%%%%%%%%%%%%%%%%%%%%%%%%%%%%%%%%%%%%%%%%%%%%%%%%%%%%%%%%%%%%%%
%%%%%%%%%%%%%%%%%%%%%%%%%%%%%%%%%%%%%%%%%%%%%%%%%%%%%%%%%%%%%%%%%%%%%%%%%%%%%
%%%%%%%%%%%%%%%%%%%%%%%%%%%%%%%%%%%%%%%%%%%%%%%%%%%%%%%%%%%%%%%%%%%%%%%%%%%%%
%%%%%%%%%%%%%%%%%%%%%%%%%%%%%%%%%%%%%%%%%%%%%%%%%%%%%%%%%%%%%%%%%%%%%%%%%%%%%
%%%%%%%%%%%%%%%%%%%%%%%%%%%%%%%%%%%%%%%%%%%%%%%%%%%%%%%%%%%%%%%%%%%%%%%%%%%%%
% Begin Document
\begin{document}
\selectlanguage{british}
\pagenumbering{gobble} % Turn page numbering off
\maketitle % Create title

\vspace{4cm}
\begin{center}
Master Thesis Presented to the\\
Department of Economics at the\\
Rheinische Friedrich-Wilhelms-Universität Bonn\\
\vspace{1cm}
In Partial Fulfillment of the Requirements for the Degree of\\
Master of Science (M.Sc.)
\end{center}

\vspace{5cm}
\begin{center}
\begin{tabular}{ l l }
Supervisors: & Prof.~Dr.~Keith Kuester \\
& Dr.~Gregor Böhl \\
& \\
& \\
Date of Submission: & \printdate{2023-8-25}\\
Author: & Andreas Koundouros\\
Matriculation Number: & 3466868
\end{tabular}
\end{center}

%%%%%%%%%%%%%%%%%%%%%%%%%%%%%%%%%%%%%%%%%%%%%%%%%%%%%%%%%%%%%%%%%%%%%%%%%%%%%
%%%%%%%%%%%%%%%%%%%%%%%%%%%%%%%%%%%%%%%%%%%%%%%%%%%%%%%%%%%%%%%%%%%%%%%%%%%%%
%%%%%%%%%%%%%%%%%%%%%%%%%%%%%%%%%%%%%%%%%%%%%%%%%%%%%%%%%%%%%%%%%%%%%%%%%%%%%
\newpage
\pagenumbering{roman} % Turn page numbering to small roman
\setcounter{tocdepth}{2}
{ \hypersetup{hidelinks} \tableofcontents } % Hide links in table of contents

\newpage
{ \hypersetup{hidelinks} \listoffigures } % Hide links in list of figures

\newpage
{ \hypersetup{hidelinks} \listoftables } % Hide links in list of tables

% Add here list of acronyms
\newpage
\setlist[description]{leftmargin=!, labelwidth=5em} % Change for glossaries
\printglossary[title=List of Acronyms]
\setlist[description]{style=standard} % Reset settings back to default


%%%%%%%%%%%%%%%%%%%%%%%%%%%%%%%%%%%%%%%%%%%%%%%%%%%%%%%%%%%%%%%%%%%%%%%%%%%%%
%%%%%%%%%%%%%%%%%%%%%%%%%%%%%%%%%%%%%%%%%%%%%%%%%%%%%%%%%%%%%%%%%%%%%%%%%%%%%
%%%%%%%%%%%%%%%%%%%%%%%%%%%%%%%%%%%%%%%%%%%%%%%%%%%%%%%%%%%%%%%%%%%%%%%%%%%%%
\newpage
\begin{abstract} % Create abstract

Household debt not only sustains borrowers' consumption but also provides lenders with liquidity. Tightening the borrowing capacity of debtors thus reduces the demand for goods and the supply of assets. How do households and the economy react to such a ``credit crunch"? In a small-scale \Gls{hank} model, I study two shocks leading to a credit crunch: a contraction in the debt limit and an increase in the borrowing premium. I find that the former disturbance produces the mechanisms identified in the literature: interest rates fall but not sufficiently to avert a recession in the short run. This is because, in the presence of incomplete markets, mechanical deleveraging by constrained borrowers is accompanied by an invigorated precautionary motive by unconstrained agents. The shock to the credit spread operates similarly but induces less pronounced short-run adjustments as it works only in tandem with the real interest rate. Both scenarios predict reduced wealth and higher consumption inequality in the long run. Finally, I document an interesting asymmetry: financial tightenings are more recessionary than financial easings are expansionary.

\begin{center}
\rule{10cm}{0.4pt}
\end{center}

\noindent
\textit{JEL Classification:} E12, E21, E32 % JEL topic classification of paper

\noindent
\textit{Keywords:} Borrowers and Lenders, Household Debt, Deleveraging, \Gls{hank} % Keywords of paper
\end{abstract}

%%%%%%%%%%%%%%%%%%%%%%%%%%%%%%%%%%%%%%%%%%%%%%%%%%%%%%%%%%%%%%%%%%%%%%%%%%%%%
%%%%%%%%%%%%%%%%%%%%%%%%%%%%%%%%%%%%%%%%%%%%%%%%%%%%%%%%%%%%%%%%%%%%%%%%%%%%%
%%%%%%%%%%%%%%%%%%%%%%%%%%%%%%%%%%%%%%%%%%%%%%%%%%%%%%%%%%%%%%%%%%%%%%%%%%%%%
\newpage
\pagenumbering{arabic} % Turn page numbering to small arabic
\pagestyle{fancy}
\fancyhf{}
\fancyhead[C]{\leftmark}
\fancyfoot[C]{\thepage}

%%%%%%%%%%%%%%%%%%%%%%%%%%%%%%%%%%%%%%%%%%%%%%%%%%%%%%%%%%%%%%%%%%%%%%%%%%%%%
%%%%%%%%%%%%%%%%%%%%%%%%%%%%%%%%%%%%%%%%%%%%%%%%%%%%%%%%%%%%%%%%%%%%%%%%%%%%%
%%%%%%%%%%%%%%%%%%%%%%%%%%%%%%%%%%%%%%%%%%%%%%%%%%%%%%%%%%%%%%%%%%%%%%%%%%%%%
\thispagestyle{plain}
\section{Introduction}
\label{sec:introduction}

The ups and downs of life prompt households to differ in their asset positions. Some households are net lenders, i.e.~their positive asset holdings exceed their liabilities, while others happen to be net borrowers, i.e.~they owe more in debt than they own in assets. Crucially, the debt holdings of borrowers are the lenders' asset holdings, reflecting the fact that debt provides creditors with the liquidity they demand. Thus, the balance sheets of the two groups of households are linked, which in turn suggests that their respective merits are closely intertwined.\footnote{Note two things. First, over time, individuals might switch from being a borrower to being a lender and vice versa. Second, entities other than households can intermediate funds from and to the household sector, e.g.~financial institutions. However, I focus on the intermediation of credit between households. The latter can be thought of as frictionless intermediation through perfectly competitive banks.}

Interconnected balance sheets imply that the consequences of a sudden tightening in the financial conditions under which borrowers obtain credit -- caused e.g.~by a loss of confidence in borrowers' ability to repay -- will not only be borne by the indebted households themselves but by the lending households as well. Thus, a so-called "credit crunch" will affect individuals with different asset positions and unequal abilities to absorb shocks. This raises the question as to whether and how such financial shocks impact not only on the consumption, lending and borrowing decisions of households, but also on the macroeconomy as a whole. 

Hence, this thesis asks: what are the short- and long-run consequences of a credit crunch? In particular, how does a financial shock work through borrowers' and lenders' balance sheets, respectively, and how does this impact, first, the two groups' consumption and savings decisions and, second, the aggregate state of the economy? In other words, how are consumption and wealth redistributed and how are macroeconomic variables disturbed when access to finance is suddenly curtailed?

To answer these questions, I follow \textcite{gl2017} by employing a heterogeneous-agent (\Gls{ha}) model, which naturally lends itself to the analysis of household-level financial shocks since borrowing and lending emerge from it endogenously and it allows for precautionary-savings motives to arise. With a small-scale New Keynesian (\Gls{nk}) state space the model accounts for the standard mechanisms of contemporary monetary economics. In the resulting heterogeneous-agent New Keynesian (\Gls{hank}) model, I incorporate two disturbances to households' debt capacity: shocks to the borrowing limit and to the borrowing premium.

I present three key results. First, confronting this model with a shock to the borrowing limit confirms the presence of the general mechanisms identified in the theoretical literature on credit tightenings. Second, I demonstrate how an equally sized shock to the borrowing spread induces qualitatively similar responses but that quantitative differences do emerge. Third, I find that financial shocks can operate asymmetrically in the present framework.

In more detail, the first result derives from reducing -- unexpectedly and permanently -- the household-level borrowing limit in a plausibly calibrated version of the model. This shock forces constrained borrowers to delever mechanically (a mechanism I call \textit{balance-sheet effect}) and elevates the fear of reaching the now more imminent constraint among the unconstrained households (a \textit{precautionary-savings effect}). Both of these forces prompt agents -- the asset-poor in particular -- to cut back on consumption. To alleviate the pressure from the shrunken supply of and the increased demand for liquidity, interest rates must fall. In the long-run, this reduced rates give rise to lower wealth inequality but higher dispersion in consumption. In the short-run of the adjustment, the economy enters recession with consumption falling strongly and the zero-lower bound (\Gls{zlb}) on nominal rates potentially inhibiting the central bank to stimulate the recovery as needed. Interestingly, outstanding debt takes exceptionally long to fall to its new steady state in the present framework.

This type of a credit crunch has been studied in the theoretical literature before, most prominently by \textcite{egg2012} and \textcite{gl2017}. Both papers show how such a shock provokes depressed levels of consumption and interest rates close or at the \Gls{zlb}, macroeconomic features that resemble those witnessed during the Great Recession of 2007--2009. Both studies emphasise the endogenous reaction of the interest rate in the economy's process of adjustment, while \textcite{gl2017} highlight how rich household heterogeneity gives rise to the precautionary-savings channel of a credit crunch.

Two related sensitivity analyses investigate how the speed of the transition to tight credit and the preference specification for households influence the results. The results are as expected. The slower financial shocks evolve, the more mildly the economy reacts and allowing agents to individually adjust their labour hours provides them with a quantitatively important tool to absorb shocks to their balance sheets.

The second main result of this thesis derives from the fact that, unlike a credit crunch through a tighter borrowing limit, an increase in the credit spread works exclusively in tandem with the real rate on lending. While triggering overall similar mechanisms -- especially in the long run -- this dampens the economic fallout of this kind of deleveraging shock.

Finally, my third result establishes that the present model is capable of producing at least some form of asymmetry in its responses to financial shocks. That is, I ask whether the recession caused by the transition from loose to tight credit is of the same absolute magnitude as the boom triggered by the reverse movement, i.e.~a credit \textit{easing}. The answer is no: a credit loosening is much less expansionary than a credit crunch is recessionary. Intuitively, an easing in financial constraints leaves households more flexibility in their response than a tightening does.

Before proceeding, be aware of the following caveat: this thesis is \textit{not} about identifying the roots of the Great Recession. In fact, the present investigation is entirely agnostic about the origins of financial constraints and the causes of why they might change over time. Rather, the focus is on the \textit{effects} that exogenous variation in these constraints has on household decisions and aggregates. Of course, understanding the mechanisms of a credit crunch is not only an interesting theoretical exercise in and for itself but should, in extension, inform empirical and policy-orientated work. This is important -- independent of whether these effects resemble stylised facts of the Great Recession or not -- but beyond the scope of the present thesis.\footnote{Indeed, even if responses coincide with what was witnessed during the Great Recession, this can not be taken as evidence in support of the hypothesis that financial shocks caused the Great Recession. Whether financial shocks were the main cause of the Great Recession or merely an accelerator of events is still a matter of discussion in the literature. For example, \textcite{justiniano2015} show that, in their two-agent model, house prices are the more likely driver of the events of the Great Recession than changes in borrowing constraints.}

%%%%%%%%%%%%%%%%%%%%%%%%%%%%%%%%%%%%%%%%%%%%%%%%%%%%%%%%%%%%%%%%%%%%%%%%%%%%%
\subsubsection*{Outline}
\label{sec:outline}
The remainder of this thesis is structured as follows. Section \ref{sec:literature} briefly reviews the literature on financial constraints and heterogeneous households from a macroeconomic perspective. Then, section \ref{sec:model} lays down the \Gls{hank} model with a focus on borrowers and lenders, their interactions and characteristics as well as the included financial constraints and shocks. Section \ref{sec:model} also deals with the solution method of the model and its calibration. Section \ref{sec:limit} studies the long- and short-term effects of a permanent tightening in the household-level borrowing constraint. Section \ref{sec:sensitivity} presents related sensitivity analyses. Subsequently, section \ref{sec:wedge} studies the increase in the interest rate wedge and section \ref{sec:asymmetry} explores the asymmetry of financial shocks. Section \ref{sec:conclusion} concludes.

%%%%%%%%%%%%%%%%%%%%%%%%%%%%%%%%%%%%%%%%%%%%%%%%%%%%%%%%%%%%%%%%%%%%%%%%%%%%%
%%%%%%%%%%%%%%%%%%%%%%%%%%%%%%%%%%%%%%%%%%%%%%%%%%%%%%%%%%%%%%%%%%%%%%%%%%%%%
%%%%%%%%%%%%%%%%%%%%%%%%%%%%%%%%%%%%%%%%%%%%%%%%%%%%%%%%%%%%%%%%%%%%%%%%%%%%%
\section{Related Literature}
\label{sec:literature}

Since at least the Great Depression of the 1930s, economists have been thinking about the role of finance and indebtedness in the macroeconomy. For example, the work of \textcite{fisher1933} points to the debt-deflation spiral as one of the key drivers of the prolonged economic weakness caused by the Great Depression. In that view, nominal debt has to be reduced once it proves unsustainably high. However, this deleveraging kicks off a vicious circle of reduced consumption, deflation, a mechanic increase in the real debt burden, which finally aggravates the need for deleveraging.

Later, formal macroeconomic models captured financial frictions on the \textit{firm} side or in the \textit{banking} sector of the economy. The typical approach within the former model class is to let entrepreneurs have private information over their projects which their lenders can not observe \textit{ex ante}, implying that the net worth of companies plays a role in their access to credit (see e.g.~\cite{bg1989}, \cite{bgg1999}). Motivated by the events of the Great Recession, the literature introduced constraints to the leverage of banks, which are argued to arise due to the possibility of banks' managers to liquidate the bank (see e.g.~\cite{gertler2011}). Studies in both directions typically find that such financial frictions prove to be accelerators of an economy's reaction to a variety of shocks.\footnote{See \textcite{brunnermeier2012} and \textcite{christiano2022} for surveys on financial frictions in macroeconomics.}

Concurrently, the literature has increasingly turned to \textit{household} financial constraints and debt as potentially important factors in business cycle analysis. This development is motivated by empirical evidence suggesting that households' asset positions, access to credit and indebtedness are key to understanding the behaviour of individual as well as macroeconomic outcomes. 

Pertaining to the former, \textcite{campbell1989} (with aggregate U.S.~time series) and \textcite{zeldes1989} (with survey data on U.S.~families) show that fractions of the population are indeed constrained in their access to liquidity.\footnote{Such findings are often taken as evidence against the permanent-income hypothesis (e.g.~\cite{kaplan2014}).} Using granular credit card data which allow to control for credit demand, \textcite{gross2002} find that households significantly increase debt upon a loosening in their borrowing limit and that these responses are heterogeneous: individuals at or close to their debt limit respond more strongly. Also, credit card debt moves inversely with interest rates. 

More recently, \textcite{kaplan2014} document how differences in the liquidity of households' portfolios of asset holdings influence their consumption behaviour. For example, agents wealthy in illiquid assets might exhibit consumption patterns resembling the ones of the wealth-poor.\footnote{For a different view on the importance of portfolio liquidity, see \textcite{aguiar2020}.} \textcite{fagereng2021mpc} find with Norwegian data that the consumption sensitivity of households with respect to income changes falls in liquid wealth but not in total wealth.

Regarding aggregate outcomes, \textcite{mian2010} show that, on the U.S.~county level, the increase in household leverage before the Great Recession and the reductions in credit card debt availability during the Great Recession significantly inform about the severity of the recession in that county. Further, in a panel-data sample on 30 mostly advanced economies, \textcite{mian2017} find that (total) household debt is a significant statistical predictor for output growth.

Prompted by the apparent relevance of household finance in the data, the theoretical literature sought to develop macroeconomic models which give centre stage to household-level financial constraints and allow for a meaningful inclusion of household debt. Among others, \textcite{egg2012}, \textcite{riosrull2015}, \textcite{justiniano2015} and \textcite{gl2017} study borrower-lender interactions and how these shape aggregate dynamics.

In \textcite{egg2012}, borrowing and lending arise from differences in patience between two household types of exogenous proportions. The impatient agents borrow from the relatively patient ones, up to a (real) debt limit. An unexpected tightening in this limit forces debtors to delever, reducing the supply of liquidity in the economy and triggering the interest rate to fall. However, if the change in the debt constraint is large enough and the \Gls{zlb} is present, the interest rate might be unable to adjust sufficiently and there is a short-run recession. As debt is nominal, debt-deflation à la \textcite{fisher1933} exacerbates the downturn.

In contrast, \textcite{riosrull2015} and \textcite{gl2017} consider \Gls{ha} models in which the distribution of agents is free to adjust after a financial disturbance. While \textcite{riosrull2015} emphasise endogenous changes in consumers' search efforts and productivity as credit tightens, \textcite{gl2017} share with \textcite{egg2012}, as do I, the focus on the endogenous adjustment of the interest rate.

\textcite{gl2017} induce borrowing and lending by exposing households to idiosyncratic income risk. This gives rise to an increased demand for precautionary savings after a sudden, permanent reduction in the debt limit. Together with mechanical deleveraging, this results in a severe mismatch between asset demand and supply, provoking a reduction of aggregate demand even in the absence of nominal rigidities and the \Gls{zlb}. As the wealth distribution adjusts, the real interest rate undershoots its lower, terminal value. The authors further show that, since the wealth-poor are more sensitive in their consumption responses and it is predominantly the wealthy who react to interest-rate changes, the former suffer much larger consumption losses than their wealthier peers. Fixed wages and the \Gls{zlb} exacerbate the downturn. 

Relative to \textcite{gl2017}, I add a small-scale \Gls{nk} structure to the \Gls{ha} block, \textit{inter alia} a Phillips Curve, a Taylor rule and a Fisher relationship. This setup allows for interesting mechanisms. For example, unexpected deflation serves as an amplification mechanism by redistributing from borrowers to lenders on impact of an adverse shock.

The insight that forced deleveraging might spark considerable and unequal consumption losses raises the question of how policy could ameliorate the fallout. \textcite{egg2012} and \textcite{gl2017} show how fiscal policy can intervene with government debt issuance and spending after the shock arises. \textcite{korinek2016} sketch how regulation, e.g.~in the form of maximum leverage or insurance subsidies for borrowers, can contain excessive leverage \textit{ex-ante}, thus preventing or dampening the deleveraging shock.

Finally, my work draws on the fast-growing literature embedding the incomplete-markets framework, pioneered \textit{inter alia} by \textcite{bewley1986}, \textcite{im1989}, \textcite{huggett1993} and \textcite{aiyagari1994}, in \Gls{nk} environments. An increasing number of macroeconomic questions is investigated using these \Gls{hank} models, e.g.~\textcite{mckay2016} and \textcite{kaplan2018} study monetary policy through the lens of \Gls{hank}, \textcite{bayer2023} investigate fiscal policy and its supply of liquid assets and \textcite{bayer2019} analyse shocks to income risk.\footnote{\textcite{heath2009} provide an overview of modelling heterogeneous households in macroeconomics. \textcite{kv2018} discuss the advances with \Gls{hank}. \textcite{cherrier2023} offer a historical overview of the different approaches in the macroeconomic analysis with heterogeneity in the 1980s and 1990s.}

%%%%%%%%%%%%%%%%%%%%%%%%%%%%%%%%%%%%%%%%%%%%%%%%%%%%%%%%%%%%%%%%%%%%%%%%%%%%%
%%%%%%%%%%%%%%%%%%%%%%%%%%%%%%%%%%%%%%%%%%%%%%%%%%%%%%%%%%%%%%%%%%%%%%%%%%%%%
%%%%%%%%%%%%%%%%%%%%%%%%%%%%%%%%%%%%%%%%%%%%%%%%%%%%%%%%%%%%%%%%%%%%%%%%%%%%%
\section{Baseline Model}
\label{sec:model}

This section puts forward the baseline model. It is heavily inspired by the one-asset \Gls{hank} models in \textcite{auclert2021} and \textcite{boehl2023econpizza}, so I base the description of the model to a large extent on these papers. Section \ref{sec:model-hh} begins with the household sector. Section \ref{sec:model-sectors} deals with the firms, the fiscal and monetary authorities as well as with equilibrium. Finally, section \ref{sec:model-solution} touches upon the numerical solution method and explains the model calibration. 

In all what follows, time, $t$, is discrete, refers to quarters of a year and runs forever.

%%%%%%%%%%%%%%%%%%%%%%%%%%%%%%%%%%%%%%%%%%%%%%%%%%%%%%%%%%%%%%%%%%%%%%%%%%%%%
%%%%%%%%%%%%%%%%%%%%%%%%%%%%%%%%%%%%%%%%%%%%%%%%%%%%%%%%%%%%%%%%%%%%%%%%%%%%%
\subsection{Heterogeneous Households and Financial Constraints}
\label{sec:model-hh}

Households in this model choose optimal paths of consumption, labour hours and savings, given prices. However, individual labour income varies stochastically. To smooth consumption in the presence of incomplete markets, households save in government bonds and lend to and borrow from each other. This private credit intermediation is subject to a debt limit and a premium on the borrowing rate. The model allows for shocks to both of these constraints.

%%%%%%%%%%%%%%%%%%%%%%%%%%%%%%%%%%%%%%%%%%%%%%%%%%%%%%%%%%%%%%%%%%%%%%%%%%%%%
\subsubsection{Environment for Households}
\label{sec:model-hh-environment}

This subsection presents the formal environment for households. The economy is populated by a mass-one continuum of \textit{ex-ante} identical households, indexed by $i$, who are subject to idiosyncratic labour productivity, $e_{i,t}$, which evolves over time as a first-order autoregressive (\Gls{ar}(1)) process:
\begin{equation}
    log(e_{i,t}) = \rho_e log(e_{i,t-1}) + \varepsilon^e_{i,t}, \label{eq:hh-income-process}
\end{equation}
where $\varepsilon^e_{i,t}$ is a mean-zero shock to agents' skill level with variance $\sigma_e$. These shocks die out with persistence $\rho_e$. Average productivity is one.

Household-specific skill, $e_{i,t}$, signifies how well household $i$ can turn its working hours into output in period $t$. For the household, process \eqref{eq:hh-income-process} is entirely exogenous and can not be influenced. $e_{i,t}$ enters the labour income of household $i$ in $t$, which is composed of idiosyncratic productivity, individual labour hours, $n_{i,t}$, and the aggregate wage rate, $w_t$.

Households face incomplete markets, i.e.~they lack access to state-contingent insurance against idiosyncratic risk stemming from $\varepsilon^e_{i,t}$. Instead, households can only self-insure by accumulating precautionary savings in a non-state-contingent and perfectly liquid asset which matures after one period.\footnote{`Non-state-contingent' means that the asset pays out in any state of the (model) world. In particular, the payment is independent of $e_{i,t}$. `Perfectly liquid' means that there are no costs involved in trading the asset.} This asset is in real terms, i.e.~it is a claim to one unit of the consumption good in the next period. I denote the period-$t$ choice on holdings of this asset for period $t+1$ as $b_{i,t}$. If agent $i$ is allowed to borrow and chooses to do so, $b_{i,t}$ is negative to denote debt.\footnote{Note that, as in most models (e.g.~\cite{kaplan2018}), asset and debt holdings alike here refer to \textit{net} positions, i.e.~$b_{i,t} > 0$ means that the asset side of the household's balance sheet is longer than its liability side.}

Given the prevailing prices in the economy, given the process in \eqref{eq:hh-income-process} and given the asset structure, households maximise their expected life-time utility:
\begin{equation}
    \max_{c_{i,t}, n_{i,t}, b_{i,t}} \ \mathbb{E}_{e,t} \Bigg\{ \sum_{t=0}^{\infty} \beta^t u(x_{i,t}) \Bigg\}, \label{eq:hh-objective}
\end{equation}
by choosing individual (non-durable) consumption in $t$, $c_{i,t}$, labour supply and next-period's asset holdings, $b_{i,t}$, for all periods. $\mathbb{E}_{e,t}$ denotes expectations taken in $t$ over \eqref{eq:hh-income-process}, $\beta$ is the factor by which households discount future utility, $u( \cdot )$ is the period-utility function and $x_{i,t}$ is a composite good of $c_{i,t}$ and $n_{i,t}$. The functional forms of $u( \cdot )$ and $x_{i,t}$ are specified below but, naturally, households derive utility from consuming and disutility from working.\footnote{That is, I require: $\frac{\partial u( \cdot )}{\partial x_{i,t}} > 0$, $\frac{\partial x_{i,t}}{\partial c_{i,t}} > 0$, $\frac{\partial x_{i,t}}{\partial n_{i,t}} < 0$.} 

In maximising \eqref{eq:hh-objective}, households are subject to a sequence of budget constraints and a sequence of borrowing constraints. The former states that, in any given period $t$, household $i$ must not spend more than it earns. The period-$t$ budget constraint for household $i$ reads as follows: 
\begin{equation}
c_{i,t} + b_{i,t} = RR (RR_t, b_{i,t-1}) b_{i,t-1} + w_t e_{i,t} n_{i,t} + \Pi_t \bar{\Pi} (e_{i,t}) - \tau_t \bar{\tau} (e_{i,t}), \label{eq:hh-budget}
\end{equation}
where $w_t$ is the aggregate real wage paid for each unit of effective labour, $e_{i,t} n_{i,t}$, and where $RR (RR_t, b_{i,t-1})$ is the real interest rate schedule determining the return on or cost of previous period's asset or debt holdings, $b_{i,t-1}$. 

$\Pi_t$ are firm dividends and $\tau_t$ is a tax rate set by the government, discussed in subsequent sections. As in \textcite{mckay2016}, $\bar{\Pi} (e_{i,t})$ and $\bar{\tau} (e_{i,t})$ are incidence rules which transform aggregate dividends and taxes into household-level values in a lump-sum fashion.\footnote{\textcite{mckay2016} explain that the present approach is analogous to lump-sum procedures in representative-agent models. \textcite{kaplan2018} note that incidence rules can influence the results of \Gls{ha} models.} Thus, the left-hand side of \eqref{eq:hh-budget} captures the expenditure of household $i$ in $t$, comprised of goods consumption and asset acquisitions. Earnings, the right-hand side of \eqref{eq:hh-budget}, are composed of asset income, labour income and dividend income, less taxes.

I specify the real interest rate schedule in \eqref{eq:hh-budget} to take the following form:
\begin{equation}
    RR (RR_t, b_{i,t-1}) = \begin{cases}
	RR_t \ \ \ \ \ \ \ \ \ \ \ \ \ \text{if} \ b_{i,t-1} \ge 0 \\
	RR_t + \kappa_t \ \ \ \ \text{if} \ b_{i,t-1} < 0,
	\end{cases} \label{eq:hh-interest-rate-wedge}
\end{equation}
where $RR_t$ is the gross \textit{ex-post} real interest rate, to be specified momentarily, and where $\kappa_t \ge 0$ is an exogenously set premium on debt holdings. That is, household $i$ earns the return $RR_t$ when being a net lender but pays $RR_t$ plus $\kappa_t$ when being a net borrower. This wedge is an additional cost to borrowing and is thus a \textit{financial constraint}, which I discuss in subsection \ref{sec:model-hh-constraints}.

As in \textcite{auclert2021}, $RR_t$ in \eqref{eq:hh-interest-rate-wedge} adheres to the following Fisher relationship:
\begin{equation}
    RR_t = \frac{R_{t-1}}{\pi_t}, \label{eq:hh-fisher-equation}
\end{equation}
where $R_{t-1}$ is last period's gross nominal interest rate and $\pi_t$ is the current gross rate of inflation. Subsections \ref{sec:model-policy} and \ref{sec:model-firms} explain the determination of these quantities, respectively. $RR_t$ is the crucial price guiding the household's consumption-savings decisions and I dwell on it in subsection \ref{sec:model-hh-constraints}.

Finally, when choosing next period's asset holding, $b_{i,t}$, household $i$ faces a borrowing constraint à la \textcite{huggett1993}:
\begin{equation}
    b_{i,t} \ge \phi_t, \label{eq:borrowing-limit}
\end{equation}
where $\phi_t \le 0$ is an exogenously imposed limit to borrowing. $\phi_t = 0$ prohibits borrowing, $\phi_t < 0$ allows borrowing up to the specified limit. This limit to indebtedness is another \textit{financial constraint}. Subsection \ref{sec:model-hh-constraints} discusses the borrowing limit in detail.

%%%%%%%%%%%%%%%%%%%%%%%%%%%%%%%%%%%%%%%%%%%%%%%%%%%%%%%%%%%%%%%%%%%%%%%%%%%%%
\subsubsection{Solution to the Household Problem and Aggregation}
\label{sec:model-hh-solution}

Equations \eqref{eq:hh-income-process} -- \eqref{eq:borrowing-limit} constitute the formal setup of households. Before I turn to a discussion of this framework, this subsection succinctly presents the solution to the household problem and aggregates household-level outcomes. Consult appendix \ref{sec-app:full-derivations-households} for detailed derivations.

To solve the optimisation problem, I follow \textcite{boehl2023econpizza} and \textcite{bayer2023} in choosing $u( \cdot )$ and $x_{i,t}$ as follows. The utility function is of the constant relative-risk aversion (\Gls{crra}) type:
\begin{equation}
    u(x_{i,t}) = \frac{x_{i,t}^{1-\sigma_c}}{1-\sigma_c}, \label{eq:hh-utility-fun}
\end{equation}
where $\sigma_c$ is the coefficient of relative risk aversion, which reflects the degree with which households want to smooth consumption over time. The composite good, $x_{i,t}$, combines $c_{i,t}$ and $n_{i,t}$ as follows:
\begin{equation}
    x_{i,t} = c_{i,t} - e_{i,t}\frac{n_{i,t}^{1+\sigma_l}}{1+\sigma_l},
    \label{eq:hh-composite-good}
\end{equation}
where $\sigma_l$ is the inverse Frisch elasticity of labour supply. Solving \eqref{eq:hh-income-process} -- \eqref{eq:hh-composite-good} yields two first-order conditions (\Gls{foc}) and an envelope condition. The first \Gls{foc} and the envelope condition determine the households' optimal consumption, $c_{i,t}$, and savings policies, $b_{i,t}$, given last period's bond holdings, $b_{i,t-1}$, and the current period's productivity realisation, $e_{i,t}$:
\begin{align}
    x_{i,t}^{-\sigma_c} &\ge \beta \mathbb{E}_{e,t} \{ V_{b,t+1} (e_{i,t+1}, b_{i,t}) \}, \label{eq:foc-euler} \\
    V_{b,t} (e_{i,t}, b_{i,t-1}) &= RR (RR_t, b_{i,t-1}) x_{i,t}^{-\sigma_c}, \label{eq:foc-envelope} 
\end{align}
where $V_{b,t}$ is the marginal continuation value of marginally increasing bond holdings for the next period. The Euler equation \eqref{eq:foc-euler} holds with equality whenever household $i$ is away from the borrowing limit. The second \Gls{foc} pins down the optimal labour supply of household $i$, given the real wage:
\begin{equation}
n_{i,t}^{\sigma_l} = w_t, \label{eq:foc-labour-supply}
\end{equation}
which is identical over households due to \Gls{ghh} preferences, see appendix \ref{sec-app:full-derivations-households}.

The fact that households transverse different paths of skill realisations from \eqref{eq:hh-income-process}, induces them to accumulate and decumulate assets at varying speeds. This gives rise to \textit{ex-post} household heterogeneity with respect to productivity states and asset holdings. That is, at time period $t$, there is a distribution of agents over skill levels and wealth, which I denote by $\Psi_t (b,e)$, where, for notational convenience, $b=b_{i,t-1}$ and $e=e_{i,t}$.\footnote{Note the timing: the distribution in $t$ is determined by how much wealth (or debt) agents in $t-1$ chose to hold in $t$ and by where the productivity process dictates them to be in $t$.} Integrate individual-level outcomes over this distribution to obtain economy-wide consumption, $C_t$, labour supply, $N_t$, and savings, $B_t^d$:
\begin{equation*}
    C_t \equiv \int c_{i,t} d \Psi_t (b,e), \ \ \ N_t \equiv \int n_{i,t} d \Psi_t (b,e), \ \ \ B_t^d \equiv \int b_{i,t} d \Psi_t (b,e),
\end{equation*}

Due to the fact that all households supply the same amount of labour, individual and aggregate supplies co-move one-to-one and \eqref{eq:foc-labour-supply} generalises to the aggregate labour supply condition:
\begin{equation}
    N_t^{\sigma_l} = w_t. \label{eq:hh-agg-labour-supply}
\end{equation}

%%%%%%%%%%%%%%%%%%%%%%%%%%%%%%%%%%%%%%%%%%%%%%%%%%%%%%%%%%%%%%%%%%%%%%%%%%%%%
\subsubsection{Incomplete Markets, Assets, Borrowers and Lenders}
\label{sec:model-hh-assets}

This subsection aims at clarifying the key implications of the chosen asset structure and how borrowers and lenders emerge in the model. Thereafter, subsection \ref{sec:model-hh-constraints} explains how the financial constraints introduced above affect individual consumption behaviour. 

In the model, the main objective of households is to smooth consumption over time. However, they are subject to idiosyncratic labour skill risk from \eqref{eq:hh-income-process} and thus fear the possibility of obtaining low labour income as it hurts their consumption. Further, households dread the possibility of reaching the borrowing limit in \eqref{eq:borrowing-limit}, e.g.~after repeatedly being in low productivity states. Being at the limit would be particularly harmful for households, as explained below.

Clearly, households have an incentive to insure themselves against labour income shocks.\footnote{The chosen \Gls{ghh} preferences eliminate the possibility of households using working hours as insurance against low productivity. Section \ref{sec:sensitivity-end-labour} explores the case where households can adjust labour supply individually.} However, in an environment of incomplete markets, contracts which insure against specific realisations of labour productivity, high or low, are unavailable.\footnote{Incidentally, assuming complete markets and hence the presence of state-contingent claims would reduce the problem to a representative-agent setup, rendering the investigation of the present issues vacuous.} As \textcite{heath2009} remark, households thus resort to self-insurance by means of precautionary savings which they can draw down when confronted with low labour income. The existence of the borrowing limit further strengthens the wish to save. Crucially, the desire for precautionary savings contributes to the \textit{demand for assets} by households. 

Some agents, however, do remain in states of low productivity for prolonged periods of time. To nevertheless sustain a certain level of consumption, these households have a high willingness to tap into debt to finance the related expenditures, even if this involves owing the borrowed funds and the return $RR_t + \kappa_t$. Thus, these households have a \textit{demand for debt}.

Now, it is important to realise that the demand for debt corresponds to a source of \textit{asset supply}. This is because a debt contract offers lenders a vehicle to postpone consumption to the future as it specifies that the borrowing household will, in the next period, return the borrowed funds plus some interest. Borrowing thus creates a private liquid asset in the form of an ``I Owe You" (\Gls{iou}) and reflects unsecured private credit intermediation (see e.g.~\cite{bayer2023}).

\Gls{iou}s are mutually beneficial since the desire of borrowers to trade consumption today against a promise to repay in the future is matched by the lenders' wish to save today in return for a claim to future consumption. On the one hand, creditors are willing to supply these funds as they have experienced a series of positive shocks and thus demand liquid savings as insurance against future shocks. On the other hand, debtors have gone through low productivity and would like to exchange some of tomorrow's consumption for consumption today.

As the debt that borrowing households have on the liability side of their balance sheet are matched one-to-one by assets that the lenders hold on the asset side of their balance sheets -- each positive \Gls{iou} of a lender is a negative \Gls{iou} of some borrower -- the two groups of households are tightly linked through their balance sheets and the asset market. This implies that when borrowers are forced, out of some reason, to curtail their leverage and thus their liabilities, lenders will be directly affected as the supply of \Gls{iou}s they can invest in is reduced. Evidently, such a credit crunch will have repercussions that outsize the initial effect on borrowers.

The real interest rate, $RR_t$, is what lenders earn on their positive \Gls{iou}s and what borrowers pay on their negative \Gls{iou}s. Intuitively, the real interest rate reflects the opportunity cost of consuming today and is thus the crucial price guiding the household's consumption-savings decisions. For example, a lower real rate in the next period implies that a net lender shifts some savings towards consumption in the present as the opportunity cost of doing so is lessened. Naturally, debtors prefer to pay low interest, while lenders like high returns on their savings. To match these contrarious interests, $RR_t$ adjusts endogenously so that, in equilibrium, saving and borrowing households have no incentive to save or borrow more or less. Hence, whenever a situation of excess demand for liquidity, the real interest rate has to adjust to restore equilibrium.

The real rate does so through the Fisher equation \eqref{eq:hh-fisher-equation}, which underlines that in the present setup, unlike in \textcite{gl2017}, developments in the asset market influence $RR_t$ through the goods market. In particular, excess liquidity demand is equivalent (via the budget constraint \eqref{eq:hh-budget}) to excess demand for tomorrow's relative to today's consumption. The latter situation implies deflation today and a counter-reaction of the central bank (see below), which in turn provokes a lower real rate tomorrow (as the real rate is an \textit{ex-post} rate), prompting more consumption today and thus ameliorating the mismatch in goods and asset markets. 

As the asset positions of lenders and borrowers net out, \Gls{iou}s are in zero net supply. Hence, up to this point, the asset structure is essentially the one found in the \textcite{huggett1993} economy. However, I follow a series of papers (e.g.\cite{gl2017}, \cite{bayer2023}) by introducing an exogenous and time-invariant supply of government bonds, $B$. These bonds are perfect substitutes to the privately intermediated \Gls{iou}s. Therefore, the entirety of liquidity available to net lenders, gross liquidity, is the sum of government bonds and \Gls{iou}s.

%%%%%%%%%%%%%%%%%%%%%%%%%%%%%%%%%%%%%%%%%%%%%%%%%%%%%%%%%%%%%%%%%%%%%%%%%%%%%
\subsubsection{Financial Constraints and Consumption Behaviour}
\label{sec:model-hh-constraints}

In obtaining credit from lenders, borrowing households are subject to a limit to debt and a borrowing premium. How do these frictions work and how do they affect household behaviour?

Equation \eqref{eq:borrowing-limit} caps the maximum \textit{quantity} of debt a household can hold to amounts smaller or equal to $\phi_t$. Households encounter such limits in everyday life e.g.~as constraints on how much they can spend with their credit cards. That is, $\phi_t$ is not an economy-wide but a household-level constraint on leverage. As \textcite{egg2012} point out, it is meant to capture the notion that lenders might perceive some levels of debt to be too excessive for borrowers to repay. In the present model, this is an entirely \textit{ad hoc} assumption, however, as default is not considered and lenders would thus lend even more than $\phi_t$ despite debt not being collateralised.\footnote{Appendix xx shows that, in all calibrations, $\phi_t$ is tighter than the natural borrowing limit.}

As stated in \textcite{carroll2021}, a borrowing limit hampers households' efforts to smooth shocks to their consumption in the wake of unfortunate income realisations. This is because the debt limit introduces an asymmetry, or kink, at $b_{i,t} = \phi_t$: households can always choose to retire debt -- e.g.~if they experience an increase in income -- but are restricted in taking up more debt in case of an unfortunate realisation of skills and income. 

This kink in households' savings decision influences their consumption behaviour. For households who are \textit{at} the constraint, a negative shock to productivity translates into a one-to-one fall in consumption, exposing these agents in an extreme way to fluctuations in their labour skills. Consequently, individuals at the limit have a very high inclination to consume if given a small debt relief, i.e.~they have very high marginal propensities to consume (\Gls{mpc}) out of a marginal increase in liquid wealth.\footnote{Note that there is substantial variation in what is understood as the marginal propensity to consume (e.g.~across horizons, amounts, type of income shocks). I follow \textcite{gl2017} to define and calculate the \Gls{mpc} as the derivative of the consumption function over bond holdings: $mpc_{i,t} = \frac{\partial c_{i,t}}{\partial b_{i,t}}$, where I restrict $mpc_{i,t} \le 1$. That is, the \Gls{mpc}s I calculate are strictly \Gls{mpc}s \textit{out of a small windfall of liquid wealth}.}

Even households away from the borrowing limit are affected by its existence. They anticipate the risk of reaching the constraint in finite time, which would expose them to the same asymmetry as the currently constrained agents, and thus engage in precautionary savings. However, the further away the household is from the limit, the less probability it assigns to the possibility of being constrained and the more it will behave like a permanent-income consumer, i.e.~fully unconstrained. Crucially, these agents feature significantly lower \Gls{mpc}s out of liquid wealth as they are able to shield their consumption well with the assets they have accumulated. In total, borrowers -- especially those close to or at the debt limit -- are expected to feature much higher \Gls{mpc}s out of liquid wealth than lenders do.

Apart from outright limits to borrowing, households may be forced to pay a premium, $\kappa_t$, on the lending rate for their borrowed funds as specified in \eqref{eq:hh-interest-rate-wedge}. This premium affects the \textit{price} of debt and operates as a stand-in for the observation that financial institutions typically charge higher rates on accounts with negative balances than they pay on accounts with positive balances. As the present model does not feature any agent who sets $\kappa_t$, the wedge reflects a wasted cost in private credit intermediation as in e.g.~\textcite{bayer2023}.

I deem this constraint an interesting one as it introduces another kink in households' savings policies at zero assets, capturing the plausible notion that agents not only fear the possibility of reaching the borrowing limit but are also reluctant to tap into debt in the first place.\footnote{As the online appendix G.3 of \textcite{achdou2022} notes, the interest rate wedge can help account for empirical facts of the wealth distribution, see: \url{https://benjaminmoll.com/wp-content/uploads/2019/07/HACT_appendix.pdf}, last retrieved \printdate{2023-07-27}.} Finally, it influences households only in combination with the real interest rate on lent funds. That is, if the spread falls but the real rate rises by the same amount, the borrowing rate and thus the behaviour of indebted agents remain unchanged.

In reality, financial constraints such as the borrowing limit and the credit spread vary across household characteristics (e.g.~income, credit-worthiness) and asset types (e.g.~credit card vs mortgage debt). Importantly however, these constraints change over time, reflecting the fact that credit becomes more or less abundant in the course of time. For example, \textcite{anenberg2018} use a frontier estimation approach to empirically document how the limits to mortgage loan amounts do not only differ across borrower types but that they also vary over time. The analysis from section \ref{sec:limit} on aims to capture the effects of the exogenous component of such time variation in order to make causal statements about the mechanisms induced by a credit crunch, which in turn is caused by something outside of the model, e.g.~lost confidence in the ability of borrowers to repay their debt.\footnote{In general, financial constraints are of course endogenous. However, setting the constraints exogenously allows to study the main channels by which variation of these constraints influence the economy. In a more comprehensive model, $\phi_t$ and $\kappa_t$ would be the result of the optimisation of some agents, e.g.~financial institutions.}

In summary, ``lucky" households lend to ``unlucky" ones, with the real interest rate intermediating between the two. In obtaining credit, debtors are subject to financial constraints, making them more sensitive in their consumption behaviour than lenders are.

%%%%%%%%%%%%%%%%%%%%%%%%%%%%%%%%%%%%%%%%%%%%%%%%%%%%%%%%%%%%%%%%%%%%%%%%%%%%%
%%%%%%%%%%%%%%%%%%%%%%%%%%%%%%%%%%%%%%%%%%%%%%%%%%%%%%%%%%%%%%%%%%%%%%%%%%%%%
\subsection{Firms, Government and Equilibrium}
\label{sec:model-sectors}

This section closes the model with a corporate sector and a government which controls a fiscal and a monetary authority. Further, I formally define equilibrium for the present model.

%%%%%%%%%%%%%%%%%%%%%%%%%%%%%%%%%%%%%%%%%%%%%%%%%%%%%%%%%%%%%%%%%%%%%%%%%%%%%
\subsubsection{Firms}
\label{sec:model-firms}

The corporate sector of the model is kept minimal and close to the \Gls{nk} literature with ingredients from \textcite{gust2017wp} and \textcite{mckay2016}. In particular, I assume that a representative and perfectly competitive firm produces the final consumption good by using inputs from a continuum of monopolistically competitive firms which employ households' labour and set prices subject to convex price adjustment costs à la \textcite{rotemberg1982} with price indexation.\footnote{Thus, I abstract from productive capital. A meaningful inclusion of capital would require a two-asset model in order to avoid a counterfactual increase in investment amid a credit crunch, see \textcite{kv2018}.}

Appendix \ref{sec-app:full-derivations-firms} formally states the firms' maximisation problems and solves for their optimal behaviour. Here, I merely state the equations relevant for the model implementation. First, production of the final good, $y_t^f$, occurs according to the following simple production function:
\begin{equation}
    y_t^f = z N_t, \label{eq:firms-production}
\end{equation}
where $z$ is technology and $N_t$ is aggregate labour input. Aggregate marginal costs are $mc_t = \frac{w_t}{z}$. 

Costly price adjustment makes the problem of intermediate-goods firms dynamic, leading to a \Gls{nk} Phillips Curve (\Gls{nkpc}) of the form:
\begin{equation}
    \psi \Bigg( \frac{\pi_t}{\tilde{\pi}_{t-1}} - 1 \Bigg) \frac{\pi_t}{\tilde{\pi}_{t-1}} = (1-\theta) + \theta mc_t + \psi \beta \Bigg( \frac{\pi_{t+1}}{\tilde{\pi}_{t}} - 1 \Bigg) \frac{\pi_{t+1}}{\tilde{\pi}_{t}} \frac{y_{t+1}^f}{y_t^f}, \label{eq:firms-nkpc}
\end{equation}
where $\pi_t \equiv \frac{P_t}{P_{t-1}}$ is the change in the price level in period $t$, i.e.~gross inflation in period $t$. $\tilde{\pi}_t \equiv \pi_{ss}^{\omega} \pi_{t}^{1-\omega}$ is indexed inflation with $\omega$ determining the degree of indexation to steady-state inflation. $\theta$ is the elasticity of substitution between the intermediate-goods varieties in the production of $y_t^f$ and $\psi$ is the parameter which determines the costs to price adjustment.  

The \Gls{nkpc} pins down inflation in this model and reflects the trade-off between changes in economic activity and changes in prices. In essence, it captures how a reduction in goods demand leads to a reduction in inflation -- or deflation -- through lower marginal costs, which come about through lower wages as labour demand adjusts to lower goods demand. This trade-off gives rise to a channel where aggregate consumption demand, through firms and their price-setting, directly influences the \textit{ex-post} real interest rate, $RR_t$, making transparent the tight link between the asset and the goods market in the model.

Finally, monopolistic competition in the inputs-market results in firm profits:
\begin{equation}
    \Pi_t = \Bigg[ 1 - \frac{\psi}{2} \Bigg( \frac{\pi_t}{\tilde{\pi}_{t-1}} - 1 \Bigg)^2 \Bigg] y_t^f - w_t N_t, \label{eq:firms-profits}
\end{equation}
as intermediate-goods firms can sell their output with a markup, $\mu_t$. As explained above, profits are distributed lump-sum to households. They can not trade their shares in the firms.

%%%%%%%%%%%%%%%%%%%%%%%%%%%%%%%%%%%%%%%%%%%%%%%%%%%%%%%%%%%%%%%%%%%%%%%%%%%%%
\subsubsection{Fiscal and Monetary Policies}
\label{sec:model-policy}

The government conducts fiscal and monetary policy. As alluded to in section \ref{sec:model-hh-assets}, the fiscal authority supplies a time-invariant quantity of government bonds, $B$, which imposes interest payments on the government. To run a balanced budget, the fiscal authority levies the already specified tax on households, $\tau_t$. Thus, the government budget constraint reads:
\begin{equation}
    \tau_t + B = RR_t B \ \ \ \Leftrightarrow \ \ \ \tau_t = (RR_t - 1)B, \label{eq:fiscal-budget}
\end{equation}
which implies that the tax rate moves one-to-one with the \textit{ex-post} real interest rate.

The monetary authority sets the notional gross nominal interest rate, $R_{t}^n$, in order to stabilise inflation and output. It does so according to a standard Taylor rule:
\begin{equation}
    R_{t}^n = ( R_{t-1}^n )^{\rho} \Bigg[ R_{ss} \Bigg( \frac{\pi_t}{\pi_{ss}} \Bigg)^{\phi_{\pi}} \Bigg( \frac{y_t}{y_{ss}} \Bigg)^{\phi_y} \Bigg]^{(1-\rho)}, \label{eq:taylorrule}
\end{equation}
where $R_{ss}$ is the steady state gross nominal interest rate, to be defined momentarily, and $y_t$ is total ``effective" output of the economy, specified in section \ref{sec:model-eq}. Equation \eqref{eq:taylorrule} implies that the central bank responds -- with persistence $\rho$ -- to deviations of inflation and output from their steady-state values. The strengths of these responses are given by $\phi_{\pi}$ and $\phi_y$, respectively. 

In setting the notional rate, however, the central bank is subject to the \Gls{zlb} on nominal interest rates. Hence, the gross nominal interest rate, $R_t$, is pinned down by: %the maximum out of the notional gross nominal interest rate and $1$: 
\begin{equation}
    R_t = \text{max} \ \{ 1, \ R_{t}^n \}. \label{eq:zlb}
\end{equation}

This completes the description of the agents in the model. For the reader's convenience, I depict key aspects of the baseline model graphically in figure \ref{fig:graphical-model} in appendix \ref{sec-app:full-graphical}.

%%%%%%%%%%%%%%%%%%%%%%%%%%%%%%%%%%%%%%%%%%%%%%%%%%%%%%%%%%%%%%%%%%%%%%%%%%%%%
\subsubsection{Equilibrium and Market Clearing}
\label{sec:model-eq}

Next, I define equilibrium and state the conditions by which markets clear. Recall that $\Psi_t (b,e)$ is the distribution of agents over $b_{i,t-1}$ and $e_{i,t}$ at $t$.

\begin{definition}[Equilibrium]
\label{def:eq}
An equilibrium in this economy is defined as a sequence of household consumption and savings policies, $\{ c_{i,t} (b, e) \}_{t=0}^{\infty}$ and $\{ b_{i,t} (b, e) \}_{t=0}^{\infty}$, a sequence of distributions, $\{ \Psi_t (b,e) \}_{t=0}^{\infty} $, and a sequence of aggregate outcomes $\{ B_t^d$, $C_t$, $mc_t$, $N_t$, $\pi_t$, $\tilde{\pi}_t$ $\Pi_t$, $RR_t$, $R_t$, $R_t^n$, $\tau_t$, $w_t$, $y_t$, $y_t^f \}$ such that:
\begin{enumerate}
    \item given prices, the policies, $c_{i,t} (b, e)$ and $b_{i,t} (b, e)$, solve the households' maximisation problem given in equations \eqref{eq:hh-income-process} -- \eqref{eq:borrowing-limit},
    \item together with the Markov process for productivity, the policies induce a law of motion for the distribution, denoted by $\Gamma$,
    \item given an initial distribution $\Psi_0$, the distribution solves the forward equation of the law of motion:
    \begin{equation*}
        \Psi_{t+1} (b_{i,t},e_{i,t+1}) = \Psi_{t} (b_{i,t-1},e_{i,t}) \Gamma,
    \end{equation*}
    \item all markets clear.
\end{enumerate}
\end{definition}

The labour market clears when \eqref{eq:hh-agg-labour-supply} holds. The bond and goods markets clear when $B_t^d = B$ and $C_t = y_t$ hold, respectively, where $y_t$ is output net of price adjustment and credit intermediation costs:
\begin{equation}
    y_t \equiv \Bigg( 1 - \frac{\psi}{2} \Bigg( \frac{\pi_t}{\tilde{\pi}_{t-1}} - 1 \Bigg)^2 \Bigg) y_t^f - \kappa_t D_t, \label{eq:effective-gdp}
\end{equation}
where $D_t$, given by:
\begin{equation}
    D_t \equiv - \int b_{i,t} \mathbb{1}_{[\phi_t \le b_{i,t} \le 0]} d \Psi_t (b,e), \label{eq:agg-debt}
\end{equation}
denotes aggregate debt holdings. $\mathbb{1}_{[\phi_t \le b_{i,t} \le 0]}$ is an indicator function for negative asset holdings.

%%%%%%%%%%%%%%%%%%%%%%%%%%%%%%%%%%%%%%%%%%%%%%%%%%%%%%%%%%%%%%%%%%%%%%%%%%%%%
%%%%%%%%%%%%%%%%%%%%%%%%%%%%%%%%%%%%%%%%%%%%%%%%%%%%%%%%%%%%%%%%%%%%%%%%%%%%%
\subsection{Solution Method and Calibration}
\label{sec:model-solution}

The \Gls{hank} model does not allow for an analytical solution. This subsection therefore briefly discusses the numerical solution approach I use and the calibration of model parameters I choose.

As is well-known to the literature, \Gls{ha} models are notoriously difficult to solve since general equilibrium and rational expectations require that agents are able to forecast market-clearing prices which depend on the sequence of distributions, $\{ \Psi_t (b,e) \}_{t=0}^{\infty} $, each element of which is an infinite-dimensional object. By now, a variety of solution methods for this challenge exists but choosing among methods still involves trade-offs. 

The present thesis makes use of the method laid out by \textcite{boehl2023econpizza} which builds on \textcite{auclert2021} and implements a perfect-foresight approach to find the non-linear transitions of \Gls{ha} models after one-time, unexpected (``MIT") shocks. To do so, the method makes use of a Newton root-finding procedure that searches in sequence space for aggregate outcomes, truncated at some distant terminal period, which are consistent with the household policies and the law of motion for the distribution, with the policies and the distribution being given on discretised grids of skills and assets. The endogenous-gridpoints method (\Gls{egm}) by \textcite{carroll2006} iterates on equations \eqref{eq:foc-euler} and \eqref{eq:foc-envelope} to find the household consumption and savings policies.

On the one hand, the implementation by \textcite{boehl2023econpizza} offers advantages over competitor methods. It flexibly accommodates non-linearities in the model's aggregate state space and is typically very fast due to the use of automatic differentiation. On the other hand, meaningful uncertainty in aggregates -- as opposed to idiosyncratic risk -- is disregarded. In particular, the method assumes that households know the exact trajectory of the economy once the shock to the model is announced. This notwithstanding, accounting for non-linearities such as the \Gls{zlb} appears important for the present use case, so the method by \textcite{boehl2023econpizza} seems appropriate. Its implementation is available through the Python package \texttt{Econpizza}.\footnote{The package is maintained by Dr.~Gregor Böhl on \href{https://github.com/gboehl/econpizza}{GitHub}, last retrieved on \printdate{2023-08-22}.} For the sake of brevity, I delegate a discussion of some caveats regarding the present application to appendix \ref{sec-app:codes}.

Numerically solving a model implies that its parameters must be calibrated. Table \ref{tab:calibration-baseline} reports the baseline calibration together with the sources or targets I use.\footnote{Calibration is always to some degree arbitrary. Estimation is preferred but beyond the scope of this thesis.} I choose $\beta$, the households' discount factor, such that the net real interest rate is $0.15\%$ in steady state. This arguably low target aims to capture the possibility of the \Gls{zlb} binding during the credit crunch, given the importance which the literature assigns to this mechanism (e.g.~\cite{egg2012}, \cite{gl2017}). Appendix \ref{sec-app:robust-no-zlb} explores a model version ignoring the \Gls{zlb}.

\begin{table}[t]
\centering
\caption{Calibration of Baseline Model}
\label{tab:calibration-baseline}
\begin{tabular}{llcc}
Parameter &   &  Value & Source/Target \\
\hline
\hline
\multicolumn{2}{l}{\textbf{Households}} & & \\
$\beta$ &  discount factor & $0.99056$ & $RR_{ss}-1 = 0.15\%$ \\
$\sigma_c$ & coefficient of relative risk aversion & $2$ & \textcite{auclert2021} \\
$\sigma_l$ & inverse Frisch elasticity & $2$ & \textcite{auclert2021} \\
$\rho_e$ & persistence in productivity shocks & $0.966$ & \textcite{auclert2021} \\
$\sigma_e$ & variance of productivity shocks & $0.5$ & \textcite{auclert2021} \\
$N_e$ & number of productivity grid points & $8$ & See appendix \ref{sec-app:codes-ep} \\
$N_b$ & number of asset grid points & $236$ & See appendix \ref{sec-app:codes-ep} \\
& & & \\
\multicolumn{2}{l}{\textbf{Firms}} & & \\
$z$ &  technology & $1$ & $y=N$ \\
$\theta$ & elasticity of substitution & $6$ & \textcite{auclert2021} \\
$\psi$ & parameter on price adjustment costs & $100$ & Flat \Gls{nkpc} \\
$\omega$ & parameter of price ineratia & $0.44$ & \textcite{gust2017wp} \\
& & & \\
\multicolumn{2}{l}{\textbf{Fiscal Policy}} & & \\
$B$ & supply of government bonds & $5.21$ & $B/y_{ss} = 5.71$ \\
& & & (\cite{mckay2016}) \\
& & & \\
\multicolumn{2}{l}{\textbf{Monetary Policy}} & & \\
$\pi_{ss}$ & steady state gross inflation & $1$ & No steady-state net inflation \\
$\rho$ & persistence in notional interest rate & $0.8$ & Standard \\
$\phi_{\pi}$ & Taylor rule coefficient on inflation & $2$ & See text \\
$\phi_{y}$ & Taylor rule coefficient on output & $0.15$ & Common \\
& & & \\
\multicolumn{2}{l}{\textbf{Financial Constraints}} & & \\
$\phi_{ss}$ & initial borrowing limit & $-2.3485$ & $D_{ss}/y_{ss} = 57\%$ \\
& & & (\cite{bayer2023}) \\
$\kappa_{ss}$ & initial interest rate wedge & 1e-8 & No initial wedge \\
\hline
\multicolumn{4}{l}{\footnotesize \multirow{2}{15cm}{\justifying \textit{Note:} Note that all parameters jointly determine the steady states. All numbers refer to quarterly values. $p.p.$ stands for percentage points.}} \\
& & & \\
\end{tabular}
\end{table}

For the household utility function \eqref{eq:hh-utility-fun}, I use a standard value of $2$ for both, the coefficient of relative risk aversion and the inverse Frisch elasticity of labour supply. For the skills process \eqref{eq:hh-income-process}, I set $\rho_e = 0.966$ and $\sigma_e = 0.5$.\footnote{Consult appendix \ref{sec-app:codes-ep} on the details for discretising the skills process and the asset grid.} All of these parameters are as in \textcite{auclert2021}.

On the production side, I choose $\theta = 6$ as in \textcite{auclert2021} and $\psi = 100$ to achieve a rather flat \Gls{nkpc} with a slope of $0.06$, which seems in line with empirical estimates.\footnote{See e.g.~\textcite{schorfheide2008}. Recent cross-regional evidence from the U.S.~by \textcite{hazell2022} suggests an even flatter Phillips Curve.} In this simple setup, $\theta$ pins down all of the following: $\mu_{ss}$, $mc_{ss}$, $w_{ss}$, $N_{ss}$ as well as $C_{ss}$ and $y_{ss}$ if $\kappa_{ss}$ does not change. This leaves the interest rate free to adjust to a permanent disturbance to the model. $z=1$ normalises output to aggregate labour hours. $\omega = 0.44$ is from \textcite{gust2017wp}. 

Regarding policy, I set $B$, the supply of real government bonds, along the lines of \textcite{mckay2016} and \textcite{gl2017} by matching it to the ratio of liquid assets held by U.S.~households to U.S.~real gross domestic product (\Gls{gdp}), averaged over $1951$ -- $2023$.\footnote{See appendix \ref{sec-app:data} for details on the data sources and the required calculations.} Appendix \ref{sec-app:robust-bonds} calibrates $B$ as in \textcite{bayer2023}. The parameters for monetary policy are mostly standard. I set steady-state net inflation to $0$, $\rho = 0.8$, $\phi_y = 0.15$ and $\phi_{\pi} = 2$. The latter is arguably a strong reaction of monetary policy to inflation but lies between the values from two recent estimations of \Gls{hank} models: \textcite{boehl2022} and \textcite{bayer2023}.

Finally, consider the financial constraints in the steady state of relatively ``loose" credit. Akin to \textcite{gl2017} and \textcite{bayer2023}, I select $\phi_{ss}$, the loose debt limit, so that the ratio of household debt to output in the model is equal to the fraction of unsecured consumer credit over \Gls{gdp} in the U.S., averaged over $1951$ -- $2023$, which turns out to be $57\%$.\footnote{See again appendix \ref{sec-app:data}.} I set $\kappa_{ss} =$ 1e-8, so that there is virtually no premium on borrowing to begin with.

%%%%%%%%%%%%%%%%%%%%%%%%%%%%%%%%%%%%%%%%%%%%%%%%%%%%%%%%%%%%%%%%%%%%%%%%%%%%%
%%%%%%%%%%%%%%%%%%%%%%%%%%%%%%%%%%%%%%%%%%%%%%%%%%%%%%%%%%%%%%%%%%%%%%%%%%%%%
%%%%%%%%%%%%%%%%%%%%%%%%%%%%%%%%%%%%%%%%%%%%%%%%%%%%%%%%%%%%%%%%%%%%%%%%%%%%%
\section{A Credit Crunch through the Borrowing Limit}
\label{sec:limit}

This section confronts the \Gls{hank} model with a credit crunch through a permanent tightening of the household-level borrowing limit. To that end, section \ref{sec:limit-stst-loose} begins by briefly characterising the steady state of relatively loose borrowing, i.e.~with $\phi_{ss}$ and $\kappa_{ss}$. Section \ref{sec:limit-stst-tight} defines a state with a tighter credit limit, denoted $\phi_{ss}'$, and compares it to the initial one. Thereafter, section \ref{sec:limit-transition} describes and explains the dynamics from easy to tight credit.

%%%%%%%%%%%%%%%%%%%%%%%%%%%%%%%%%%%%%%%%%%%%%%%%%%%%%%%%%%%%%%%%%%%%%%%%%%%%%
%%%%%%%%%%%%%%%%%%%%%%%%%%%%%%%%%%%%%%%%%%%%%%%%%%%%%%%%%%%%%%%%%%%%%%%%%%%%%
\subsection{A State of Loose Credit}
\label{sec:limit-stst-loose}

Figure \ref{fig:init-stst-dist-pol} displays key characteristics of the initial steady state. Panel \ref{fig:init-stst-dist-pol-a-dist} shows the steady-state distribution of households over bond and \Gls{iou} holdings. Agents situated to the left of $0$ are net debtors: they owe more in debt than they own in assets. These households have gone through unfortunate realisations of labour skills and subsequently tap into debt to sustain consumption. In the present calibration, $33.66\%$ of households are net borrowers, see table \ref{tab:stst_comparison_baseline_limit_permanent}. $4.89\%$ of the population end up directly at the credit limit.\footnote{Note that these fractions might not be close to the data (e.g.~table 5 of \cite[p.~722]{kaplan2018} reports $15\%$ as the borrowing fraction in the population). However, the initial stationary distribution in the baseline model of \textcite{gl2017} features a comparable fraction of borrowers, about $33\%$, but their model predicts only roughly one percent of the population to be at the initial constraint. These values are from own calculations based on the MATLAB replication codes by \textcite{gl2017}, available under: \url{https://sites.northwestern.edu/guidolor/}, last retrieved \printdate{2023-6-28}.} These individuals might want to borrow more but are prohibited from doing so because of this constraint.

\begin{figure}[t]
    \caption{Baseline Model: Initial Steady State}
    \label{fig:init-stst-dist-pol}
    \centering
    \begin{subfigure}[b]{0.49\textwidth}
    \caption{Asset Distribution}
    \label{fig:init-stst-dist-pol-a-dist}
         \centering
         \includesvg[inkscapelatex=false, width=\textwidth]{Results/stst_dist_2d_bar_baseline_limit_permanent_initial.svg}
     \end{subfigure}
     \hfill
     \begin{subfigure}[b]{0.49\textwidth}
     \caption{Consumption Policies}
     \label{fig:init-stst-dist-pol-c}
         \centering
         \includesvg[inkscapelatex=false, width=\textwidth]{Results/stst_policies_c_baseline_limit_permanent_initial.svg}
     \end{subfigure}

    \vspace{10pt}
     
     \justifying
     \footnotesize
	\textit{Note}: In figure \ref{fig:init-stst-dist-pol-a-dist}, the distribution is cut off at an asset level of $50$ for readability. As the model is solved on a logarithmic grid, grid points are very dense at the lower end of the grid, which makes the colour appear bleak. The true density at high asset levels is overestimated as grid points are sparse there. The width of the bar at the borrowing limit is exaggerated. In figure \ref{fig:init-stst-dist-pol-c}, the labels $\theta\_0$ -- $\theta\_7$ represent the eight skill states from \eqref{eq:hh-income-process}.
\end{figure}

\begin{table}[t]
\centering
\caption{Baseline Model -- Shock to Borrowing Limit: Comparison of Steady States}
\label{tab:stst_comparison_baseline_limit_permanent}
\begin{tabular}{lccc}
Variable & Initial & Terminal &  Change \\
\hline
\hline
\multicolumn{2}{l}{\textbf{Borrowing Limit}} & & \\
$\phi$ & $-2.3485$ & $ -2.1775$ &  $0.1710$ \\
& & & \\
\multicolumn{2}{l}{\textbf{Aggregates}} & & \\
$C$ &  $0.9129$ &   $0.9129$ &  -- \\
$D / y$ &    $57\%$ & $52\%$ & $-5 \ p.p.$ \\
$B + D$ &  $5.73$ &   $5.68$ & $-0.80\%$ \\
$RR - 1$ &  $0.15\%$ &  $0.13\%$ & $-0.02 \ p.p.$ \\
& & & \\
\multicolumn{2}{l}{\textbf{Cross-Section}} & & \\
                  Share of Assets Held by Top-$10\%$ &  $48.34$ &   $47.74$ & $-0.6 \ p.p.$ \\
                  Share of Consumption by Top-$10\%$ &  $17.29$ &   $17.31$ & $0.02 \ p.p.$ \\
Fraction of Borrowers & $33.66$ & $32.9$ & $-0.76 \ p.p.$ \\
Fraction at Borrowing Limit & $ 4.89$ & $5.07$ &  $0.18 \ p.p.$ \\
\Gls{mpc} of Borrowers &    $0.16$ &     $0.17$ &   \\
\Gls{mpc} of Lenders &    $0.02$ &     $0.02$ & \\
\hline
\multicolumn{4}{l}{\footnotesize \multirow{2}{12cm}{\justifying \textit{Note:} Selected values of the steady states with $\phi_{ss}$ and $\phi_{ss}'$. Numbers are rounded and refer to quarterly values. $p.p.$ stands for percentage points.}} \\
& & & \\
\end{tabular}
\end{table}

The rest of the population holds assets in excess of its liabilities. Net lenders have experienced a series of positive productivity realisations and build positive asset positions in order to smooth consumption over time and to save in precaution of potentially worse skill endowments. Some of these households invest their funds in bonds, others in the \Gls{iou}s originated by borrowers. Gross liquidity, $B+D$, equals $6.3$ times quarterly output (table \ref{tab:stst_comparison_baseline_limit_permanent}).

As section \ref{sec:model-hh} explains, balance sheets are expected to have implications for consumption behaviour. Figure \ref{fig:init-stst-dist-pol-c} shows that this is the case by plotting the household consumption policies over asset holdings, separately for each productivity state. Naturally, lower asset holdings and/or lower productivity states translate into lower consumption. Importantly, the curvature of the consumption policies also depends on the asset level. For very low levels of wealth and high debt burdens, the consumption policies are concave and much steeper than for higher asset levels. This confirms that the consumption of a household at or near the borrowing limit strongly responds to changes in its asset positions and that consequently, i.e.~its \Gls{mpc} is high.

Indeed, table \ref{tab:stst_comparison_baseline_limit_permanent} shows that, in the initial steady state, the average borrower has an \Gls{mpc} of $0.16$, while the average lender features an \Gls{mpc} of merely $0.02$. Thus, an average borrower would increase consumption by $14$ cents more if given one dollar more in assets (i.e.~less in debt) than the average lender would. This distinction still masks substantial heterogeneity in the \Gls{mpc}s among borrowers, see figure \ref{fig:baseline-limit-initial-mpcs} in appendix \ref{sec-app:figures}. The \Gls{mpc} of borrowers at the debt limit can equal one, while borrowers with only little debt have \Gls{mpc}s comparable to those of lenders. Overall, this suggests that debtors -- especially those at or very close to $\phi_{ss}$ -- are much more sensitive to changes in their asset positions and incomes, as alluded to in section \ref{sec:model-hh}.\footnote{The fact that \Gls{mpc}s are heterogeneous is empirically plausible (see e.g.~\cite{gross2002}), but their magnitudes might not be exactly matched by this simple model. The aggregate \Gls{mpc} equals $0.0663$, which is low compared to empirical estimates such as \textcite{fagereng2021mpc} or the studies cited in \textcite{kaplan2018}.}

The aggregate dimension of the steady state shows that the real and nominal interest rates are $0.15\%$ and that debt over output equals $57\%$, as calibrated. Aggregate wage and labour hours imply an average quarterly labour income of $0.76$, so the initial debt limit allows households to borrow up to $3.1$ times their average quarterly labour income. 

%%%%%%%%%%%%%%%%%%%%%%%%%%%%%%%%%%%%%%%%%%%%%%%%%%%%%%%%%%%%%%%%%%%%%%%%%%%%%
%%%%%%%%%%%%%%%%%%%%%%%%%%%%%%%%%%%%%%%%%%%%%%%%%%%%%%%%%%%%%%%%%%%%%%%%%%%%%
\subsection{Loose vs Tight Credit: Comparing Steady States}
\label{sec:limit-stst-tight}

Next, I specify a new steady state of the economy, in which credit is tighter due to a reduced borrowing limit, $\phi_{ss}' < \phi_{ss}$. To that end, let $\phi_{ss}' = -2.1775$ so that the (quarterly) debt-to-output ratio is five percentage points lower in the terminal steady state than it is initially, i.e.~$\frac{D_{ss}'}{y_{ss}'} = 52\%$. Taking into account the observed developments in unsecured credit over the last decades (figure \ref{fig:app-calibration-debt} in appendix \ref{sec-app:data}), a change of five percentage points appears to be an empirically realistic magnitude.\footnote{\textcite{gl2017} consider a reduction by ten percentage points of the ratio of debt to \textit{annual} \Gls{gdp}. This translates into a drop of $40$ percentage points in quarterly terms, i.e.~a much larger change than I study.} 

How does the steady state with a confined debt limit compare to the initial steady state? What forces and induces households to behave differently in an environment of tight credit relative to one of loose credit? To contrast steady states, consider plot \ref{fig:comparison-stst-baseline-policies} and appendix figure \ref{fig:comparison-stst-baseline-dist}, which are analogous to figure IV in \textcite[p.1443]{gl2017}, as well as table \ref{tab:stst_comparison_baseline_limit_permanent}. 

\begin{figure}[t]
    \caption{Baseline Model -- Shock to Borrowing Limit: Household Policies}
    \label{fig:comparison-stst-baseline-policies}
    \centering
    \begin{subfigure}[b]{0.49\textwidth}
    \caption{Average Asset Policy}
    \label{fig:comparison-stst-baseline-policies-a}
         \centering
         \includesvg[inkscapelatex=false, width=\textwidth]{Results/stst_a_baseline_limit_permanent.svg}
     \end{subfigure}
     \hfill
     \begin{subfigure}[b]{0.49\textwidth}
     \caption{Average Consumption Policy}
     \label{fig:comparison-stst-baseline-policies-c}
         \centering
         \includesvg[inkscapelatex=false, width=\textwidth]{Results/stst_c_baseline_limit_permanent.svg}
     \end{subfigure}

    \vspace{10pt}
     
     \justifying
     \footnotesize
	\textit{Note}: The policies are cut off at an asset level of $30$ for readability. In both figures, the differences are difficult to discern. A close look reveals the aspects discussed in the text.
\end{figure}

Figure \ref{fig:comparison-stst-baseline-dist} shows that the stationary distributions of agents over their asset positions look very similar across steady states. Trivially, no mass of agents is admissible between the old and the new borrowing limit in the new steady state, so the density there is zero. The distribution seems to have been moved to the right. However, as in \textcite{gl2017}, the tighter borrowing limit actually compresses the distribution rather than pushing it to the right. On the one hand, at the remaining admissible levels of debt and at low positive asset holdings, the terminal density lies above the initial one. On the other hand, less agents are located at high bond holdings in an environment of tight credit relative to before.

Figure \ref{fig:comparison-stst-baseline-policies-a} sheds light on why this compression takes place by contrasting the average asset accumulation in each steady state.\footnote{Asset accumulation is $b_{i,t} - b_{i,t-1}$. The average is over the productivity levels induced by \eqref{eq:hh-income-process}.} Compared to the initial steady state, the average household reduces debt faster when being a borrower and draws down positive asset holdings more rapidly when being a wealthy lender. Asset-poor lenders accumulate bonds quicker. Why do agents behave in this new way? Consider first the indebted and wealth-poor. As the borrowing limit has tightened, the danger of reaching it is especially elevated for those households. To maintain a safe distance to the limit, these agents forgo consumption (see the low-wealth portion of figure \ref{fig:comparison-stst-baseline-policies-c}) and aim at reducing borrowing or build buffers faster whenever given the chance to do so.

All else equal, asset-rich households would likewise aim for higher savings due to the debt limit being a more imminent threat for them as well. These households nevertheless decumulate assets faster rather than doing the opposite as the real interest rate is permanently lower in the terminal steady state (table \ref{tab:stst_comparison_baseline_limit_permanent}). As these agents are interest-rate sensitive -- recall from subsection \ref{sec:model-hh-constraints} that the asset-rich behave akin to permanent-income consumers -- the lower real interest rate induces them to substitute away from savings towards consumption (see the right end of figure \ref{fig:comparison-stst-baseline-policies-c}). In total, in an environment of tight credit, asset-poor individuals consume less to reduce the possibility of becoming constrained, while wealthy households reduce asset holdings in order to consume more due to a lower real rate. These developments result in reduced long-term wealth inequality but elevated consumption inequality as captured by the shares that the top-$10\%$ save and consume, respectively, as illustrated in table \ref{tab:stst_comparison_baseline_limit_permanent}.

These observations raise the question as to why the long-run real interest rate is reduced. Crucially, this is a consequence of two adjustments in the asset market: a drop in the supply of liquidity and a simultaneous increase in the demand for it. The former is the result of what I call the \textit{balance-sheet effect}: the deleveraging forced onto the households between the initial and the terminal borrowing limits prohibits these agents to hold their initial levels of debt, thus preventing them to issue as many \Gls{iou}s as they did before. Crucially, this mechanical restriction in the quantity of \Gls{iou}s out of debtors' balance sheets reduces the supply of liquid assets in the economy, see the change of $B + D$ in table \ref{tab:stst_comparison_baseline_limit_permanent}. The balance-sheet effect is present in the models of both, \textcite{egg2012} and \textcite{gl2017}. 

However, as alluded to above, a tighter borrowing limit increases the desire of all agents for precautionary savings. This is the second development in the asset market. The heightened fear of reaching the borrowing limit thus further reduces asset supply by unconstrained debtors and boosts asset demand from unconstrained creditors. This is the \textit{precautionary-savings effect} of a credit crunch. It is muted in \textcite{egg2012} but features prominently in \Gls{ha} models such as the one here and in \textcite{gl2017}. 

Taken together, the curtailed supply of and the heightened demand for liquidity call for a lower real interest rate to equilibrate the asset market. In particular, the reduction in the real rate depresses the desire of borrowers to shorten the liability sides of their balance sheets, thus stabilising the supply of \Gls{iou}s, and prompts lenders to demand less assets as they substitute towards consumption today. By consequently dampening the balance-sheet and precautionary-savings effects, the lower real rate ensures equilibrium. This is the crucial change occurring in an economy which adjusts through the interest rate to tight credit.\footnote{Note that, as expected, aggregate consumption is identical across steady states since, in this setup, it is pinned down by the supply side of the model. Hence, the model can not speak to considerations about the long-term connection of credit availability and the level of aggregate consumption.}

%%%%%%%%%%%%%%%%%%%%%%%%%%%%%%%%%%%%%%%%%%%%%%%%%%%%%%%%%%%%%%%%%%%%%%%%%%%%%
%%%%%%%%%%%%%%%%%%%%%%%%%%%%%%%%%%%%%%%%%%%%%%%%%%%%%%%%%%%%%%%%%%%%%%%%%%%%%
\subsection{Transitioning to Tight Credit}
\label{sec:limit-transition}

How does the economy evolve during its transition from the steady state of subsection \ref{sec:limit-stst-loose} to the one of subsection \ref{sec:limit-stst-tight}? That is, how does the shock to the borrowing limit work through household balance sheets and then create distributional and macroeconomic dynamics? To answer these questions, section \ref{sec:limit-transition-specific} defines the transition path of the credit limit. Section \ref{sec:limit-transition-summ} briefly summarises the responses of the aggregate variables and section \ref{sec:limit-transition-mechanisms} explains the underlying mechanisms. Section \ref{sec:limit-transition-debt} briefly dwells on the prolonged transition of aggregate debt.

%%%%%%%%%%%%%%%%%%%%%%%%%%%%%%%%%%%%%%%%%%%%%%%%%%%%%%%%%%%%%%%%%%%%%%%%%%%%%
\subsubsection{Specifying the Transition}
\label{sec:limit-transition-specific}

Let the economy be in the initial steady state with $\phi_{ss}$ in $t=0$. Latest by $t=200$, the economy settles in the terminal steady state with $\phi_{ss}'$. Inbetween the two endpoints, the shock arises unexpectedly in $t=1$ and $\phi_t$ evolves deterministically thereafter according to an \Gls{ar}(1) process:
\begin{equation}
    \phi_t = \phi_{ss} \Bigg( \frac{\phi_{t-1}}{\phi_{ss}} \Bigg)^{\rho_{\phi}} exp(\varepsilon_{\phi}), \label{eq:borrowing-limit-process}
\end{equation}
where $\rho_{\phi}$ is the persistence of changes in the borrowing limit. $\varepsilon_{\phi}$ is a one-time shock to the borrowing limit but in this thesis I only consider permanent shocks and thus $\varepsilon_{\phi} = 0$ $\forall t$.

I follow \textcite{gl2017} in choosing a duration of the credit crunch of six quarters, which requires $\rho_{\phi} = 0.5$. Doing so is meant to capture the notion that a financial shock might not force households to delever fully within one period but might take some time to manifest itself in household balance sheets. Section \ref{sec:sensitivity-persistence} explores alternative choices for $\rho_{\phi}$.

%%%%%%%%%%%%%%%%%%%%%%%%%%%%%%%%%%%%%%%%%%%%%%%%%%%%%%%%%%%%%%%%%%%%%%%%%%%%%
\subsubsection{Summary of Aggregate Responses}
\label{sec:limit-transition-summ}

Figure \ref{fig:baseline-permanent-limit-agg} presents the dynamics of some key aggregate variables to the credit crunch. For the sake of brevity, I refer the reader to figure \ref{fig:baseline-permanent-limit-agg-extra} in appendix \ref{sec-app:figures-limit-transition} for some supplementary responses. First, plot \ref{fig:baseline-permanent-limit-agg-phi} shows the evolution of the borrowing limit from $\phi_{ss}$ to $\phi_{ss}'$. In $t=1$, the shock is announced -- think of it as the moment when lenders suddenly loose confidence in borrowers -- and the debt limit contracts by almost half of its total change.

\begin{figure}[!t]
    \centering
    \caption{Baseline Model -- Shock to Borrowing Limit: Aggregate Dynamics}
    \label{fig:baseline-permanent-limit-agg}
     \begin{subfigure}[b]{0.49\textwidth}
     \caption{Borrowing Limit, $\phi$}
     \label{fig:baseline-permanent-limit-agg-phi}
    \centering
      \includesvg[inkscapelatex=false, width=\textwidth]{Results/transition_phi_baseline_limit_permanent.svg}
     \end{subfigure}
     \vspace{5pt}
     \hfill
    \begin{subfigure}[b]{0.49\textwidth}
    \caption{Consumption, $C$}
    \label{fig:baseline-permanent-limit-agg-C}
         \centering
         \includesvg[inkscapelatex=false, width=\textwidth]{Results/transition_C_baseline_limit_permanent.svg}
     \end{subfigure}
     \vspace{5pt}
    \hfill
    \begin{subfigure}[b]{0.49\textwidth}
    \caption{Debt, $D$}
    \label{fig:baseline-permanent-limit-agg-d}
         \centering
         \includesvg[inkscapelatex=false, width=\textwidth]{Results/transition_D_baseline_limit_permanent.svg}
     \end{subfigure}
     \hfill
     \begin{subfigure}[b]{0.49\textwidth}
     \caption{Debt-to-Output, $\frac{D}{y}$}
     \label{fig:baseline-permanent-limit-agg-DY}
         \centering
         \includesvg[inkscapelatex=false, width=\textwidth]{Results/transition_DY_baseline_limit_permanent.svg}
     \end{subfigure}
     \hfill
     \begin{subfigure}[b]{0.49\textwidth}
     \caption{\textit{Ex-Post} Real Interest Rate, $RR$}
     \label{fig:baseline-permanent-limit-agg-RR}
         \centering
         \includesvg[inkscapelatex=false, width=\textwidth]{Results/transition_Rr_baseline_limit_permanent.svg}
     \end{subfigure}
     \hfill
     \begin{subfigure}[b]{0.49\textwidth}
     \caption{Nominal and Notional Interest Rates, $R$ and $R^n$}
     \label{fig:baseline-permanent-limit-agg-RN}
         \centering
         \includesvg[inkscapelatex=false, width=\textwidth]{Results/transition_R_Rn_baseline_limit_permanent.svg}
     \end{subfigure}

     \vspace{5pt}
     
     \justifying
     \footnotesize
	\textit{Note}: Wherever the terminal steady state of a variable differs from the initial steady state, the red dotted line displays the former. Percent deviations mean percent deviations from the respective initial steady state. The time horizon in all plots is fixed to twelve quarters, including the initial steady state in $t=0$.
\end{figure}

Upon announcement of the shock, the economy enters recession. While aggregate debt holdings fall only slightly, consumption drops strongly by roughly three percent. These different speeds imply that debt over output increases by $1.5$ percentage points.\footnote{The results in \textcite{gl2017} do not show such an increase. \textcite{justiniano2015} argue that an account of the Great Recession must be consistent with the observed surge in debt-to-\Gls{gdp}.} The \textit{ex-post} real interest rate also spikes on impact. The monetary authority would like to move the nominal interest rate into negative territory but the \Gls{zlb} starts binding. Figure \ref{fig:baseline-permanent-limit-agg-extra} shows that labour hours, wages, marginal costs and inflation fall. Taxes as well as dividends surge.

After the initial shock, the debt limit continues to tighten in a slower pace and the economy rebounds. Consumption exceeds its steady-state value for a short period of time, before reaching its original level roughly by period seven. Labour hours, wages, marginal costs, inflation and dividends follow along. The real and nominal interest rates undershoot their lower, terminal values and show quite some persistence on their path to those. Outstanding debt takes an exceptionally prolonged period of time -- roughly $15$ years -- to reach its terminal state. 

%%%%%%%%%%%%%%%%%%%%%%%%%%%%%%%%%%%%%%%%%%%%%%%%%%%%%%%%%%%%%%%%%%%%%%%%%%%%%
\subsubsection{Mechanisms}
\label{sec:limit-transition-mechanisms}

Next, I probe into the mechanisms giving rise to the observed responses, thus answering the following questions: why does aggregate consumption fall? What explains the evolution of the interest rates? How does the economy return to steady state?

Consider first the economy's instantaneous response to the credit crunch. On impact, the agents are surprised by the tighter borrowing limit which applies for their debt choices in the next period. The balance-sheet and precautionary-savings effects described in section \ref{sec:limit-stst-tight} begin to transpire. The tightening in the borrowing limit forces borrowers with debt between $\phi_{ss}$ and $\phi_1$ to delever. Through their budget constraints \eqref{eq:hh-budget}, the affected borrowers must cut back on consumption to consolidate their balance sheets. This balance-sheet effect results in reduced goods demand and in a contracting supply of \Gls{iou}s.

As the credit limit moves closer to all individuals, the precautionary-savings effect prompts even unconstrained borrowers to reduce debt, making \Gls{iou}s even more scarce, and unconstrained lenders to demand more safe assets. To achieve these higher wealth targets, both groups of agents are willing to forgo consumption. Importantly, recall from section \ref{sec:limit-stst-loose} that \Gls{mpc}s are heterogeneous and so, the degree by which agents reduce expenditures varies along their balance sheet positions. Individuals close to the borrowing limit thus curb consumption by much more than agents far away from the constraint.

As \textcite{gl2017} document, the balance-sheet and the precautionary-savings effects together provoke a strong partial-equilibrium reduction in aggregate consumption. Of course, in general equilibrium, there are amplifying as well as off-setting forces. Regarding the former, a dwindling goods demand reduces production, labour demand by firms and thus wages. Hence, labour incomes fall for all households but this hurts especially the low-skill agents with inadequate insurance against such shortfalls. As they feature high \Gls{mpc}s, they strongly reduce their consumption, amplifying the downturn.

Further, depressed wages lead to falling marginal costs and, via the \Gls{nkpc}, to deflation as firms try to maintain demand by adjusting prices downward. Through the Fisher equation \eqref{eq:hh-fisher-equation}, deflation in $t=1$ induces the \textit{ex-post} real interest rate on initial asset positions to increase sharply. An unexpectedly high real rate redistributes income from high-\Gls{mpc} borrowers to low-\Gls{mpc} lenders as the former face higher interest payments on last period's debt, which translate into capital gains for creditors. This is a channel not captured by the timing convention in \textcite{gl2017}.\footnote{This redistribution is not to be mistaken with a debt-deflation spiral à la \textcite{fisher1933}. Unexpected deflation here strains borrowers' budgets but does not increase their real debt burden.} The fiscal authority raises the tax burden on households to meet its obligations, contributing to the pressure on households' budgets.

Consider now the dampening forces in general equilibrium. One of those is that firms' dividends to households increase during the recession.\footnote{This is a typical result in a textbook \Gls{nk} environment with flexible wages, as the latter imply that marginal costs fall faster than the rigid prices do, giving rise to higher margins.} However, the most important adjustment is the one of the real interest rate. As the credit crunch induces a state of excess demand for bonds and \Gls{iou}s or, equivalently, a surplus of demand for consumption tomorrow relative to demand for consumption today, the real interest rate must fall after its unexpected surge. 

The real interest rate in $t=2$ is the key determinant for the consumption-savings decisions of households in $t=1$ and is driven by the nominal rate and inflation. With the advent of the shock, the monetary authority immediately adjusts its target rate to $R_{ss}' - 1 = 0.13\%$. Concurrently, the pronounced downward deviations of inflation and output from their steady-state levels let the Taylor rule prescribe a nominal rate of roughly $-0.3$ percent \textit{per annum} but the \Gls{zlb} starts binding. Still, this achieves a downward adjustment in the real rate in $t=2$. 

The lower interest rate induces intertemporal substitution by dampening the desire of unconstrained agents to achieve lower debt or higher asset holdings, respectively, in favour of consumption today. This reduces the excess-demand pressure in the asset market in $t=1$. The \Gls{zlb} prevents a more forceful reduction of the real rate and thus confines the incentive to expand consumption. However, unlike in \textcite{egg2012} and as demonstrated in appendix \ref{sec-app:robust-no-zlb}, the \Gls{zlb} here is indeed only an amplifier and not the main mechanism behind the recession. At the same time, those agents who are most interest-rate sensitive feature low \Gls{mpc}s. This implies that, even amid falling rates, these households do not substitute away from their increased savings demand sufficiently and consequently do not aid much in dampening the recession. The result of all the above forces is the strong general-equilibrium reduction of aggregate consumption witnessed in figure \ref{fig:baseline-permanent-limit-agg-C}.

Lower interest rates feedback to unconstrained debtors and their decisions on how much debt to hold are thus influenced by counteracting forces: falling consumption and dampened rates encourage borrowing but ever closer limits discourage it. Figure xxx displays the reaction of the average borrower across debt holdings between $\phi_1$ and zero. Evidently, unconstrained borrowers increase their indebtedness on average. Thus, in the present calibration, the former of the above forces slightly dominates the latter and unconstrained households take up more debt in the admissible range of debt. This also explains the mild impact-decline of total debt, driven by constrained and unconstrained borrowers in the aggregate, in plot \ref{fig:baseline-permanent-limit-agg-d}.\footnote{Some calibrations, not shown, indeed produce an increase of aggregate debt on impact.}

%As aggregate debt is determined by the policies of \textit{all} borrowers. 
%As mentioned above, constrained borrowers must delever by the amount prescribed by the exogenous movement in $\phi_t$. Unconstrained borrowers are in principle free to choose more or less debt. As 
%lower interest rate apply to the liability side of balance sheets, too, debt becomes relatively cheaper. Thus, unconstrained debtors' decisions on how much debt to hold are influenced by counteracting forces: falling consumption and dampened rates encourage borrowing but ever closer limits discourage it. Plot \ref{fig:baseline-permanent-limit-agg-d} shows a mild decline of aggregate debt on impact in the present calibration, so the latter effect slightly dominates the former and unconstrained households do take up more debt in the admissible range of indebtedness.\footnote{An increase of total debt would be possible as the shock restricts the maximum quantity of each household's debt and not economy-wide indebtedness. Some calibrations, not shown, indeed produce an increase on impact.}

This leads to a noteworthy result: despite the borrowing limit tightening most strongly in $t=1$, total debt does not decline much and yet, the economy enters recession. Thus, one might infer that the model features strong internal amplification through heterogeneous \Gls{mpc}s: a financial shock might not force all debtors to delever but it affects the most vulnerable ones, i.e.~those at and close to the initial limit, and thus provokes a strong reaction in aggregate consumption, which in turn sets into motion the amplifying effects described above.

Figure \ref{fig:baseline-permanent-limit-dist-c-response} confirms that the credit crunch chiefly hurts households at the lower end of the asset distribution by plotting the consumption responses of the bottom-$25\%$, the bottom-$50\%$ and the top-$25\%$ in the distribution. In line with the above deliberations and the findings in \textcite{gl2017}, the recession harms the bottom-$25\%$ the most: they reduce expenditures by over eight percent on impact of the shock. In contrast, the consumption losses for the top-$25\%$ are less than those in the aggregate as these agents are better insured and react most strongly to falling interest rates. The figure also foreshadows the increased long-run consumption inequality.

\begin{figure}[t]
    \centering
    \caption{Baseline Model -- Shock to Borrowing Limit: \\ Consumption Response by Percentiles}
    \label{fig:baseline-permanent-limit-dist-c-response}
    \includesvg[inkscapelatex=false, width=0.75\textwidth]{Results/percentile_transitions_baseline_limit_permanent_C.svg}

    \vspace{10pt}
    
    \begin{minipage}{0.75\textwidth} 
    \footnotesize
    \textit{Note}: The responses are calculated as the product of aggregate consumption in period $t$ and the share that the respective household percentile has in consumption in $t$. Percent deviations mean percent deviations from the respective initial steady state. The time horizon in the plot is fixed to twelve quarters, including the initial steady state in $t=0$.
    \end{minipage}
\end{figure}

How does the economy evolve after the initial shock? The debt limit keeps tightening over the next five quarters, albeit at an ever slower pace. Consistent with the findings of \textcite{gl2017}, the real interest rate continues its downward adjustment and undershoots its terminal value before approaching it from below. The intuition for the undershooting is as follows: while the debt limit contracts, the asset distribution features, in each period, too many debtors to the left of the final debt limit in comparison to the terminal density.\footnote{The code file \texttt{main.py} visualises the transition of the asset distribution. I omit it here for the sake of brevity.} These agents are subject to the balance-sheet effect as they must mechanically delever. An undershooting real interest rate guarantees equilibrium amid the pressure from this shortfall in \Gls{iou} supply.

The anticipation of ever lower real rates stimulates consumption for the unconstrained agents and lets the economy rebound from $t=2$ on. As demand picks up, so does inflation, which in turn amplifies the recovery by further lowering expected real rates. Inflation even pushes the real interest rate into negative territory for two quarters. This instigates a small and short-lived boom in consumption, which provokes a counter-reaction of the central bank in the form of higher nominal interest rates and finally a return of the economy to the steady state.

In summary, a shock to the debt limit triggers mechanical deleveraging, which reduces the supply of liquidity, and an enforced demand for precautionary savings. Unexpected inflation and heterogeneous \Gls{mpc}s amplify the strong drop in consumption initiated by these forces. To maintain equilibrium, the real interest rate falls and undershoots its terminal value as the asset distribution adjusts towards the state of tight credit.

%%%%%%%%%%%%%%%%%%%%%%%%%%%%%%%%%%%%%%%%%%%%%%%%%%%%%%%%%%%%%%%%%%%%%%%%%%%%%
\subsubsection{The Transition of Debt}
\label{sec:limit-transition-debt}

As mentioned above, aggregate debt takes a very long time to attain its terminal value. By the time the borrowing limit completes its transition in $t=6$, the debt-to-\Gls{gdp} ratio has merely reached $55.5\%$ and the rest of the transition takes until $15$ years after the shock arose, see figure \ref{fig:baseline-permanent-limit-longrun} in the appendix. This is a robust feature of all model versions in this thesis.\footnote{For example, I confirmed that this feature persists also with $\omega = 0$ or $\psi = 10$.}

Interestingly, the slow adjustment of debt seems to be a quite general aspect of the \Gls{hank} model. Appendix \ref{sec-app:robust-beta} shows that after a permanent shock to households' discount factor, the debt quantities similarly take exceptionally long to reach their terminal levels. Thus, agents in this model find it optimal to gradually change their asset positions. Presumably, this is because the solution method lets agents perfectly know the path of the economy and there is no immediate incentive for households (e.g.~in the utility function) to quickly adjust their balance sheets.

Figures \ref{fig:baseline-permanent-limit-agg-d} and \ref{fig:baseline-permanent-limit-agg-DY} further reveal that the speed of debt reduction changes over time. Between $t=1$ and $t=5$, outstanding debt falls faster than before and after. This is due to the demand for and the supply of debt being particularly dampened during that period. Falling rates weaken lenders' asset demand, constrained borrowers continue to delever mechanically, while unconstrained borrowers require less debt amid rising consumption and incomes. As the real rate as well as consumption reverse course and the borrowing limit stops tightening, the efforts to delever lose speed and debt settles into a steady downward path. I come back to this issue throughout the following sections, \textit{inter alia} during the sensitivity analyses conducted next.

%%%%%%%%%%%%%%%%%%%%%%%%%%%%%%%%%%%%%%%%%%%%%%%%%%%%%%%%%%%%%%%%%%%%%%%%%%%%%
%%%%%%%%%%%%%%%%%%%%%%%%%%%%%%%%%%%%%%%%%%%%%%%%%%%%%%%%%%%%%%%%%%%%%%%%%%%%%
%%%%%%%%%%%%%%%%%%%%%%%%%%%%%%%%%%%%%%%%%%%%%%%%%%%%%%%%%%%%%%%%%%%%%%%%%%%%%
\section{Sensitivity Analyses}
\label{sec:sensitivity}

This section explores how two choices made in the preceding analysis influence its results. First, subsection \ref{sec:sensitivity-persistence} varies the persistence of the shock process \eqref{eq:borrowing-limit-process} to show how the economy's responses differ across varying speeds of the transition to tight credit. Second, subsection \ref{sec:sensitivity-end-labour} alters $u( \cdot )$ in \eqref{eq:hh-objective} so that households can self-insure by choosing their own working hours to investigate whether this is an important insurance mechanism for borrowers.

%%%%%%%%%%%%%%%%%%%%%%%%%%%%%%%%%%%%%%%%%%%%%%%%%%%%%%%%%%%%%%%%%%%%%%%%%%%%%
%%%%%%%%%%%%%%%%%%%%%%%%%%%%%%%%%%%%%%%%%%%%%%%%%%%%%%%%%%%%%%%%%%%%%%%%%%%%%
\subsection{Duration of the Credit Crunch}
\label{sec:sensitivity-persistence} 

Section \ref{sec:limit} lets the debt limit tighten within six quarters. It motivates this duration with the notion that, in reality, deleveraging shocks might take time to take full effect, e.g.~because household balance sheets have a richer maturity structure than in the model. The choice for six quarters is still arbitrary, so this subsection aims at quantifying the effect of different speeds. Intuitively, I find that faster evolving shocks are considerably more harmful.

Compare the baseline scenario from section \ref{sec:limit-transition} to two alternatives: first, the shock resolves in ten quarters and second, the shock lasts for only four quarters. The former experiment requires setting the persistence of \eqref{eq:borrowing-limit-process}, $\rho_{\phi}$, equal to $0.62$ (``slow" shock), while the latter implies $\rho_{\phi} = 0.3$ (``fast" shock).\footnote{Changing $\rho_{\phi}$ affects the asset grid on which the model is solved. This is due to the computational challenges associated with shocks to the borrowing limit, see appendix \ref{sec-app:codes-ep}. This should not influence the results much.} Figure \ref{fig:baseline-permanent-limit-duration} contrasts the responses of aggregate consumption and debt across persistence parameters. Appendix figure \ref{fig:baseline-permanent-limit-duration-extra} contains some supplementary responses.

\begin{figure}[t]
    \centering
    \caption{Baseline Model -- Varying the Duration of a Credit Crunch}
    \label{fig:baseline-permanent-limit-duration}
     \begin{subfigure}[b]{0.49\textwidth}
     \caption{Consumption, $C$}
     \label{fig:baseline-permanent-limit-duration-C}
         \centering
         \includesvg[inkscapelatex=false, width=\textwidth]{Results/compare_transitions/comparison_C_fast_shock_limit_permanent_baseline_limit_permanent_slow_shock_limit_permanent.svg}
     \end{subfigure}
     \hfill
     \begin{subfigure}[b]{0.49\textwidth}
     \caption{Debt, $D$}
     \label{fig:baseline-permanent-limit-duration-D}
         \centering
         \includesvg[inkscapelatex=false, width=\textwidth]{Results/compare_transitions/comparison_D_fast_shock_limit_permanent_baseline_limit_permanent_slow_shock_limit_permanent.svg}
     \end{subfigure}

    \vspace{10pt}
     
     \justifying
     \footnotesize
	\textit{Note}: Percent deviations mean percent deviations from the respective initial steady state. The time horizon in all plots is fixed to twelve quarters, including the initial steady state in $t=0$.
\end{figure}

Clearly, the quicker the transition to the tight credit limit, the harsher the recession on impact. Consumption drops by over four percent relative to its initial level in case of a quickly evolving shock. In contrast, allowing for an adjustment period of ten quarters results in a drop of about two percent on impact. The baseline scenario lies inbetween the two extreme cases. The findings are intuitive: a quicker shock reduces the borrowing limit more forcefully, see plot \ref{fig:baseline-permanent-limit-duration-extra-phi}. This mechanically intensifies the deleveraging forced on constrained agents and their consumption reductions, consequently reinforcing the balance-sheet effect. As the debt constraint moves more quickly towards the unconstrained agents, the precautionary-savings effect is similarly more pronounced. The result is that consumption drops more strongly. 

At the same time, plot \ref{fig:baseline-permanent-limit-duration-C} reveals that the recovery is more pronounced in the scenario of a fast shock. This is because the real rate falls by more and does so more quickly (picture \ref{fig:baseline-permanent-limit-duration-Rr}), as it must achieve equilibrium amid a larger fraction of households deleveraging and an increase in the precautionary-savings motive over and above the one present in the baseline.

Plot \ref{fig:baseline-permanent-limit-duration-D} shows how the transition speed of the credit limit influences the pace by which aggregate debt adjusts. As expected, a faster contraction in the limit lets outstanding debt fall more vigorously. However, beyond $t=10$, the paths of debt are identical across the three cases, see figure \ref{fig:baseline-permanent-limit-duration-long-run} in the appendix. These observations complement the discussion of section \ref{sec:limit-transition-debt}. Faster and stronger contractions in the debt limit provoke more pronounced disequilibria and thus force the real interest rate to fall faster, which shrinks demand for assets by the wealthy more quickly. Yet, the stronger recovery prompts unconstrained borrowers to require less debt and the supply of debt is not as large. Consistent with this logic, the speeds of deleveraging coincide once the paths for consumption and the real rate are identical across scenarios, i.e.~roughly from $t=10$ onward.

Overall, different speeds of adjustment to the regime of tight credit produce the same short-run patterns but with varying quantitative intensity. Quicker transitions enforce faster deleveraging and leave agents less time to adjust to the tighter credit limit, making the immediate downturn more painful, while leaving the long-run steady state unaffected.

%%%%%%%%%%%%%%%%%%%%%%%%%%%%%%%%%%%%%%%%%%%%%%%%%%%%%%%%%%%%%%%%%%%%%%%%%%%%%
%%%%%%%%%%%%%%%%%%%%%%%%%%%%%%%%%%%%%%%%%%%%%%%%%%%%%%%%%%%%%%%%%%%%%%%%%%%%%
\subsection{Insurance through Endogenous Labour Supply}
\label{sec:sensitivity-end-labour}

The \Gls{ghh} preferences in the baseline model preclude households from adjusting their individual working hours to absorb shocks. This section explores an extension of the model which allows agents to do so. The analysis shows that, amid a credit crunch, the wealth-poor indeed expand their labour effort, thus dampening the aggregate response of the economy significantly.

Consider households with \Gls{crra}-type preferences:
\begin{equation}
    u(c_{i,t}, n_{i,t}) = \frac{c_{i,t}^{1-\sigma_c}}{1-\sigma_c} - \phi \frac{n_{i,t}^{1+\sigma_l}}{1+\sigma_l}, \label{eq:hh-crra-utility}
\end{equation}
where $\phi$ is the weight on the disutility of labour. Consult appendix \ref{sec-app:full-derivations-households} for the implied modifications of the formal aspects of the model. 

I follow \textcite{mckay2016} and set $\phi=1$.\footnote{\textcite{gl2017} also use \Gls{crra} preferences. In their baseline, they set $\psi = 12.48$ but choose $\psi = 1.389$ in a ``low-$\psi$" calibration.} I re-calibrate $\beta$, $\phi_{ss}$ and $\phi_{ss}'$ to match the same targets as before, in particular: debt to output is $57\%$ initially and equals $52\%$ in the terminal steady state.\footnote{Specifically: $\beta = 0.98801$, $\phi_{ss} = -1.7956$, $\phi_{ss}' = -1.655$ and $B = 5.39$. $\rho_{\phi}$ is again $0.5$. Note that leaving $\sigma_l$ unchanged is likely to change the average Frisch elasticity in the economy.} The rest of the model remains unchanged, but of course the steady states can be quite different to those described in sections \ref{sec:limit-stst-loose} and \ref{sec:limit-stst-tight}. In the appendix, figure \ref{fig:init-stst-end-L} and table \ref{tab:stst_comparison_end_L_limit_permanent} reproduce plot \ref{fig:init-stst-dist-pol} and table \ref{tab:stst_comparison_baseline_limit_permanent}, respectively.

Figures \ref{fig:comparison-end-labour-agg} and \ref{fig:comparison-end-labour-agg-extra} compare the model extension to the baseline after a tightening in the debt limit. Stark differences are visible. Plot \ref{fig:comparison-end-labour-c} shows that, on impact, consumption drops by only slightly more than half a percent, roughly six times less than with \Gls{ghh} preferences. As consumption does not fall as strongly, goods demand deteriorates by less, implying that wages, marginal costs and prices do not drop as much as they do in the baseline.\footnote{Plots \ref{fig:comparison-end-labour-agg-extra-N} and \ref{fig:comparison-end-labour-agg-extrag-w} suggest that, indeed, the average Frisch elasticity might differ from the baseline.} The central bank reduces the nominal interest rate but the \Gls{zlb} does not bind. Further, consumption quickly returns to its pre-shock level by $t=3$ and does not exhibit the same boom as in the baseline. 

\begin{figure}[t]
    \centering
    \caption{Endogenous Labour Supply -- Shock to the Borrowing Limit: Aggregate Dynamics}
    \label{fig:comparison-end-labour-agg}
    \begin{subfigure}[b]{0.49\textwidth}
    \caption{Consumption, $C$}
    \label{fig:comparison-end-labour-c}
         \centering
         \includesvg[inkscapelatex=false, width=\textwidth]{Results/compare_transitions/comparison_C_baseline_limit_permanent_end_L_limit_permanent.svg}
     \end{subfigure}
     \hfill
    \begin{subfigure}[b]{0.49\textwidth}
    \caption{Debt-to-Output, $\frac{D}{y}$}
    \label{fig:comparison-end-labour-DY}
         \centering
         \includesvg[inkscapelatex=false, width=\textwidth]{Results/compare_transitions/comparison_DY_baseline_limit_permanent_end_L_limit_permanent.svg}
     \end{subfigure}
     
     \vspace{10pt}
     
     \justifying
     \footnotesize
	\textit{Note}: Percent deviations mean percent deviations from the respective initial steady state. The time horizon in all plots is fixed to twelve quarters, including the initial steady state in $t=0$.
\end{figure}

Why is the short-run fallout so much less pronounced? Table \ref{tab:stst_comparison_end_L_limit_permanent} could have suggested the opposite: more households are in debt and at the constraint and borrowers exhibit much higher \Gls{mpc}s than in the baseline, insinuating a stronger response of the economy to the credit crunch. However, these statistics can of course not capture the total consumption response. 

Crucially, the preferences in \eqref{eq:hh-crra-utility} equip households with labour supply choice as a tool to absorb shocks. This margin of adjustment is used particularly by those hardest hit by the financial shock. To see this, consider figure \ref{fig:end-L-permanent-limit-percentiles} which displays households' consumption and labour supply responses separately for the bottom-$25\%$, bottom-$50\%$ and top-$25\%$ in the distribution. While the bottom-$25\%$ experience the largest consumption losses as in the baseline (refer back to figure \ref{fig:baseline-permanent-limit-dist-c-response}), they can now individually adjust their labour hours, which they do as plot \ref{fig:end-L-permanent-limit-percentiles-n} reveals: the wealth-poor expand the amount of work they supply by over three percent on impact. The bottom-$50\%$ also increase labour hours relative to steady state.

\begin{figure}[t]
    \centering
    \caption{Endogenous Labour Model -- Shock to Borrowing Limit: \\ Consumption and Labour Responses by Percentiles}
    \label{fig:end-L-permanent-limit-percentiles}
    \begin{subfigure}[b]{0.49\textwidth}
    \caption{Consumption}
    \label{fig:end-L-permanent-limit-percentiles-c}
         \centering
         \includesvg[inkscapelatex=false, width=\textwidth]{Results/percentile_transitions_end_L_limit_permanent_C.svg}
     \end{subfigure}
     \hfill
    \begin{subfigure}[b]{0.49\textwidth}
    \caption{Labour Supply}
    \label{fig:end-L-permanent-limit-percentiles-n}
         \centering
         \includesvg[inkscapelatex=false, width=\textwidth]{Results/percentile_transitions_end_L_limit_permanent_N.svg}
     \end{subfigure}
     
     \vspace{10pt}
     
     \justifying
     \footnotesize
	\textit{Note}: The responses are calculated as the product of aggregate consumption (or labour) in period $t$ and the share that the respective household percentile has in consumption (or labour) in $t$. Percent deviations mean percent deviations from the respective initial steady state. The time horizon in all plots is fixed to twelve quarters, including the initial steady state in $t=0$.
\end{figure}

By working more, those experiencing the harshest financial troubles -- predominantly those agents who are mechanically forced to delever -- can shield their consumption from even bigger strains. As these agents have high \Gls{mpc}s, their efforts to do so stabilise the aggregate very effectively, leading to the observed milder responses. In turn, a weaker consumption reduction, together with a subsequently dampened need of unconstrained borrowers to tap into debt, induces a \textit{fall} in the ratio of debt to output instead of the increase observed in section \ref{sec:limit}, see figure \ref{fig:comparison-end-labour-DY}. Overall, debt completes its transition to its terminal level in roughly half the time it needs in the baseline, hinting at the fact that idiosyncratic labour supply allows aggregate deleveraging to proceed more swiftly. However, as the models are quite different, a definite conclusion is hard to reach.

These results suggest that idiosyncratic labour supply might be a quantitatively important device for households to insure against the fallout of financial shocks. This is, admittedly, a natural result. However, it crucially depends on a number of aspects, the choice for $\psi$ in particular. $\psi$ guides how much disutility agents sustain when supplying labour (working more is still costly in this model) and thus greatly affects the shape of the responses. Exploring the effect of this parameter in the present use case is, however, beyond the scope of this thesis.

%%%%%%%%%%%%%%%%%%%%%%%%%%%%%%%%%%%%%%%%%%%%%%%%%%%%%%%%%%%%%%%%%%%%%%%%%%%%%
%%%%%%%%%%%%%%%%%%%%%%%%%%%%%%%%%%%%%%%%%%%%%%%%%%%%%%%%%%%%%%%%%%%%%%%%%%%%%
%%%%%%%%%%%%%%%%%%%%%%%%%%%%%%%%%%%%%%%%%%%%%%%%%%%%%%%%%%%%%%%%%%%%%%%%%%%%%
\section{A Credit Crunch through the Interest Rate Wedge}
\label{sec:wedge}

Having understood the working of a credit crunch through the borrowing limit, this section considers a credit crunch through a permanent increase in the premium paid on debt holdings, $\kappa_t$. By doing so, I aim to identify how this financial shock differs in its short- and long-term consequences from the shock to the debt limit. I deem this interesting as it might not be clear \textit{ex-ante} whether the two shocks work similarly.

On the one hand, as remarked by \textcite{gl2017}, the wedge shock affects all borrowers, suggesting more pronounced responses. On the other hand, unlike the shock to the debt limit, it does not eliminate the possibility for households to take large debt positions. This might ameliorate the severity of the shock. Moreover, remember the shock is to the credit \textit{spread}, not to the level of the borrowing rate \textit{per se}. The effects crucially depend on the endogenous evolution of the real rate. As a result, the findings point to similar overall mechanisms across the two credit-crunch scenarios but also to considerable quantitative divergences in the dynamics. 

Let the state of loose credit be characterised by $\kappa_{ss} =$ 1e-8 and $\phi_{ss}$ as before. Setting $\kappa_{ss}' = 0.00203$, i.e.~a premium of $0.203\%$ per quarter on the lending rate, yields a debt-to-output ratio which is five percentage points lower in the state of tight credit, making the comparison to the debt limit shock fair. Between the two steady states, the interest rate wedge evolves as $\kappa_t = \kappa_{ss} ( \kappa_{t-1} / \kappa_{ss} )^{\rho_{\kappa}} exp(\varepsilon_{\kappa})$, where choosing the persistence of the spread, $\rho_{\kappa}$, equal to $0.3$ results in an adjustment within six periods. As before, $\varepsilon_{\kappa} = 0$ $\forall t$.

Consider first the steady states in table \ref{tab:stst_comparison_baseline_wedge_permanent}. The increase of the spread by roughly $0.8$ percentage points in annual terms achieves the desired long-run reduction in debt-to-output, which, as is the case with a tighter credit limit, is accompanied by a permanently lower real rate on lending. However, in the present scenario, this rate differs from the rate paid on debt, which now equals $1.32\%$ \textit{per annum}. Through \eqref{eq:effective-gdp}, this mechanically decreases long-run aggregate consumption as the credit spread reflects a wasted cost in the present model.

\begin{table}[t]
\centering
\caption{Baseline Model -- Shock to Interest Rate Wedge: \\ Comparison of Steady States}
\label{tab:stst_comparison_baseline_wedge_permanent}
\begin{tabular}{lccc}
Variable & Initial & Terminal &  Change \\
\hline
\hline
\multicolumn{2}{l}{\textbf{Interest Rate Wedge}} & & \\
$\kappa$ & $0\%$ & $0.203\%$ & $0.203 \ p.p.$ \\
& & & \\
\multicolumn{2}{l}{\textbf{Aggregates}} & & \\
$C$ &  $0.9129$ &   $0.9119$ & $-0.11\%$ \\
 $D / y$ &  $57\%$ &  $52\%$ & $-5 \ p.p.$ \\
$B+D$ &  $5.73$ &   $5.68$ & $-0.81\%$ \\
$Rr-1$ &  $0.15\%$ & $0.12\%$ & $-0.03 \ p.p.$ \\
$(RR+\kappa)-1$ &  $0.15\%$ & $0.33\%$ &  $0.18 \ p.p.$ \\
& & & \\
\multicolumn{2}{l}{\textbf{Cross-Section}} & & \\
Share of Assets Held by Top-$10\%$ &  $48.34$ &   $47.42$ & $-0.92 \ p.p.$ \\
Share of Consumption by Top-$10\%$ &  $17.29$ &   $17.34$ &  $0.05 \ p.p.$ \\
Fraction of Borrowers &   $33.66$ &    $31.87$ & $-1.79 \ p.p.$ \\
Fraction at Borrowing Limit &    $4.89$ &     $4.17$ & $-0.72 \ p.p.$ \\
\Gls{mpc} of Borrowers &    $0.16$ &     $0.14$ &  \\
\Gls{mpc} of Lenders &    $0.02$ &     $0.02$ & \\
\hline
\multicolumn{4}{l}{\footnotesize \multirow{2}{12cm}{\justifying \textit{Note:} Selected values of the steady states with $\kappa_{ss}$ and $\kappa_{ss}'$. Numbers are rounded and refer to quarterly values. $p.p.$ stands for percentage points.}} \\
& & & \\
\end{tabular}
\end{table}

Despite the shock working through the price on debt rather than its quantity, the overall mechanisms at play here are a mirror-image of those identified in section \ref{sec:limit}. As borrowers pay permanently more on their liabilities, they can not afford to issue debt to the same extent as they did in an environment of cheap credit, prompting them to delever. This gives rise to the balance-sheet effect by which gross liquidity in the economy is reduced. Concurrently, the precautionary-savings motive induces lenders to increase their asset demand in order to avoid being forced to use the now more costly option of borrowing when confronted with unfortunate productivity realisations. Analogously to before, the balance-sheet and precautionary-savings effects call for lower real interest rates to clear the asset market. 

However, as I calibrate terminal debt-to-output to be the same, debtors have to pay a permanently increased rate on borrowing and thus have to reduce long-run consumption by more in comparison to the scenario of a tighter debt limit, see figure \ref{fig:comparison-stst-baseline-wedge-policies-c} in the appendix. In the present case, the interest rate on lending must hence fall slightly more in order to entice expanded consumption by the wealthy and thus ensure equilibrium. This implies that consumption inequality rises more than after the debt limit shock, although still in economically small terms. Also, for the same reduction in the debt-to-\Gls{gdp} ratio, the fraction of borrowers falls stronger ($-1.79$ here versus $-0.76$ percentage points before) and instead of higher, the fraction of households at the debt limit is now lower. Clearly, this is due to borrowing being especially costly at very high levels of indebtedness.

Having established that the two shocks operate overall analogously in the long run, I next ask what differences arise between the two credit-crunch scenarios in the short run. Are the responses similar to those identified in \ref{sec:limit}? If not, why? Figures \ref{fig:baseline-permanent-wedge-agg} and \ref{fig:baseline-permanent-wedge-agg-extra} present selected dynamics from the low to the high credit spread, while also depicting the response to the credit limit for reference. Plot \ref{fig:baseline-permanent-wedge-agg-kappa} visualises how the shock process starts very close to zero, implying that the \Gls{ar}(1) process exhibits only a small increase in $\kappa_1$ relative to $\kappa_{ss}$. 

\begin{figure}[t]
    \centering
    \caption{Baseline Model -- Shock to Interest Rate Wedge: Aggregate Dynamics}
    \label{fig:baseline-permanent-wedge-agg}
     \begin{subfigure}[b]{0.49\textwidth}
     \caption{Interest Rate Wedge, $\kappa$}
     \label{fig:baseline-permanent-wedge-agg-kappa}
         \centering
         \includesvg[inkscapelatex=false, width=\textwidth]{Results/compare_transitions/comparison_Rbar_baseline_limit_permanent_baseline_wedge_permanent.svg}
     \end{subfigure}
     \vspace{5pt}
     \hfill
     \begin{subfigure}[b]{0.49\textwidth}
     \caption{\textit{Ex-Post} Real Rate on Debt, $RR + \kappa$}
     \label{fig:baseline-permanent-wedge-agg-Rr-on-debt}
         \centering
         \includesvg[inkscapelatex=false, width=\textwidth]{Results/compare_transitions/comparison_Rrminus_baseline_limit_permanent_baseline_wedge_permanent.svg}
     \end{subfigure}
     \vspace{5pt}
     \hfill
    \begin{subfigure}[b]{0.49\textwidth}
    \caption{Consumption, $C$}
    \label{fig:baseline-permanent-wedge-agg-C}
         \centering
         \includesvg[inkscapelatex=false, width=\textwidth]{Results/compare_transitions/comparison_C_baseline_limit_permanent_baseline_wedge_permanent.svg}
     \end{subfigure}
    \hfill
    \begin{subfigure}[b]{0.49\textwidth}
    \caption{Debt, $D$}
    \label{fig:baseline-permanent-wedge-agg-D}
         \centering
         \includesvg[inkscapelatex=false, width=\textwidth]{Results/compare_transitions/comparison_D_baseline_limit_permanent_baseline_wedge_permanent.svg}
     \end{subfigure}

    %\vspace{10pt}
     
     \justifying
     \footnotesize
	\textit{Note}: Percent deviations mean percent deviations from the respective initial steady state. The time horizon in all plots is fixed to twelve quarters, including the initial steady state in $t=0$.
\end{figure}

At first, a rise in the premium is merely an unexpected additional obligation for debtors. To meet this payment, borrowing households -- especially those at extreme levels of debt with high \Gls{mpc}s -- reduce consumption. This is not compensated for by lenders as the premium is a wasted cost. Of course, upon announcement of the shock, agents get to know that the wedge will be permanently higher and adjust optimally. The balance-sheet and precautionary-savings effects start to apply. First, debtors aim to start deleveraging as they face elevated debt servicing costs in the future. To do so, they target for lower consumption. Second, lenders immediately strive for higher precautionary buffers and less consumption. As in section \ref{sec:limit-transition}, this partial-equilibrium reduction in consumption kicks off amplifying forces: falling labour incomes, deflation fuelling redistribution from borrowers to lenders through a higher \textit{ex-post} real rate and higher taxes. 

The crucial counteracting force in general equilibrium is the real interest rate. To clear markets, it must fall in $t=2$. It does so as the central bank lowers its policy rate amid the drop in output. Falling rates dampen the wish of households to forgo consumption and weaken the partial-equilibrium response of it. The general-equilibrium result is that consumption drops on impact (plot \ref{fig:baseline-permanent-wedge-agg-C}) but, evidently, much less so than after the shock to the borrowing limit. Thus, the recession is less pronounced and the nominal interest rate stays away from the \Gls{zlb}.\footnote{\textcite{gl2017} also perform a shock to the credit spread which has much larger quantitative effects than their shock to the credit limit. However, it is hard to compare the results as they consider a transitory shock and a much more involved model with durable goods which serve as collateral.} 

Interestingly, the drop in the real rate is even sufficient to let the borrowing rate, $RR + \kappa$, decrease in $t=2$ despite an accelerating premium, see figure \ref{fig:baseline-permanent-wedge-agg-Rr-on-debt}. That is, even as the \textit{spread} of the borrowing over the lending rate rises in $t=2$, the endogenous adjustment of the real rate makes borrowing relatively cheaper. Effectively, this off-setting development in the real rate gives households more time to adjust their balance sheets to the new terminal conditions. 

The conclusion that the shock to the spread leaves agents a considerable amount of time to adjust can be reached by considering figure \ref{fig:comparison-wedge-analogy-slow} in the appendix. The figure compares the shock to the borrowing limit to a shock to the debt limit with a persistence parameter, $\rho_{\phi}$, calibrated such that the latter shock produces the same impact-response of consumption as the former shock. This requires $\rho_{\phi} = 0.86$, i.e.~a very slow shock. Indeed, the impact-fall in aggregate debt is almost indiscernible in plot \ref{fig:baseline-permanent-wedge-agg-D}. 

Of course, in the scenario of section \ref{sec:limit}, interest rates also fall. However, in that case, the credit limit keeps tightening, enforcing more deleveraging and even lower rates. This is absent in the spreas-scenario and it is exclusively the interest rate on debt, the spread \textit{plus} the real rate, which inhibits borrowers' debt capacity. Thus, the change in the real rate affects the process of deleveraging much more directly and attenuates it. This lessens the excess demand for bonds, rendering the economic fallout smaller than after the borrowing limit shock. As a result, the real rate does not undershoot its terminal value by much (figure \ref{fig:baseline-permanent-wedge-agg-Rr}). 

Yet, it is enough to induce consumption to recover from its trough, which thereafter converges to a level permanently below its initial steady state by $0.11\%$, as remarked above.\footnote{The consumption responses are again heterogeneous. Figure \ref{fig:baseline-permanent-wedge-dist-c-response} in the appendix reproduces figure \ref{fig:baseline-permanent-limit-dist-c-response} and demonstrates that, despite the losses being smaller, they seem to be concentrated more among the bottom-$25\%$ and $50\%$ than after a borrowing limit shock.} The transition of aggregate debt is again very slow, even more so than in response to the credit limit shock. This has two reasons. First, the endogenous adjustment of the real rate gives borrowers more time to delever. Second, the fact that the interest rate does not undershoot its terminal level as much lets asset demand to remain rather elevated.

One could argue that the weak response uncovered here is due to the \Gls{ar}(1) process implying a meagre increase of the spread on impact. However, computing another hike from $\kappa_{ss}'$ to $\kappa_{ss}'' = 0.0041$ -- yielding a further drop in debt-to-output by five percentage points -- generates an immediate increase in $\kappa_1$ but almost identical responses for the overall consumption effect.

All in all, for the same long-term reduction in debt-to-\Gls{gdp}, the credit crunch through the credit spread produces more benign responses in comparison to the one caused by a tightening debt limit. I argue that this is because the former works in conjunction with the real interest rate, thus granting agents more time to adjust and thus resembles a slowly evolving shock to the credit limit.

%%%%%%%%%%%%%%%%%%%%%%%%%%%%%%%%%%%%%%%%%%%%%%%%%%%%%%%%%%%%%%%%%%%%%%%%%%%%%
%%%%%%%%%%%%%%%%%%%%%%%%%%%%%%%%%%%%%%%%%%%%%%%%%%%%%%%%%%%%%%%%%%%%%%%%%%%%%
%%%%%%%%%%%%%%%%%%%%%%%%%%%%%%%%%%%%%%%%%%%%%%%%%%%%%%%%%%%%%%%%%%%%%%%%%%%%%
\section{The Asymmetry of Financial Shocks}
\label{sec:asymmetry}

This section explores the asymmetry of financial shocks, i.e.~the possibility that the sign of the
shock is relevant for the absolute magnitude of its effect on some outcome. \textcite{kocherlakota2000} remarks that one of the stylised features of business cycles is that they are asymmetric. Can financial shocks -- with ``financial shocks" not necessarily corresponding exactly to the ones analysed in the present thesis -- explain such behaviour in principle? 

There is some evidence that the answer might be affirmative. For example, \textcite{guerrieri2017asym} estimate a dynamic, two-agent general-equilibrium model and find that the occasionally binding housing collateral constraint accounts for the asymmetric relationship between house prices and consumption before and during the Great Recession. In an econometric framework, \textcite{barnichon2022} show that economic activity adjusts more strongly following an adverse shock to a measure of excess bond premiums than after a favourable shock. %According to this analysis, house price growth did not overly stimulate consumption before the Great Recession but that it contributed significantly to the downturn during the crisis.

To explore the potential sign dependency of disturbances to debtors' borrowing capacity in the present framework, I consider a simple experiment: what if the borrowing limit, instead of tightening from $\phi_{ss}$ to $\phi_{ss}'$, unexpectedly loosens from $\phi_{ss}'$ to $\phi_{ss}$?\footnote{I call this a ``simple" experiment because a more complete analysis would also consider an expansion from $\frac{D}{y} = 57\%$ to $\frac{D}{y} = 62\%$. I avoid doing so because this would require further adjustments to the asset grid.} Would the economy experience the same dynamics just with their sign reversed? To answer these questions, I use the baseline model from section \ref{sec:limit} and compute the transition from $\phi_{ss}'$ to $\phi_{ss}$. %, thus inducing a credit easing. 

Figure \ref{fig:comparison-asymmetry-C} plots the response of consumption after such a credit easing and compares it with the dynamics after the reverse transition.\footnote{Note that the figure masks that the impulse responses from the credit easing exhibit some wiggle around $t=8$ and $t=9$, which I could not eliminate even with the procedure creating the asset grid described in appendix \ref{sec-app:codes-ep}.} The results are telling: the boom from a credit easing is much less pronounced than the recession caused by a credit tightening. Specifically, in the former case, consumption increases by only slightly more than $0.5\%$ on impact, significantly less than the three percent drop provoked by tightening financial conditions. The less pronounced boom of course extends to all quantities in the economy, e.g.~labour hours, inflation, interest rates.

\begin{figure}[t]
    \caption{Baseline Model -- Shock to Borrowing Limit: Asymmetry}
    \label{fig:comparison-asymmetry}
    \centering
    \begin{subfigure}[b]{0.49\textwidth}
    \caption{Borrowing Limit, $\phi$}
    \label{fig:comparison-asymmetry-phi}
         \centering
         \includesvg[inkscapelatex=false, width=\textwidth]{Results/compare_transitions/comparison_phi_baseline_limit_permanent_baseline_limit_permanent_asymmetric.svg}
     \end{subfigure}
    \hfill
    \begin{subfigure}[b]{0.49\textwidth}
    \caption{Consumption, $C$}
    \label{fig:comparison-asymmetry-C}
         \centering
         \includesvg[inkscapelatex=false, width=\textwidth]{Results/compare_transitions/comparison_C_baseline_limit_permanent_baseline_limit_permanent_asymmetric.svg}
     \end{subfigure}
     
     \vspace{10pt}
     
     \justifying
     \footnotesize
	\textit{Note}: Percent deviations mean percent deviations from the respective initial steady state. The time horizon in all plots is fixed to twelve quarters, including the initial steady state in $t=0$.
\end{figure}

These findings align well with the evidence cited above but the question remains as to what exactly is behind this sign dependence. One candidate is the \Gls{zlb}. However, as appendix \ref{sec-app:robust-no-zlb} shows, the contraction in consumption in a model ignoring the \Gls{zlb} still significantly exceeds, in absolute magnitude, the expansion after credit easing (plot \ref{fig:comparison-no-ZLB-agg-C}). Thus, the aggregate state space of the model can not explain the largest fraction of the observed sign dependency.

This implies that the results stem mainly from the asymmetry of the borrowing limit itself (as in \cite{guerrieri2017asym}). Recall that subsection \ref{sec:model-hh-constraints} describes this limit as a kink in the household's decision problem as it precludes increasing leverage, while always allowing for debt repayment. Hence, while a tightening forces constrained households to reduce debt and consumption, an easing in the credit limit does not oblige agents to take up more debt or to adjust expenditures.

This flexibility implies that the balance-sheet effect has less bite. Intuitively, households will only slowly tap into more debt when exposed to  unfortunate productivity realisations.\footnote{This is consistent with the fact that the distribution does not feature any agents directly at the limit in the first periods of the transition to easy credit. This is in stark contrast to a tightening, in the case of which the shock pushes agents to the increasingly tight credit limits. See the visualisation produced by the code file \texttt{main.py}.} The precautionary-savings motive is of course still present and should be of the same magnitude as after a credit crunch. Since agents see the debt limit moving further away, they aim for lower precautionary buffers and adjust towards higher consumption targets. 

The subsequent adjustment of the economy is similar to what was discussed before, just in reverse. For example, the expansion in demand leads to unexpected inflation in $t=1$. This mechanically reduces the \textit{ex-post} real interest rate on impact, which redistributes income from lenders to borrowers. Taxes and dividends fall. As the initial consumption response is weaker, all these effects are less pronounced. Thereafter, rates must rise to contain the excess supply of bonds and the economy returns to steady state.

Finally, let me comment on the transition of aggregate debt after a credit easing. As before, debt takes a prolonged time to reach its terminal (now higher) level. The overall duration of this transition is identical to the one after a credit crunch, which confirms that the long-run behaviour of debt is not influenced by the short-run and is more of an inherent feature of the model.

In the short-run, however, a difference is visible: the transition of debt is more smooth from the very beginning as it does not exhibit the same pattern of faster adjustment in the first quarters after the shock. This is consistent with the observations in the preceding sections: aggregate debt in the short-run is also influenced by other conditions. For example, as the real interest rate adjusts less strongly in the credit-easing than in the credit-crunch scenario, debt transitions more gradually in the short term.

%%%%%%%%%%%%%%%%%%%%%%%%%%%%%%%%%%%%%%%%%%%%%%%%%%%%%%%%%%%%%%%%%%%%%%%%%%%%%
%%%%%%%%%%%%%%%%%%%%%%%%%%%%%%%%%%%%%%%%%%%%%%%%%%%%%%%%%%%%%%%%%%%%%%%%%%%%%
%%%%%%%%%%%%%%%%%%%%%%%%%%%%%%%%%%%%%%%%%%%%%%%%%%%%%%%%%%%%%%%%%%%%%%%%%%%%%
\section{Concluding Remarks}
\label{sec:conclusion}

I study credit-crunch scenarios in an incomplete-markets environment with standard \Gls{nk} features in which private credit intermediation arises endogeneously. In this framework, households borrow from and lend to each other as debt sustains consumption for debtors, while providing insurance to creditors. A credit crunch curtails the amount that debtors can borrow and thus disturbs the goods and asset markets in the economy. 

Three key results emerge. First, the small-scale \Gls{hank} model accounts for the mechanisms identified in the theoretical macroeconomic literature on credit tightenings, in particular \textcite{egg2012} and \textcite{gl2017}. A shock that tightens the household-level borrowing limit forces constrained debtors to mechanically delever, giving rise to a balance-sheet effect. At the same time, a closer limit invigorates the unconstrained agents' demand for safe assets, the precautionary-savings effect. Both mechanisms reduce consumption and lead to excess demand for liquidity. In the present model, this leads to a short-term recession and a real rate which undershoots its terminal, lower level to ensure equilibrium. As a fairly general result, outstanding debt takes over a prolonged period of time to transition to lower levels.

Second, financial tightenings through the debt limit and the credit spread give rise to qualitatively similar mechanisms. However, the latter's effects depend much more on the endogenous adjustment of the real interest rate, implying that the short-run downturn is attenuated quantitatively. Third, consistent with the notion that finance allows for flexibility when expanded but not when curtailed, shocks to the borrowing limit operate asymmetrically. That is, the recession provoked by a tightening in this limit is much larger in absolute terms than the boom caused by the reverse transition.

As the model predicts potentially large and unequally distributed consumption losses after financial tightenings, it is natural to ask whether these findings carry policy implications. Yet, the model in its present form appears too simple to speak to such considerations. For example, the model does not address the important issues as to why constraints to household finance exist and as to why they change. Importantly, it does not feature any criterion as to whether a certain debt level is associated with less risk or even ``optimal". In any case, as \textcite{korinek2016} remark, the reflex to curtail \textit{ex-ante} leverage has to be weighed against the positive effects of private credit intermediation. 

This provides the basis for a possible extension of the present analysis. Accounting for a richer framework of household finance -- e.g.~by explicitly including default risk and asset positions in gross terms and with varying maturities -- would be a fruitful extension and could provide sharper results as to how policy can intervene most effectively when confronted with a credit crunch. As mentioned above, \textcite{gl2017} study an extension of their framework with a durable good which serves as collateral when borrowing. \textcite{kaplan2018} develop a two-asset \Gls{hank} model with a rich representation of household balance sheets in terms of their liquidity composition. \textcite{guerrieri2020} consider mortgage debt with different maturities in a partial-equilibrium setting.

Furthermore, an episode of tight credit is unlikely to impact households alone. Firms, banks and the government will most certainly be affected as well. Thus, studying these agents' qualitative and quantitative reactions is important. \textcite{buera2020} consider heterogeneous entrepreneurs and the response of investment and factor productivity to a credit crunch similar to the one studied here. \textcite{lee2021} put forward a \Gls{hank} economy with a frictional banking sector. Clearly however, more work is needed to understand how sectors amplify or dampen each other's responses so as to gain an understanding of how the entire economy reacts to financial shocks of various sources.

Finally, an empirical analysis of the mechanisms discussed should be a top priority. Ideally, one would proceed similarly to \textcite{bayer2019} on shocks to income risk. That is, one would identify shocks to household-level borrowing limits from micro data to estimate how such disturbances affect macroeconomic and cross-sectional outcomes. Thereafter, one would match these responses in an estimated \Gls{hank} model. Doing so would reveal whether and if so, to what extent, the issues of the present thesis have an impact on people's lives and well-being in the data.

%%%%%%%%%%%%%%%%%%%%%%%%%%%%%%%%%%%%%%%%%%%%%%%%%%%%%%%%%%%%%%%%%%%%%%%%%%%%%
%%%%%%%%%%%%%%%%%%%%%%%%%%%%%%%%%%%%%%%%%%%%%%%%%%%%%%%%%%%%%%%%%%%%%%%%%%%%%
%%%%%%%%%%%%%%%%%%%%%%%%%%%%%%%%%%%%%%%%%%%%%%%%%%%%%%%%%%%%%%%%%%%%%%%%%%%%%
% References section
\newpage
\thispagestyle{plain}
\pagenumbering{Roman}
\printbibliography[heading=bibintoc] % Insert references

%%%%%%%%%%%%%%%%%%%%%%%%%%%%%%%%%%%%%%%%%%%%%%%%%%%%%%%%%%%%%%%%%%%%%%%%%%%%%
%%%%%%%%%%%%%%%%%%%%%%%%%%%%%%%%%%%%%%%%%%%%%%%%%%%%%%%%%%%%%%%%%%%%%%%%%%%%%
%%%%%%%%%%%%%%%%%%%%%%%%%%%%%%%%%%%%%%%%%%%%%%%%%%%%%%%%%%%%%%%%%%%%%%%%%%%%%
% Appendices section
\newpage
\begin{refsection}
\thispagestyle{plain}
\pagenumbering{arabic}  % Turn page numbering to arabic
\renewcommand*{\thepage}{A-\arabic{page}} % Add 'A' to each page number for appendices section
\addtocontents{toc}{\protect\setcounter{tocdepth}{1}} % This hides the appendix subsections in the table of contents
\begin{appendices}
%%%%%%%%%%%%%%%%%%%%%%%%%%%%%%%%%%%%%%%%%%%%%%%%%%%%%%%%%%%%%%%%%%%%%%%%%%%%%
%%%%%%%%%%%%%%%%%%%%%%%%%%%%%%%%%%%%%%%%%%%%%%%%%%%%%%%%%%%%%%%%%%%%%%%%%%%%%
\section{Full Models}
\label{sec-app:full}

This appendix present the full models employed throughout the main text and the subsequent appendices. Appendix \ref{sec-app:full-derivations} derives the equations characterising optimal behaviour of households and firms. Thereafter, appendix \ref{sec-app:full-equations} displays the full system of equations passed to the \texttt{Econpizza} package (see appendix \ref{sec-app:codes} for details). Finally, appendix \ref{sec-app:full-graphical} presents a graphical representation of the baseline model.

%%%%%%%%%%%%%%%%%%%%%%%%%%%%%%%%%%%%%%%%%%%%%%%%%%%%%%%%%%%%%%%%%%%%%%%%%%%%%
%%%%%%%%%%%%%%%%%%%%%%%%%%%%%%%%%%%%%%%%%%%%%%%%%%%%%%%%%%%%%%%%%%%%%%%%%%%%%
\subsection{Derivations for Households and Firms}
\label{sec-app:full-derivations}

This subsection derives the equations that characterise optimal behaviour by households and firms.

%%%%%%%%%%%%%%%%%%%%%%%%%%%%%%%%%%%%%%%%%%%%%%%%%%%%%%%%%%%%%%%%%%%%%%%%%%%%%
\subsubsection{Households}
\label{sec-app:full-derivations-households}

The derivations for the household sector depend on the preference specification. I begin with the baseline case with \Gls{ghh} preferences as detailed in section \ref{sec:model-hh-solution}. Thereafter, I present the changes necessary for the environment in which agents have \Gls{crra} preferences.

%%%%%%%%%%%%%%%%%%%%%%%%%%%%%%%%%%%%%%%%%%%%%%%%%%%%%%%%%%%%%%%%%%%%%%%%%%%%%
\subsubsection*{The Case of \Gls{ghh} Preferences (Baseline Model; Section \ref{sec:model})}

The households' environment in the baseline case is given by equations \eqref{eq:hh-income-process} -- \eqref{eq:hh-composite-good}. This setup can be cast into the following Bellman equation:
\begin{equation}
    V_t (e_{i,t}, b_{i,t-1}) = \max_{c_{i,t}, n_{i,t}, b_{i,t}} \Bigg\{ \frac{1}{1-\sigma_c} \Bigg( c_{i,t} - e_{i,t}\frac{n_{i,t}^{1+\sigma_l}}{1+\sigma_l} \Bigg)^{1-\sigma_c} + \beta \mathbb{E}_{e,t} \{ V_{t+1} (e_{i,t+1}, b_{i,t}) \} \Bigg\} \label{eq:app-hh-bellman}
\end{equation}
where households maximise subject to the budget constrain \eqref{eq:hh-budget} and the borrowing limit \eqref{eq:borrowing-limit}. Using \eqref{eq:hh-budget}, this yields:
\begin{align}
    \max_{c_{i,t}, n_{i,t}, b_{i,t} \ge \phi_t} \Bigg\{ \frac{1}{1-\sigma_c} \Bigg( &RR_t b_{i,t-1} + w_t e_{i,t} n_{i,t} + \Pi_t \bar{\Pi} (e_{i,t}) - \tau_t \bar{\tau} (e_{i,t}) - b_{i,t} \nonumber \\ 
    &- e_{i,t}\frac{n_{i,t}^{1+\sigma_l}}{1+\sigma_l} \Bigg)^{1-\sigma_c} \nonumber \\ 
    &+ \beta \mathbb{E}_{e,t} \{ V_{t+1} (e_{i,t+1}, b_{i,t}) \} \Bigg\}, \label{eq:app-hh-max}
\end{align}
where I abbreviate $RR (RR_t, b_{i,t-1})$ by $RR_t$ to ease notation. Differentiating \eqref{eq:app-hh-max} with respect to $b_{i,t}$ yields:
\begin{align}
    - x_{i,t}^{-\sigma_c} + \beta \mathbb{E}_{e,t} \{ V_{b,t+1} (e_{i,t+1} b_{i,t} ) \} + \lambda_t &\overset{!}{=} 0 \nonumber \\
    \Leftrightarrow \ x_{i,t}^{-\sigma_c} &\ge \beta \mathbb{E}_{e,t} \{ V_{b,t+1} (e_{i,t+1} b_{i,t} ) \}, \label{eq:app-hh-foc-euler}
\end{align}
where $x_{i,t}$ is given by \eqref{eq:hh-composite-good} and $\lambda_t \ge 0$ is the Lagrange multiplier on the borrowing constraint. Thus, the inequality in \eqref{eq:app-hh-foc-euler} holds with equality whenever $b_{i,t} > \phi_t$, i.e.~when the borrowing constraint is slack and $\lambda_t = 0$. The corresponding Envelope condition is:
\begin{equation}
    V_{b,t} = RR (RR_t, b_{i,t-1}) x_{i,t}^{-\sigma_c}. \label{eq:app-hh-foc-envelope}
\end{equation}

Equations \eqref{eq:app-hh-foc-euler} and \eqref{eq:app-hh-foc-envelope} are equations \eqref{eq:foc-euler} and \eqref{eq:foc-envelope}. To obtain \eqref{eq:foc-labour-supply}, differentiate \eqref{eq:app-hh-max} with respect to $n_{i,t}$ to obtain:
\begin{align}
    x_{i,t}^{-\sigma_c} \big( w_t e_{i,t} - e_{i,t} n_{i,t}^{\sigma_l} \big) &\overset{!}{=} 0 \nonumber \\
    \Leftrightarrow \ w_t e_{i,t} - e_{i,t} n_{i,t}^{\sigma_l} &= 0 \nonumber \\
    \Leftrightarrow \  w_t &= n_{i,t}^{\sigma_l}, \label{eq:app-hh-foc-labour}
\end{align}
which holds for all $i$. This derivation shows how \Gls{ghh} preferences dramatically simplify the households' decision problem. As these preferences are such that agent $i$'s labour supply, $n_{i,t}$, only depends on $w_t$, all households work the same hours, despite differing in consumption levels and balance sheet positions. Intuitively, this is because, with \Gls{ghh} preferences, the marginal utility of consumption and the marginal disutility of working move proportionally, thus leaving $n_{i,t}$ unchanged when $c_{i,t}$ changes, see \textcite{auclert2017}.

%%%%%%%%%%%%%%%%%%%%%%%%%%%%%%%%%%%%%%%%%%%%%%%%%%%%%%%%%%%%%%%%%%%%%%%%%%%%%
\subsubsection*{The Case of \Gls{crra} Preferences (Endogenous Labour Model; Section \ref{sec:sensitivity-end-labour})}

Now, let households have \Gls{crra} preferences as given in \eqref{eq:hh-crra-utility}. This implies the following Bellman equation:
\begin{equation}
    V_t (e_{i,t}, b_{i,t-1}) = \max_{c_{i,t}, n_{i,t}, b_{i,t}} \Bigg\{ \frac{c_{i,t}^{1-\sigma_c}}{1-\sigma_c} - \phi \frac{n_{i,t}^{1+\sigma_l}}{1+\sigma_l} + \beta \mathbb{E}_{e,t} \{ V_{t+1} (e_{i,t+1}, b_{i,t}) \} \Bigg\}, \label{eq:app-hh-bellman-crra}
\end{equation}
again subject to \eqref{eq:hh-budget} and \eqref{eq:borrowing-limit}. Similar to above, I substitute the borrowing constraint for $c_{i,t}$ and differentiate with respect to $b_{i,t}$, which yields:
\begin{align}
    -c_{i,t}^{-\sigma_c} + \beta \mathbb{E}_{e,t} \{ V_{b,t+1} (e_{i,t+1} b_{i,t} ) \} + \lambda_t &\overset{!}{=} 0 \nonumber \\
    \Leftrightarrow \ c_{i,t}^{-\sigma_c} \ge \beta \mathbb{E}_{e,t} \{ V_{b,t+1} (e_{i,t+1} b_{i,t} ) \}, \label{eq:app-hh-foc-euler-crra}
\end{align}
which again holds with equality whenever $b_{i,t} > \phi_t$, i.e.~$\lambda_t = 0$. The envelope condition now is:
\begin{equation}
    V_{b,t} = RR (RR_t, b_{i,t-1}) c_{i,t}^{-\sigma_c}. \label{eq:app-hh-foc-envelope-crra}
\end{equation}

To obtain the intra-temporal condition on labour supply, differentiate \eqref{eq:app-hh-bellman-crra} with respect to $n_{i,t}$:
\begin{align}
    c_{i,t}^{-\sigma_c} w_t e_{i,t} - \phi n_{i,t}^{\sigma_l} \overset{!}{=} 0 \ \Leftrightarrow \ n_{i,t} = \Bigg( \frac{c_{i,t}^{-\sigma_c} w_t e_{i,t}}{\phi} \Bigg)^{\frac{1}{\sigma_l}}. \label{eq:app-hh-foc-labour-crra}
\end{align}

Equations \eqref{eq:app-hh-foc-euler-crra}, \eqref{eq:app-hh-foc-envelope-crra} and \eqref{eq:app-hh-foc-labour-crra} are the counterparts of equations \eqref{eq:app-hh-foc-euler}, \eqref{eq:app-hh-foc-envelope} and \eqref{eq:app-hh-foc-labour} and thus constitute the formal solution to the households setup in the case that agents have \Gls{crra} preferences. Aggregate labour supply is found by: $N_t = \int n_{i,t} d \Psi_t (b,e)$.

Note that, as the labour supply decision now occurs on the individual level, it is part of the \Gls{egm} step in the solution algorithm of the household block.

%%%%%%%%%%%%%%%%%%%%%%%%%%%%%%%%%%%%%%%%%%%%%%%%%%%%%%%%%%%%%%%%%%%%%%%%%%%%%
\subsubsection{Firms}
\label{sec-app:full-derivations-firms}

This section describes the corporate sector of the model in more detail and derives its optimal behaviour, in particular the \Gls{nkpc} \eqref{eq:firms-nkpc}. The corporate sector consists of a final-good producer and a continuum of intermediate-goods producers, indexed by $j$.

%%%%%%%%%%%%%%%%%%%%%%%%%%%%%%%%%%%%%%%%%%%%%%%%%%%%%%%%%%%%%%%%%%%%%%%%%%%%%
\subsubsection*{Final-Good Firm}

Consider first the representative, perfectly competitive firm which produces the final good, $y_t^f$, by combining intermediate goods, denoted by $Y_{j,t}$. This firm uses the following technology with constant returns-to-scale:
\begin{equation}
    y_t^f = \Bigg( \int_0^1 Y_{j,t}^{\frac{\theta - 1}{\theta}} dj \Bigg)^{\frac{\theta}{\theta-1}}, \label{eq:app-firms-technology-final}
\end{equation}
where $\theta$ is the elasticity of substitution for the intermediate goods in the production process. The objective of the final-good firm is to maximise its profits, $\Pi_t^f$, i.e.:
\begin{equation*}
    \max_{Y_{j,t}} \ \Pi_t^f = P_t y_t^f - \int_0^1 P_{j,t} Y_{j,t} dj,
\end{equation*}
subject to its technology \eqref{eq:app-firms-technology-final}. $P_t$ is the price of the final good and $P_{j,t}$ is the price of intermediate good $Y_{j,t}$. The firm maximises its profits by choosing the inputs optimally. Thus, the result of this optimisation will be the demand of the final-good firm for intermediate goods.  

Inserting \eqref{eq:app-firms-technology-final} into the objective and differentiating with respect to $Y_{j,t}$ results in the \Gls{foc} which yields the input demand. Formally:
\begin{align}
    \max_{Y_{j,t}} \ \Pi_t^f &= P_t y_t^f - \int_0^1 P_{j,t} Y_{j,t} dj \nonumber \\
    \max_{Y_{j,t}} \ \Pi_t^f &= P_t \Bigg( \int_0^1 Y_{j,t}^{\frac{\theta - 1}{\theta}} dj \Bigg)^{\frac{\theta}{\theta-1}} - \int_0^1 P_{j,t} Y_{j,t} dj \nonumber \\
    \Rightarrow \ \frac{\partial \Pi_t^f}{\partial Y_{j,t}} &= \frac{\theta}{\theta - 1} P_t \Bigg( \int_0^1 Y_{j,t}^{\frac{\theta - 1}{\theta}} dj \Bigg)^{\frac{\theta}{\theta-1}-1} \frac{\theta-1}{\theta} Y_{j,t}^{-\frac{1}{\theta}} - P_{j,t} \overset{!}{=} 0 \nonumber \\
    \Leftrightarrow \ \frac{P_{j,t}}{P_t} &= \Bigg[ \Bigg( \int_0^1 Y_{j,t}^{\frac{\theta - 1}{\theta}} dj \Bigg)^{\frac{\theta}{\theta-1}} \Bigg]^{\frac{1}{\theta}} Y_{j,t}^{-\frac{1}{\theta}} \nonumber  \\
    \Leftrightarrow \ \frac{P_{j,t}}{P_t} &= (y_t^f)^{\frac{1}{\theta}} Y_{j,t}^{-\frac{1}{\theta}} \nonumber \\
    \Leftrightarrow \ \Bigg( \frac{P_{j,t}}{P_t} \Bigg)^{-\theta} &= (y_t^f)^{-1} Y_{j,t} \nonumber \\
    \Leftrightarrow \ Y_{j,t} &= \Bigg( \frac{P_{j,t}}{P_t} \Bigg)^{-\theta} y_t^f, \label{eq:app-firms-input-demand}
\end{align}
which is, as desired, demand for inputs by the final-good firm. As this firm is perfectly competitive, we impose a zero-profits restriction, implying:
\begin{align}
    \Pi_t^f \overset{!}{=} 0 \ &\Rightarrow \ P_t y_t^f - \int_0^1 P_{j,t} Y_{j,t} dj = 0 \nonumber \\
    &\Leftrightarrow \ P_t y_t^f - \int_0^1 P_{j,t} \Bigg( \frac{P_{j,t}}{P_t} \Bigg)^{-\theta} y_t^f dj = 0 \nonumber \\
    &\Leftrightarrow \ y_t^f \Bigg( P_t - \int_0^1 P_{j,t} \Bigg( \frac{P_{j,t}}{P_t} \Bigg)^{-\theta} dj \Bigg) = 0 \nonumber \\
    &\Leftrightarrow \ P_t = \int_0^1 P_{j,t} \Bigg( \frac{P_{j,t}}{P_t} \Bigg)^{-\theta} dj \nonumber \\
    &\Leftrightarrow \ P_t = \frac{1}{P_t^{-\theta}} \int_0^1 P_{j,t}^{1-\theta} dj \nonumber \\ 
    &\Leftrightarrow \ P_t = \Bigg( \int_0^1 P_{j,t}^{1-\theta} dj \Bigg)^{\frac{1}{1-\theta}}, \nonumber
\end{align}
where the last equation is the solution for the price of the final good.

%%%%%%%%%%%%%%%%%%%%%%%%%%%%%%%%%%%%%%%%%%%%%%%%%%%%%%%%%%%%%%%%%%%%%%%%%%%%%
\subsubsection*{Intermediate-Goods Firms}

Consider next the intermediate-goods firms, of which there is a continuum. They sell their output, $Y_{j,t}$, produced with labour input $N_{j,t}$ and with the production function:
\begin{equation}
    Y_{j,t} = z N_{j,t}, \label{eq:app-firms-input-production}
\end{equation}
to the representative final-good firm in a monopolistically competitive market by setting the nominal price $P_{j,t}$. In adjusting their prices in period $t$, the intermediate-goods firms are subject to quadratic price adjustment costs à la \textcite{rotemberg1982} in the form of:
\begin{equation}
    \frac{\psi}{2} \Bigg( \frac{P_{j,t}}{P_{j,t-1} \tilde{\pi}_{t-1}} - 1 \Bigg)^2 y_t^f, \label{eq:app-firms-price-adj}
\end{equation}
as in \textcite{gust2017wp}.

The objective of intermediate-goods firm $j$ is to maximise the discounted stream of its profits, i.e.~revenues minus labour input costs and price adjustment costs \eqref{eq:app-firms-price-adj}, by setting the price $P_{j,t}$ in $t$.\footnote{How to discount firm profits is not entirely straightforward in \Gls{ha} economies. I let firms discount profits at rate $\beta$. In most studies the results remain robust under alternative specifications.} Formally, they maximise:
\begin{equation}
    \max_{P_{j,t}} \ \Pi_t = \sum_{t=0}^{\infty} \Bigg( \beta^t \Bigg[ \frac{P_{j,t} Y_{j,t} - W_t N_{j,t}}{P_t} - \frac{\psi}{2} \Bigg( \frac{P_{j,t}}{P_{j,t-1} \tilde{\pi}_{t-1}} - 1 \Bigg)^2 y_t^f \Bigg] \Bigg), \label{eq:app-firms-profits-input}
\end{equation}
where $W_t$ is the nominal wage, defined as $W_t \equiv w_t P_t$, subject to the demand for their inputs from the final-good firm \eqref{eq:app-firms-input-demand}. Using \eqref{eq:app-firms-input-demand}, $mc_t = \frac{w_t}{z}$ and \eqref{eq:app-firms-input-production}, expression \eqref{eq:app-firms-profits-input} can be simplified a bit:
\begin{align}
    &\max_{P_{j,t}} \sum_{t=0}^{\infty} \Bigg( \beta^t \Bigg[ \frac{P_{j,t} Y_{j,t} - W_t N_{j,t}}{P_t} - \frac{\psi}{2} \Bigg( \frac{P_{j,t}}{P_{j,t-1} \tilde{\pi}_{t-1}} - 1 \Bigg)^2 y_t^f \Bigg] \Bigg) \nonumber \\
    \Leftrightarrow \ &\max_{P_{j,t}} \sum_{t=0}^{\infty} \Bigg( \beta^t \Bigg[ \frac{P_{j,t} Y_{j,t} - W_t \frac{Y_{j,t}}{z}}{P_t} - \frac{\psi}{2} \Bigg( \frac{P_{j,t}}{P_{j,t-1} \tilde{\pi}_{t-1}} - 1 \Bigg)^2 y_t^f \Bigg] \Bigg) \nonumber \\
    \Leftrightarrow \ &\max_{P_{j,t}} \sum_{t=0}^{\infty} \Bigg( \beta^t \Bigg[ \Bigg( \frac{P_{j,t}}{P_t} - mc_t \Bigg) Y_{j,t} - \frac{\psi}{2} \Bigg( \frac{P_{j,t}}{P_{j,t-1} \tilde{\pi}_{t-1}} - 1 \Bigg)^2 y_t^f \Bigg] \Bigg) \nonumber \\
    \Leftrightarrow \ &\max_{P_{j,t}} \sum_{t=0}^{\infty} \Bigg( \beta^t \Bigg[ \Bigg( \frac{P_{j,t}}{P_t} - mc_t \Bigg) \Bigg( \frac{P_{j,t}}{P_t} \Bigg)^{-\theta} y_t^f - \frac{\psi}{2} \Bigg( \frac{P_{j,t}}{P_{j,t-1} \tilde{\pi}_{t-1}} - 1 \Bigg)^2 y_t^f \Bigg] \Bigg) \nonumber \\
    \Leftrightarrow \ &\max_{P_{j,t}} \sum_{t=0}^{\infty} \Bigg( \beta^t y_t^f \Bigg[ \Bigg( \frac{P_{j,t}}{P_t} - mc_t \Bigg) \Bigg( \frac{P_{j,t}}{P_t} \Bigg)^{-\theta} - \frac{\psi}{2} \Bigg( \frac{P_{j,t}}{P_{j,t-1} \tilde{\pi}_{t-1}} - 1 \Bigg)^2 \Bigg] \Bigg) \label{eq:app-firms-input-max}
\end{align}

Solving \eqref{eq:app-firms-input-max} yields the \Gls{nkpc}, but requires some algebra. The first step is to differentiate with respect to $P_{j,t}$:
\begin{alignat}{3}
    \frac{\partial \Pi_t}{\partial P_{j,t}} &= \beta^t y_t^f \Bigg( &&(1-\theta) \Bigg( \frac{P_{j,t}}{P_t} \Bigg)^{-\theta} \frac{1}{P_t} + \theta \frac{mc_t}{P_t} \Bigg( \frac{P_{j,t}}{P_t} \Bigg)^{-\theta-1} \nonumber \\
    & \ &&- \psi \Bigg( \frac{P_{j,t}}{P_{j,t-1} \tilde{\pi}_{t-1}} - 1 \Bigg) \frac{1}{P_{j,t-1} \tilde{\pi}_{t-1}} \Bigg) \nonumber \\
    & \ &&+\beta^{t+1} y_{t+1}^f \Bigg( -\psi \Bigg( \frac{P_{j,t+1}}{P_{j,t} \tilde{\pi}_{t}} - 1 \Bigg) (-1) \frac{P_{j,t+1}}{P_{j,t}^2 \tilde{\pi}_{t}} \Bigg) \overset{!}{=} 0, \nonumber
\end{alignat}
which becomes:
\begin{align}
    \Leftrightarrow \ 0 = \frac{1-\theta}{P_t} \Bigg( \frac{P_{j,t}}{P_t} \Bigg)^{-\theta} &+ \frac{\theta mc_t}{P_t} \Bigg( \frac{P_{j,t}}{P_t} \Bigg)^{-\theta-1} - \frac{\psi}{P_{j,t-1} \tilde{\pi}_{t-1}} \Bigg( \frac{P_{j,t}}{P_{j,t-1} \tilde{\pi}_{t-1}} - 1 \Bigg) \nonumber \\
    &+\psi \beta \frac{P_{j,t+1}}{P_{j,t}^2 \tilde{\pi}_{t}} \Bigg( \frac{P_{j,t+1}}{P_{j,t} \tilde{\pi}_t} - 1 \Bigg) \frac{y_{t+1}^f}{y_t^f} \nonumber \\
    \Leftrightarrow \ \psi \Bigg( \frac{\pi_t}{\tilde{\pi}_{t-1}} - 1 \Bigg) \frac{\pi_t}{\tilde{\pi}_{t-1}} &= (1-\theta) + \theta mc_t + \psi \beta \Bigg( \frac{\pi_{t+1}}{\tilde{\pi}_{t}} - 1 \Bigg) \frac{\pi_{t+1}}{\tilde{\pi}_{t}} \frac{y_{t+1}^f}{y_t^f}, \label{eq:app-firms-nkpc}
\end{align}
which is finally the desired result and where I used that, in the symmetric equilibrium, $P_{j,t} = P_t$ $\forall j$ holds. This is because all intermediate-goods firms face the same problem. \eqref{eq:app-firms-nkpc} is equation \eqref{eq:firms-nkpc} in the main text. 

As all intermediate-goods firms behave identically, aggregate production can be described by $y_t^f = zN_t$ and aggregate profits by \eqref{eq:firms-profits}. Thus, equations \eqref{eq:firms-production}, \eqref{eq:firms-nkpc} and \eqref{eq:firms-profits} constitute the corporate block. 

%%%%%%%%%%%%%%%%%%%%%%%%%%%%%%%%%%%%%%%%%%%%%%%%%%%%%%%%%%%%%%%%%%%%%%%%%%%%%
%%%%%%%%%%%%%%%%%%%%%%%%%%%%%%%%%%%%%%%%%%%%%%%%%%%%%%%%%%%%%%%%%%%%%%%%%%%%%
\subsection{Full System of Equations}
\label{sec-app:full-equations}

This subsection contains the full system of equations passed to the \texttt{Econpizza} package. The equations for the household block are either:
\begin{align}
    x_{i,t}^{-\sigma_c} &\ge \beta \mathbb{E}_{e,t} \{ V_{b,t+1} (e_{i,t+1} b_{i,t} ) \} \\
    V_{b,t} &= RR (RR_t, b_{i,t-1}) x_{i,t}^{-\sigma_c} \\
    w_t &= N_t^{\sigma_l}
\end{align}
in the case of \Gls{ghh} preferences or:
\begin{align}
    c_{i,t}^{-\sigma_c} &\ge \beta \mathbb{E}_{e,t} \{ V_{b,t+1} (e_{i,t+1} b_{i,t} ) \} \\
    V_{b,t} &= RR (RR_t, b_{i,t-1}) c_{i,t}^{-\sigma_c} \\
    n_{i,t} &= \Bigg( \frac{c_{i,t}^{-\sigma_c} w_t e_{i,t}}{\phi} \Bigg)^{\frac{1}{\sigma_l}}
\end{align}
in the case of \Gls{crra} preferences.

The rest of the model reads:\footnote{I omit some equations which have the main purpose of defining certain quantities or percentiles.}
\begin{align}
    C_t &= \int c_{i,t} d \Psi_t (b,e) \\
    N_t &= \int n_{i,t} d \Psi_t (b,e) \\ 
    B &= \int b_{i,t} d \Psi_t (b,e) \\
    C_t &= y_t \\
    D_t &= - \int b_{i,t} \mathbb{1}_{[\phi_t \le b_{i,t} \le 0]} d \Psi_t (b,e) \\
    y_t^f &= zN_t \\
    mc_t &= \frac{w_t}{z} \\
    \psi \Bigg( \frac{\pi_t}{\tilde{\pi}_{t-1}} - 1 \Bigg) \frac{\pi_t}{\tilde{\pi}_{t-1}} &= (1-\theta) + \theta mc_t + \psi \beta \Bigg( \frac{\pi_{t+1}}{\tilde{\pi}_{t}} - 1 \Bigg) \frac{\pi_{t+1}}{\tilde{\pi}_{t}} \frac{y_{t+1}^f}{y_t^f} \\
    \tilde{\pi}_t &= \pi_{ss}^{\omega} \pi_{t}^{1-\omega} \\
    \Pi_t &= \Bigg[ 1 - \frac{\psi}{2} \Bigg( \frac{\pi_t}{\tilde{\pi}_{t-1}} - 1 \Bigg)^2 \Bigg] y_t^f - w_t N_t \\
    y_t &= \Bigg( 1 - \frac{\psi}{2} \Bigg( \frac{\pi_t}{\tilde{\pi}_{t-1}} - 1 \Bigg)^2 \Bigg) y_t^f - \kappa_t D_t \\
    \tau_t &= (RR_t - 1) B \\
    R_{t}^n &= ( R_{t-1}^n )^{\rho} \Bigg[ R_{ss} \Bigg( \frac{\pi_t}{\pi_{ss}} \Bigg)^{\phi_{\pi}} \Bigg( \frac{y_t}{y_{ss}} \Bigg)^{\phi_y} \Bigg]^{(1-\rho)} \\
    R_t &= \text{max} \ \{ 1, \ R_{t}^n \} \\
    RR (RR_t, b_{i,t-1}) &= \begin{cases}
	\frac{R_{t-1}}{\pi_t} \ \ \ \ \ \ \ \ \ \ \ \ \ \text{if} \ b_{i,t-1} \ge 0 \\
	\frac{R_{t-1}}{\pi_t} + \kappa_t \ \ \ \ \text{if} \ b_{i,t-1} < 0,
	\end{cases} \\
    \phi_t &= \phi_{ss} \Bigg( \frac{\phi_{t-1}}{\phi_{ss}} \Bigg)^{\rho_{\phi}} exp(\varepsilon_{\phi}) \\
    \kappa_t &= \kappa_{ss} \Bigg( \frac{\kappa_{t-1}}{\kappa_{ss}} \Bigg)^{\rho_{\kappa}} exp(\varepsilon_{\kappa}).
\end{align}

%%%%%%%%%%%%%%%%%%%%%%%%%%%%%%%%%%%%%%%%%%%%%%%%%%%%%%%%%%%%%%%%%%%%%%%%%%%%%
%%%%%%%%%%%%%%%%%%%%%%%%%%%%%%%%%%%%%%%%%%%%%%%%%%%%%%%%%%%%%%%%%%%%%%%%%%%%%
\subsection{Graphical Summary of the Model}
\label{sec-app:full-graphical}

This subsection presents a visualisation of the baseline model. The figure is meant to be an aid in understanding the flows and mechanisms of the model and clearly does not capture essential aspects of the model, such as its dynamic nature.

\begin{figure}[H]
\centering
\caption{Graphical Representation of the Baseline Model}
\label{fig:graphical-model}
\begin{tikzpicture}
  % Big rectangle
  \draw[fill=white] (0,) rectangle (15,8);
  \node at (7.5,7.5) {\Large Households};

  \node at (3,6) {``unlucky"};
  \node at (12,6) {``lucky"};

  \node at (2.25,4) {Borrowers};
  \node at (2.25,3.6) {\scriptsize negative \Gls{iou}s};
  \node at (8,4) {Lenders};
  \node at (8,3.6) {\scriptsize positive \Gls{iou}s};
  \node at (12,4) {Bond Holders};

  \node at (13.85,-2) {Bonds $B$};
  \node at (14.1,-3) {Interest $RR$};
  
  \node at (7.5,9) {Productivity Risk $\varepsilon^e$};

  \node at (2,9) {Financial Constraints $\phi$, $\kappa$};
  
  \node at (2.5,-4.25) {Firms};

  \node at (4.3,-4.5) {$\pi$};
  \node at (6.1,-4.5) {$R^n$};
  
  \node at (8.25,-4.25) {Monetary Policy};

  \node at (13,-4.25) {Fiscal Policy};
  
  % Arrows
  \draw[->] (7.5,8.75) -- (7.5,8.25); % to households

  \draw[->] (7.5,7.25) -- (3,6.25); % to unlucky
  \draw[->] (7.5,7.25) -- (12,6.25); % to lucky

  \draw[->] (3,5.75) -- (3,4.25); % to borrowers
  \draw[->] (12,5.75) -- (8,4.25); % to lenders
  \draw[->] (12,5.75) -- (12,4.25); % to bond holders
  
  \draw[->, red] (0.5,8.75) -- (2.25,4.25); % to borrowers

  \draw[->] (7,4.1) -- (3.75,4.1);
  \draw[->] (3.75,3.9) -- (7,3.9); 

  \draw[->] (13,-4) -- (13,3.75); 

  \draw[-] (3.2,-4.25) -- (6.5,-4.25); % btw MP an firms
  \draw[->] (5.5,-4.25) -- (5.5,2.5); % to interest rate btw borr and lend
  
  \draw[->] (5.5,-3) -- (13,-3); 

  \draw[->] (2.65,1) -- (2.65,-4); % to firms
  \draw[->] (2.4,-4) -- (2.4,1); % from firms
  \node at (1.4,-1) {Income $wN$};
  \node at (3.6,-1) {Labour $N$};

  \node at (1.2,-2) {Consumption};
  \node at (1.4,-2.5) {Goods $C$};
  
  \node at (1.4,0) {Profits $\Pi$};

  \draw[->] (12.2,1) -- (12.2,-4); % to FP
  \node at (11.4,-2) {Taxes $\tau$};

  \node at (5.5,4.5) {\footnotesize ``Consumption"};
  \node at (5.5,3.5) {\footnotesize ``Insurance"};
  \node at (5.5,3) {\footnotesize + Interest $RR$};

\end{tikzpicture}
\begin{flushleft}
\footnotesize
\textit{Note}: This figure displays the agents and some key mechanisms of the baseline model from section \ref{sec:model}. To ease notation, I suppress time subscripts.
\end{flushleft}
\end{figure}

%%%%%%%%%%%%%%%%%%%%%%%%%%%%%%%%%%%%%%%%%%%%%%%%%%%%%%%%%%%%%%%%%%%%%%%%%%%%%
%%%%%%%%%%%%%%%%%%%%%%%%%%%%%%%%%%%%%%%%%%%%%%%%%%%%%%%%%%%%%%%%%%%%%%%%%%%%%
% \subsection{The Natural Borrowing Limit}
% \label{sec:app-hh-borrowing}
% The natural borrowing limit is the borrowing limit which just marginally ensures that borrowers can repay and not have negative consumption. This natural borrowing limit can be calculated from forward iterating the household budget constraint \eqref{eq:hh-budget}:
% \begin{equation*}
%     \frac{wne_{L} + \Pi_t \bar{\Pi} (e_{L}) - \tau_t \bar{\tau} (e_{L})}{RR - 1},
% \end{equation*}
% where $e_L$ denotes the lowest level of skills attainable from \eqref{eq:hh-income-process}.

%%%%%%%%%%%%%%%%%%%%%%%%%%%%%%%%%%%%%%%%%%%%%%%%%%%%%%%%%%%%%%%%%%%%%%%%%%%%%
%%%%%%%%%%%%%%%%%%%%%%%%%%%%%%%%%%%%%%%%%%%%%%%%%%%%%%%%%%%%%%%%%%%%%%%%%%%%%
%%%%%%%%%%%%%%%%%%%%%%%%%%%%%%%%%%%%%%%%%%%%%%%%%%%%%%%%%%%%%%%%%%%%%%%%%%%%%
\section{Computational Implementation}
\label{sec-app:codes}

This appendix has two parts. The first one notes some caveats when implementing the simulations of the main text and the present appendices in \texttt{Econpizza}. The second part provides a small guide to the computer codes which produce the results (figures, tables, etc.) for this thesis. The full set of codes is provided together with the thesis and is available as a GitHub repository: \url{https://github.com/andkound98/master-thesis}, last retrieved \printdate{2023-08-25}.

%%%%%%%%%%%%%%%%%%%%%%%%%%%%%%%%%%%%%%%%%%%%%%%%%%%%%%%%%%%%%%%%%%%%%%%%%%%%%
%%%%%%%%%%%%%%%%%%%%%%%%%%%%%%%%%%%%%%%%%%%%%%%%%%%%%%%%%%%%%%%%%%%%%%%%%%%%%
\subsection{Notes on the Implementation in \texttt{Econpizza}}
\label{sec-app:codes-ep}

As mentioned in the main text of this thesis, the \texttt{Econpizza} package is maintained by Dr.~Gregor Böhl on GitHub under this link: \url{https://github.com/gboehl/econpizza/tree/master}, last retrieved \printdate{2023-08-22}, and documented in detail here: \url{https://econpizza.readthedocs.io/en/stable/index.html}, last retrieved \printdate{2023-08-22}.

I next note some important aspects regarding the implementation of this thesis' simulations with the package.

%%%%%%%%%%%%%%%%%%%%%%%%%%%%%%%%%%%%%%%%%%%%%%%%%%%%%%%%%%%%%%%%%%%%%%%%%%%%%
\subsubsection*{Discretisation of Asset and Productivity Grids}

As remarked in section \ref{sec:model-solution}, the asset and productivity grids have to be discretised. The package does this internally but the user must specify some characteristics of the grids, e.g.~the number of grid points. Moreover, the asset grid requires special attention in my use case. Therefore, I here briefly explain my approach to discretisation with an emphasis on the asset grid and the pitfalls associated with specifying it.

Consider first the productivity grid. Given the persistence of the skills process, $\rho_e$, and the variance of productivity shocks, $\sigma_e$, I need to specify the number of grid points, $N_e$. Choosing the number of grid points of course involves an accuracy-speed trade-off. In general, the more grid points, the more accurate the solution, but also the more computationally demanding the computation of that solution. I try to proceed pragmatically and choose $N_e = 8$. Changing this to $N_e=6$ or $N_e=10$ did not meaningfully affect my results for the aggregate dynamics after a credit crunch through the borrowing limit. 

Given $N_e = 8$ and the calibration of $\rho_e$ and $\sigma_e$ from table \ref{tab:calibration-baseline}, the \texttt{Econpizza} package employs the \textcite{rouwenhorst1995} method to convert the \Gls{ar}(1) process \eqref{eq:hh-income-process} into a grid of skills levels, a matrix of transition probabilities and a stationary skills distribution.

Next, consider the asset grid. Its specification is more involved in the present case because the correct implementation of a shock to the borrowing limit requires that the asset grid is the same in every time period, see \textcite{mellior2016mimeo}. If it was not, the fact that the lowest admissible grid point (i.e.~the borrowing limit) changes over time would imply that household policies are not given on the same grid in every instance of time. This would invalidate the dynamics because interpolation would occur between differing asset grids.\footnote{Recall that the policies are given as interpolants over discretised grids.}

To meet this requirement in the implementation, I use a function (named \texttt{create\_grid()} in the \texttt{hank\_functions.py} file, see section \ref{sec-app:codes-guide}) which creates a standard logarithmic grid (as is the default in the package) and then extend it as follows. Given the process for the credit limit \eqref{eq:borrowing-limit-process} together with values for the initial borrowing limit, $\phi_{ss}$ and, in the case of a permanent shock, the terminal borrowing limit, $\phi_{ss}'$, the persistence parameter, $\rho_{\phi}$, as well as the terminal horizon by which the model has settled in steady state (see below), it first calculates the specific transition that the borrowing limit undertakes from the initial period to the terminal period within the given horizon. I round these values to the sixth decimal place and thus remove points which are indistinguishably close to each other to avoid unnecessarily large grids. Then, the function adds the remaining points to the initial logarithmic grid. Finally, I compute the reverse transition of the borrowing limit, i.e.~from $\phi_{ss}'$ to $\phi_{ss}$, and add the resulting values to the grid, too, so that I can fairly compare an easing credit limit to a tightening one as in done in section \ref{sec:asymmetry} of the thesis.

To guarantee that the solution of the household block indeed uses valid asset grid points, I add a further step to the \Gls{egm} function of the implementation (\texttt{egm\_step()} and \texttt{egm\_step\_labour()} in the \texttt{hank\_functions.py} file), which finds for a given $\phi_s$ in period $s$ the closest point to $\phi_{ss}$ which is on the grid.

Note that the above procedure of creating the asset grid implies that the asset grids differ whenever one of the following parameters changes: $\phi_{ss}$, $\phi_{ss}'$, $\rho_{\phi}$ -- apart from the obvious ones such as the number of grid points (for the logarithmic grid) or the terminal horizon. This has to be kept in mind when interpreting the results, especially when comparing across models. 

In the baseline model with $\phi_{ss}$, $\phi_{ss}'$ and $\rho_{\phi}$ as in table \ref{tab:calibration-baseline}, the asset grid has $N_b = 236$ grid points as I specify the basic logarithmic grid to begin with $200$ grid points.\footnote{Changing this to $500$ did not affect the aggregate dynamics after a shock to the borrowing limit as in section \ref{sec:limit-transition}.} I set the maximum grid point to $150$, which ensures that no experiment considered in this thesis has negative values in the distribution in steady state.

%%%%%%%%%%%%%%%%%%%%%%%%%%%%%%%%%%%%%%%%%%%%%%%%%%%%%%%%%%%%%%%%%%%%%%%%%%%%%
\subsubsection*{Transition to New Steady State}

Throughout this thesis, I consider \textit{permanent} shocks. That is, once disturbed, the economy does not return to its initial steady state but converges to a new steady state, in which a new environment applies. To calculate the perfect-foresight transition of a model to the terminal steady state, this transition path must be calculated using the Jacobian of the terminal steady state instead of the one from the initial steady state, see \textcite{auclert2021}.\footnote{The steady-state Jacobian of a model is the crucial object in the solution approaches by \textcite{auclert2021} and \textcite{boehl2023econpizza}. It contains information as to how the economy behaves when outside of its steady state.} This is because, from $t=1$, the new regime applies. Implementing this in \texttt{Econpizza} is done by applying the function \texttt{.find\_path()} to the terminal model and passing to that function the initial steady state (i.e.~the aggregate steady-state values) and the initial distribution of agents over assets and skills.

Note that another crucial choice in the implementation is on the terminal period in which the economy is assumed to have completed its transition to the terminal steady state. I follow the default of the \texttt{Econpizza} package, which sets this horizon to $200$. See \textcite{auclert2021} for the importance of choosing this horizon sufficiently large.

%%%%%%%%%%%%%%%%%%%%%%%%%%%%%%%%%%%%%%%%%%%%%%%%%%%%%%%%%%%%%%%%%%%%%%%%%%%%%
%%%%%%%%%%%%%%%%%%%%%%%%%%%%%%%%%%%%%%%%%%%%%%%%%%%%%%%%%%%%%%%%%%%%%%%%%%%%%
\subsection{Guide to Computer Codes}
\label{sec-app:codes-guide}

I now shortly describe the Python code files employed. Essentially, this description is the \href{https://github.com/andkound98/master-thesis/blob/main/README.md}{README} file in the GitHub repository. 

%%%%%%%%%%%%%%%%%%%%%%%%%%%%%%%%%%%%%%%%%%%%%%%%%%%%%%%%%%%%%%%%%%%%%%%%%%%%%
\subsubsection*{\texttt{main.py}}

The file \texttt{main.py} is the main file of the project.

%%%%%%%%%%%%%%%%%%%%%%%%%%%%%%%%%%%%%%%%%%%%%%%%%%%%%%%%%%%%%%%%%%%%%%%%%%%%%
\subsubsection*{\texttt{custom\_functions.py}}

%%%%%%%%%%%%%%%%%%%%%%%%%%%%%%%%%%%%%%%%%%%%%%%%%%%%%%%%%%%%%%%%%%%%%%%%%%%%%
\subsubsection*{\texttt{plot\_functions.py}}

%%%%%%%%%%%%%%%%%%%%%%%%%%%%%%%%%%%%%%%%%%%%%%%%%%%%%%%%%%%%%%%%%%%%%%%%%%%%%
\subsubsection*{\texttt{calibration.py}}

This file uses the data described in appendix \ref{sec-app:data} to calculate the desired calibration targets and to create the figures \ref{fig:app-calibration} and \ref{fig:app-calibration-bonds}.

\bigskip

All these codes were run using the Spyder IDE $5.4.3$ with Python $3.9.12$ and \texttt{Econpizza} $0.6.1$ on macOS $12.6.5$. 

%%%%%%%%%%%%%%%%%%%%%%%%%%%%%%%%%%%%%%%%%%%%%%%%%%%%%%%%%%%%%%%%%%%%%%%%%%%%%
%%%%%%%%%%%%%%%%%%%%%%%%%%%%%%%%%%%%%%%%%%%%%%%%%%%%%%%%%%%%%%%%%%%%%%%%%%%%%
%%%%%%%%%%%%%%%%%%%%%%%%%%%%%%%%%%%%%%%%%%%%%%%%%%%%%%%%%%%%%%%%%%%%%%%%%%%%%
\section{Data for Model Calibration}
\label{sec-app:data}

This appendix describes the data series and sources used for the calibration of the baseline model in section \ref{sec:model-solution} and for the calibration of the model in section \ref{sec-app:robust-bonds}.

%%%%%%%%%%%%%%%%%%%%%%%%%%%%%%%%%%%%%%%%%%%%%%%%%%%%%%%%%%%%%%%%%%%%%%%%%%%%%
\subsubsection*{Matching $B$}

First, to match $B$ to total liquid assets held by households as in \textcite{mckay2016} and \textcite{gl2017}, I follow these authors by using information on aggregate household balance sheets, available through the Financial Accounts of the U.S.~(Z1), known as ``Flow of Funds" (\Gls{fof}) and provided by the Board of Governors of the Federal Reserve System. The data are available online through the website: \url{https://www.federalreserve.gov/releases/z1/}, last retrieved \printdate{2023-08-21}. I download the CSV file with the data from that website.

To obtain the time series of the measure of total liquid assets, I use the information from the table B.101, titled ``Balance Sheet of Households and Nonprofit Organizations (1)", as the aforementioned studies do. The code file \texttt{calibration.py} reads the CSV file and takes the sum of the seven following series to obtain total liquid assets (series IDs in brackets):\footnote{I try to be as close as possible to the choices of \textcite{mckay2016} and \textcite{gl2017} but some instances do not allow for this as series' names have likely changed.}
\begin{enumerate}
    \item 'Foreign deposits' (LM153091003.Q),
    \item 'Checkable deposits and currency' (FL153020005.Q),
    \item 'Time and savings deposits' (FL153030005.Q),
    \item 'Money market fund shares' (FL153034005.Q),
    \item 'Debt securities' (LM154022005.Q),
    \item 'Corporate equities' (LM153064105.Q),
    \item 'Mutual fund shares' (LM153064205.Q).
\end{enumerate}

I require data on the price level and \Gls{gdp} in the U.S., so that I can match $B$. I acquire data on U.S.~nominal \Gls{gdp} and the U.S.~price level from the FRED data base, which is maintained by the Federal Reserve Bank of St.~Louis and accessible through this website: \url{https://fred.stlouisfed.org}, last retrieved \printdate{2023-08-21}. Nominal \Gls{gdp} is titled ``Gross Domestic Product" (GDP) and the price level is ``Gross Domestic Product: Implicit Price Deflator" (GDPDEF). To obtain that data, \texttt{calibration.py} requires a FRED API key, which can be created for free through the FRED website. However, the data can be downloaded without that key as well.

I deflate the result for total liquid assets by the \Gls{gdp} deflator and convert it to a share of real \Gls{gdp} (nominal \Gls{gdp} deflated by the same deflator). Next, I multiply the result by four (to transform it into quarterly terms as ``GDP" is in annualised terms) and by steady-state output, $y_{ss}$ (to have a result which is in model units). Finally, I take the simple mean over the sample from the fourth quarter of 1951 to the first quarter of 2023. This procedure results in $B=5.21$ for the baseline model and $B=5.39$ for the model of section \ref{sec:sensitivity-end-labour}.\footnote{These differ because, in the model with endogenous labour supply, $y_{ss}$ is different than in the baseline.} I depict the data series together with its mean for the baseline model in figure \ref{fig:app-calibration-liquid}.

\begin{figure}[t]
    \caption{Data for the Baseline Model Calibration}
    \label{fig:app-calibration}
    \centering
    \begin{subfigure}[b]{0.49\textwidth}
    \caption{Liquid Assets Held by Households}
    \label{fig:app-calibration-liquid}
         \centering
         \includesvg[inkscapelatex=false, width=\textwidth]{Results/calibration_liquid.svg}
     \end{subfigure}
     \hfill
     \begin{subfigure}[b]{0.49\textwidth}
     \caption{Consumer Credit over \Gls{gdp}} 
    \label{fig:app-calibration-debt}
         \centering
         \includesvg[inkscapelatex=false, width=\textwidth]{Results/calibration_d.svg}
     \end{subfigure}

    \vspace{10pt}
    
    \begin{minipage}{\textwidth} 
    \footnotesize
    \textit{Note}: Data taken from the \href{https://www.federalreserve.gov/releases/z1/}{\Gls{fof} statistics} and the \href{https://fred.stlouisfed.org}{FRED data base}. Sample range: 1951:Q4 -- 2023:Q1. The dotted line is the full sample mean of the respective series. The series in panel \ref{fig:app-calibration-liquid} is in terms of the units from the \textit{baseline} calibration of the model in section \ref{sec:model}, in which $y_{ss} = 0.9129$. 
    \end{minipage}
\end{figure}

Second, in the robustness exercise of appendix \ref{sec-app:robust-bonds}, I follow \textcite{bayer2023} to calibrate $B$ such that it corresponds to the market value of gross U.S.~federal debt. This data is also available on FRED, specifically as ``Market Value of Gross Federal Debt" (MVGFD027MNFRBDAL). I take three-month averages to convert this monthly series into a quarterly one. Thereafter, I proceed analogously to before, I deflate the series, divide it by real \Gls{gdp}, multiply it by four and by $y_{ss}$. For the baseline model, the result is $B=2.24$, which is used in appendix \ref{sec-app:robust-bonds}. Figure \ref{fig:app-calibration-bonds} plots the series.

\begin{figure}[H]
    \centering
    \caption{U.S.~Federal Debt}
    \label{fig:app-calibration-bonds}
    \includesvg[inkscapelatex=false, width=0.75\textwidth]{Results/calibration_b.svg}

    \vspace{10pt}
    
    \begin{minipage}{0.75\textwidth} 
    \footnotesize
    \textit{Note}: Data taken from \href{https://fred.stlouisfed.org}{FRED data base}. Sample range: 1951:Q4 -- 2023:Q1. The dotted line is the full sample mean of the series. The series is in terms of the units from the \textit{baseline} calibration of the model in section \ref{sec:model}, in which $y_{ss} = 0.9129$.
    \end{minipage}
\end{figure}

%%%%%%%%%%%%%%%%%%%%%%%%%%%%%%%%%%%%%%%%%%%%%%%%%%%%%%%%%%%%%%%%%%%%%%%%%%%%%
\subsubsection*{Matching $\frac{D}{y}$}

As in \textcite{gl2017} and \textcite{bayer2023}, I use data on unsecured household credit to calibrate the (loose) debt limit. This is a quantity from the \Gls{fof} but I obtain it through the FRED data base. Therein, it is titled ``Households and Nonprofit Organizations; Consumer Credit; Liability, Level" (HCCSDODNS). Again, I deflate the series and divide by real \Gls{gdp}. Finally, I multiply by four to convert into quarterly terms and take the mean over 1951 (fourth quarter) to 2023 (first quarter). The result is $57\%$, which is the target for (model) debt over output used throughout the thesis. 

See figure \ref{fig:app-calibration-debt} for a visualisation of the time series. It reflects the remark in section \ref{sec:limit-stst-tight}, namely that reductions in debt-to-\Gls{gdp} of around five percentage points are not unrealistic. However, in reality, these changes are rather transitory around what appears to be a deterministic upward trend.

%%%%%%%%%%%%%%%%%%%%%%%%%%%%%%%%%%%%%%%%%%%%%%%%%%%%%%%%%%%%%%%%%%%%%%%%%%%%%
%%%%%%%%%%%%%%%%%%%%%%%%%%%%%%%%%%%%%%%%%%%%%%%%%%%%%%%%%%%%%%%%%%%%%%%%%%%%%
%%%%%%%%%%%%%%%%%%%%%%%%%%%%%%%%%%%%%%%%%%%%%%%%%%%%%%%%%%%%%%%%%%%%%%%%%%%%%
\section{Supplementary Results}
\label{sec-app:figures}

This appendix contains supplementary figures and tables, placed here for the sake of brevity in the main text. It is ordered according to the corresponding sections in the text.

%%%%%%%%%%%%%%%%%%%%%%%%%%%%%%%%%%%%%%%%%%%%%%%%%%%%%%%%%%%%%%%%%%%%%%%%%%%%%
%%%%%%%%%%%%%%%%%%%%%%%%%%%%%%%%%%%%%%%%%%%%%%%%%%%%%%%%%%%%%%%%%%%%%%%%%%%%%
\subsection{Figures for Section \ref{sec:limit-stst-loose}}
\label{sec-app:figures-limit-loose}

\begin{figure}[H]
    \centering
    \caption{\Gls{mpc}s Across Low Asset Holdings}
    \label{fig:baseline-limit-initial-mpcs}
    \includesvg[inkscapelatex=false, width=0.75\textwidth]{Results/stst_policies_mpc_baseline_limit_permanent_initial.svg}

    \vspace{10pt}
    
    \begin{minipage}{0.75\textwidth}
    \footnotesize
    \textit{Note}: The figure is cut off at an asset level of $2$. \Gls{mpc}s calculated as $mpc_{i,t} = \frac{\partial c_{i,t}}{\partial b_{i,t}}$ and are restricted to be weakly smaller than one. The labels $\theta\_0$ -- $\theta\_7$ represent the idiosyncratic labour skill states from \eqref{eq:hh-income-process}.
    \end{minipage}
\end{figure}

%%%%%%%%%%%%%%%%%%%%%%%%%%%%%%%%%%%%%%%%%%%%%%%%%%%%%%%%%%%%%%%%%%%%%%%%%%%%%
%%%%%%%%%%%%%%%%%%%%%%%%%%%%%%%%%%%%%%%%%%%%%%%%%%%%%%%%%%%%%%%%%%%%%%%%%%%%%
\subsection{Figures for Section \ref{sec:limit-stst-tight}}
\label{sec-app:figures-limit-tight}

The figure is placed on a separate page.

\begin{sidewaysfigure}
    \caption{Baseline Model -- Shock to Borrowing Limit: Comparison of Asset Distributions}
    \label{fig:comparison-stst-baseline-dist}
    \centering
\includesvg[inkscapelatex=false, width=0.8\textwidth]{Results/stst_dist_2d_comparison_baseline_limit_permanent.svg}
    \\
     \centering
	\textit{Note}: The distribution is cut off at a value of $30$.
\end{sidewaysfigure}

%%%%%%%%%%%%%%%%%%%%%%%%%%%%%%%%%%%%%%%%%%%%%%%%%%%%%%%%%%%%%%%%%%%%%%%%%%%%%
%%%%%%%%%%%%%%%%%%%%%%%%%%%%%%%%%%%%%%%%%%%%%%%%%%%%%%%%%%%%%%%%%%%%%%%%%%%%%
\subsection{Figures for Section \ref{sec:limit-transition}}
\label{sec-app:figures-limit-transition}

\begin{figure}[H]
    \caption{Supplement for Figure \ref{fig:baseline-permanent-limit-agg}}%{\hypersetup{hidelinks}\ref{fig:baseline-permanent-limit-agg}}}
    \label{fig:baseline-permanent-limit-agg-extra}
    \begin{subfigure}[b]{0.49\textwidth}
     \caption{Labour Hours, $N$}
     \label{fig:baseline-permanent-limit-agg-N}
         \centering
         \includesvg[inkscapelatex=false, width=\textwidth]{Results/transition_N_baseline_limit_permanent.svg}
     \end{subfigure}
     \hfill
     \begin{subfigure}[b]{0.49\textwidth}
     \caption{Wage, $w$}
     \label{fig:baseline-permanent-limit-agg-w}
         \centering
         \includesvg[inkscapelatex=false, width=\textwidth]{Results/transition_w_baseline_limit_permanent.svg}
     \end{subfigure}
     \hfill
     \begin{subfigure}[b]{0.49\textwidth}
     \centering
     \caption{Marginal Costs, $mc$}
     \label{fig:baseline-permanent-limit-agg-mc}
         \includesvg[inkscapelatex=false, width=\textwidth]{Results/transition_mc_baseline_limit_permanent.svg}
     \end{subfigure}
     \hfill
     \begin{subfigure}[b]{0.49\textwidth}
     \centering
     \caption{Inflation, $\pi$}
     \label{fig:baseline-permanent-limit-agg-pi}
         \includesvg[inkscapelatex=false, width=\textwidth]{Results/transition_pi_baseline_limit_permanent.svg}
     \end{subfigure}
     \hfill
    \begin{subfigure}[b]{0.49\textwidth}
    \caption{Tax Rate, $\tau$}
    \label{fig:baseline-permanent-limit-agg-tau}
         \centering
         \includesvg[inkscapelatex=false, width=\textwidth]{Results/transition_tau_baseline_limit_permanent.svg}
     \end{subfigure}
     \hfill
     \begin{subfigure}[b]{0.49\textwidth}
     \caption{Dividends, $\Pi$}
     \label{fig:baseline-permanent-limit-agg-Pi}
         \centering
         \includesvg[inkscapelatex=false, width=\textwidth]{Results/transition_div_baseline_limit_permanent.svg}
     \end{subfigure}

    \vspace{10pt}
     
     \begin{flushleft}
     \footnotesize
	\textit{Note}: Percent deviations mean percent deviations from the respective initial steady state. The time horizon in all plots is fixed to twelve quarters, including the initial steady state in $t=0$.
	\end{flushleft}
\end{figure}

\begin{figure}[H]
    \caption{Baseline Model -- Shock to Borrowing Limit: Debt in the Long-Run}
    \label{fig:baseline-permanent-limit-longrun}
    \begin{subfigure}[b]{0.49\textwidth}
     \caption{Debt, $D$}
     \label{fig:baseline-permanent-limit-longrun-D}
         \centering
         \includesvg[inkscapelatex=false, width=\textwidth]{Results/transition_D_baseline_limit_permanent_long_run_debt.svg}
     \end{subfigure}
     \hfill
     \begin{subfigure}[b]{0.49\textwidth}
     \caption{Debt-to-Output, $\frac{D}{y}$}
     \label{fig:baseline-permanent-limit-longrun-DY}
         \centering
         \includesvg[inkscapelatex=false, width=\textwidth]{Results/transition_DY_baseline_limit_permanent_long_run_debt.svg}
     \end{subfigure}

     \vspace{10pt}

     \begin{minipage}{\textwidth} 
    \footnotesize
    \textit{Note}: The time horizon in all plots is fixed to $80$ quarters, including the initial steady state in $t=0$.
    \end{minipage}
\end{figure}

%%%%%%%%%%%%%%%%%%%%%%%%%%%%%%%%%%%%%%%%%%%%%%%%%%%%%%%%%%%%%%%%%%%%%%%%%%%%%
%%%%%%%%%%%%%%%%%%%%%%%%%%%%%%%%%%%%%%%%%%%%%%%%%%%%%%%%%%%%%%%%%%%%%%%%%%%%%
\subsection{Figures for Section \ref{sec:sensitivity-persistence}}
\label{sec-app:figures-persistence}

\begin{figure}[H]
    \centering
    \caption{Supplement for Figure \ref{fig:baseline-permanent-limit-duration}}
    \label{fig:baseline-permanent-limit-duration-extra}
     \begin{subfigure}[b]{0.49\textwidth}
     \caption{Borrowing Limit, $\phi$}
     \label{fig:baseline-permanent-limit-duration-extra-phi}
         \centering
         \includesvg[inkscapelatex=false, width=\textwidth]{Results/compare_transitions/comparison_phi_fast_shock_limit_permanent_baseline_limit_permanent_slow_shock_limit_permanent.svg}
     \end{subfigure}
     \hfill
     \begin{subfigure}[b]{0.49\textwidth}
     \caption{\textit{Ex-Post} Real Interest Rate, $RR$}
     \label{fig:baseline-permanent-limit-duration-Rr}
         \centering
         \includesvg[inkscapelatex=false, width=\textwidth]{Results/compare_transitions/comparison_Rr_fast_shock_limit_permanent_baseline_limit_permanent_slow_shock_limit_permanent.svg}
     \end{subfigure}

    \vspace{10pt}
     
     \begin{minipage}{\textwidth} 
    \footnotesize
    \textit{Note}: The blue dashed line corresponds to $\rho_{\phi} = 0.3$ ("fast shock"), the orange continuous line corresponds to $\rho_{\phi} = 0.5$ (baseline) and the green dotted line corresponds to $\rho_{\phi} = 0.65$ ("slow shock"). Percent deviations mean percent deviations from the respective initial steady state. The time horizon in all plots is fixed to twelve quarters, including the initial steady state in $t=0$.
    \end{minipage}
\end{figure}

%%%%%%%%% INSERT FIGURES OF LONG-RUN DEBT DEVELOPMENT
\begin{figure}[H]
    \caption{Baseline Model -- Varying the Duration of a Credit Crunch: \\ Debt in the Long-Run}
    \label{fig:baseline-permanent-limit-duration-long-run}
    \begin{subfigure}[b]{0.49\textwidth}
     \caption{Debt, $D$}
     \label{fig:baseline-permanent-limit-duration-long-run-D}
         \centering
         \includesvg[inkscapelatex=false, width=\textwidth]{Results/compare_transitions/comparison_D_fast_shock_limit_permanent_baseline_limit_permanent_slow_shock_limit_permanent_long_run_debt}
     \end{subfigure}
     \hfill
     \begin{subfigure}[b]{0.49\textwidth}
     \caption{Debt-to-Output, $\frac{D}{y}$}
     \label{fig:baseline-permanent-limit-duration-long-run-DY}
         \centering
         \includesvg[inkscapelatex=false, width=\textwidth]{Results/compare_transitions/comparison_DY_fast_shock_limit_permanent_baseline_limit_permanent_slow_shock_limit_permanent_long_run_debt}
     \end{subfigure}

     \vspace{10pt}

     \begin{minipage}{\textwidth} 
    \footnotesize
    \textit{Note}: The blue dashed line corresponds to $\rho_{\phi} = 0.3$ ("fast shock"), the orange continuous line corresponds to $\rho_{\phi} = 0.5$ (baseline) and the green dotted line corresponds to $\rho_{\phi} = 0.65$ ("slow shock"). Percent deviations mean percent deviations from the respective initial steady state. The time horizon in all plots is fixed to $80$ quarters, including the initial steady state in $t=0$.
    \end{minipage}
\end{figure}

%%%%%%%%%%%%%%%%%%%%%%%%%%%%%%%%%%%%%%%%%%%%%%%%%%%%%%%%%%%%%%%%%%%%%%%%%%%%%
%%%%%%%%%%%%%%%%%%%%%%%%%%%%%%%%%%%%%%%%%%%%%%%%%%%%%%%%%%%%%%%%%%%%%%%%%%%%%
\subsection{Results for Section \ref{sec:sensitivity-end-labour}}
\label{sec-app:figures-end-L}

\begin{figure}[H]
    \caption{Endogenous Labour -- The Initial Steady State}
    \label{fig:init-stst-end-L}
    \centering
    \begin{subfigure}[b]{0.49\textwidth}
    \caption{Asset Distribution}
    \label{fig:init-stst-end-L-dist}
         \centering
         \includesvg[inkscapelatex=false, width=\textwidth]{Results/stst_dist_2d_bar_end_L_limit_permanent_initial.svg}
     \end{subfigure}
     \hfill
     \begin{subfigure}[b]{0.49\textwidth}
     \caption{Consumption Policies}
     \label{fig:init-stst-end-L-pol-c}
         \centering
         \includesvg[inkscapelatex=false, width=\textwidth]{Results/stst_policies_c_end_L_limit_permanent_initial.svg}
     \end{subfigure}

    \vspace{10pt}
     
     \justifying
     \footnotesize
	\textit{Note}: In figure \ref{fig:init-stst-end-L-dist}, the distribution is cut off at an asset level of $50$ for readability. As the model is solved on a logarithmic grid, grid points are distributed unevenly. Grid points are very dense at the lower end of the grid, which makes the colour appear bleak. At high levels of assets, grid points are very sparse, overestimating the true density at high bond holdings. The size of the bar at the borrowing limit is exaggerated. In figure \ref{fig:init-stst-end-L-pol-n}, the labels $\theta\_0$ -- $\theta\_7$ represent the idiosyncratic labour skill states from \eqref{eq:hh-income-process}.
\end{figure}

\begin{figure}[H]
    \centering
    \caption{Endogenous Labour -- Labour Supply Policies}
    \label{fig:init-stst-end-L-pol-n}
    \includesvg[inkscapelatex=false, width=0.75\textwidth]{Results/stst_policies_n_end_L_limit_permanent_initial.svg}

    \vspace{10pt}
    
    \begin{minipage}{0.75\textwidth}
    \footnotesize
    \textit{Note}: The labels $\theta\_0$ -- $\theta\_7$ represent the idiosyncratic labour skill states from \eqref{eq:hh-income-process}.
    \end{minipage}
\end{figure}

\begin{table}[H]
\centering
\caption{Endogenous Labour -- Shock to the Borrowing Limit: \\ Comparison of Steady States}
\label{tab:stst_comparison_end_L_limit_permanent}
\begin{tabular}{lccc}
Variable & Initial & Terminal &  Change \\
\hline
\hline
\multicolumn{2}{l}{\textbf{Borrowing Limit}} & & \\
$\phi$ & $-1.7956$ &   $-1.655$ &  $0.1406$ \\
& & & \\
\multicolumn{2}{l}{\textbf{Aggregates}} & & \\
$C$ &   $0.944$ &   $0.9438$ & $-0.02\%$ \\
$N$ &   $0.944$ &   $0.9438$ & $-0.02\%$ \\
$\frac{D}{y}$ & $57\%$ & $52\%$ & $-5 \ p.p.$ \\
$B+D$ &  $5.93$ &   $5.88$ & $-0.80\%$ \\
$Rr-1$ &  $0.15\%$ & $0.14\%$ & $-0.01 \ p.p.$ \\
& & & \\
\multicolumn{2}{l}{\textbf{Cross-Section}} & & \\
Share of Assets Held by Top-$10\%$ &  $53.22$ &   $52.64$ & $-0.58 \ p.p.$ \\
Share of Consumption by Top-$10\%$ &  $15.63$ &   $15.64$ &  $0.01 \ p.p.$ \\
Share of Labour Hours by Top-$10\%$ &  $21.23$ &   $21.21$ & $-0.02 \ p.p.$ \\
Fraction of Borrowers &   $39.14$ &    $38.46$ & $-0.68 \ p.p.$ \\
Fraction at Borrowing Limit &   $11.21$ &    $11.24$ &  $0.03 \ p.p.$ \\
\Gls{mpc} of Borrowers &    $0.33$ &     $0.34$ &  \\
\Gls{mpc} of Lenders &    $0.02$ &     $0.02$ &  \\
\hline
\multicolumn{4}{l}{\footnotesize \multirow{2}{12cm}{\justifying \textit{Note:} The table contains selected values of the initial steady state with $\phi_{ss}$ and the terminal steady state with $\phi_{ss}'$. All numbers are rounded and refer to quarterly values. $p.p.$ stands for percentage points.}} \\
& & & \\
\end{tabular}
\end{table}

\begin{figure}[H]
    \caption{Supplement for Figure \ref{fig:comparison-end-labour-agg}}
    \label{fig:comparison-end-labour-agg-extra}
    \begin{subfigure}[b]{0.49\textwidth}
     \caption{Debt, $D$}
     \label{fig:comparison-end-labour-agg-extra-D}
         \centering
         \includesvg[inkscapelatex=false, width=\textwidth]{Results/compare_transitions/comparison_D_baseline_limit_permanent_end_L_limit_permanent.svg}
     \end{subfigure}
     \hfill
     \begin{subfigure}[b]{0.49\textwidth}
     \centering
     \caption{Inflation, $\pi$}
     \label{fig:comparison-end-labour-agg-extra-pi}
         \includesvg[inkscapelatex=false, width=\textwidth]{Results/compare_transitions/comparison_pi_baseline_limit_permanent_end_L_limit_permanent.svg}
     \end{subfigure}
     \hfill
    \begin{subfigure}[b]{0.49\textwidth}
     \caption{\textit{Ex-Post} Real Interest Rate, $RR$}
     \label{fig:comparison-end-labour-agg-extra-Rr}
         \centering
         \includesvg[inkscapelatex=false, width=\textwidth]{Results/compare_transitions/comparison_Rr_baseline_limit_permanent_end_L_limit_permanent.svg}
     \end{subfigure}
     \hfill
     \begin{subfigure}[b]{0.49\textwidth}
     \caption{Nominal and Notional Interest Rates, $R$ and $R^n$}
     \label{fig:comparison-end-labour-agg-extrag-Rn}
         \centering
         \includesvg[inkscapelatex=false, width=\textwidth]{Results/compare_transitions/comparison_R_Rn_baseline_limit_permanent_end_L_limit_permanent.svg}
     \end{subfigure}
     \hfill
    \begin{subfigure}[b]{0.49\textwidth}
     \caption{Labour Hours, $N$}
     \label{fig:comparison-end-labour-agg-extra-N}
         \centering
         \includesvg[inkscapelatex=false, width=\textwidth]{Results/compare_transitions/comparison_N_baseline_limit_permanent_end_L_limit_permanent.svg}
     \end{subfigure}
     \hfill
     \begin{subfigure}[b]{0.49\textwidth}
     \caption{Wage, $w$}
     \label{fig:comparison-end-labour-agg-extrag-w}
         \centering
         \includesvg[inkscapelatex=false, width=\textwidth]{Results/compare_transitions/comparison_w_baseline_limit_permanent_end_L_limit_permanent.svg}
     \end{subfigure}

     \vspace{10pt}

     \begin{flushleft}
     \footnotesize
	\textit{Note}: Percent deviations mean percent deviations from the respective initial steady state. The time horizon in all plots is fixed to twelve quarters, including the initial steady state in $t=0$.
	\end{flushleft}
\end{figure}

%%%%%%%%%%%%%%%%%%%%%%%%%%%%%%%%%%%%%%%%%%%%%%%%%%%%%%%%%%%%%%%%%%%%%%%%%%%%%
%%%%%%%%%%%%%%%%%%%%%%%%%%%%%%%%%%%%%%%%%%%%%%%%%%%%%%%%%%%%%%%%%%%%%%%%%%%%%
\subsection{Results for Section \ref{sec:wedge}}
\label{sec-app:figures-wedge}

\begin{figure}[H]
    \caption{Baseline Model -- Shock to the Interest Rate Wedge: \\ Household Policies}
    \label{fig:comparison-stst-baseline-wedge-policies}
    \centering
    \begin{subfigure}[b]{0.75\textwidth}
    \caption{Average Asset Policy}
    \label{fig:comparison-stst-baseline-wedge-policies-a}
         \centering
         \includesvg[inkscapelatex=false, width=\textwidth]{Results/stst_a_baseline_wedge_permanent.svg}
     \end{subfigure}
     
     \vspace{10pt}
     
     \begin{subfigure}[b]{0.75\textwidth}
     \caption{Average Consumption Policy}
     \label{fig:comparison-stst-baseline-wedge-policies-c}
         \centering
         \includesvg[inkscapelatex=false, width=\textwidth]{Results/stst_c_baseline_wedge_permanent.svg}
     \end{subfigure}

    \vspace{10pt}

    \begin{minipage}{0.75\textwidth} 
    \footnotesize
    \textit{Note}: The policies are cut off at an asset level of $30$ for readability. In both figures, the differences are a bit difficult to discern. A close look reveals the aspects discussed in the text.
    \end{minipage}
\end{figure}

\begin{sidewaysfigure}
    \caption{Baseline Model -- Shock to the Interest Rate Wedge: \\ Comparison of Asset Distributions}
    \label{fig:comparison-stst-baseline-wedge-dist}
    \centering
\includesvg[inkscapelatex=false, width=0.8\textwidth]{Results/stst_dist_2d_comparison_baseline_wedge_permanent.svg}
    \\
     \centering
	\textit{Note}: The distribution is cut off at a value of $30$.
\end{sidewaysfigure}

\begin{figure}[H]
    \caption{Supplement for Figure \ref{fig:baseline-permanent-wedge-agg}}%{\hypersetup{hidelinks}\ref{fig:baseline-permanent-limit-agg}}}
    \label{fig:baseline-permanent-wedge-agg-extra}
    \begin{subfigure}[b]{0.49\textwidth}
     \caption{Debt-to-Output, $\frac{D}{y}$}
     \label{fig:baseline-permanent-wedge-agg-DY}
         \centering
         \includesvg[inkscapelatex=false, width=\textwidth]{Results/compare_transitions/comparison_DY_baseline_limit_permanent_baseline_wedge_permanent.svg}
     \end{subfigure}
     \hfill
     \begin{subfigure}[b]{0.49\textwidth}
     \centering
     \caption{Inflation, $\pi$}
     \label{fig:baseline-permanent-wedge-agg-pi}
         \includesvg[inkscapelatex=false, width=\textwidth]{Results/compare_transitions/comparison_pi_baseline_limit_permanent_baseline_wedge_permanent.svg}
     \end{subfigure}
     \hfill
    \begin{subfigure}[b]{0.49\textwidth}
     \caption{\textit{Ex-Post} Real Interest Rate, $RR$}
     \label{fig:baseline-permanent-wedge-agg-Rr}
         \centering
         \includesvg[inkscapelatex=false, width=\textwidth]{Results/compare_transitions/comparison_Rr_baseline_limit_permanent_baseline_wedge_permanent.svg}
     \end{subfigure}
     \hfill
     \begin{subfigure}[b]{0.49\textwidth}
     \caption{Nominal and Notional Interest Rates, $R$ and $R^n$}
     \label{fig:baseline-permanent-wedge-agg-Rn}
         \centering
         \includesvg[inkscapelatex=false, width=\textwidth]{Results/compare_transitions/comparison_R_Rn_baseline_limit_permanent_baseline_wedge_permanent.svg}
     \end{subfigure}

     \vspace{10pt}

     \begin{flushleft}
     \footnotesize
	\textit{Note}: Percent deviations mean percent deviations from the respective initial steady state. The time horizon in all plots is fixed to twelve quarters, including the initial steady state in $t=0$.
	\end{flushleft}
\end{figure}

\begin{figure}[H]
    \centering
    \caption{Baseline Model -- Shock to Interest Rate Wedge: \\ Consumption Response by Percentiles}
    \label{fig:baseline-permanent-wedge-dist-c-response}
    \includesvg[inkscapelatex=false, width=0.75\textwidth]{Results/percentile_transitions_baseline_wedge_permanent_C.svg}
    
    \vspace{10pt}
    
    \begin{minipage}{0.75\textwidth} 
    \footnotesize
    \textit{Note}: The responses are calculated as the product of aggregate consumption in period $t$ and the share that the respective household percentile has in consumption in $t$. Percent deviations mean percent deviations from the respective initial steady state. The time horizon in the plot is fixed to twelve quarters, including the initial steady state in $t=0$.
    \end{minipage}
\end{figure}

%%%%%%%%%%%%%%%%%%%%%%%%%%%%%%%%%%%%%%%%%%%%%%%%%%%%%%%%%%%%%%%%%%%%%%%%%%%%%
% \subsubsection{Analogy of Interest Rate Wedge Shock to Slow Borrowing Limit Shock}
% \label{sec-app:figures-wedge-analogy}

\begin{figure}[H]
    \caption{Baseline Model -- Shock to Interest Rate Wedge vs \\ Slow Shock to Borrowing Limit ($\rho_{\phi} = 0.86$): Aggregate Dynamics}
    \label{fig:comparison-wedge-analogy-slow}
    \centering
    \begin{subfigure}[b]{0.49\textwidth}
    \caption{Borrowing Limit, $\phi$}
    \label{fig:comparison-wedge-analogy-slow-phi}
         \centering
         \includesvg[inkscapelatex=false, width=\textwidth]{Results/compare_transitions/comparison_phi_baseline_wedge_permanent_very_slow_phi_limit_permanent.svg}
     \end{subfigure}
     \hfill
    \begin{subfigure}[b]{0.49\textwidth}
    \caption{Interest Rate Wedge, $\kappa$}
    \label{fig:comparison-wedge-analogy-slow-kappa}
         \centering
         \includesvg[inkscapelatex=false, width=\textwidth]{Results/compare_transitions/comparison_kappa_baseline_wedge_permanent_very_slow_phi_limit_permanent.svg}
     \end{subfigure}
     \hfill
    \begin{subfigure}[b]{0.49\textwidth}
    \caption{Consumption, $C$}
    \label{fig:comparison-wedge-analogy-slow-C}
         \centering
         \includesvg[inkscapelatex=false, width=\textwidth]{Results/compare_transitions/comparison_C_baseline_wedge_permanent_very_slow_phi_limit_permanent.svg}
     \end{subfigure}
     \hfill
    \begin{subfigure}[b]{0.49\textwidth}
    \caption{\textit{Ex-Post} Real Rate on Debt, $RR+\kappa$}
    \label{fig:comparison-wedge-analogy-slow-Rrminus}
         \centering
         \includesvg[inkscapelatex=false, width=\textwidth]{Results/compare_transitions/comparison_Rrminus_baseline_wedge_permanent_very_slow_phi_limit_permanent.svg}
     \end{subfigure}
     \hfill
    \begin{subfigure}[b]{0.49\textwidth}
     \caption{Debt, $D$}
     \label{fig:comparison-wedge-analogy-slow-D}
         \centering
         \includesvg[inkscapelatex=false, width=\textwidth]{Results/compare_transitions/comparison_D_baseline_wedge_permanent_very_slow_phi_limit_permanent.svg}
     \end{subfigure}
     \hfill
     \begin{subfigure}[b]{0.49\textwidth}
     \caption{Debt-to-Output, $\frac{D}{y}$}
     \label{fig:comparison-wedge-analogy-slow-DY}
         \centering
         \includesvg[inkscapelatex=false, width=\textwidth]{Results/compare_transitions/comparison_DY_baseline_wedge_permanent_very_slow_phi_limit_permanent.svg}
     \end{subfigure}
     
     \vspace{10pt}
     
     \justifying
     \footnotesize
	\textit{Note}: Percent deviations mean percent deviations from the respective initial steady state. The time horizon in all plots is fixed to twelve quarters, including the initial steady state in $t=0$.
\end{figure}

%%%%%%%%%%%%%%%%%%%%%%%%%%%%%%%%%%%%%%%%%%%%%%%%%%%%%%%%%%%%%%%%%%%%%%%%%%%%%
%%%%%%%%%%%%%%%%%%%%%%%%%%%%%%%%%%%%%%%%%%%%%%%%%%%%%%%%%%%%%%%%%%%%%%%%%%%%%
%%%%%%%%%%%%%%%%%%%%%%%%%%%%%%%%%%%%%%%%%%%%%%%%%%%%%%%%%%%%%%%%%%%%%%%%%%%%%
\section{Robustness and Supplementary Experiments}
\label{sec-app:robust}

This appendix presents various robustness exercises and supplementary experiments alluded to in the main text.

%%%%%%%%%%%%%%%%%%%%%%%%%%%%%%%%%%%%%%%%%%%%%%%%%%%%%%%%%%%%%%%%%%%%%%%%%%%%%
%%%%%%%%%%%%%%%%%%%%%%%%%%%%%%%%%%%%%%%%%%%%%%%%%%%%%%%%%%%%%%%%%%%%%%%%%%%%%
\subsection{Abscence of the \Gls{zlb}}
\label{sec-app:robust-no-zlb}

This appendix conducts the same exercise as section \ref{sec:limit-transition} but ignores the \Gls{zlb}. The \Gls{zlb} is usually motivated by the argument that households would not hold any asset with a return strictly below zero as money would have a strictly higher return.\footnote{As is common in the literature, the present model does not feature money explicitly.} The literature has highlighted the \Gls{zlb} either as the trigger (as in \cite{egg2012}) or as an amplifier (as in \cite{gl2017}) of financial shocks, contractions in the debt limit in particular.

Given the fact that the \Gls{zlb} does bind in my baseline experiment -- in fact, I calibrated $\beta$ to be such that it does -- I here aim to gauge the quantitative effect that the \Gls{zlb} has on the results from section \ref{sec:limit-transition}. to that end, I leave the model used in that section unchanged but exchange \eqref{eq:zlb} for:
\begin{equation}
    R_t = R^n_t, \label{eq:no-zlb}
\end{equation}
i.e.~the modified model ignores the \Gls{zlb} on nominal interest rates and lets the nominal rate fall below zero.

Figure \ref{fig:comparison-no-ZLB-agg} replicates figure \ref{fig:baseline-permanent-limit-agg}. In particular, it presents some selected aggregate dynamics after a permanent contraction in the debt limit and compares these to the baseline case, i.e.~the responses from plot \ref{fig:baseline-permanent-limit-agg}. Clearly, the \Gls{zlb} does have a noticable impact on the economy's dynamics. Aggregate consumption does not drop as much in the absence of this constraint as it does in its presence. Specifically, consumption drops by roughly half a percentage point less on impact, see figure \ref{fig:comparison-no-ZLB-agg-C}.

This is intuitive because, as explained in the main text, once the \Gls{zlb} becomes binding, the real interest rate cannot fall as much, reducing the incentive for the unconstrained agents to reduce their higher wealth targets, i.e.~less debt in the case of an unconstrained net borrower and more savings in the case of an unconstrained net lender. Thus, in the absence of the \Gls{zlb}, the real interest rate in $t=2$ falls by more, stimulating more consumption as well as less savings or more debt, respectively. This dampens the recession in $t=1$, which again feeds back into a nominal interest rate in $t=1$ which is lower than the notional rate in the same period in the presence of the \Gls{zlb}, see plot \ref{fig:comparison-no-ZLB-agg-Rn}.

Note however, that the evidence presented also supports the claim in section \ref{sec:limit-transition} that the \Gls{zlb} is merely an amplifier and not the cause of the recession in the present model, consistent with the findings by \textcite{gl2017}. That is, a rough calculation suggests that from the three percent drop observed in the \Gls{zlb} scenario, only $0.5$ percentage points, i.e.~around one sixth, is due to the \Gls{zlb} binding.

\begin{figure}[H]
    \caption{Baseline w/o \Gls{zlb} -- Shock to the Borrowing Limit: Aggregate Dynamics}
    \label{fig:comparison-no-ZLB-agg}
    \centering
    \begin{subfigure}[b]{0.49\textwidth}
    \caption{Borrowing Limit, $\phi$}
    \label{fig:comparison-no-ZLB-agg-phi}
         \centering
         \includesvg[inkscapelatex=false, width=\textwidth]{Results/compare_transitions/comparison_phi_baseline_limit_permanent_no_ZLB_limit_permanent.svg}
     \end{subfigure}
     \hfill
    \begin{subfigure}[b]{0.49\textwidth}
    \caption{Consumption, $C$}
    \label{fig:comparison-no-ZLB-agg-C}
         \centering
         \includesvg[inkscapelatex=false, width=\textwidth]{Results/compare_transitions/comparison_C_baseline_limit_permanent_no_ZLB_limit_permanent.svg}
     \end{subfigure}
     \hfill
    \begin{subfigure}[b]{0.49\textwidth}
    \caption{Debt, $D$}
    \label{fig:comparison-no-ZLB-agg-D}
         \centering
         \includesvg[inkscapelatex=false, width=\textwidth]{Results/compare_transitions/comparison_D_baseline_limit_permanent_no_ZLB_limit_permanent.svg}
     \end{subfigure}
     \hfill
    \begin{subfigure}[b]{0.49\textwidth}
    \caption{Debt-to-Output, $\frac{D}{y}$}
    \label{fig:comparison-no-ZLB-agg-DY}
         \centering
         \includesvg[inkscapelatex=false, width=\textwidth]{Results/compare_transitions/comparison_DY_baseline_limit_permanent_no_ZLB_limit_permanent.svg}
     \end{subfigure}
     \hfill
    \begin{subfigure}[b]{0.49\textwidth}
     \caption{\textit{Ex-Post} Real Interest Rate, $RR$}
     \label{fig:comparison-no-ZLB-agg-Rr}
         \centering
         \includesvg[inkscapelatex=false, width=\textwidth]{Results/compare_transitions/comparison_Rr_baseline_limit_permanent_no_ZLB_limit_permanent.svg}
     \end{subfigure}
     \hfill
     \begin{subfigure}[b]{0.49\textwidth}
     \caption{Nominal and Notional Interest Rates, $R$ and $R^n$}
     \label{fig:comparison-no-ZLB-agg-Rn}
         \centering
         \includesvg[inkscapelatex=false, width=\textwidth]{Results/compare_transitions/comparison_R_Rn_baseline_limit_permanent_no_ZLB_limit_permanent.svg}
     \end{subfigure}
     
     \vspace{10pt}
     
     \justifying
     \footnotesize
	\textit{Note}: Percent deviations mean percent deviations from the respective initial steady state. The time horizon in all plots is fixed to twelve quarters, including the initial steady state in $t=0$.
\end{figure}

%%%%%%%%%%%%%%%%%%%%%%%%%%%%%%%%%%%%%%%%%%%%%%%%%%%%%%%%%%%%%%%%%%%%%%%%%%%%%
%%%%%%%%%%%%%%%%%%%%%%%%%%%%%%%%%%%%%%%%%%%%%%%%%%%%%%%%%%%%%%%%%%%%%%%%%%%%%
\subsection{Calibration of Bonds à la \textcite{bayer2023}}
\label{sec-app:robust-bonds}

This appendix re-calibrates $B$ following the approach of \textcite{bayer2023}. Specifically, $B$ here targets the market value of gross federal government debt instead of total liquid assets. To that end, I use -- as \textcite{bayer2023} do -- data from FRED, see appendix \ref{sec-app:data} for details. The result is that $B$ must equal $2.24$. Further, I re-set $\beta$ so that the net steady-state interest rate is $0.15\%$, as before.

Note that the corresponding steady states, the wealth distributions in particular, is different. For example, both the fraction of borrowers ($46.71\%$) and the aggregate \Gls{mpc} ($0.14$) are much higher in the present case than in the baseline. The latter observation is consistent with the observation by \textcite{kaplan2018} that \Gls{mpc}s increase when the wealth target is smaller.

Now, \Gls{iou}s make up a larger share of gross liquidity. 

\begin{figure}[H]
    \caption{Baseline with Low $B$ -- Shock to the Borrowing Limit: Aggregate Dynamics}
    \label{fig:comparison-low-B-agg}
    \centering
    \begin{subfigure}[b]{0.49\textwidth}
    \caption{Borrowing Limit, $\phi$}
    \label{fig:comparison-low-B-agg-phi}
         \centering
         \includesvg[inkscapelatex=false, width=\textwidth]{Results/compare_transitions/comparison_phi_baseline_limit_permanent_low_B_limit_permanent.svg}
     \end{subfigure}
     \hfill
    \begin{subfigure}[b]{0.49\textwidth}
    \caption{Consumption, $C$}
    \label{fig:comparison-low-B-agg-C}
         \centering
         \includesvg[inkscapelatex=false, width=\textwidth]{Results/compare_transitions/comparison_C_baseline_limit_permanent_low_B_limit_permanent.svg}
     \end{subfigure}
     \hfill
    \begin{subfigure}[b]{0.49\textwidth}
    \caption{Debt, $D$}
    \label{fig:comparison-low-B-agg-D}
         \centering
         \includesvg[inkscapelatex=false, width=\textwidth]{Results/compare_transitions/comparison_D_baseline_limit_permanent_low_B_limit_permanent.svg}
     \end{subfigure}
     \hfill
    \begin{subfigure}[b]{0.49\textwidth}
    \caption{Debt-to-Output, $\frac{D}{y}$}
    \label{fig:comparison-low-B-agg-DY}
         \centering
         \includesvg[inkscapelatex=false, width=\textwidth]{Results/compare_transitions/comparison_DY_baseline_limit_permanent_low_B_limit_permanent.svg}
     \end{subfigure}
     \hfill
    \begin{subfigure}[b]{0.49\textwidth}
     \caption{\textit{Ex-Post} Real Interest Rate, $RR$}
     \label{fig:comparison-low-B-agg-Rr}
         \centering
         \includesvg[inkscapelatex=false, width=\textwidth]{Results/compare_transitions/comparison_Rr_baseline_limit_permanent_low_B_limit_permanent.svg}
     \end{subfigure}
     \hfill
     \begin{subfigure}[b]{0.49\textwidth}
     \caption{Nominal and Notional Interest Rates, $R$ and $R^n$}
     \label{fig:comparison-low-B-agg-Rn}
         \centering
         \includesvg[inkscapelatex=false, width=\textwidth]{Results/compare_transitions/comparison_R_Rn_baseline_limit_permanent_low_B_limit_permanent.svg}
     \end{subfigure}
     
     \vspace{10pt}
     
     \justifying
     \footnotesize
	\textit{Note}: Percent deviations mean percent deviations from the respective initial steady state. The time horizon in all plots is fixed to twelve quarters, including the initial steady state in $t=0$.
\end{figure}

%%%%%%%%%%%%%%%%%%%%%%%%%%%%%%%%%%%%%%%%%%%%%%%%%%%%%%%%%%%%%%%%%%%%%%%%%%%%%
%%%%%%%%%%%%%%%%%%%%%%%%%%%%%%%%%%%%%%%%%%%%%%%%%%%%%%%%%%%%%%%%%%%%%%%%%%%%%
\subsection{Shock to the Discount Factor}
\label{sec-app:robust-beta}

This section briefly presents the results of the baseline model of section \ref{sec:model} to a shock to the household discount factor, $\beta$. I do so to show that the sluggish and gradual response of aggregate debt holdings is a feature of the \Gls{hank} which is present also in other setting than merely in the credit-crunch scenarios studied in the main text.

In particular, I permanently increase $\beta$ from the baseline $0.99056$ to $0.992$, where $\beta$ evolves analogously to \eqref{eq:borrowing-limit-process} as:\footnote{The terminal value of $\beta$ is not meant to achieve any particular target. The long-run dynamics of debt are telling in any case.}
\begin{equation*}
    \beta_t = \beta_{ss} \Bigg( \frac{\beta_{t-1}}{\beta_{ss}} \Bigg)^{\rho_{\beta}} exp(\varepsilon_{\beta}),
\end{equation*}
where $\rho_{\beta} = 0.5$ to achieve a transition of six quarters and $\varepsilon_{\beta} = 0$ $\forall t$.

Figure \ref{fig:baseline-permanent-beta-agg} presents the resulting dynamics of the discount factor itself, of aggregate consumption and of the debt quantities over a horizon of $100$ quarters. The findings clearly point to what was argued in the main text: agents in the present \Gls{hank} model find it optimal to adjust their balance sheets over prolonged periods of time, as witnessed by the lengthy transitions of debt and debt-to-output in figures \ref{fig:baseline-permanent-beta-agg-D} and \ref{fig:baseline-permanent-beta-agg-DY}.

\begin{figure}[H]
    \caption{Baseline Model -- Shock to Discount Factor: Aggregate Dynamics}
    \label{fig:baseline-permanent-beta-agg}
    \begin{subfigure}[b]{0.49\textwidth}
     \caption{Household Discount Factor, $\beta$}
     \label{fig:baseline-permanent-beta-agg-beta}
         \centering
         \includesvg[inkscapelatex=false, width=\textwidth]{Results/transition_beta_baseline_beta_permanent.svg}
     \end{subfigure}
     \hfill
     \begin{subfigure}[b]{0.49\textwidth}
     \centering
     \caption{Consumption, $C$}
     \label{fig:baseline-permanent-beta-agg-C}
         \includesvg[inkscapelatex=false, width=\textwidth]{Results/transition_C_baseline_beta_permanent.svg}
     \end{subfigure}
     \hfill
    \begin{subfigure}[b]{0.49\textwidth}
     \caption{Debt, $D$}
     \label{fig:baseline-permanent-beta-agg-D}
         \centering
         \includesvg[inkscapelatex=false, width=\textwidth]{Results/transition_D_baseline_beta_permanent.svg}
     \end{subfigure}
     \hfill
     \begin{subfigure}[b]{0.49\textwidth}
     \caption{Debt-to-Output, $\frac{D}{y}$}
     \label{fig:baseline-permanent-beta-agg-DY}
         \centering
         \includesvg[inkscapelatex=false, width=\textwidth]{Results/transition_DY_baseline_beta_permanent.svg}
     \end{subfigure}

     \vspace{10pt}

     \begin{flushleft}
     \footnotesize
	\textit{Note}: Percent deviations mean percent deviations from the respective initial steady state. The time horizon in all plots is fixed to $100$ quarters, including the initial steady state in $t=0$.
	\end{flushleft}
\end{figure}

%%%%%%%%%%%%%%%%%%%%%%%%%%%%%%%%%%%%%%%%%%%%%%%%%%%%%%%%%%%%%%%%%%%%%%%%%%%%%
%%%%%%%%%%%%%%%%%%%%%%%%%%%%%%%%%%%%%%%%%%%%%%%%%%%%%%%%%%%%%%%%%%%%%%%%%%%%%
\end{appendices}
\newpage
\thispagestyle{plain}
\pagenumbering{Roman} % Start Roman page numbering for appendix references
\renewcommand*{\thepage}{A-\Roman{page}} % Add 'A' to each page number for appendix references

\printbibliography[heading=subbibliography, title={Appendix References}]
\addcontentsline{toc}{section}{Appendix References}

\cleardoublepage % Ensure the next content starts on an odd page (if your document is two-sided)

\pagenumbering{arabic} % Switch back to Arabic page numbering
\renewcommand*{\thepage}{\arabic{page}} % Reset the page numbering format

\end{refsection}

%%%%%%%%%%%%%%%%%%%%%%%%%%%%%%%%%%%%%%%%%%%%%%%%%%%%%%%%%%%%%%%%%%%%%%%%%%%%%
%%%%%%%%%%%%%%%%%%%%%%%%%%%%%%%%%%%%%%%%%%%%%%%%%%%%%%%%%%%%%%%%%%%%%%%%%%%%%
%%%%%%%%%%%%%%%%%%%%%%%%%%%%%%%%%%%%%%%%%%%%%%%%%%%%%%%%%%%%%%%%%%%%%%%%%%%%%
\newpage
\thispagestyle{plain}
\pagenumbering{gobble} % Turn page numbering off
\section*{Statement of Authorship} % Include statement of authorship
I hereby confirm that the work presented has been performed and interpreted solely by myself except for where I explicitly identified the contrary. I assure that this work has not been presented in any other form for the fulfillment of any other degree or qualification. Ideas taken from other works in letter and in spirit are identified in every single case.

\vspace{2cm}
\noindent
\rule{8cm}{0.4pt}\\
Andreas Koundouros\\
Bonn, the \printdate{2023-8-25}
\end{document}
% End Document
%%%%%%%%%%%%%%%%%%%%%%%%%%%%%%%%%%%%%%%%%%%%%%%%%%%%%%%%%%%%%%%%%%%%%%%%%%%%%
%%%%%%%%%%%%%%%%%%%%%%%%%%%%%%%%%%%%%%%%%%%%%%%%%%%%%%%%%%%%%%%%%%%%%%%%%%%%%
%%%%%%%%%%%%%%%%%%%%%%%%%%%%%%%%%%%%%%%%%%%%%%%%%%%%%%%%%%%%%%%%%%%%%%%%%%%%%
%%%%%%%%%%%%%%%%%%%%%%%%%%%%%%%%%%%%%%%%%%%%%%%%%%%%%%%%%%%%%%%%%%%%%%%%%%%%%
%%%%%%%%%%%%%%%%%%%%%%%%%%%%%%%%%%%%%%%%%%%%%%%%%%%%%%%%%%%%%%%%%%%%%%%%%%%%%
%%%%%%%%%%%%%%%%%%%%%%%%%%%%%%%%%%%%%%%%%%%%%%%%%%%%%%%%%%%%%%%%%%%%%%%%%%%%%
