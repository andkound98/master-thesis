\documentclass[12pt]{article} % Set documentclass article and set font size

%%%%%%%%%%%%%%%%%%%%%%%%%%%%%%%%%%%%%%%%%%%%%%%%%%%%%%%%%%%%%%%%%%%%%%%%%%%%%
%%%%%%%%%%%%%%%%%%%%%%%%%%%%%%%%%%%%%%%%%%%%%%%%%%%%%%%%%%%%%%%%%%%%%%%%%%%%%
%%%%%%%%%%%%%%%%%%%%%%%%%%%%%%%%%%%%%%%%%%%%%%%%%%%%%%%%%%%%%%%%%%%%%%%%%%%%%
% Packages
\usepackage{fontspec} % If used, change compiler to XeLaTex
\usepackage[utf8]{inputenc}
\usepackage[british]{babel}
\usepackage[british]{isodate}
\usepackage{csquotes}
\usepackage{mathtools}
\usepackage{geometry}
\usepackage[
backend=biber,
style=apa,
sorting=nyt, % Sort by name, year, title
backref=true,
dateabbrev=false,
language=british,
urldate=long
]{biblatex}
\usepackage{xcolor}
\usepackage[toc,page]{appendix}
\usepackage{svg}
\usepackage{float}
\usepackage{longtable}
\usepackage{multirow}
\usepackage{caption}
\usepackage{subcaption}
\usepackage{setspace}
\usepackage[symbols, nopostdot, automake, nonumberlist]{glossaries}
\usepackage{fancyhdr}
\usepackage{enumitem}

%%%%%%%%%%%%%%%%%%%%%%%%%%%%%%%%%%%%%%%%%%%%%%%%%%%%%%%%%%%%%%%%%%%%%%%%%%%%%
%%%%%%%%%%%%%%%%%%%%%%%%%%%%%%%%%%%%%%%%%%%%%%%%%%%%%%%%%%%%%%%%%%%%%%%%%%%%%
%%%%%%%%%%%%%%%%%%%%%%%%%%%%%%%%%%%%%%%%%%%%%%%%%%%%%%%%%%%%%%%%%%%%%%%%%%%%%
% Custom Settings
\addbibresource{master_thesis.bib} % Add bibliography source

\setlength\bibitemsep{1.5\itemsep} % Adjust distance of reference entries

\renewcommand*\appendixpagename{\Large Appendices} % Adjust size of title of the appendices section

\renewcommand*{\finentrypunct}{}
\renewbibmacro*{pageref}{%
  \addperiod% NEW
  \iflistundef{pageref}
    {}
    {\newline\footnotesize\printtext[parens]{%
       \ifnumgreater{\value{pageref}}{1}
         {\bibstring{backrefpages}\ppspace}
     {\bibstring{backrefpage}\ppspace}%
       \printlist[pageref][-\value{listtotal}]{pageref}}}}%
\DefineBibliographyStrings{english}{
  backrefpage={Cited on page},
  backrefpages={Cited on pages}
}

\setmainfont{Times New Roman} % If the usepackage fontspec is active, you can set Times New Roman here

\geometry{ % Adjust paper geometry
 a4paper,
 right=25mm,
 bottom=20mm,
 left=25mm,
 top=25mm}
\urlstyle{same} % Make URL font the same with the rest of the document
\onehalfspacing % Adjust spacing
\allowdisplaybreaks % Allow for breaks in equations across pages
\addto\captionsbritish{ % Replace "english" with the language you use
  \renewcommand{\contentsname}%
    {Table of Contents}%
} % Adjust title of the table of contents
\numberwithin{equation}{section} % Number equations with sections
\setcounter{tocdepth}{2} % Set maximum level to which sections are displayed in the table of contents

\pagestyle{plain}

% Define a custom name format that prints names of authors in references in bold
\DeclareNameWrapperFormat{sortname}{\mkbibbold{#1}}
\DeclareNameWrapperAlias{author}{sortname}

%%%%%%%%%%%%%%%%%%%%%%%%%%%%%%%%%%%%%%%%%%%%%%%%%%%%%%%%%%%%%%%%%%%%%%%%%%%%%
%%%%%%%%%%%%%%%%%%%%%%%%%%%%%%%%%%%%%%%%%%%%%%%%%%%%%%%%%%%%%%%%%%%%%%%%%%%%%
%%%%%%%%%%%%%%%%%%%%%%%%%%%%%%%%%%%%%%%%%%%%%%%%%%%%%%%%%%%%%%%%%%%%%%%%%%%%%
% Glossary
\makeglossaries 

% Glossary entries
\newglossaryentry{nk}{
name=NK,
description={New Keynesian}
}

\newglossaryentry{ha}{
name=HA,
description={Heterogeneous Agent}
}

\newglossaryentry{hank}{
name=HANK,
description={Heterogeneous Agent New Keynesian}
}

\newglossaryentry{gfc}{
name=GFC,
description={Great Financial Crisis}
}

\newglossaryentry{ltv}{
name=LTV,
description={Loan-to-Value}
}

\newglossaryentry{dsge}{
name=DSGE,
description={Dynamic Stochastic General Equilibrium}
}

\newglossaryentry{egm}{
name=EGM,
description={Endogenous Gridpoints Method}
}

\newglossaryentry{iou}{
name=IOU,
description={I Owe You}
}

%%%%%%%%%%%%%%%%%%%%%%%%%%%%%%%%%%%%%%%%%%%%%%%%%%%%%%%%%%%%%%%%%%%%%%%%%%%%%
%%%%%%%%%%%%%%%%%%%%%%%%%%%%%%%%%%%%%%%%%%%%%%%%%%%%%%%%%%%%%%%%%%%%%%%%%%%%%
%%%%%%%%%%%%%%%%%%%%%%%%%%%%%%%%%%%%%%%%%%%%%%%%%%%%%%%%%%%%%%%%%%%%%%%%%%%%%
% hyperref package
\usepackage{hyperref} % Load after everything else
\usepackage{footnotebackref} % Brings you back to where the footnote is in the text
\hypersetup{ 
    colorlinks,
    linkcolor=blue,
    filecolor=blue,  
    citecolor=blue,
    urlcolor=blue} % Setup with the hyperref package

%%%%%%%%%%%%%%%%%%%%%%%%%%%%%%%%%%%%%%%%%%%%%%%%%%%%%%%%%%%%%%%%%%%%%%%%%%%%%
%%%%%%%%%%%%%%%%%%%%%%%%%%%%%%%%%%%%%%%%%%%%%%%%%%%%%%%%%%%%%%%%%%%%%%%%%%%%%
%%%%%%%%%%%%%%%%%%%%%%%%%%%%%%%%%%%%%%%%%%%%%%%%%%%%%%%%%%%%%%%%%%%%%%%%%%%%%
% Formalities
\title{Financial Constraints and Household Heterogeneity\\in the Macroeconomy}
\author{}
\date{}

%%%%%%%%%%%%%%%%%%%%%%%%%%%%%%%%%%%%%%%%%%%%%%%%%%%%%%%%%%%%%%%%%%%%%%%%%%%%%
%%%%%%%%%%%%%%%%%%%%%%%%%%%%%%%%%%%%%%%%%%%%%%%%%%%%%%%%%%%%%%%%%%%%%%%%%%%%%
%%%%%%%%%%%%%%%%%%%%%%%%%%%%%%%%%%%%%%%%%%%%%%%%%%%%%%%%%%%%%%%%%%%%%%%%%%%%%
%%%%%%%%%%%%%%%%%%%%%%%%%%%%%%%%%%%%%%%%%%%%%%%%%%%%%%%%%%%%%%%%%%%%%%%%%%%%%
%%%%%%%%%%%%%%%%%%%%%%%%%%%%%%%%%%%%%%%%%%%%%%%%%%%%%%%%%%%%%%%%%%%%%%%%%%%%%
%%%%%%%%%%%%%%%%%%%%%%%%%%%%%%%%%%%%%%%%%%%%%%%%%%%%%%%%%%%%%%%%%%%%%%%%%%%%%
% Begin Document
\begin{document}
\selectlanguage{british}
\pagenumbering{gobble} % Turn page numbering off
\maketitle % Create title

\vspace{4cm}
\begin{center}
Master Thesis Presented to the\\
Department of Economics at the\\
Rheinische Friedrich-Wilhelms-Universität Bonn\\
\vspace{1cm}
In Partial Fulfillment of the Requirements for the Degree of\\
Master of Science (M.Sc.)
\end{center}

\vspace{5cm}
\begin{center}
\begin{tabular}{ l l }
Supervisors: & Prof.~Dr.~Keith Kuester \\
& Dr.~Gregor Böhl \\
& \\
& \\
Date of Submission: & \printdate{2023-8-25}\\
Author: & Andreas Koundouros\\
Matriculation Number: & 3466868
\end{tabular}
\end{center}

%%%%%%%%%%%%%%%%%%%%%%%%%%%%%%%%%%%%%%%%%%%%%%%%%%%%%%%%%%%%%%%%%%%%%%%%%%%%%
%%%%%%%%%%%%%%%%%%%%%%%%%%%%%%%%%%%%%%%%%%%%%%%%%%%%%%%%%%%%%%%%%%%%%%%%%%%%%
%%%%%%%%%%%%%%%%%%%%%%%%%%%%%%%%%%%%%%%%%%%%%%%%%%%%%%%%%%%%%%%%%%%%%%%%%%%%%
\newpage
\pagenumbering{roman} % Turn page numbering to small roman
{ \hypersetup{hidelinks} \tableofcontents } % Hide links in table of contents

\newpage
{ \hypersetup{hidelinks} \listoffigures } % Hide links in list of figures
{ \hypersetup{hidelinks} \listoftables } % Hide links in list of tables

% Add here list of abbreviations and list of symbols 
\newpage
\setlist[description]{leftmargin=!, labelwidth=5em} % Change for glossaries
\printglossary[title=List of Abbreviations]
\setlist[description]{style=standard} % Reset settings back to default


%%%%%%%%%%%%%%%%%%%%%%%%%%%%%%%%%%%%%%%%%%%%%%%%%%%%%%%%%%%%%%%%%%%%%%%%%%%%%
%%%%%%%%%%%%%%%%%%%%%%%%%%%%%%%%%%%%%%%%%%%%%%%%%%%%%%%%%%%%%%%%%%%%%%%%%%%%%
%%%%%%%%%%%%%%%%%%%%%%%%%%%%%%%%%%%%%%%%%%%%%%%%%%%%%%%%%%%%%%%%%%%%%%%%%%%%%
\newpage
\begin{abstract} % Create abstract
Abstract goes here.

\begin{center}
\rule{10cm}{0.4pt}
\end{center}

% \bigskip
\noindent
\textit{JEL Classification:}

\noindent
\textit{Keywords:} Heterogeneous Households, Borrowing Limits, Household Debt, Deleveraging
\end{abstract}

%%%%%%%%%%%%%%%%%%%%%%%%%%%%%%%%%%%%%%%%%%%%%%%%%%%%%%%%%%%%%%%%%%%%%%%%%%%%%
%%%%%%%%%%%%%%%%%%%%%%%%%%%%%%%%%%%%%%%%%%%%%%%%%%%%%%%%%%%%%%%%%%%%%%%%%%%%%
%%%%%%%%%%%%%%%%%%%%%%%%%%%%%%%%%%%%%%%%%%%%%%%%%%%%%%%%%%%%%%%%%%%%%%%%%%%%%
\newpage
\pagenumbering{arabic} % Turn page numbering to small arabic
\pagestyle{fancy}
\fancyhf{}
\fancyhead[C]{\leftmark}
\fancyfoot[C]{\thepage}

%%%%%%%%%%%%%%%%%%%%%%%%%%%%%%%%%%%%%%%%%%%%%%%%%%%%%%%%%%%%%%%%%%%%%%%%%%%%%
%%%%%%%%%%%%%%%%%%%%%%%%%%%%%%%%%%%%%%%%%%%%%%%%%%%%%%%%%%%%%%%%%%%%%%%%%%%%%
\thispagestyle{plain}
\section{Introduction}
\label{sec:introduction}

Households face limits to borrowing. For example, there is usually a limit to how much a credit card holder can spend with one such card. Also, home buyers usually are not able to take loans at the same value of the house they want to purchase. Clearly, such borrowing constraints vary across at least three dimensions: first, household characteristics (e.g.~income, wealth, credit-worthiness), second, asset type (e.g.~mortgage debt vs credit card debt) and third, time.\footnote{This is by no means meant to be a formal or sufficient description of dimensions along which borrowing limits may vary.}

Financial constraints as the ones described above have important implications for the behaviour of households.\footnote{Of course, firms and governments can be subject to borrowing constraints, too. The focus of this thesis, however, is on households.} 

How do borrowers and lenders interact in environments with different and time-varying borrowing limits and what do those interactions imply for the aggregate?

In the absence of complete insurance markets, the presence of borrowing limits constrains households in their ability to smooth shocks to consumption. Moreover, borrowing limits combined with idiosyncratic income risk induce precautionary savings, i.e.~a build-up of assets in order to prevent running into the borrowing constraints. 

Borrowing limits play an important role not only for individuals but for the aggregate economy as well.

The following question arises: how does time-variation in those limits affect the economic outcomes on aggregate and household levels? To answer this question, this thesis follows and extends the work by \textcite{gl2017} and analyses shocks to the household-level borrowing limits in a heterogeneous agent New Keynesian model (\Gls{hank}).\footnote{The term \enquote{HANK} was coined by \textcite{kaplan2018}.}

This limit to borrowing is usually much harsher than the so-called natural borrowing limit, which is commonly defined as the expected net present value of an individual's income stream. 

Since at least the Great Depression economists have aknowledged the importance of debt. The Great Financial Crisis (\Gls{gfc}) was a vivid reminder of how developments in finance, including credit and limits to credit, affect households and the macroeconomy. 

A heterogeneous-agent (\Gls{ha}) model naturally lends itself to the analysis of household-level borrowing limits for at least two reasons. First, models which neglect or only parsimoniously model household heterogeneity cannot speak to interactions between borrowers and lenders.

Debt is inherently connected with heterogeneity as borrowers and lenders crucially differ in their nature.

First, unlike their representative-agent counterparts, prototypical \Gls{ha} models explicitly incorporate borrowing limits. Second, \Gls{ha} models speak to the differential effect of borrowing constraints across the distribution of households.

The present model is not so much about the \textit{origins} of financial constraints in form of borrowing limits but rather on the \textit{effects} of changes in such limits and their differential consequences across household groups (i.e.~borrowers and lenders).

% I further include some financial constraints on the side of the firms and banks as done in \textcite{zoch2022}.

The model and its dynamics are solved with the Python package \texttt{Econpizza} by \textcite{boehl2023econpizza} and the methods therein.

\subsubsection*{Outline}
\label{sec:outline}
The remainder of this thesis is structured as follows. Section \ref{sec:literature} reviews the existing literature on financial constraints and household heterogeneity in the macroeconomy. Section \ref{sec:model} lays down the macroeconomic model employed in the present analysis of shocks to household-level borrowing limits. The numerical implementation and calibration of the model is discussed in section \ref{sec:implementation}. Thereafter, section \ref{sec:results} presents the results on the steady states of lose and tight credit, respectively, as well as the results on the aggregate and disaggregated transitions between those steady states. Section \ref{sec:conclusion} concludes.

%%%%%%%%%%%%%%%%%%%%%%%%%%%%%%%%%%%%%%%%%%%%%%%%%%%%%%%%%%%%%%%%%%%%%%%%%%%%%
%%%%%%%%%%%%%%%%%%%%%%%%%%%%%%%%%%%%%%%%%%%%%%%%%%%%%%%%%%%%%%%%%%%%%%%%%%%%%
\section{Related Literature}
\label{sec:literature}

This section reviews the literature which this thesis is most closely related to. I briefly dwell on the advances the macroeconomic literature has accomplished towards considering financial frictions and household heterogeneity, respectively. Then, I highlight the results of studies which analysed borrowing constraints and their effects on the macroeconomy.

In the real world, it is not only households who face financial constraints. Other agents, such as firms and financial intermediaries, can be subject to limits to their borrowing capacity.

BG and BGG are among the earliest and most widely used approaches to the study of financial constraints on the firm side.

GK focus on the economy's financial intermediaries. In their framework, banks are limited in the amount of leverage they can hold because of the possibility of banks' owners to liquidate the business.

Heterogeneous banks are modeled in the paper Bewley banks by.

The literature has investigated shocks to borrowing constraints in a macroeconomic setting before, see e.g.~\textcite{kv2018}. 

Shocks to borrowing constrains have been considered in the literature before. This subsection reviews some of these works.

At least since \textcite{fisher1933} macroeconomists have been thinking about debt and its influence on features and developments of the macroeconomy.

An early empirircal paper on the importance of borrowing constraints for conusmer behaviour is \textcite{gross2002}.

HANK models fuse two basic modelling approaches in macroeconomics. The first is the incomplete-markets framework, pioneered \textit{inter alia} by \textcite{bewley1986}, \textcite{im1989}, \textcite{huggett1993} and \textcite{aiyagari1994}.\footnote{\textcite{im1989} compares the consumption costs of business cycles in imperfect-insurance models to the costs implied by perfect-insurance models and finds that the former exhibits higher costs than the latter in the case of no borrowing and lower costs when borrowing is permitted. \textcite{aiyagari1994} studies asset market equilibrium where asset supply is given by households who are subject to labour income risk and asset demand by firms with a neoclassical production function. The result is that the degree by which the presence of idiosynratic risk influences aggregate savings and the equilibrium interest rate depends on the calibration of the parameters governing the desire for precautionary savings.} The second one, pertaining to the aggregate economy in the model, is the New Keynsian dynamic stochastic and general equilibrium (\Gls{dsge}) framework.

\textcite{gl2017} build a household-focused incomplete-markets model and find that a permanent tightening in the household-level borrowing constraint evokes deleveraging and increased precuationary savings. Combined, these two effects depress output and interest rates in the economy. The recession is aggravated and the economy falls into a liquidity trap when the model is augmented with nominal wage rigidities and the zero lower bound. 

Relative to \textcite{gl2017}, I add an aggregate state space, featuring production by firms, a \Gls{nk} Phillips Curve and monetary policy which follows the Taylor rule.

\textcite{egg2012} analyse a permanent shock to an exogenously set debt limit in a two-agent economy with a constant fraction of borrowers and lenders. The advantage of this approach is that the authors can derive analytical results. The paper shows that reductions in the borrowing limit can significantly affect macroeconomic outcomes; for example, the economy can end up in a liquidity trap.

Importantly, \textcite{egg2012} consider real debt, which leads to a debt-deflation spiral as in \textcite{fisher1933}.

\textcite{egg2012} also find that a deleveraging shock can be completely undone by fiscal policy by targeting short-run tax cuts and long-run tax increases at the borrowing fraction of the population, thereby effectively using its own ability to borrow to help the borrwers who are affected by the deleveraging shock. This result, however, relies very much on the assumption that the debt limit is exogenously set.

As \textcite{egg2012} note, it is reasonable to assume that the debt limit depends to some degree on current conditions, such as current income.

\textcite{heath2009} provide an overview of modelling heterogenous households in quantitative macroeconomics. See \textcite{cherrier2023} for a historical overview of the advances and different approaches towards incorporating heterogeneous agents into macroeconomic analysis in the 1980s and 1990s.

\textcite{riosrull2015} have an exogenous interest rate and two types of goods.

%%%%%%%%%%%%%%%%%%%%%%%%%%%%%%%%%%%%%%%%%%%%%%%%%%%%%%%%%%%%%%%%%%%%%%%%%%%%%
%%%%%%%%%%%%%%%%%%%%%%%%%%%%%%%%%%%%%%%%%%%%%%%%%%%%%%%%%%%%%%%%%%%%%%%%%%%%%
\section{Model}
\label{sec:model}

This section puts forward the model used to analyse shocks to borrowing constraints. Time, denoted by $t$, is discrete and refers to quarters of a year.

\subsection{Household Sector}
\label{sec:model-hh}

The model allows households explicitly to lend and to borrow. This is at the heart of the present analysis. Thus, the households can issue unsecured \enquote{I Owe You}'s (\Gls{iou}).

Default is not considered in this model environment.

Households $i$ are ex-ante identical but subject to idiosyncratic, i.e.~household-level, income shocks.\footnote{As \textcite{cherrier2023} note, there has been a strand of the literature considering also ex-ante heterogeneity in addition to ex-post heterogeneity. Doing so is beyond the scope of the present thesis.} Markets are incomplete and thus, households can only self-insure by saving and borrowing in a one-period, rsik-free bond. This bond is the only asset in the economy and is perfectly liquid, i.e.~there are no costs involved when purchasing or selling bonds. The bond is to be understood as unsecured borrowing.\footnote{Secured borrowing would be collateralised.}

Households have utility function $U$, which is strictly increasing in household consumption $c_i$. Households choose consumption, labour and asset holdings so as to maximise their utiliy, given the prices they face in the economy.

The bond is supplied by households (i.e.~when households lend to and borrow from each other) and by the government. %and is in zero net supply.

Thus, I use wealth always to refer to liqiud wealth, which is the only form of wealth available in this setup, while being aware that wealth in reality comprises illiquid wealth, too.

A note on the asset structure is in order. The one-asset framework seems to be in contradiction to what we find in reality, where households can choose from a variety of assets to self-insure. What is more, there is government insurance and insurance from employers, see Heathcote et al.

\textcite{kaplan2018} provide an insightful critique of the one-asset structure. 

Note that in incomplete markets models it is required that the interest rate times the discount fatcor is smaller than one.

Negative asset holdings are subject to wasteful intermediation costs, $\bar{R}$. Thus, the aggregate constraint must look different too, i.e.~$C + \bar{R} debt = Y$, as also seen in \textcite{kaplan2018}.

\subsection{Remaining Sectors and Equilibrium}
\label{sec:model-sectors}

\subsubsection{Firms}
\label{sec:model-firms}

Firm profits bear the problem that they have to be distributed. The way this is done might matter for households and their differential consumption-savings behaviour. 

\subsubsection{Fiscal and Monetary Policies}
\label{sec:model-policy}

The monetary authority sets the notional gross nominal interest rate according to a standard Taylor rule which features persistence and which responds to the deviations of inflation from its steady state and of output from its lagged value. The strengths of these responses are given by the parameters $\phi_{\pi}$ and $\phi_y$, respectively. In setting the notional interest rate, the central bank is subject to the zero lower bound on nominal interest rates. Thus, equation \eqref{eq:zlb} pins down the effective nominal interest rate by selecting the maximum out of the notional gross nominal interest rate and 1:
\begin{align}
    R_{n,t} &= ( R_{n,t-1} )^{\rho} \Bigg( R_{ss} \Bigg( \frac{\pi_t}{\pi_{ss}} \Bigg)^{\phi_{\pi}} \Bigg( \frac{y_t}{y_{t-1}} \Bigg)^{\phi_y} \Bigg)^{(1-\rho)} \label{eq:taylorrule} \\
    R_t &= \text{max} \ \{ 1, \ R_{n,t} \}, \label{eq:zlb}
\end{align}
where $\rho$ is the persistence in the response of the central bank.

\subsubsection{Equilibrium}
\label{sec:model-eq}

\subsection{Model Extensions}
\label{sec:model-extensions}

\subsubsection{Nominal Debt}
\label{sec:model-extensions-realdebt}

Another model extension is to consider nominal instead of real, i.e.~inflation-indexed, debt. 

%%%%%%%%%%%%%%%%%%%%%%%%%%%%%%%%%%%%%%%%%%%%%%%%%%%%%%%%%%%%%%%%%%%%%%%%%%%%%
%%%%%%%%%%%%%%%%%%%%%%%%%%%%%%%%%%%%%%%%%%%%%%%%%%%%%%%%%%%%%%%%%%%%%%%%%%%%%
\section{Shocks to Financial Constraints}
\label{sec:shocks}

This section discusses the shock to the borrowing limit that I confront the above model with and sketches some consequences for the model's dynamics one might expect.

A shock to the model is effectively an exogenous shock to the supply of IOUs. This can be due to a lot of reasons, \textit{inter alia} due to lost confidence in the ability of borrowers to repay their debt.

As noted by \textcite{gl2017}, this shock has the effect of forcing some to save more and the desire of others to save more, too. This reduced the bond market-clearing real interest rate. However, note that this reduction again incentivises consumption.

%%%%%%%%%%%%%%%%%%%%%%%%%%%%%%%%%%%%%%%%%%%%%%%%%%%%%%%%%%%%%%%%%%%%%%%%%%%%%
%%%%%%%%%%%%%%%%%%%%%%%%%%%%%%%%%%%%%%%%%%%%%%%%%%%%%%%%%%%%%%%%%%%%%%%%%%%%%
\section{Implementation and Calibration}
\label{sec:implementation-calibration}

\subsubsection*{Numerical Implementation}
\label{sec:implementation}

As is well-known to the macroeconomic literature, models with heterogeneous agents are notoriously difficult to solve as the computation of steady states as well as transition dynamics involves the infinite-dimensional distribution. Overcoming this challenge has always involved trade-offs. The approach here is to use the \texttt{Econpizza} package by \textcite{boehl2023econpizza}, which implements a perfect-foresight approach for representative- and heteroegenous-agent models.

The heterogeneous-agent block of the model is solved using the endogenous-gridpoints method (\Gls{egm}) by \textcite{carroll2006}.

The grid is not allowed to change over time as this mimeo notes. Otherwise, the interpolation would be between repeatedly different grid points, making the comparison of over time inaccurate. 

The purpose of the present exercise is not to accurately match all moments of the empirical data but rather to flesh out the main mechanisms by which a shock to the borrowing limit affects individual-level and aggregate behaviour.

As \textcite{auclert2021} note, it is important to calculate the transition using the Jacobian of the terminal steady state.

Note that the perfect solution implies that aggregates move deterministically and agents know this perfectly.

\subsubsection*{Calibration of Model Parameters}
\label{sec:calibration}

Having presented the details of the model and its practical implementation, I now turn to the results upon a shock to the borrowing limit as discussed in \ref{sec:shocks}.

\section{Results}
\label{sec:results}

Note however that the responses in \textcite{gl2017} are also not very persistent. Their model also lacks capital, hanit formation or adjustment costs.

\subsection{Comparison of Steady States}
\label{sec:results-stst}

\begin{figure}[H]
    \centering
    \caption{Consumption Policies Across Steady States}
    \label{fig:c-policies-across-stst}
    \begin{subfigure}[b]{0.49\textwidth}
         \centering
         \caption{Initial Steady State}
         \includesvg[inkscapelatex=false, width=\textwidth]{Initial/stst_consumption_policies.svg}
         \label{fig:c-policies-initial}
     \end{subfigure}
     \hfill
     \begin{subfigure}[b]{0.49\textwidth}
         \centering
         \caption{Terminal Steady State}
         \includesvg[inkscapelatex=false, width=\textwidth]{Terminal/stst_consumption_policies.svg}
         \label{fig:c-policies-terminal}
     \end{subfigure}
    \begin{footnotesize}
	\begin{flushleft}
	%\textit{Note}: All values in percent deviations from the respective steady state. The time horizon is fixed to 50 quarters.
	\end{flushleft}
	\end{footnotesize}
\end{figure}

\begin{table}
\centering
\caption{Comparison of Steady States}
\label{tab:stst}
\begin{tabular}{lccc}
\hline
     Variable &  Initial Steady State &  Terminal Steady State &  Percent Change \\
\hline
            B &                1.6000 &                 1.6000 &            0.00 \\
       Bot10A &               -0.0469 &                -0.0312 &          -33.48 \\
       Bot10C &                0.0325 &                 0.0326 &            0.31 \\
            C &                1.0000 &                 1.0000 &            0.00 \\
        D\_o\_Y &                0.3013 &                 0.1928 &          -36.01 \\
          div &                0.1815 &                 0.1815 &            0.00 \\
lower\_bound\_a &               -0.7500 &                -0.5000 &          -33.33 \\
          MPC &                0.1670 &                 0.1715 &            2.69 \\
            n &                0.9354 &                 0.9354 &            0.00 \\
           pi &                1.0000 &                 1.0000 &            0.00 \\
            R &                1.0020 &                 1.0003 &           -0.17 \\
           Rn &                1.0020 &                 1.0003 &           -0.17 \\
           Rr &                1.0020 &                 1.0003 &           -0.17 \\
          tax &                0.0032 &                 0.0004 &          -87.50 \\
       Top10A &                0.7141 &                 0.6656 &           -6.79 \\
       Top10C &                0.2186 &                 0.2189 &            0.14 \\
        Top1A &                0.1531 &                 0.1434 &           -6.34 \\
        Top1C &                0.0317 &                 0.0317 &            0.00 \\
            w &                0.8750 &                 0.8750 &            0.00 \\
            y &                1.0000 &                 1.0000 &            0.00 \\
       y\_prod &                1.0000 &                 1.0000 &            0.00 \\
            z &                1.0690 &                 1.0690 &            0.00 \\
\hline
\end{tabular}
\end{table}

\subsection{Transitional Dynamics}
\label{sec:results-transition}

\subsubsection{Aggregate Dynamics}
\label{sec:results-transition-agg}

\subsubsection{Individual-Level Dynamics}
\label{sec:results-transition-ind}

\subsection{Sensitivity Analyses}
\label{sec:results-sensitivity}

%%%%%%%%%%%%%%%%%%%%%%%%%%%%%%%%%%%%%%%%%%%%%%%%%%%%%%%%%%%%%%%%%%%%%%%%%%%%%
%%%%%%%%%%%%%%%%%%%%%%%%%%%%%%%%%%%%%%%%%%%%%%%%%%%%%%%%%%%%%%%%%%%%%%%%%%%%%
\section{Conclusion}
\label{sec:conclusion}

Would be interesting to extend the model so  that there is also borrowing by other entities, especially firms.

An interesting extension of the model would be a small open economy as then funds could potentially be obtained elsewhere.

Borrowing limits are, of course, endogenous outcomes. The present approach abstracted from this endogeneity. Uncovering the \textit{sources} of time-variations in the borrowing limits (also in the cross-section) requires addressing the endogeneity and explicitly modelling borrowing limits as an outcome of decisions by market participants.

%%%%%%%%%%%%%%%%%%%%%%%%%%%%%%%%%%%%%%%%%%%%%%%%%%%%%%%%%%%%%%%%%%%%%%%%%%%%%
%%%%%%%%%%%%%%%%%%%%%%%%%%%%%%%%%%%%%%%%%%%%%%%%%%%%%%%%%%%%%%%%%%%%%%%%%%%%%
%%%%%%%%%%%%%%%%%%%%%%%%%%%%%%%%%%%%%%%%%%%%%%%%%%%%%%%%%%%%%%%%%%%%%%%%%%%%%
% References section
\newpage
\thispagestyle{plain}
\pagenumbering{Roman}
\printbibliography[heading=bibintoc] % Insert references

%%%%%%%%%%%%%%%%%%%%%%%%%%%%%%%%%%%%%%%%%%%%%%%%%%%%%%%%%%%%%%%%%%%%%%%%%%%%%
%%%%%%%%%%%%%%%%%%%%%%%%%%%%%%%%%%%%%%%%%%%%%%%%%%%%%%%%%%%%%%%%%%%%%%%%%%%%%
%%%%%%%%%%%%%%%%%%%%%%%%%%%%%%%%%%%%%%%%%%%%%%%%%%%%%%%%%%%%%%%%%%%%%%%%%%%%%
% Appendices section
\newpage
\thispagestyle{plain}
\pagenumbering{arabic}  % Turn page numbering to arabic
\renewcommand*{\thepage}{A-\arabic{page}} % Add 'A' to each page number for appendices section
\begin{appendices}
%%%%%%%%%%%%%%%%%%%%%%%%%%%%%%%%%%%%%%%%%%%%%%%%%%%%%%%%%%%%%%%%%%%%%%%%%%%%%
%%%%%%%%%%%%%%%%%%%%%%%%%%%%%%%%%%%%%%%%%%%%%%%%%%%%%%%%%%%%%%%%%%%%%%%%%%%%%
\section{Full Model}
\label{sec-app:full}

%%%%%%%%%%%%%%%%%%%%%%%%%%%%%%%%%%%%%%%%%%%%%%%%%%%%%%%%%%%%%%%%%%%%%%%%%%%%%
%%%%%%%%%%%%%%%%%%%%%%%%%%%%%%%%%%%%%%%%%%%%%%%%%%%%%%%%%%%%%%%%%%%%%%%%%%%%%
\newpage
\thispagestyle{plain}
\section{Description of Computer Codes}
\label{sec-app:codes}

%%%%%%%%%%%%%%%%%%%%%%%%%%%%%%%%%%%%%%%%%%%%%%%%%%%%%%%%%%%%%%%%%%%%%%%%%%%%%
%%%%%%%%%%%%%%%%%%%%%%%%%%%%%%%%%%%%%%%%%%%%%%%%%%%%%%%%%%%%%%%%%%%%%%%%%%%%%
\newpage
\thispagestyle{plain}
\section{Analogy of the Borrowing Limit Shock to an Income Risk Shock}
\label{sec-app:income}
The literature has given quite some attention to shocks to labour income risk in \Gls{hank} models in order to produce fluctuations which resemble business cycles. In this appendix, I show that a permanent shock to labour income risk produces impulse responses which are similar to the ones obtained from the analysis of a permanent tightening of the borrowing limit.

\end{appendices}

%%%%%%%%%%%%%%%%%%%%%%%%%%%%%%%%%%%%%%%%%%%%%%%%%%%%%%%%%%%%%%%%%%%%%%%%%%%%%
%%%%%%%%%%%%%%%%%%%%%%%%%%%%%%%%%%%%%%%%%%%%%%%%%%%%%%%%%%%%%%%%%%%%%%%%%%%%%
%%%%%%%%%%%%%%%%%%%%%%%%%%%%%%%%%%%%%%%%%%%%%%%%%%%%%%%%%%%%%%%%%%%%%%%%%%%%%
\newpage
\thispagestyle{plain}
\pagenumbering{gobble} % Turn page numbering off
\section*{Statement of Authorship} % Include statement of authorship
I hereby confirm that the work presented has been performed and interpreted solely by myself except for where I explicitly identified the contrary. I assure that this work has not been presented in any other form for the fulfillment of any other degree or qualification. Ideas taken from other works in letter and in spirit are identified in every single case.

\vspace{2cm}
\noindent
\rule{8cm}{0.4pt}\\
Andreas Koundouros\\
Bonn, the \printdate{2023-8-25}
\end{document}
% End Document
%%%%%%%%%%%%%%%%%%%%%%%%%%%%%%%%%%%%%%%%%%%%%%%%%%%%%%%%%%%%%%%%%%%%%%%%%%%%%
%%%%%%%%%%%%%%%%%%%%%%%%%%%%%%%%%%%%%%%%%%%%%%%%%%%%%%%%%%%%%%%%%%%%%%%%%%%%%
%%%%%%%%%%%%%%%%%%%%%%%%%%%%%%%%%%%%%%%%%%%%%%%%%%%%%%%%%%%%%%%%%%%%%%%%%%%%%
%%%%%%%%%%%%%%%%%%%%%%%%%%%%%%%%%%%%%%%%%%%%%%%%%%%%%%%%%%%%%%%%%%%%%%%%%%%%%
%%%%%%%%%%%%%%%%%%%%%%%%%%%%%%%%%%%%%%%%%%%%%%%%%%%%%%%%%%%%%%%%%%%%%%%%%%%%%
%%%%%%%%%%%%%%%%%%%%%%%%%%%%%%%%%%%%%%%%%%%%%%%%%%%%%%%%%%%%%%%%%%%%%%%%%%%%%
